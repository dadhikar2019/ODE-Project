

\chapter{Differentiation Under the Integral Sign}\label{Leibnitz Rule-Improper Integrals}
We start with the following basic result from the multivariate calculus.

\begin{theorem}[Fubini Theorem-Version I]\label{fubini-v1}
	Let $f(x, y)$ be a continuous function on a rectangle $R=\{(x, y): a\le x \le b, \; c\le  y\le d\}.$ Then %$f$ is integrable on $R$ and its double integral on $R$ equals
	\[\int_a^b \int_c^d  f(x, y)\, dx\, dy = \int_c^d \int_a^b f(x, y) \, dy\, dx.\]
\end{theorem}
As an application of Theorem~\ref{fubini-v1} and fundamental theorem of calculus, we have a basic result for differentiating under the integral sign.

\begin{theorem}[Leibniz Rule]\label{DUI-v1}
	Let $f(x, y)$ be a  function on a rectangle $R=\{(x, y): a\le x \le b, \; c\le  y\le d\}$ such that both $f$ and $\frac{\partial f}{\partial y}$ are continuous on $R.$ Then 
	\[\frac{d}{dy}\int_a^b  f(x, y)\, dx =  \int_a^b \frac{\partial f}{\partial y}(x, y) \,  dx.\]
\end{theorem}
\begin{proof}
Applying Theorem~\ref{fubini-v1} to $\frac{\partial f}{\partial y},$ we have
\[\frac{d}{dy}\int_c^y \int_a^b \frac{\partial f}{\partial z}(x, z) \, dz\, dx=\frac{d}{dy}\int_a^b \int_c^y  \frac{\partial f}{\partial z}(x, z)\, dx\, dz \]
for all $y\in [c, d].$  Applying the fundamental theorem of calculus, we obtain
\[\int_a^b \frac{\partial f}{\partial y}(x, y) dx =\frac{d}{dy}\int_a^b   \big(f(x, y) - f(x, c)\big)\, dx  =\frac{d}{dy}\int_a^b   f(x, y) \, dx. \qedhere\]
\end{proof}

\begin{theorem}[General Leibniz Rule]\label{thm:DUI-v2}
	Let $f(x, y)$ be a  function on a rectangle $R=\{(x, y): a\le x \le b, \; c\le  y\le d\}$ such that both $f$ and $\frac{\partial f}{\partial y}$ are continuous on $R.$  Let $a$ and $b$ be differentiable functions on $[c, d]$ such that $a\le a(y), b(y) \le b$ for all $y$ in $[c, d].$
	Then 
	\[\frac{d}{dy}\int_{a(y)}^{b(y)}  f(x, y)\, dx =   b'(y)\, f(b(y), y) - a'(y) \,f((y), y)+\int_{a(y)}^{b(y)} \frac{\partial f}{\partial y}(x, y) \,  dx.\]
\end{theorem}

\begin{proof}
	Define  a function $J:  [a, b]\times [a, b]\times [c, d]\to \mathbf R$ by
	\[J(u, v, y) = \int_u^v f(x, y)\, dx. \]
	Then, by Theorem~\ref{DUI-v1}, we have
	\[\frac{\partial J}{\partial y} = \int_u^v \frac{\partial f}{\partial y} f(x, y) \, dx.\]
	Also, we have
	\[\frac{\partial J}{\partial u}=  - f(u, y) \quad\text{ and }\quad 
	\frac{\partial J}{\partial v}=   f(v, y).\]
	By the chain rule, we have
	\[\begin{split}
		\frac{dJ}{dy} (a(y), b(y), y)& =\frac{\partial J}{\partial u} (a(y), b(y), y)\, a'(y) + \frac{\partial J}{\partial v} (a(y), b(y), y)\, b'(y) \\
		&+ \frac{\partial J}{\partial y} (a(y), b(y), y),
	\end{split}
	\]
	which gives
		\[\frac{d}{dy}\int_{a(y)}^{b(y)}  f(x, y)\, dx =   b'(y) f(b(y), y) - a'(y) f(a(y), y) +\int_{a(y)}^{b(y)} \frac{\partial f}{\partial y}(x, y) \,  dx.\qedhere\]
	\end{proof}
	
	\begin{theorem}[Leibniz Rule for Improper Integrals]\label{thm:Leibnitz Improper Integrals}
		Let $f(x, y)$ be a  function on a rectangle $R=\{(x, y): a\le x <\infty, \; c\le  y\le d\}$ such that, for each $y$ in $I,$ both $f$ and $\frac{\partial f}{\partial y}$ have  improper integrals on $[a, \infty)$ and such that, for each $x$ in $[a, \infty),$ $f(x, y)$ is differentiable in $y$ for each $y$ in $I.$ Suppose, further, that there exists a function $g$ on $[a, \infty)$ such that 
		\[\abs{\frac{\partial f}{\partial y}(x, y)}\le g(x)\] for all $x\in [a, \infty)$ and for all $y$ in $I$ and such that 
		\[\int_a^\infty g(x) \,dx\] is convergent.
	Then 
	\[\frac{d}{dy}\int_a^\infty  f(x, y)\, dx =  \int_a^\infty \frac{\partial f}{\partial y}(x, y) \,  dx.\]
	\end{theorem}
	

	The dominated convergence theorem (whose proof is beyond the scope of this book) for improper integrals over $[a, \infty)$ is required to prove Theorem~\ref{thm:Leibnitz Improper Integrals}. The reader may consult the book\footnote{J.~W. Lewin and M. Lewin, {\it An introduction to mathematical analysis}, The Random House/Birkh\"auser Mathematics Series, Random House, New York, 1988; MR1019088} for the dominated convergence theorem for improper integrals.





\chapter{Partial Fraction Decomposition}\label{partial-fraction-decomposition}


%
%% --------------------------------------------------------------------
%% Packages
%% --------------------------------------------------------------------
%\usepackage{amsmath,amssymb,amsfonts}
%\usepackage{enumitem}
%\usepackage{tikz}
%\usetikzlibrary{arrows.meta, positioning, shapes.geometric}
%
%% Hyperbolic operators (for consistency in examples)
%\DeclareMathOperator{\sech}{sech}
%\DeclareMathOperator{\csch}{csch}
%
%% Counter for auto-numbered items
%\newcounter{tableitem}
%\newcommand{\resetitems}{\setcounter{tableitem}{0}}
%\newcommand{\nextitem}{\stepcounter{tableitem}\arabic{tableitem}.}
%
%\section*{Summary of Partial Fraction Decomposition Techniques}

Let $P$ and $Q$ be general polynomials in $x,$  and let $m$ and $n$ be  the degrees of $P$ and $Q$, respectively. Then the \textbf{\textit{rational}} function
\[
R(x)=\frac{P(x)}{Q(x)}
\]
is said to be \textbf{\textit{proper }} if \(m<n\) and \textbf{\textit{improper}} if  \(m\ge n.\)
When  $R(x)$ is improper, we   perform polynomial division to obtain
\[
R(x)=q(x)+\frac{r(x)}{Q(x)},
\]
where $q(x)$ is the \textbf{\textit{quotient}} polynomial of degree  $m-n,$ and $r(x)$ is called the \textbf{\textit{remainder}} polynomial and it is of degree less than $n.$ For example, if $P(x) = x^3, Q(x) = x^2-1,$ then 
\[\frac{x^3+2}{x^2-1} = x+ \frac{x+2}{x^2-1},\] so that \(q(x) = x\) and $r(x) = x+2.$

After performing polynomial division, the term \(q(x)\) is already a polynomial and requires no  decomposition. The  term \(\dfrac{r(x)}{Q(x)}\) is a proper rational function and the only part of $R(x)$ that is usually decomposed into partial fractions. Therefore, in what follows, we assume that \(R(x)\) itself is  proper. The  method of partial fraction decomposition  then depends entirely  on the factorization of the
denominator $Q(x)$.  We now examine  all possible cases that may occur.

\bigskip

%--------------------------------------------------------------
\subsection*{1. Linear Factors \texorpdfstring{$(x-a)$}{(x-a)}}

If the denominator contains distinct linear factors,
\[
Q(x)=(x-a_1)(x-a_2)\cdots(x-a_n),
\]
then the decomposition has the form
\[
\frac{P(x)}{Q(x)}
= \frac{A_1}{x-a_1}+\frac{A_2}{x-a_2}+\cdots+\frac{A_n}{x-a_n},
\]
where  $A_1, \dots, A_n$ are constants to be determined. Let us illustrate this with an example below.

\paragraph{Example 1.} Perform  partial fraction decomposition of 
\[
\frac{3x+5}{(x-1)(x+2)}.
\]
Suppose there are constants $A$ and $B$ such that 
\[
\frac{3x+5}{(x-1)(x+2)}
= \frac{A}{x-1}+\frac{B}{x+2}.
\]
Multiplying through by $(x-1)(x+2)$ gives
\[
3x+5 = A(x+2)+B(x-1)
\] which must be an identity in $x.$ For convenience, we do allow $x=1$ or $x=-2$ in this identity.
We find $A=2$, $B=1$, so
\[
\frac{3x+5}{(x-1)(x+2)}
= \frac{2}{x-1}+\frac{1}{x+2}.
\]

\bigskip

%--------------------------------------------------------------
\subsection*{2. Repeated Linear Factors \texorpdfstring{$(x-a)^n$}{(x-a)^n}}

If 
\[
Q(x)=(x-a)^n,
\]
then the decomposition includes every power of the linear factor:
\[
\frac{P(x)}{(x-a)^n}
= \frac{A_1}{x-a} + \frac{A_2}{(x-a)^2}
+ \cdots + \frac{A_n}{(x-a)^n},
\]
$A_1, \dots, A_n$ are constants to be determined.

\paragraph{Example 2.}  Perform the partial fraction decomposition of 
\[
\frac{x+1}{(x-1)^2}.
\]
Suppose there exist constants $A$ and $B$ such that
\[\frac{x+1}{(x-1)^2}= \frac{A}{x-1}+\frac{B}{(x-1)^2}.\]
Multiplying through by  $(x-1)^2$ gives
\[
x+1 = A(x-1) + B.
\]
We find $A=1$ and $B=2.$ Then
\[
\frac{x+1}{(x-1)^2}
= \frac{1}{x-1}+\frac{2}{(x-1)^2}.
\]

\bigskip

%--------------------------------------------------------------
\subsection*{3. Irreducible Quadratic Factors \texorpdfstring{$(x^2+bx+c)$}{(x²+bx+c)}}

Suppose that $Q(x)$ contains  a factor $x^2+bx+c$ that is irreducible  over the real numbers.  The partial fraction decomposition of \(\displaystyle\frac{P(x)}{Q(x)}\) contains a term of the form
\[
\frac{Ax+B}{x^2+bx+c},
\]  where $A$ and $B$ are constants to be determined.
We illustrate this in the next an example.

\paragraph{Example 3.} To perform the partial fraction decomposition of
\[
\frac{x}{(x^2+1)(x-2)},
\]
we find $A$,$B$ and $C$ such that 
\[\frac{x}{(x^2+1)(x-2)}= \frac{Ax+B}{x^2+1} + \frac{C}{x-2}.\] Notice the term \(\dfrac{C}{x-2}\) corresponding to the linear factor $x-2$ of the denominator.
Multiplying through by $(x^2+1)(x-2)$ yields
\[
x = (Ax+B)(x-2) + C(x^2+1).
\] 
Comparing coefficients gives
\[
A = \dfrac13, \qquad B = \dfrac23, \qquad C = -\dfrac13,
\]
so
\[
\frac{x}{(x^2+1)(x-2)}
= \frac{\tfrac13 x + \tfrac23}{x^2+1} - \frac{1}{3(x-2)}.
\]

\bigskip

%--------------------------------------------------------------
\subsection*{4. Repeated Irreducible Quadratic Factors}

If \(Q(x) = (x^2+bx+c)^k,\) where the quadratic $x^2+bx+c$ is irreducible and repeated with multiplicity $k,$  
then the decomposition \(\displaystyle\frac{P(x)}{Q(x)}\)  is of the form
\[
\frac{P(x)}{(x^2+bx+c)^k}
= \frac{A_1x+B_1}{x^2+bx+c}
+ \frac{A_2x+B_2}{(x^2+bx+c)^2}
+ \cdots +
\frac{A_kx+B_k}{(x^2+bx+c)^k},
\]
with the constants $A_1, \dots, A_k$ and $B_1, \dots, B_k$ are to be determined. We illustrate this in the next example.

\paragraph{Example 4.} To perform the partial fraction decomposition of 
\[
\frac{3x+5}{(x^2+1)^2},
\]
we find constants $A, B, C, D$ such that 
\[\frac{3x+5}{(x^2+1)^2}= \frac{Ax+B}{x^2+1} + \frac{Cx+D}{(x^2+1)^2}.\]
Multiplying both sides by $(x^2+1)^2$ gives
\[
3x+5 = (Ax+B)(x^2+1) + (Cx+D).
\]
Expanding and matching coefficients yields
\[
A=0,\quad B=3,\quad C=-3,\quad D=5,
\]
and hence
\[
\frac{3x+5}{(x^2+1)^2}
= \frac{3}{x^2+1} + \frac{-3x+5}{(x^2+1)^2}.
\]

\bigskip

%--------------------------------------------------------------
\subsection*{5. Mixed Factors (General Case)}

If the denominator $Q(x)$ factors as\footnote[1]{The symbol $\prod_{i=1}^\ell p_i$ denotes product $p_1p_2 \dots p_\ell$.}
\[
Q(x)=
\prod_{i=1}^{\ell_1} (x-a_i)^{k_i}
\prod_{j=1}^{\ell_2} (x^2 + b_jx + c_j)^{m_j},
\] 
then the full partial fraction decomposition of \(\displaystyle\frac{P(x)}{Q(x)}\) contains terms

\begin{itemize}
	\item    $ \dfrac{A}{x-a}$, 
	$\dfrac{A}{(x-a)^2}$, etc., and
	\item  $ \dfrac{Ax+B}{x^2+bx+c}$,
	$\dfrac{Ax+B}{(x^2+bx+c)^2}$, etc.
\end{itemize}
Let us illustrate this in the next example

\paragraph{Example 5.} To perform the partial fraction decomposition of 
\[
\frac{2x^2+3x+1}{(x-1)(x^2+1)},
\]
we find constants $A, B, C$ such that 
\[\frac{2x^2+3x+1}{(x-1)(x^2+1)}= \frac{A}{x-1} + \frac{Bx+C}{x^2+1}.\]
Multiplying both sides by $(x-1)(x^2+1)$ gives
\[
2x^2+3x+1 = A(x^2+1) + (Bx+C)(x-1).
\]
Expanding and comparing coefficients yields
\[
A = 1,\qquad B = 1,\qquad C = 2,
\]
so
\[
\frac{2x^2+3x+1}{(x-1)(x^2+1)}
= \frac{1}{x-1} + \frac{x+2}{x^2+1}.
\]

\bigskip



% ====================================================================
\section*{Exercises}
% ====================================================================

\resetitems
\textbf{Decompose each rational function into partial fractions.}

\begin{enumerate}
	\item $\displaystyle \frac{5x+3}{(x-1)(x+4)}$
	\item $\displaystyle \frac{7}{(x+2)^2}$
	\item $\displaystyle \frac{3x^2+2x+1}{x(x^2+4)}$
	\item $\displaystyle \frac{x^2+5}{(x^2+1)^2}$
	\item $\displaystyle \frac{4x+9}{(x-1)^3}$
	\item $\displaystyle \frac{2x^2-1}{(x^2+1)(x-3)}$
	\item $\displaystyle \frac{x^3+1}{x^2-1}$
	\item $\displaystyle \frac{6x}{(x^2+2x+5)(x-1)}$
\end{enumerate}

\subsection*{Answers}

\begin{enumerate}[label=\arabic*.]
	\item $\displaystyle \frac{2}{x-1}+\frac{3}{x+4}$
	\item $\displaystyle \frac{7}{(x+2)^2}$
	\item $\displaystyle \frac{1}{x}+\frac{x+2}{x^2+4}$
	\item $\displaystyle \frac{5}{x^2+1}-\frac{x}{(x^2+1)^2}$
	\item $\displaystyle \frac{A}{x-1}+\frac{B}{(x-1)^2}+\frac{C}{(x-1)^3}$ (determine $A, B, C$)
	\item $\displaystyle \frac{1}{x-3}+\frac{x+3}{x^2+1}$
	\item Divide first: $x+1+\frac{2}{x^2-1}$
	\item $\displaystyle \frac{2}{x-1}+\frac{-2x+1}{x^2+2x+5}$
\end{enumerate}
%	
%	% ====================================================================
%	\section*{Flashcard Summary (Cheat Sheet)}
%	% ====================================================================
%	
%	\textbf{Goal:} Decompose $\displaystyle \frac{P(x)}{Q(x)}$ into partial fractions when the degree of $P(x)$ is less than that of $Q(x).$
%	
%	\begin{itemize}[noitemsep]
	%		\item Distinct linear factors:  
	%		$\displaystyle \frac{A}{x-a}$
	%		\item Repeated linear factors:  
	%		$\displaystyle \frac{A_1}{x-a}+\frac{A_2}{(x-a)^2}+\cdots$
	%		\item Irreducible quadratic factor:  
	%		$\displaystyle \frac{Ax+B}{x^2+bx+c}$
	%		\item Repeated irreducible quadratic:  
	%		$\displaystyle \sum \frac{A_ix+B_i}{(x^2+bx+c)^i}$
	%	\end{itemize}
%	
%\noindent\textbf{Steps:}
%	\begin{enumerate}[noitemsep]
	%		\item Factor denominator.
	%		\item Write correct partial fraction decomposition  form.
	%		\item Multiply through to eliminate denominators.
	%		\item Solve for unknown coefficients.
	%		\item Write down the full decomposition.
	%	\end{enumerate}

% ====================================================================




\chapter{Formulas}
\section{Fundamental Laplace Transforms}

%\begin{table}[htbp]
	%\centering
	\renewcommand{\arraystretch}{2.25}
	\setlength{\tabcolsep}{6pt}
	%\setcounter{rownum}{5}
	\rowcolors{2}{tablegray}{white}

	\begin{tabular}{llc} 
	%\begin{tabular}{ll@{\hskip .4cm}c}
		%\toprule
		\specialrule{1pt}{5pt}{-2pt} % Thick rule, 5pt space above, -2pt below (example)
 		& $f(t)$  & $\mathscr{L}\{f(t)\}=F(s)$ \\ [-1ex]
		\midrule
		%\specialrule{0.4pt}{0pt}{1pt}
		\nextitem  & $1$   &$\dfrac{1}{s},$ \quad $s>0$ \\ 
		\nextitem  & $t$ & $\dfrac{1}{s^{2}},$ \quad $s>0$  \\ 
		\nextitem  & $t^{n},\;  n\mbox{ being a natural number}$ & $\dfrac{n!}{s^{\,n+1}},$ \quad $s>0$  \\
		\nextitem   & $t^{-1/2}$ & $\dfrac{\sqrt{\pi}}{\sqrt{s}},$ \quad $s>0$  \\
		\nextitem   & $t^{1/2}$ & $\dfrac{\sqrt{\pi}}{2\,s^{3/2}},$ \quad $s>0$  \\
		\nextitem   & $t^{\alpha},\ \alpha>-1$ & $\dfrac{\Gamma(\alpha+1)}{s^{\alpha+1}},$ \quad $s>0$  \\
		\nextitem   & $\sin(kt)$ & $\dfrac{k}{s^{2}+k^{2}},$ \quad $s>0$  \\
		\nextitem  & $\cos(kt)$ & $\dfrac{s}{s^{2}+k^{2}},$ \quad $s>0$ \\
		\nextitem   & $\sin^{2}(kt)$ & $\dfrac{2k^{2}}{s(s^{2}+4k^{2})},$ \quad $s>0$  \\
		\nextitem  & $\cos^{2}(kt)$ & $\dfrac{s^{2}+2k^{2}}{s(s^{2}+4k^{2})},$ \quad $s>0$  \\
		%\bottomrule
	\end{tabular}
	%\caption{Basic Laplace transforms (including error-function-related formulas).}
%\end{table}

%\begin{table}[htbp]
\renewcommand{\arraystretch}{2.25}
\rowcolors{2}{tablegray}{white}
\setlength{\tabcolsep}{6pt}
	\begin{tabular}{llc} 
		%\begin{tabular}{ll@{\hskip .4cm}c}
		%\toprule 
		\specialrule{1pt}{5pt}{-2pt} % Thick rule, 5pt space above, -2pt below (example)
		 & $f(t)$  & $\mathscr{L}\{f(t)\}=F(s)$ \\[-1ex]
		\midrule
		\nextitem  & $e^{at}$ & $\dfrac{1}{s-a},$ \quad $s>a$  \\
		\nextitem  & $\sinh(kt)$ & $\dfrac{k}{s^{2}-k^{2}},$ \quad $s>\abs{k}$  \\ 
		\nextitem  & $\cosh(kt)$ & $\dfrac{s}{s^{2}-k^{2}},$ \quad $s>\abs{k}$ \\ 
		\nextitem  & $\sinh^{2}(kt)$ & $\dfrac{2k^{2}}{s(s^{2}-4k^{2})},$ \quad $s>\abs{k}$ \\ 
		\nextitem  & $\cosh^{2}(kt)$ & $\dfrac{s^{2}-2k^{2}}{s(s^{2}-4k^{2})},$ \quad $s>\abs{k}$ \\
		\nextitem  & $t e^{at}$ & $\dfrac{1}{(s-a)^{2}},$ \quad $s>a$ \\ 
		\nextitem & $t^{n}e^{at},\;  n\mbox{ being a natural number}$ & $\dfrac{n!}{(s-a)^{n+1}},$ \quad $s>a$ \\ 
		\nextitem & $e^{at}\sin(kt)$ & $\dfrac{k}{(s-a)^{2}+k^{2}},$ \quad $s>a$ \\ 
		\nextitem & $e^{at}\cos(kt)$ & $\dfrac{s-a}{(s-a)^{2}+k^{2}},$ \quad $s>a$ \\ 
		\nextitem & $e^{at}\sinh(kt)$ & $\dfrac{k}{(s-a)^{2}-k^{2}},$ \quad $s>a+\abs{k}$ \\ 
		\nextitem & $e^{at}\cosh(kt)$ & $\dfrac{s-a}{(s-a)^{2}-k^{2}},$ \quad $s>a+\abs{k}$ \\ 
		\nextitem & $t\sin(kt)$ & $\dfrac{2ks}{(s^{2}+k^{2})^{2}},$ \quad $s>0$  \\ 
		\nextitem & $t\cos(kt)$ & $\dfrac{s^{2}-k^{2}}{(s^{2}+k^{2})^{2}},$ \quad $s>0$  \\ 
		\nextitem & $\sin(kt)+kt\cos(kt)$ &
		$\dfrac{2k s^{2}}{(s^{2}+k^{2})^{2}},$ \quad $s>0$  \\ 
		\nextitem & $\sin(kt)-kt\cos(kt)$ &
		$\dfrac{2k^{3}}{(s^{2}+k^{2})^{2}},$ \quad $s>0$  \\ 
			\nextitem & $t\sinh(kt)$ & $\dfrac{2ks}{(s^{2}-k^{2})^{2}},$ \quad $s>\abs{k}$ \\ 
		\nextitem & $t\cosh(kt)$ & $\dfrac{s^{2}+k^{2}}{(s^{2}-k^{2})^{2}},$ \quad $s>\abs{k}$ \\ 
		\nextitem & $\dfrac{e^{at}-e^{bt}}{a-b}$ &
		$\dfrac{1}{(s-a)(s-b)},$ \quad $s>\max\{a, b\}$  \\ 
\end{tabular}
%\caption{Basic Laplace transforms (including error-function-related formulas).}
%\end{table}

%\begin{table}[htbp]
	\renewcommand{\arraystretch}{2.25}
	\rowcolors{2}{tablegray}{white}
	\setlength{\tabcolsep}{10pt}
	\begin{tabular}{llc} 
		%\begin{tabular}{ll@{\hskip .4cm}c}
		%\toprule
		\specialrule{1pt}{5pt}{-2pt} % Thick rule, 5pt space above, -2pt below (example) 
		& $f(t)$  & $\mathscr{L}\{f(t)\}=F(s)$ \\[-1ex]
		\midrule
	
		\nextitem & $\dfrac{ae^{at}-be^{bt}}{a-b}$ &
		$\dfrac{s}{(s-a)(s-b)},$ \quad $s>\max\{a, b\}$ \\ 
		\nextitem & $1-\cos(kt)$ & $\dfrac{k^{2}}{s(s^{2}+k^{2})}, $ \quad $s>0$ \\ 
		\nextitem & $kt-\sin(kt)$ & $\dfrac{k^{3}}{s^{2}(s^{2}+k^{2})}, $ \quad $s>0$ \\ 
		\nextitem & $\dfrac{a\sin(bt)-b\sin(at)}{ab(a^{2}-b^{2})}$ &
		$\dfrac{1}{(s^{2}+a^{2})(s^{2}+b^{2})}, $ \quad $s>0$\\ 
		\nextitem  & $\dfrac{\cos(bt)-\cos(at)}{a^{2}-b^{2}}$ &
		$\dfrac{s}{(s^{2}+a^{2})(s^{2}+b^{2})},$ \quad $s>0$ \\ 
		\nextitem & $\sin(kt)\sinh(kt)$ &
		$\dfrac{2k^{2}s}{s^{4}-4k^{4}}$ \\ 
		\nextitem & $\sin(kt)\cosh(kt)$ &
		$\dfrac{k(s^{2}+2k^{2})}{s^{4}-4k^{4}}$ \\ 
		\nextitem & $\cos(kt)\sinh(kt)$ &
		$\dfrac{k(s^{2}-2k^{2})}{s^{4}-4k^{4}}$ \\ 
		\nextitem & $\cos(kt)\cosh(kt)$ &
		$\dfrac{s^{3}}{s^{4}-4k^{4}}$ \\ 
%		\nextitem & $J_{0}(kt)$ &
%		$\dfrac{1}{\sqrt{s^{2}+k^{2}}}$ \\ 
		\nextitem & $\dfrac{e^{bt}-e^{at}}{t}$ &
		$\ln\!\left(\dfrac{s-a}{s-b}\right)$ \\ 
		\nextitem & $\dfrac{2(1-\cos(kt))}{t}$ &
		$\ln\!\left(\dfrac{s^{2}+k^{2}}{s^{2}}\right)$ \\ 
		\nextitem & $\dfrac{2(1-\cosh(kt))}{t}$ &
		$\ln\!\left(\dfrac{s^{2}}{s^{2}-k^{2}}\right)$ \\ 
		\nextitem & $\dfrac{\sin(at)}{t}$ &
		$\arctan\!\left(\dfrac{a}{s}\right)$ \\ 
		\nextitem & $\dfrac{\sin(at)\cos(bt)}{t}$ &
		$\dfrac12\!\left[
		\arctan\!\left(\dfrac{a+b}{s}\right)
		+\arctan\!\left(\dfrac{a-b}{s}\right)
		\right]$ \\ 
		\nextitem & $\displaystyle \operatorname{erf}\!\left(\frac{a}{2\sqrt{t}}\right)$ &
		$\dfrac{1 - e^{-a\sqrt{s}}}{s}$ \\ 
		\nextitem & $\displaystyle \operatorname{erfc}\!\left(\frac{a}{2\sqrt{t}}\right)$ &
		$\dfrac{e^{-a\sqrt{s}}}{s}$ \\ 
		\nextitem & $\displaystyle \frac{1}{\sqrt{\pi t}}\,
		\exp\!\left(-\frac{a^{2}}{4t}\right)$ &
		$\dfrac{e^{-a\sqrt{s}}}{\sqrt{s}}$ \\ 
%		\nextitem & $\displaystyle t^{-1/2}\exp\!\left(-\frac{a^{2}}{4t}\right)$ &
% 	$\sqrt{\pi}\,\dfrac{e^{-a\sqrt{s}}}{\sqrt{s}}$ \\ 
%		\nextitem & $\displaystyle \frac{a}{2\sqrt{\pi}}\,
%		t^{-3/2}\exp\!\left(-\frac{a^{2}}{4t}\right)$ &
%		$e^{-a\sqrt{s}}$ \\ 
%		\nextitem & $\displaystyle
%		\left(\frac{a^{2}}{4t}-\frac12\right)
%		\frac{\exp\!\left(-\frac{a^{2}}{4t}\right)}%
%		{\sqrt{\pi}\,t^{3/2}}$ &
%		$\sqrt{s}\,e^{-a\sqrt{s}}$ \\ 
		%\bottomrule
\end{tabular}
%\caption{Basic Laplace transforms (including error-function-related formulas).}
%\end{table}

%\begin{table}[htbp]
\renewcommand{\arraystretch}{2.25}
\rowcolors{2}{tablegray}{white}
\setlength{\tabcolsep}{6pt}
	\begin{tabular}{cll}
		%\toprule
		\specialrule{1pt}{5pt}{-2pt} % Thick rule, 5pt space above, -2pt below (example)
			& $f(t)$  & $\mathscr{L}\{f(t)\}=F(s)$ \\[-1ex]
		\midrule
	\nextitem & $e^{at}f(t)$ &
		$F(s-a)$ \\ 
		\nextitem & $\mathcal{U}(t-a)$ &
		$\dfrac{e^{-as}}{s}$ \\ 
		\nextitem & $f(t-a)\,\mathcal{U}(t-a)$ &
		$e^{-as}F(s)$ \\ 
		\nextitem & $g(t)\,\mathcal{U}(t-a)$ &
		$e^{-as}\,\mathscr{L}\{g(t+a)\}$ \\ 
		\nextitem & $y^{(n)}(t)$ &
		$s^{n}Y(s)-s^{n-1}y(0)-\cdots-y^{(n-1)}(0)$ \\ 
		\nextitem & $t^{n}f(t)$ &
		$(-1)^{n}F^{(n)}(s)$ \\ 
		\nextitem &$\displaystyle\frac{f(t)}{t}$& $\displaystyle\int_s^\infty F(p)\,dp$\\
		\nextitem & $(f*g)(t)=\displaystyle\int_{0}^{t}
		f(\tau)g(t-\tau)\,d\tau$ &
		$F(s)G(s)$ \\ 
		\nextitem & $f\left( {t + T} \right) = f\left( t \right)$&$\displaystyle\frac{1}{1- e^{-sT}}\int_0^T f(t) e^{-st}\, dt$\\
		\nextitem & $\delta(t)$ & $1$ \\ 
		\nextitem & $\delta(t-a)$ & $e^{-as}$ \\ 
		%\bottomrule
		\specialrule{1pt}{5pt}{-2pt} % Thick rule, 5pt space above, -2pt below (example)
	\end{tabular}
	%\caption{Operational properties of the Laplace transform.}
%\end{table}



\clearpage

\section{Derivatives}

We notations $\dfrac{d}{dx}f(x)$ and $f'(x)$  synonymously denote the derivative of $f(x)$ with respect to $x.$
\resetitems
\vspace{1em}

%-------------------------------------------------
\noindent\textbf{Differentiation Rules}\\
\renewcommand{\arraystretch}{2.25}
\normalsize
\begin{tabular}{l}
	\nextitem\ 
	Constant:\quad
	$\displaystyle \frac{d}{dx}c = 0$\\
	\nextitem\ 
	Constant Multiple:\quad
	$\displaystyle \frac{d}{dx}\big(cf(x)\big) = c f'(x)$\\
	
	\nextitem\ 
	Sum:\quad
	$\displaystyle \frac{d}{dx}\!\big(f(x)\pm g(x)\big)
	= f'(x)\pm g'(x)$\\	
	\nextitem\ 
	Product:
	$\displaystyle \frac{d}{dx}\big(f(x)g(x)\big)
	= f'(x)g(x) + g'(x)f(x)$\\

	\nextitem\ 
	Quotient:\quad
	$\displaystyle \frac{d}{dx}\left(\frac{f(x)}{g(x)}\right)
	= \frac{g(x)f'(x)-f(x)g'(x)}{[g(x)]^{2}}$\\
	
	\nextitem\ 
	Chain:\quad
	$\displaystyle \frac{d}{dx}f(g(x)) = f'(g(x))\,g'(x)$\\
	
	\nextitem\ 
	Power:\quad
	$\displaystyle \frac{d}{dx}x^{n} = n x^{n-1}$\\
	\nextitem\ 
	Power:\quad
	$\displaystyle \frac{d}{dx}[g(x)]^{n}
	= n[g(x)]^{n-1}g'(x)$
\end{tabular}

\vspace{1em}

%-------------------------------------------------

\noindent\textbf{Derivatives of Trigonometric Functions}\\
\newcolumntype{?}{!{\vrule width 1.5pt}}
\begin{tabular}{@{}l?l@{}}
	\nextitem\ $\displaystyle \frac{d}{dx}\sin x = \cos x$	&
	\nextitem\ $\displaystyle \frac{d}{dx}\cos x = -\sin x$\\

	\nextitem\ $\displaystyle \frac{d}{dx}\tan x = \sec^{2}x$
&
	\nextitem\ $\displaystyle \frac{d}{dx}\cot x = -\csc^{2}x$\\
	
	\nextitem\ $\displaystyle \frac{d}{dx}\sec x = \sec x\tan x$
	&
	\nextitem\ $\displaystyle \frac{d}{dx}\csc x = -\csc x\cot x$\\
\end{tabular}

\vspace{0.8em}

\noindent\textbf{Derivatives of Inverse Trigonometric Functions}\\
\newcolumntype{?}{!{\vrule width 1.5pt}}
\begin{tabular}{@{}l?l@{}}
	\nextitem\ $\displaystyle \frac{d}{dx}\sin^{-1}x
	= \frac{1}{\sqrt{1-x^{2}}}$
	&
	\nextitem\ $\displaystyle \frac{d}{dx}\cos^{-1}x
	= -\frac{1}{\sqrt{1-x^{2}}}$\\

	\nextitem\ $\displaystyle \frac{d}{dx}\tan^{-1}x
	= \frac{1}{1+x^{2}}$
	&
	
	\nextitem\ $\displaystyle \frac{d}{dx}\cot^{-1}x
	= -\frac{1}{1+x^{2}}$\\
	
	\nextitem\ $\displaystyle \frac{d}{dx}\sec^{-1}x
	= \frac{1}{|x|\sqrt{x^{2}-1}}$
	&
	\nextitem\ $\displaystyle \frac{d}{dx}\csc^{-1}x
	= -\frac{1}{|x|\sqrt{x^{2}-1}}$
\end{tabular}

\vspace{0.8em}
\clearpage
\noindent\textbf{Derivatives of Hyperbolic Functions}\\
\newcolumntype{?}{!{\vrule width 1.5pt}}
\begin{tabular}{@{}l?l@{}}
	\nextitem\ $\displaystyle \frac{d}{dx}\sinh x = \cosh x$
	&
	\nextitem\ $\displaystyle \frac{d}{dx}\cosh x = \sinh x$\\
	
	\nextitem\ $\displaystyle \frac{d}{dx}\tanh x = \sech^{2}x$
	&
	
	\nextitem\ $\displaystyle \frac{d}{dx}\coth x = -\csch^{2}x$\\
	
	\nextitem\ $\displaystyle \frac{d}{dx}\sech x = -\sech x\tanh x$
	&
	\nextitem\ $\displaystyle \frac{d}{dx}\csch x = -\csch x\coth x$
\end{tabular}

\vspace{0.8em}

\noindent\textbf{Derivatives of Inverse Hyperbolic Functions}\\
\newcolumntype{?}{!{\vrule width 1.5pt}}
\begin{tabular}{@{}l?l@{}}
	\nextitem\ $\displaystyle \frac{d}{dx}\sinh^{-1}x
	= \frac{1}{\sqrt{x^{2}+1}}$
	&
	\nextitem\ $\displaystyle \frac{d}{dx}\cosh^{-1}x
	= \frac{1}{\sqrt{x^{2}-1}}$\\
	
	\nextitem\ $\displaystyle \frac{d}{dx}\tanh^{-1}x
	= \frac{1}{1-x^{2}}$
	&
	
	\nextitem\ $\displaystyle \frac{d}{dx}\coth^{-1}x
	= \frac{1}{1-x^{2}}$\\
	
	\nextitem\ $\displaystyle \frac{d}{dx}\sech^{-1}x
	= -\frac{1}{x\sqrt{1-x^{2}}}$
	&
	\nextitem\ $\displaystyle \frac{d}{dx}\csch^{-1}x
	= -\frac{1}{|x|\sqrt{x^{2}+1}}$
\end{tabular}

\vspace{0.8em}

\noindent\textbf{Derivatives of Exponential Functions}\\
\begin{tabular}{@{}ll@{}}
	\nextitem\ $\displaystyle \frac{d}{dx}e^{x} = e^{x}$
	&
	\nextitem\ $\displaystyle \frac{d}{dx}b^{x} = b^{x}\ln b$
\end{tabular}

\vspace{0.8em}

\noindent\textbf{Derivatives of Logarithmic Functions}\\
\begin{tabular}{@{}l l@{}}
	\nextitem\ $\displaystyle \frac{d}{dx}\ln|x| = \frac{1}{x}$
	&
	\nextitem\ $\displaystyle \frac{d}{dx}\log_{b}x = \frac{1}{x\ln b}$
\end{tabular}
\clearpage

\section{Integrals}
\resetitems
\newcolumntype{?}{!{\vrule width 1.5pt}}
\begin{tabular}{|@{}l?l@{}}
%\begin{tabular}{l}
	% 1–2
	\nextitem\ $\displaystyle
	\int u^{n}\,du = \frac{u^{n+1}}{n+1}+C,\quad n\neq -1$
	&
	\nextitem\ $\displaystyle
	\int \frac{1}{u}\,du = \ln|u| + C$\\
	
	% 3–4
	\nextitem\ $\displaystyle
	\int e^{u}\,du = e^{u}+C$
	&
	\nextitem\ $\displaystyle
	\int a^{u}\,du = \frac{1}{\ln a}\,a^{u}+C$\\
	
	% 5–6
	\nextitem\ $\displaystyle
	\int \sin u\,du = -\cos u + C$
	&
	\nextitem\ $\displaystyle
	\int \cos u\,du = \sin u + C$\\
	
	% 7–8
	\nextitem\ $\displaystyle
	\int \sec^{2}u\,du = \tan u + C$
	&
	\nextitem\ $\displaystyle
	\int \csc^{2}u\,du = -\cot u + C$\\
	
	% 9–10
	\nextitem\ $\displaystyle
	\int \sec u \tan u\,du = \sec u + C$
	&
	\nextitem\ $\displaystyle
	\int \csc u \cot u\,du = -\csc u + C$\\
	
	% 11–12
	\nextitem\ $\displaystyle
	\int \tan u\,du = -\ln|\cos u| + C$
	&
	\nextitem\ $\displaystyle
	\int \cot u\,du = \ln|\sin u| + C$\\
	
	% 13–14
	\nextitem\ $\displaystyle
	\int \sec u\,du = \ln|\sec u + \tan u| + C$
	&
	\nextitem\ $\displaystyle
	\int \csc u\,du = \ln|\csc u - \cot u| + C$\\
	
	% 15–16
	\nextitem\ $\displaystyle
	\int u\sin u\,du = \sin u - u\cos u + C$
	&
	\nextitem\ $\displaystyle
	\int u\cos u\,du = \cos u + u\sin u + C$\\
	
	% 17–18
	\nextitem\ $\displaystyle
	\int \sin^{2}u\,du = \frac12 u - \frac14 \sin 2u + C$
	&
	\nextitem\ $\displaystyle
	\int \cos^{2}u\,du = \frac12 u + \frac14 \sin 2u + C$\\
	
	% 19–20
	\nextitem\ $\displaystyle
	\int \tan^{2}u\,du = \tan u - u + C$
	&
	\nextitem\ $\displaystyle
	\int \cot^{2}u\,du = -\cot u - u + C$\\
	
	
	% 31–32
	\nextitem\ $\displaystyle
	\int \sinh u\,du = \cosh u + C$
	&
	\nextitem\ $\displaystyle
	\int \cosh u\,du = \sinh u + C$\\
	
	% 33–34
	\nextitem\ $\displaystyle
	\int \sech^{2}u\,du = \tanh u + C$
	&
	\nextitem\ $\displaystyle
	\int \csch^{2}u\,du = -\coth u + C$\\
	
	% 35–36
	\nextitem\ $\displaystyle
	\int \tanh u\,du = \ln(\cosh u) + C$
	&
	\nextitem\ $\displaystyle
	\int \coth u\,du = \ln|\sinh u| + C$\\

 \multicolumn{2}{@{}l@{}}{\nextitem\
		  $\displaystyle
		 \int \sin^{3}u\,du =
		 -\frac13\bigl(2+\sin^{2}u\bigr)\cos u + C$}\\
  \multicolumn{2}{@{}l@{}}{\nextitem\
	$\displaystyle
	\int \cos^{3}u\,du =
	\frac13\bigl(2+\cos^{2}u\bigr)\sin u + C$ }\\
\multicolumn{2}{@{}l@{}}{\nextitem\
	$\displaystyle
	\int \tan^{3}u\,du =
	\frac12\tan^{2}u + \ln|\cos u| + C$}\\
\multicolumn{2}{@{}l@{}}{\nextitem\
	$\displaystyle
	\int \cot^{3}u\,du =
	-\frac12\cot^{2}u - \ln|\sin u| + C$}\\
\multicolumn{2}{@{}l@{}}{\nextitem\
	$\displaystyle
	\int \sec^{3}u\,du =
	\frac12\sec u\tan u + \frac12\ln|\sec u + \tan u| + C$}\\
\end{tabular}



\begin{tabular}{l}
\nextitem\ $\displaystyle
\int \csc^{3}u\,du =
-\frac12\csc u\cot u + \frac12\ln|\csc u - \cot u| + C$\\

\nextitem\ $\displaystyle
\int \sin(au)\cos(bu)\,du =
\frac{\sin(a-b)u}{2(a-b)} -
\frac{\sin(a+b)u}{2(a+b)} + C$\\
\nextitem\ $\displaystyle
\int \cos(au)\cos(bu)\,du =
\frac{\sin(a-b)u}{2(a-b)} +
\frac{\sin(a+b)u}{2(a+b)} + C$\\

% 29–30
\nextitem\ $\displaystyle
\int e^{au}\sin(bu)\,du =
\frac{e^{au}}{a^{2}+b^{2}}
\bigl(a\sin bu - b\cos bu\bigr) + C$\\
\nextitem\ $\displaystyle
\int e^{au}\cos(bu)\,du =
\frac{e^{au}}{a^{2}+b^{2}}
\bigl(a\cos bu + b\sin bu\bigr) + C$\\



% 37–38
\nextitem\ $\displaystyle
\int \ln u\,du = u\ln u - u + C$\\
\nextitem\ $\displaystyle
\int u\ln u\,du =
\frac12 u^{2}\ln u - \frac14 u^{2} + C$\\

% 39–40
\nextitem\ $\displaystyle
\int \frac{1}{\sqrt{a^{2}-u^{2}}}\,du =
\frac{1}{a}\sin^{-1}\!\frac{u}{a} + C$\\
\nextitem\ $\displaystyle
\int \frac{1}{\sqrt{a^{2}+u^{2}}}\,du =
\ln\bigl|u+\sqrt{a^{2}+u^{2}}\bigr| + C$\\

% 41–42
\nextitem\ $\displaystyle
\int \sqrt{a^{2}-u^{2}}\,du =
\frac12 u\sqrt{a^{2}-u^{2}}
+ \frac{a^{2}}{2}\sin^{-1}\!\frac{u}{a} + C$\\
\nextitem\ $\displaystyle
\int \sqrt{a^{2}+u^{2}}\,du =
\frac12 u\sqrt{a^{2}+u^{2}}
+ \frac{a^{2}}{2}
\ln\bigl|u+\sqrt{a^{2}+u^{2}}\bigr| + C$\\

% 43–44
\nextitem\ $\displaystyle
\int \frac{1}{a^{2}+u^{2}}\,du =
\frac{1}{a}\tan^{-1}\!\frac{u}{a} + C$\\
\nextitem\ $\displaystyle
\int \frac{1}{a^{2}-u^{2}}\,du =
\frac{1}{2a}\ln\left|\frac{a+u}{\,a-u\,}\right| + C$\\


\end{tabular}



%\begin{tabular}{lll}
%	\hline
%	\multicolumn{2}{l}{Merged Row Content} & Column 2, Row 1 Column 3, Row 1 \\
%	\cline{1-2} % Partial line for columns 2 and 3
%	& Column 2, Row 2 & Column 3, Row 2 \\
%	\hline
%	Regular Row & More Data & Even More Data \\
%	\hline
%\end{tabular}
%\begin{tabular}{ |p{3cm}||p{3cm}|p{3cm}|p{3cm}|  }
%	\hline
%	\multicolumn{4}{|l|}{Country List} \\
%	\hline
%	Country Name or Area Name& ISO ALPHA 2 Code &ISO ALPHA 3 Code&ISO numeric Code\\
%	\hline
%	Afghanistan   & AF    &AFG&   004\\
%	Aland Islands&   AX  & ALA   &248\\
%	Albania &AL & ALB&  008\\
%	Algeria    &DZ & DZA&  012\\
%	American Samoa&   AS  & ASM&016\\
%	Andorra& AD  & AND   &020\\
%	Angola& AO  & AGO&024\\
%	\hline
%\end{tabular}
	

