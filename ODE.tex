\documentclass[pagenumber]{book}
%\usepackage{times}
%\usepackage{newtx}
%\usepackage{pagecolor}
%\usepackage[tagged, highstructure]{accessibility}

\usepackage{makeidx}
\makeindex % Initialize an index so we can add entries with \index

\usepackage[bindingoffset=0.625in,
left=.5in, right=.5in,
top=.8125in, bottom=.9375in,
paperwidth=6.375in, paperheight=9.25in]{geometry}

%for electrical circuits
\usepackage[american, cuteinductors]{circuitikz}
\usepackage{subcaption}
\usepackage{caption}


\usepackage{booktabs}      % for \toprule, \midrule, \bottomrule
\usepackage[table]{xcolor} % for row coloring (\rowcolors) and table shading
\usepackage{array} 
\definecolor{tablegray}{gray}{0.93}
\newcounter{tableitem}
\newcommand{\resetitems}{\setcounter{tableitem}{0}}
\newcommand{\rownumber}{\stepcounter{rownum}\arabic{rownum}.}
\newcommand{\nextitem}{\stepcounter{tableitem}\arabic{tableitem}.}
\usepackage{multirow}


%\usepackage[left=3cm,right=2.5cm,top=2.7cm,bottom=3cm]{geometry}

\usepackage[titletoc]{appendix}
\usepackage{booktabs}
\usepackage{xparse}
\usepackage{tikz}
\usepackage{tkz-fct} 
\usepackage{mathrsfs}
\usetikzlibrary{calc}
\usetikzlibrary{decorations.markings}
\usepackage{pgfplots}
\pgfplotsset{compat=1.18}



%%%titlepage
\usepackage{xcolor}
\usepackage{lipsum} % for dummy text
\usepackage{titlesec}
\usepackage{graphicx}
%%% For exercises
\usepackage{tcolorbox}
\usepackage{answers}
\usepackage{mdframed}

\usetikzlibrary{matrix}



%%%TOC%%%
% \usepackage{etoc}
\usepackage[english]{babel}
% \usepackage{titlesec}
\usepackage[undotted]{minitoc}
% \usepackage{lipsum}
\addto{\captionsenglish}{% Making babel aware of special titles
  \renewcommand{\mtctitle}{} %Add title
  }


%Exercises and solutions
\newcommand\dunderline[3][-1pt]{{%
		\sbox0{#3}%
		\ooalign{\copy0\cr\rule[\dimexpr#1-#2\relax]{\wd0}{#2}}}}

\usepackage{ifthen}
\newboolean{firstanswerofthechapter}  

\usepackage{xcolor}
\colorlet{lightcyan}{cyan!40!white}

\usepackage{chngcntr}
\usepackage{stackengine}

\usepackage{tasks}
\newlength{\longestlabel}
\settowidth{\longestlabel}{\bfseries viii.}
\settasks{label=\roman*., label-format={\bfseries}, label-width=\longestlabel,
	item-indent=20pt, label-offset=2pt, column-sep={10pt}}

\usepackage[lastexercise,answerdelayed]{exercise}
\counterwithin{Exercise}{chapter}
\counterwithin{Answer}{chapter}
\renewcounter{Exercise}[chapter]
\newcommand{\QuestionNB}{\bfseries\arabic{Question}.\ }
\renewcommand{\ExerciseName}{\textbf{Exercises}}
\renewcommand{\ExerciseHeader}{\noindent\def\stackalignment{l}% code from https://tex.stackexchange.com/a/195118/101651
	\stackunder[0pt]{\textcolor{blue}{\dunderline[-3pt]{3pt}{\colorbox{yellow}{\textcolor{black}{\textbf{\LARGE\ExerciseName\;\LARGE\ExerciseHeaderNB}}}}}}{\textcolor{blue}{\rule{\linewidth}{1.5pt}}}\medskip\medskip\medskip}
%\stackunder[0pt]{\colorbox{cyan}{\textcolor{white}{\textbf{\LARGE\ExerciseHeaderNB\;\large\ExerciseName}}}}{\textcolor{lightcyan}{\rule{\linewidth}{2pt}}}\medskip}
\renewcommand{\AnswerName}{Exercises}
\renewcommand{\AnswerHeader}{\ifthenelse{\boolean{firstanswerofthechapter}}%
{\bigskip\noindent\textcolor{cyan}{\textbf{CHAPTER \thechapter}}\newline\newline%
	\noindent\bfseries\emph{\textcolor{cyan}{\AnswerName\ \ExerciseHeaderNB, page %
			\pageref{\AnswerRef}}}\smallskip}
{\noindent\bfseries\emph{\textcolor{cyan}{\AnswerName\ \ExerciseHeaderNB, page \pageref{\AnswerRef}}}\smallskip}}
\setlength{\QuestionIndent}{16pt}



%%Bibliography%%%

 \usepackage[semicolon, square, sort&compress, numbers]{natbib}  %
 \usepackage[sectionbib]{chapterbib}  

%%%%%

%%%Fonts%%%%
\usepackage{lmodern}

%\usepackage{newtxtext}
%\usepackage{newtxmath} 
%\usepackage{mathptmx}
%%%%%
\usepackage{amsmath}
\usepackage{amssymb}
\usepackage{amsthm}
\usepackage{amsfonts,amssymb,mathrsfs} 
\usepackage{physics}
\usepackage{mathtools}
\usepackage{multicol}
\usepackage[lastexercise, answerdelayed]{exercise}
\usepackage{hyperref}
\usepackage{makeidx}
\usepackage{bbm}
\usepackage{cases}


\usepackage[T1]{fontenc}
\usepackage[font=small,skip=0pt]{caption}
\usepackage{float}

%%%%%
\usepackage{cleveref} %added by Dhruba
\renewcommand{\qedsymbol}{\ensuremath{\blacksquare}}
%%%%%

\newcommand{\ds}{\displaystyle}

%%%%%

%%%%Add hard section with *
\usepackage{tocloft}
\usepackage{lipsum} % for dummy text
\newenvironment{hardsec}
{\renewcommand{\thesection}{\thechapter.\arabic{section}*}}
{}
\newenvironment{hardsubsec}
{\renewcommand{\thesubsection}{\thesection.\arabic{subsection}*}}
{}
% set section numbers in TOC flush right (from the tocloft documentation)
\newlength{\extralen}
\setlength{\extralen}{0em}    % need some extra space at end of number
%\renewcommand{\cftsecpresnum}{\hfill} % note the double ‘l’ 
\renewcommand{\cftsecpresnum}{} % note the double ‘l’ This is to flushleft the section number
\renewcommand{\cftsecaftersnum}{\hspace*{\extralen}}
\addtolength{\cftsecnumwidth}{\extralen}

\setlength{\extralen}{-0.5em}    % need some extra space at end of number
%\renewcommand{\cftsubsecpresnum}{\hfill} % note the double ‘l’ 
\renewcommand{\cftsecpresnum}{} % note the double ‘l’ This is to flushleft the subsection number
\renewcommand{\cftsubsecaftersnum}{\hspace*{\extralen}}
\addtolength{\cftsubsecnumwidth}{\extralen}

%%%%Geogebra input%%%
\usepackage{pgfplots}
\pgfplotsset{compat=1.15}
\usepackage{mathrsfs}
\usetikzlibrary{arrows}


%%%%
%\usepackage{tgschola}
%\usepackage{lmodern}
%font
\usepackage{microtype}

%\usepackage[lf]{venturis}
\RequirePackage{relsize}
\RequirePackage[shortlabels,inline]{enumitem}
\title{Elementary Ordinary Differential Equations}
\newcommand{\booksubtitle}{Methods and Applications}
\newcommand{\booklicense}{Creative Commons Zero 1.0 Universal}

\makeatletter
\newcommand*{\defeq}{\mathrel{\rlap{%
                     \raisebox{0.3ex}{$\m@th\cdot$}}%
                     \raisebox{-0.27ex}{$\m@th\cdot$}}%
                     =}
\makeatother

\author{Dhruba R. Adhikari}

% \institute{Kennesaw State University, Georgia, USA \&
%   Fort Valley State University, Georgia, USA}
% Author subtitle could be a university or a geographical location, for example
\newcommand{\authorsubtitle}{City, Country}

% Create convenient commands \booktitle and \bookauthor
\makeatletter
\newcommand{\booktitle}{\@title}
\newcommand{\bookauthor}{\@author}
\makeatother

% This utf8 declaration is not needed for versions of latex > 2018 but may
% be helpful for older software. Eventually it may not be worth keeping.
\usepackage[utf8]{inputenc}  
\usepackage{fix-cm}  % this package allows large \fontsize
\usepackage{tikz}    % this is for graphics. e.g. rectangle on title page



\setitemize{labelsep=0.25cm,labelindent=1.5em,leftmargin=*}
\setenumerate{labelsep=0.25cm,labelindent=1.5em,leftmargin=*}
% Used by equations

% The following dimensions specify 4.75" X 7.5" content on 6 3/8" by 9 1/4"
% paper. The paper width and height can be tweaked as required and the content
% should size to fit within the margins accordingly.
%
% The (inside) bindingoffset should be larger for books with more pages. Some
% standard recommended sizes are .375in minimum up to 1in for 600+ page books.
% Sizes .75in and .875in are also recommended roughly at 150 and 400 pages.

% Here is an alternative geometry for reading on letter size paper:
% \usepackage[margin=.75in, paperwidth=8.5in, paperheight=11in]{geometry}

\renewcommand{\contentsname}{Contents} % default is {Contents}


% The next few commands are for creating fake content to fill out the template.
% You should delete this (e.g.  everything up to, but not including,
% \begin{document}) after you insert your own content.
% Example content from Einstein's Meaning of Relativity.
% Public domain book: http://www.gutenberg.org/ebooks/36276
\newcommand{\fakeparagraph}{The theory of relativity is intimately connected
with the theory of space and time. I shall therefore begin with a brief
investigation of the origin of our ideas of space and time, although in
doing so I know that I introduce a controversial subject. The object of all
science, whether natural science or psychology, is to co-ordinate our
experiences and to bring them into a logical system. How are our customary
ideas of space and time related to the character of our experiences?}

\newcommand{\fakecontent}{
\fakeparagraph{}
\fakeparagraph{}

\fakeparagraph{}}

% \newcommand{\fakesections}{
% \fakecontent{}
% \section{First Principles}\fakecontent{}
% \subsection{Examples}\fakecontent{}
% \subsection{Excessive Elaborations}\fakecontent{}
% \subsection{Long Winded Conclusion}\fakecontent{}
% \subsection{Exercises}\fakecontent{}
% \section{Theory vs Practice}\fakecontent{}
% \subsection{Examples}\fakecontent{}
% \subsection{Excessive Elaborations}\fakecontent{}
% \subsection{Long Winded Conclusion}\fakecontent{}
% \subsection{Exercises}\fakecontent{}
% \foreach \x in {A,B,C,D,E,F,G,H,I,J,K,L,M,N,O,P,Q,R,S,T,U,V,W,X,Y,Z}
%     {\index{\x 1}\index{\x 2}}
% }

\numberwithin{section}{chapter}
\numberwithin{equation}{chapter}
\renewcommand{\mathbf}{\mathbb}
\renewcommand{\b}{\mathbf}
\renewcommand{\d}{\:{\rm d}}
\renewcommand{\l}{\langle}
\renewcommand{\r}{\rangle}
\renewcommand{\epsilon}{\varepsilon}
\renewcommand{\L}{\mathscr{L}}


% Content Starts Here


%\renewcommand{\vec}{\boldsymbol}
\renewcommand{\vec}{\textbf}


%\theoremstyle{theorem}
\newtheorem{theorem}{Theorem}[section]
\newtheorem{prop}[theorem]{Proposition}
%\newtheorem{principle}[theorem]{Principle}
\newtheorem{lemma}[theorem]{Lemma}
%\newtheorem{corollary}[theorem]{Corollary}
%\newtheorem{corollary}{Corollary}[theorem]

\theoremstyle{definition}
\newtheorem{definition}[theorem]{Definition}%[theorem]
 \newtheorem{example}{Example}[section]
 \newtheorem{problem}{Problem}[section]
 %\newtheorem{remark}[theorem]{Remark}
 \newtheorem{remark}{Remark}[section]
 \newtheorem{xca}{\includegraphics[width=0.035\linewidth]{ex2.jpg}\hspace{1ex}Exercise}[section]

%%%
\newenvironment{solution}
{\renewcommand\qedsymbol{$\clubsuit$ }\begin{proof}[Solution]}
	{\end{proof}}
	
	


\newenvironment{change}{
  \begin{enumerate}[label=\small\protect\circled{\arabic*}]}{
  \end{enumerate}}
\def\upint{\mathchoice%
    {\mkern13mu\overline{\vphantom{\intop}\mkern7mu}\mkern-20mu}%
    {\mkern7mu\overline{\vphantom{\intop}\mkern7mu}\mkern-14mu}%
    {\mkern7mu\overline{\vphantom{\intop}\mkern7mu}\mkern-14mu}%
    {\mkern7mu\overline{\vphantom{\intop}\mkern7mu}\mkern-14mu}%
  \int}
\def\lowint{\mkern3mu\underline{\vphantom{\intop}\mkern7mu}\mkern-10mu\int}

\allowdisplaybreaks[1]
%\setlength\parindent{0pt}
%\setlength{\parskip}{-0.2 em}



%%%%%%%
\usepackage{framed}
\renewenvironment{leftbar}[1][\hsize]
{% 
	\def\FrameCommand 
	{%
	{\hspace{-2pt}\color[rgb]{0,0,0}\vrule width 1pt}% {.5, 0,0}
		\hspace{1pt}%must no space.
		\fboxsep=\FrameSep\colorbox[rgb]{0.9,0.95,0.95}
  %{0.8,1,1}%{0.95,0.95,0.95}%
	}%
	\MakeFramed{\hsize#1\advance\hsize-\width\FrameRestore}%
}
{\endMakeFramed}
\setlength{\FrameSep}{4pt}
%%%%%%%


\newtheorem{thmbar}{Theorem}[chapter]
\usepackage{framed}
\renewenvironment{thmbar}[1][\hsize]
{% 
	\def\FrameCommand 
	{%
		{\hspace{-2pt}\color[rgb]{0,0,0}\vrule width 1pt}%
		\hspace{1pt}%must no space.
		\fboxsep=\FrameSep\colorbox[rgb]{ 0.8,0.9,0.98}%{0.98,0.98,0.9}%
	}%
	\MakeFramed{\hsize#1\advance\hsize-\width\FrameRestore}%
}
{\endMakeFramed}
\setlength{\FrameSep}{4pt}

%\maketitle


%%%%%%%%%%%

\definecolor{coverblue}{RGB}{0, 56, 123}
\definecolor{goldenyellow}{RGB}{255, 204, 0}

%Evronment for tabular method of integration by parts
	\tikzset{Arrow Style/.style={text=black, font=\boldmath}}

\newcommand{\tikzmark}[1]{%
	\tikz[overlay, remember picture, baseline] \node (#1) {};%
}

\newcommand*{\XShift}{0.5em}
\newcommand*{\YShift}{0.5ex}

\NewDocumentCommand{\DrawArrow}{s O{} m m m}{%
	\begin{tikzpicture}[overlay,remember picture]
		\draw[->, thick, Arrow Style, #2] 
		($(#3.west)+(\XShift,\YShift)$) -- 
		($(#4.east)+(-\XShift,\YShift)$)
		node [midway,above] {#5};
	\end{tikzpicture}%
}	





\begin{document}
\frontmatter

% ---- Half Title Page ----
% current geometry will be restored after title page
\newgeometry{top=1.75in,bottom=.5in}
\begin{titlepage}
%\begin{flushleft}
%
%% Title
%\textbf{\fontfamily{qcs}\fontsize{14}{25}\selectfont Elementary Ordinary Differential Equations\\}
%
%% Draw a line 4pt high
%\par\noindent\rule{\textwidth}{5pt}\\
%
%% Subtitle
%% Shaded box from left to right. Text node is midway (centered).
%\begin{tikzpicture}
%\shade[bottom color=lightgray,top color=white]
%    (0,0) rectangle (\textwidth, 1)
%    node[midway] {\textbf{\Large \textit{\booksubtitle}}};
%\end{tikzpicture}
%
%% Edition Number
%
%\begin{flushright}
%\Large\textbf {First Edition}
%\end{flushright}
%
%\vspace{\fill}
%\author{D. R. Adhikari}
%\begin{flushright}
%\textbf{
%\LARGE Dhruba R. Adhikari
%}
%\end{flushright}
%
%
%
%% \vspace{\fill}
%\vspace{\fill}
%
%\end{flushleft}
% TODO: \usepackage{graphicx} required
% Blue background with TikZ


%Book Cover
\textcolor{red}{open this cover page}

%\definecolor{ao}{rgb}{0, 0.35, 0.68}
%\pagecolor{ao}
%\begin{figure}
%	\vspace{-3.2cm}
%\hspace{-1cm}
%	\includegraphics[scale=.39]{cover}
%
%\end{figure}



\end{titlepage}
\restoregeometry
% ---- End of Half Title Page ----

\newpage
\nopagecolor
% No page numbers on the Frontispiece page
\thispagestyle{empty}


% ---- Title Page ----
% current geometry will be restored after title page
\newgeometry{top=1in,bottom=.5in}
\begin{titlepage}
\begin{flushleft}

% Title
%\textbf{\fontfamily{qcs}\fontsize{45}{54}\selectfont Real Analysis\\}
\textbf{\fontfamily{qcs}\fontsize{14}{25}\selectfont Elementary Ordinary Differential Equations\\}

% Draw a line 4pt high
\par\noindent\rule{\textwidth}{4pt}\\

% Shaded box from left to right with Subtitle
% The text node is midway (centered).

%\textcolor{red}{open this inside page}

\begin{tikzpicture}
\shade[bottom color=lightgray,top color=white]
    (0,0) rectangle (\textwidth, 1)
    node[midway] {\textbf{\Large \textit{\booksubtitle}}};
\end{tikzpicture}

% Edition Number
\begin{flushright}
\textbf{
\Large First Edition}
\end{flushright}
\vspace{\fill}
\begin{flushright}
\textbf{
\LARGE Dhruba  R. Adhikari\\[1ex]
\large Kennesaw State University\\
Georgia, USA
}
\end{flushright}

\vspace{\fill}

% % Author and Location
% \textbf{\large \bookauthor}\\[3.5pt]
% \textbf{\large \textit{\authorsubtitle}}

\vspace{\fill}

% Self Publishing Logo. Free to use: CC0 license.
% The source file is book.svg. If you change the svg, you must then convert
% it to pdf. There are many online and offline tools available to do that.
\begin{center}
%\includegraphics{booksvg.pdf}\\[4pt]
\fontfamily{lmtt}\small{KendallHunt Publishing Company\\
	Dubuque, Iowa}
%Seattle San Francisco New York\\
%London Paris Rome Beijing Barcelona}
\end{center}

\end{flushleft}

\end{titlepage}
\restoregeometry
% ---- End of Title Page ----

% Do not show page numbers on colophon page
\thispagestyle{empty}

\begin{flushleft}

%Dedication
\vspace*{\fill}
%\centering{
%In the loving memory of my parents\\
%\smallskip\smallskip
%To my beloved wife, Bhagabati\\
%\smallskip\smallskip
%To our children, Shreya and Shrish\\}



%\smallskip\smallskip
%To my beloved wife, Bhagabati,\\
%whose unwavering support has uplifted me always.\\
%\smallskip\smallskip
%To my children, Shreya and Shrish,\\
%whose patience and love have been my strength and motivation.\\
%%This book was typeset using \LaTeX{} software.\\
\vspace{\fill}
Copyright \textcopyright{} \the\year{}  \bookauthor\\
%License: \booklicense
\end{flushleft}

% A title page resets the page # to 1, but the second title page
% was actually page 3. So add two to page counter.
\addtocounter{page}{2}

% The asterisk excludes chapter from the table of contents.
\chapter*{Preface}
% \fakeparagraph{}
% \[
% \left.
% \begin{aligned}
%   &\sqrt{{dX_{1}}^{2} + {dX_{2}}^{2} + {dX_{3}}^{2}} \\
%   &\qquad
%   \begin{aligned}
%   &= \left(1 + \frac{\kappa}{8\pi} \int \frac{\sigma\, dV_{0}}{r}\right)
%      \sqrt{{dx_{1}}^{2} + {dx_{2}}^{2} + {dx_{3}}^{2}}, \\
% dT &= \left(1 - \frac{\kappa}{8\pi} \int \frac{\sigma\, dV_{0}}{r}\right) dl.
% \end{aligned}
% \end{aligned}
% \right\}
% \]
% \fakecontent{}
% \fakecontent{}

% Three-level Table of Contents
\clearpage
% \dominitoc %this is to create mini TOC
\setcounter{tocdepth}{1} %change {1} to {2} to show subsections in toc  and {3} to show subsubsection, etc
\setcounter{minitocdepth}{2} \setlength{\mtcindent}{24pt} \renewcommand{\mtcfont}{\small\rm}% default default default 
\renewcommand{\mtcSfont}{\small\bf}
\addtocounter{chapter}{0} % was replaced from -1
\dominitoc
\tableofcontents



\mainmatter
\chapter{Introduction}


\section{What is an Ordinary Differential Equation?}
Let's begin with an example. Consider the function $ y= f(x) = x^2$  on the interval $(-1, 1).$ We know that $f$ is differentiable on $(-1, 1)$ with its first order derivative 
\begin{equation}\label{chap1:1}
	\frac{dy}{dx} =2x.
\end{equation}
If we treat (\ref{chap1:1}) as an equation with \( y \) as the unknown, then we see that the function \( y = x^2 \) satisfies the equation, and therefore  \( y = x^2 \) is considered a solution of (\ref{chap1:1}) on the interval \( (-1, 1) \).
The equation (\ref{chap1:1}) serves as an example of an \textbf{\textit{ordinary differential equation}} which involves a function of one independent variable and  ordinary derivatives of the function. 



 
 A classical example of an ordinary differential equation is Newton’s law of gravity, which states that \textit{an object  in a free fall from a point near the Earth's surface experiences a constant acceleration, assuming air resistance is negligible}. This motion is described mathematically by the differential equation  
\begin{equation}\label{chap1:3}
	\frac{d^2x}{dt^2} = g,
\end{equation}
where \( g \) represents the gravitational acceleration near the Earth's surface, and \( x = x(t) \) denotes the object's height above the ground as a function of time $t$. Equivalently, the object  of mass $m$ experiences a constant weight while near Earth’s surface described by 
\begin{equation}\label{chap1:4}
	m\frac{d^2x}{dt^2} = mg.
\end{equation}
  In order to make this motion a little bit more realistic, suppose that air exerts  a resisting force  proportional to the velocity of the object.  With this assumption,  the differential equation (\ref{chap1:4}) of  motion becomes

\begin{equation*}
	m\frac{d^2x}{dt^2} = mg- \beta\,	\frac{dx}{dt},
\end{equation*}
where $\beta>0$ denotes the constant of proportionality of the air resistance to the velocity.
Rearranging the terms and setting $b=\beta/m,$ we have
\begin{equation*}
	\frac{d^2x}{dt^2}+ b\,	\frac{dx}{dt}= g, 
\end{equation*}
  that is, 
  \begin{equation}\label{chap1:5}
  	x''(t)+ b \, x'(t)= g.
  \end{equation}
  Let $v(t)$ denote $x'(t),$ the velocity of the object.  Then (\ref{chap1:5}) can also
   be expressed as  the following system of differential equations:
   \begin{equation}\label{chap1:6}
  \begin{cases}
  	x'(t) =v(t),\\
  	v'(t)+ b\, v(t)= g.
  	\end{cases} 	
  \end{equation}
  We will study more systems of ordinary differential equations in Chapter~\ref{Systems}.


\subsection{Definitions and Vocabulary}
Suppose we are given the differential equation (\ref{chap1:1}) and asked to find all functions   $y$ with derivative equal to $2x$. To answer this, we  use the \textbf{\textit{fundamental theorem of calculus}} on the interval $[0, t]$ with $t\in (-1, 1)$ and integrate both sides of (\ref{chap1:1})  to obtain
\[f(t) =\int_0^t	\frac{dy}{dx} \, dx = \int_0^t 2x\, dx = t^2 + C,\] 
where  $C$ denotes an arbitrary  real constant. Renaming $t$ as $x,$  we find that $f(x) = x^2 +C,$   and we see that for each value of $C$ the function $y = x^2+C$ has  derivative equal to $2x$ for all $x$ in $(-1, 1).$   The collection of functions given by 
\begin{equation}\label{chap1:7}
	y = x^2+C,
\end{equation}
one function for each value of $C$, comprises all solutions of the equation (\ref{chap1:1}). The collection (\ref{chap1:7}) is called the \textbf{\textit{general solution}} of (\ref{chap1:1}) and each member of (\ref{chap1:7}) corresponding to a specific value of $C$ is called a \textbf{\textit{particular solution}} of (\ref{chap1:1}). The concepts of particular and general solutions apply to all differential equations discussed in this book and will be used consistently without further explanation.



$\bullet$ DEs, solutions, domain of solutions, goal future predictions or long-term behavior of dependent variables, why intervals are so important for solutions?\\
$\bullet$ Emphasis on the rate of change of a quantity both mathematically and geometrically. 
\subsection{Geometry of First Order Differential Equations}
\subsubsection{Slope field;  sketch of solutions}
\subsection{Numerical Approximation of Solutions: Euler's Method}

\subsection{Mathematical Models as Linear Differential Equations}
Evolution of a quantity ...
\subsubsection{Population Growth}
\subsubsection{Mixing  Solutions}
\subsubsection{Electric Circuits}
\subsection{Mathematical Models as Nonlinear Differential Equations}
\subsubsection{Logistic Population Growth}
\subsubsection{Modeling the Spread of an Infectious Disease}
\subsection{Scientific Principles Leading to Differential Equations}
\subsubsection{Newton's law of motion, Newton's law of cooling/warming, Spring/Mass System with Linear/nonlinear Damping}

\subsection{Integration Methods}
\subsubsection{The $u-$subsitution Method}

\subsubsection{Integration by parts}
Given two differentiable functions $u$ and $v$ on an interval $I= [a, b],$ we have 
\[\frac{d}{dx}(u(x) v(x))= u(x)\frac{d}{dx}v(x) + v(x)\frac{d}{dx}u(x).\]
Then 
\begin{equation}\label{IBP1}
	\int_a^b \frac{d}{dx}(u(x) v(x))\, dx = \int_a^b  u(x)\frac{d}{dx}(v(x))\, dx + \int_a^bv(x)\frac{d}{dx}(u(x))\, dx.
\end{equation}

By using the fundamental theorem of calculus that states: given a differentiable $g:[a, b]\to\mathbb R$ with $g'$ continuous on $[a, b],$ we have 
\[\int_a^b g'(x) dx = g(b)- g(a) = [g(x)]_a^b.\] We  recall that $[g(x)]_a^b$ is also denoted by $g(x)|_a^b.$ With these notations  used in (\ref{IBP1}), we have
\begin{equation}\label{IBP2}
	[u(x) v(x)]_a^b = \int_a^b  u(x)\frac{d}{dx}(v(x))\, dx + \int_a^bv(x)\frac{d}{dx}(u(x))\, dx.
\end{equation}
Using the Leibniz notations, the formula (\ref{IBP2}) can be written as 
\begin{equation}\label{IBP3}
	\boxed{
	\int_a^b  u(x)v'(x)\, dx=[u(x) v(x)]_a^b- \int_a^b u'(x) v(x)\, dx.}
\end{equation}
The integration by part formula (\ref{IBP3}) can be written for antiderivatives as 
\begin{equation}\label{IBP4}
	\boxed{
		\int u(x)v'(x)\, dx=u(x) v(x)- \int u'(x) v(x)\, dx.}
\end{equation}



To compute integral \[\int x^2 e^{2x}\, dx,\] we will need to use the integration parts repeatedly. We set
 \( u = x^2 \), which we differentiate and obtain $du = 2x\, dx$, and  we set
\( dv = e^{2x}\, dx \) which we integrate and obtain $v = \frac{e^{2x}}{2}.$
Applying integration by parts formula (\ref{IBP4}), we have 

\[
\int x^2 e^{2x}\, dx =\displaystyle x^2\,  \frac{e^{2x}}{2} - \int 2x \frac{e^{2x}}{2} dx+A,
\]
where $A$ is the constant of integration.
Using integration by parts again on \( \displaystyle\int 2x\, \frac{e^{2x}}{2} dx \), we  have

\[
\int 2x\, \frac{e^{2x}}{2} dx  = 2x \frac{e^{2x}}{4} - \int 2\, \frac{e^{2x}}{4} dx
\]

\[
=  \frac{2xe^{2x}}{4} - \frac{2e^{2x}}{8}+B,
\]
where $B$ is a constant of integration.
Substituting this back:
\[
\begin{array}{lc}
	\displaystyle 
\int x^2 e^{2x}\, dx  &= \displaystyle \frac{x^2 e^{2x}}{2} - \left( \frac{2x e^{2x}}{4} - \frac{2e^{2x}}{8} +B\right)+A\\
&= \displaystyle x^2 \,\frac{e^{2x}}{2} - 2x\,\frac{e^{2x}}{4} + 2\, \frac{e^{2x}}{8}+C,
\end{array}
\]
where $C= A-B$ is an arbitrary constant.
 This procedure can be  organized in a table and is known as the \textbf{\textit{tabular method }}(also referred to as the \textbf{\textit{Kronecker method}}).
 More precisely,
the formula in \eqref{IBP4} can be written in the tabular form as follows: 


\[
\renewcommand{\arraystretch}{1.5}
\begin{array}{c @{\hspace*{2.0cm}} c}\toprule
	\text{Derivative } &  \text{Integral} \\\cmidrule{1-2}
	u(x)\tikzmark{Left 1} & \tikzmark{Right 1}v'(x) \\ \\
	u'(x) \tikzmark{Left 2} & \tikzmark{Right 2}v(x) \\      
	%	u''(x)  \tikzmark{Left 3} & \tikzmark{Right 3}\int v(x)\, dx  \\\bottomrule
\end{array}
\]
%-----------------------------------------
\DrawArrow[draw=red]{Left 1}{Right 2}{$+$}%
\DrawArrow[draw=brown]{Left 2}{Right 2}{$-$}%
%\DrawArrow[draw=blue]{Left 3}{Right 3}{$+$}%
We observe that along the horizontal arrow we still have the integral of the signed product.
For one more iteration  of the integration by parts, the tabular form is as follows:
\[
\renewcommand{\arraystretch}{1.5}
\begin{array}{c @{\hspace*{2.0cm}} c}\toprule
	\text{Derivative} &  \text{Integral} \\\cmidrule{1-2}
	u(x)\tikzmark{Left 1} & \tikzmark{Right 1}v'(x) \\
	u'(x) \tikzmark{Left 2} & \tikzmark{Right 2}v(x) \\      \\
	u''(x)  \tikzmark{Left 3} & \tikzmark{Right 3}\int v(x)\, dx  \\\bottomrule
\end{array}
\]
%-----------------------------------------
\DrawArrow[draw=red]{Left 1}{Right 2}{$+$}%
\DrawArrow[draw=brown]{Left 2}{Right 3}{$-$}%
\DrawArrow[draw=blue]{Left 3}{Right 3}{$+$}%
and the formula for the integration by parts then becomes
\begin{equation}\label{eq:IBP-extended}
	\int u(x)v'(x)\, dx=u(x) v(x)-u'(x)  \int  v(x)\, dx+ \int \left(u''(x)\, \int  v(x)\, dx\right)\, dx.
\end{equation}
This process can be continued further. The entries in the first column are obtained through successive differentiation, while the entries in the second column are obtained by repeatedly computing antiderivatives. For example, we compute $\int x^2 e^{2x}\, dx$ discussed above by using the tabular method as follows:

%\tikzset{Arrow Style/.style={text=black, font=\boldmath}}
%
%\newcommand{\tikzmark}[1]{%
%	\tikz[overlay, remember picture, baseline] \node (#1) {};%
%}
%
%\newcommand*{\XShift}{0.5em}
%\newcommand*{\YShift}{0.5ex}
%
%\NewDocumentCommand{\DrawArrow}{s O{} m m m}{%
%	\begin{tikzpicture}[overlay,remember picture]
%		\draw[->, thick, Arrow Style, #2] 
%		($(#3.west)+(\XShift,\YShift)$) -- 
%		($(#4.east)+(-\XShift,\YShift)$)
%		node [midway,above] {#5};
%	\end{tikzpicture}%
%}

	\[
	\renewcommand{\arraystretch}{1.5}
	\begin{array}{c @{\hspace*{1.0cm}} c}\toprule
		\text{Derivative} &  \text{Integral} \\\cmidrule{1-2}
		x^2\tikzmark{Left 1} & \tikzmark{Right 1}e^{2x} \\
		2x \tikzmark{Left 2} & \tikzmark{Right 2}\frac{1}{2} e^{2x} \\      
		2  \tikzmark{Left 3} & \tikzmark{Right 3}\frac{1}{4} e^{2x} \\      
		0  \tikzmark{Left 4} & \tikzmark{Right 4}\frac{1}{8} e^{2x} \\\bottomrule
	\end{array}
	\]
	%-----------------------------------------
	\DrawArrow[draw=red]{Left 1}{Right 2}{$+$}%
	\DrawArrow[draw=brown]{Left 2}{Right 3}{$-$}%
	\DrawArrow[draw=blue]{Left 3}{Right 4}{$+$}%
	
\[\int x^2 e^{2x}\, dx = \displaystyle \frac{1}{2} x^2 \, e^{2x} - \frac{1}{2}x\,e^{2x} + \frac{1}{4}\,e^{2x}+C,\]
where $C$ is an arbitrary constant.

The construction of the table in a tabular form can also be bypassed by applying a suitable version of \eqref{eq:IBP-extended} directly, as illustrated in the following example.
We compute
\[
\begin{array}{ll}
	\displaystyle\int x^4 e^{2x}\, dx& = \displaystyle x^4 \frac{e^{2x}}{2} - 4x^3\, \frac{e^{2x}}{4} + 12x^2 \, \frac{e^{2x}}{8} - 24x\, \frac{e^{2x}}{16}+24 \frac{e^{2x}}{32} +C\\
	&=\displaystyle \frac14e^{2x}\left(2 x^4 - 4x^3 +6x^2-6x+ 3\right) +C,
\end{array}
\]
where $C$ is an arbitrary constant.
%Bibliography
\medskip
\medskip
%\addcontentsline{toc}{section}{Bibliography}
%List the items cited from the references.bib file
%\nocite{Royden, Rudin}
\bibliographystyle{abbrvnat}
\bibliography{references}

\setcounter{chapter}{1}
\chapter{First Order Differential Equations}

\section{Separation of Variables}

\section{Integrating Factors}

\section{Slope Fields}

\section{Euler’s Method}

\section{Applications} 	
\setcounter{chapter}{2}


\chapter{Second and Higher Order Linear Differential Equations}\label{chap:higher-order}
\mtcsetoffset{minitoc}{-1em}
% \mtcsetpagenumbers{minitoc}{off}

\minitoc
\section{Introduction}
The $n^\text{th}$ order linear differential equations are of the form
\begin{equation}\label{eq:nonhomogeneous-0}
	a_n(x)y^{(n)} + a_{n-1}(x)y^{(n-1)} + \cdots + a_1(x)y' + a_0(x)y = f(x)
\end{equation}
where   the  functions \( a_0(x), \dots, a_{n-1}(x), a_n(x) \) and   \( f(x) \) are continuous on some interval \( I .\)
We say that \eqref{eq:nonhomogeneous-0} is \textbf{\textit{homogeneous}}\index{homogeneous} on $I$ if $f(x) = 0$ for all $x$ in $I$ and \textbf{\textit{nonhomogeneous}}\index{nonhomogeneous} on $I$ if $f(x)\ne 0$ for some $x$ in $I.$  We will usually have $a_n(x) \ne 0$ on $I.$ If $a_n(x=0$ for some point $x$ of $I,$ we will simply restrict our attention to a smaller interval, denoted again by \(I\), on which $a_n(x) \ne 0.$  With these consideration, \eqref{eq:nonhomogeneous-0} can be   rewritten in the form
\begin{equation}\label{eq:nonhomogeneous0}
	y^{(n)} + a_{n-1}(x)y^{(n-1)} + \cdots + a_1(x)y' + a_0(x)y = f(x),
\end{equation}
 \( a_0(x), \dots, a_{n-1}(x) \) and   \( f(x) \) are continuous on some interval \( I .\) This is achieved by dividing both sides of \eqref{eq:nonhomogeneous-0} by $a_n(x)$ to obtain
 \[y^{(n)}+\frac{a_{n-1}(x)}{a_n(x)}y^{(n-1)}+ \cdot+\frac{a_1(x)}{a_n(x)}y'+\frac{a_0(x)}{a_n(x)}y= \frac{f(x)}{a_n(x)} \]
  and redefining the   functions \[ \frac{a_{n-1}(x)}{a_n(x)}, \dots,\frac{a_1(x)}{a_n(x)}, \frac{a_0(x)}{a_n(x)}, \frac{f(x)}{a_n(x)} \] as \[ a_{n-1}(x), \dots,a_1(x), a_0(x), f(x), \] respectively.
 
\begin{definition}[Solution]\label{def:solutions}
	A function $y = \phi(x)$ defined on $I$ is said to be a \textbf{\textit{solution}} of \eqref{eq:nonhomogeneous0} if 
	\[\phi^{(n)}(x) + a_{n-1}(x)\phi^{(n-1)}(x) + \cdots + a_1(x)\phi'(x) + a_0(x)\phi(x) = f(x)\] for all $x$ in $I,$ that is, if
	\eqref{eq:nonhomogeneous0} is satisfied for all $x$ in $I$ when we substitute $\phi(x)$  for $y.$
\end{definition}

%In Definition~\ref{def:solutions},  the derivatives of $\phi(x)$ at the endpoints (if any) of \(I\) are interpreted as one-sided derivatives.

\begin{remark}
	In Definition~\ref{def:solutions}, we note the following.
	\begin{enumerate}[label=(\roman*), noitemsep]
		\item Derivatives of $\phi(x)$ at the endpoints (if any) of \(I\)  are understood as one-sided.
		\item Solutions $y = \phi(x)$ may be complex-valued, though real solutions are preferred in  applications. For $\phi(x) = \phi_1(x) + i \phi_2(x)$ with real $\phi_1$ and $\phi_2$, 
		\[
		\phi'(x) = \phi_1'(x) + i \phi_2'(x),
		\]
		where $i = \sqrt{-1}$, and higher derivatives are computed similarly.
	\end{enumerate}
\end{remark}


%\begin{remark}
%	In Definition~\ref{def:solutions}, we note the following:
%	\begin{enumerate}[label=(\roman*), noitemsep]
%		\item Derivatives of $\phi(x)$ at the endpoints (if any) of \(I\) are interpreted as one-sided derivatives.
%		\item The solution $y = \phi(x)$ are allowed to be complex-valued although real-valued functions are desirable for solutions. If $\phi(x) = \phi_1(x) + i \phi_2(x)$, with real-valued $\phi_1$ and $\phi_2$, then 
%		\[
%		\phi'(x) = \phi_1'(x) + i \phi_2'(x),
%		\]
%		 and higher order derivatives are computed similarly.
%		Here $i = \sqrt{-1}$ is the imaginary unit.
%	\end{enumerate}
%\end{remark}

\begin{definition}[Initial Value Problem]
	Let   \( a_0(x), \dots, a_{n-1}(x), f(x) \)  be continuous functions on some interval \( I \), $x_0$ be a point in $I$, and $y_1,\dots, y_{n-1}$ be fixed numbers.  The $n^\text{th}$ order linear differential equation
\[
	y^{(n)} + a_{n-1}(x)y^{(n-1)} + \cdots + a_1(x)y' + a_0(x)y = f(x)
\]
subject to the conditions (called initial conditions) 
\[\\ y(x_0) = y_0,\;  y'(x_0) = y_1, \dots, y^{(n-1)}(x_0) = y_{n-1},\]
is called an \textbf{\textit{initial value problem} (IVP)}.
\end{definition}

It is essential to know when an initial value problem has a solution. This is addressed by the existence and uniqueness theorem stated below, whose proof requires advanced calculus beyond the scope of this book.


\begin{theorem}[Existence and Uniqueness]\label{thm:existence-theorem}
	Let    \( a_0(x), \dots, a_{n-1}(x), f(x) \)  be continuous functions on some interval \( I \), $x_0$ be a point in $I$, and $y_1,\dots, y_{n-1}$ be fixed numbers. Then the
	 initial value problem (IVP)  
	\[ 
		\begin{cases}
		y^{(n)} + a_{n-1}(x)y^{(n-1)} + \cdots + a_1(x)y' + a_0(x)y = f(x)\\ y(x_0) = y_0,\;  y'(x_0) = y_1, \dots, y^{(n-1)}(x_0) = y_{n-1},
		\end{cases}
	\]
	 has a unique solution  \(y = \phi(x) \) defined on $I.$ 
\end{theorem}

\begin{remark}
It follows from Theorem~\ref{thm:existence-theorem} that the  function $\phi(x) =0$ for all $x$ in $I$ is the unique solution of the IVP when $y_1=y_2 =\cdots = y_{n-1} = 0.$  By the uniqueness,
if at least one of $y_1, \dots, y_{n-1}$ is nonzero, then the unique solution of the IVP in Theorem~\ref{thm:existence-theorem} must be a nonzero function.
\end{remark}

 


The superposition principle discussed below provides a general procedure for constructing new solutions from known ones.

\begin{framed}
\noindent\textbf{Superposition Principle}:\label{superposition} Suppose that $y_1, \dots, y_k$ are solutions  of the  homogeneous linear equation 
\[y^{(n)} + a_{n-1}(x)y^{(n-1)} + \cdots + a_1(x)y' + a_0(x)y = 0,\]
where   \( a_0(x), \dots, a_{n-1}(x) \)  are continuous on some interval \( I .\) Then, for any constants $c_1, \dots, c_k$ (possibly  complex), the function 
\[y(x) = c_1y_1(x) + \cdots+c_k y_k(x) = \sum_{j=1}^k c_j y_j(x)\] is a solution on $I.$  The expression \(\sum_{j=1}^k c_j y_j(x)\) is called a \textbf{\textit{linear combination}} (or \textbf{\textit{superposition}}) of $y_1, \dots, y_k,$ with \textbf{\textit{weights}}  $c_1, \dots, c_k.$  
\end{framed}
\begin{proof}
	We treat here the special case  $n=2$ for clarity, and leave the general case as an exercise. Let $y_1$ and $y_2$  be solutions on $I$ of 
	\begin{equation}\label{eq:homogeneous-little}
		y'' + a_1(x)y' + a_0(x)y = 0,
	\end{equation}
	and let $y = c_1 y_1 + c_2y_2$ for constants $c_1$ and $c_2.$
Then
	\[y' = c_1 y'_1 + c_2y'_2\quad\mbox{and}\quad y'' = c_1 y''_1 + c_2y''_2\] on $I.$  We note that
	the differentiation rules are the same whether \(c_1\) and \(c_2\) are real or complex.
Substituting these into the left side of the differential equation gives
	\[\begin{split}
&y'' + a_1(x)y' + a_0(x)y\\
&=c_1\big(y''_1 + a_1(x)y'_1 + a_0(x)y_1\big)+c_2\big(y''_2 + a_1(x)y'_2 + a_0(x)y_2\big)\\
\end{split}\] on $I.$
	 Since $y_j$ satisfies
	 \[y''_j + a_1(x)y'_j + a_0(x)y_j = 0\] on $I$ for $j = 1, 2.$ Thus,
	 $y= c_1y_1 +c_2y_2$ is a solution of \eqref{eq:homogeneous-little} on $I.$ 
\end{proof}

\begin{remark} The superposition principle provides two  basic operations for constructing new solutions from known solutions.
	\begin{enumerate}[label=(\roman*),noitemsep]
		\item If $y_1$ is a solution, then $cy_1$ is also a solution for any constant $c$.
		\item If $y_1$ and  $y_2$ are solutions, then so is $y_1 +y_2.$
	\end{enumerate}
\end{remark}

\begin{example}
	We can verify that $y_1 = \cos x$ and $y_2 = \sin x$ are solutions of $y''+y = 0,$ which is a  homogeneous linear equation. Moreover, $y= 7 \cos x$  and $y = 2\cos x -3 \sin x$ are also solutions. In fact, $y = c_1 \cos x$ and $ y = c_1 \cos x+c_2 \sin x$  are solutions for any constants $c_1$ and $c_2.$
\end{example}

When building new solutions a  homogeneous linear differential equation from a collection of known solutions, it is desirable to use a minimal set of solutions as building blocks. Even if several solutions are  available, we may not actually need all of them, and some solutions can be discarded based on their properties. This important situation is addressed in the next subsection.

\subsection{Linear Independence}

Two functions \( y_1 \) and \( y_2 \) defined on an interval \( I \) are said to be linearly independent on \( I \) if neither function is a constant multiple of the other. For instance, the functions \( e^x \) and \( e^{-x} \) are linearly independent on \( (-\infty, \infty) \), as are \( \sin x \) and \( \cos x \) on \( (-\infty, \infty) \). The concept of linear independence of functions extends naturally to any finite collection of functions, as formalized in the following definition.


\begin{definition}[Linear Independence and Dependence]\label{defn:linear-indepednence}
	
	The functions \( y_1, \dots, y_k \) defined on an interval \( I \), where \( k \) is a positive integer, are said to be \textbf{\textit{linearly independent on \( I \)}} if the relation
	\[
	c_1 y_1(x) + \cdots + c_k y_k(x) = 0
	\]
	holds for all $x$ in $I$ \textit{only when} the constants $c_1, \dots, c_k$ satisfy \( c_1 = \cdots = c_k = 0 \).
	If this condition does not hold, that is, if there exist constants \( c_1, \dots, c_k \), \textit{not all zero}, satisfying the above relation for all $x$ in 
	\( I \), then the functions 
	\( y_1, \dots, y_k\) are said to be \textbf{\textit{linearly dependent on \( I \)}}.
\end{definition}

We say that the set \(\{y_1, \dots, y_k\}\) is linearly independent on $I$ if the functions $y_1, \dots, y_k$  are linearly independent on $I.$ Similarly,  we say that the set \(\{y_1, \dots, y_k\}\) is linearly dependent on $I$ if the functions $y_1, \dots, y_k$  are linearly dependent on $I.$


The  linear dependence or independence of a set of functions on an interval may change when the interval is changed or when functions are added or removed. It follows directly from the definition that if the set \(\{y_1, \dots, y_k\}\) is linearly independent on \(I,\) then it remains linearly independent on any larger interval \(J\) containing \(I,\) provided the functions are defined on \(J.\) Similarly, if the set \(\{y_1, \dots, y_k\}\) is linearly dependent on \(I,\) it will also be linearly dependent on every subinterval \(J\) of \(I.\) 

 Moreover, using the definition above, any subset of a linearly independent set of functions is itself linearly independent, whereas  a set that contains a linearly independent set can  be linearly dependent. Similarly, a set containing a linearly dependent set of functions is necessarily linearly dependent, whereas a subset of a linearly dependent set can be linearly independent.






Suppose the functions \(y_1, \dots, y_k\) are linearly dependent on an interval \(I\). Then there exist constants \(c_1, \dots, c_k\), not all zero, such that  
\begin{equation}\label{eq:linear-dependence}
	c_1 y_1(x) + \cdots + c_k y_k(x) = 0
\end{equation}
for all \(x\) in \(I.\)
Since the constants are not all zero, at least one of them must be nonzero. Suppose, for instance, that \(c_1\ne 0.\)
Then, from \eqref{eq:linear-dependence}, we can express 
\[
y_1(x) = \left(-\frac{c_2}{c_1}\right) y_2(x) 
+ \left(-\frac{c_3}{c_1}\right) y_3(x) 
+ \cdots 
+ \left(-\frac{c_k}{c_1}\right) y_k(x)
\]
on $I$. This 
 shows that \(y_1\) is a linear combination of the remaining functions.
Similarly, if \(c_2 \ne 0,\) then  
\[
y_2(x) = \left(-\frac{c_1}{c_2}\right) y_1(x) 
+ \left(-\frac{c_3}{c_2}\right) y_3(x) 
+ \cdots 
+ \left(-\frac{c_k}{c_2}\right) y_k(x)
\]
on $I$, and 
so \(y_2\) is a linear combination of the remaining functions.
In general, for any integer \(k_0\) with \(2 \le k_0 \le k\) and \(c_{k_0} \ne 0,\) 
\eqref{eq:linear-dependence} implies that \(y_{k_0}\) is a linear combination of the remaining functions. 
Thus, if $\{y_1, \dots, y_k\}$ is a linearly dependent set of functions $y_1, \dots, y_k$ defined on $I,$ then at least one of the functions can be expressed as a linear combination of the others. In particular, this means that any set $\{y_1, \dots, y_k\}$ containing the zero function is necessarily linearly dependent.

From a linearly dependent set, it is often desirable to remove any function that can be expressed as a linear combination of the renaming functions.  By repeatedly removing such functions, we eventually obtain a linearly independent set. The functions in this linearly independent set can then be used to construct new functions by forming linear combinations.



\begin{example}\label{ex:test-for-indepedence-1} 
Determine whether $1, \cos^2x,$ and $\sin^2x$ are linearly independent on \((-\infty, \infty).\)
\end{example}
\begin{solution}
	Suppose there are constants $c_1, c_2,$ and $ c_3$ that satisfy 
	\[c_1 + c_2\cos^2x + c_3\sin^2x=0\] for all $x$ in $(-\infty, \infty).$ The main question here is: can we have at least one of $c_1, c_2, c_3$ nonzero? In view of $\cos^2 x+ \sin^2 x=1$, we can take $c_1 = -1, c_2=c_3 =1.$  Therefore $1, \cos^2x,$ and $\sin^2x$ are linearly dependent on \((-\infty, \infty).\)
	\end{solution}
	
	
\begin{example}\label{ex:test-for-indepedence-2} 
	Determine whether $1, e^{x}$ and $x e^{x}$ are linearly independent on \((-\infty, \infty).\)
\end{example}
\begin{solution}
	Suppose there are constants $c_1, c_2,$ and $ c_3$ that satisfy 
	\begin{equation}\label{eq:linear-indepedence-example-1}
		c_1 + c_2e^x + c_3 xe^x=0
	\end{equation} for all $x$ in $(-\infty, \infty).$ Again, can we have at least one of $c_1, c_2, c_3$ nonzero? Differentiating \eqref{eq:linear-indepedence-example-1} both sides with respect to $x$ gives
	\[	c_2e^x + c_3 e^x+ c_2 x e^x=0\]
 for all $x$ in $(-\infty, \infty).$ Since $e^x$ is never zero,
 we have \begin{equation}\label{eq:linear-indepedence-example-2}
 	c_2 + c_3 + c_3 x =0
 \end{equation}
 for all $x$ in $(-\infty, \infty).$ Differentiating \eqref{eq:linear-indepedence-example-2} both sides with respect to $x$ yields \(c_3=0.\)
	   Using $c_3=0$ in \eqref{eq:linear-indepedence-example-2} yields  $c_2 = 0.$ Using $c_2=c_3=0$ in \eqref{eq:linear-indepedence-example-1}, we get $c_1=0.$ Therefore $1, e^{x}$ and $x e^{x}$are linearly independent on \((-\infty, \infty).\) 
\end{solution}

\subsection{The Wronskian}
The methods used in Example~\ref{ex:test-for-indepedence-1} and Example~\ref{ex:test-for-indepedence-2} to test the linear independence of functions can become cumbersome, especially when the functions involved are more complicated than those in the examples. In this book, we will typically need to determine whether a  set of solutions of    homogeneous linear differential equations is linearly independent.  For such  sets of solutions, the linear independence of  solutions  can be determined by using the \textbf{Wronskian}, a determinant involving solutions and their derivatives,  introduced in 1812 by  the Polish mathematician Józef Wroński. 
\begin{definition}[Wronskian]\label{defn:wronskian}
Let  $y_1, y_2, \dots, y_n$ be functions that are $n-1$ times differentiable on an interval $I.$
  Their  \textbf{\textit{Wronskian}}, denoted by \(W(y_1, \dots, y_{n-1})(x)\) (or simply by $W(x)$),   is a function on $I$ defined by
\[
W(y_1,\dots,y_n)(x)
=
\begin{vmatrix}
	y_1(x) & y_2(x) & \cdots & y_n(x)\\
	y_1'(x) & y_2'(x) & \cdots & y_n'(x)\\
	\vdots & \vdots & \ddots & \vdots\\
	y_1^{(n-1)}(x) & y_2^{(n-1)}(x) & \cdots & y_n^{(n-1)}(x)
\end{vmatrix}
\] on \(I.\)
\end{definition}

\begin{example}\label{eg:wronskian-0}
	Compute the Wronskian of \(y_1=e^x\) and $y_2=xe^{-x}$ on 
\((-\infty, \infty).\)
\end{example}
\begin{solution}
	The Wronskian \(W(y_1,y_2)(x)\) of \(y_1=e^x\) and $y_2=xe^{-x}$ is 
	\[
	 W(e^x,xe^{-x})(x)
	=
	\begin{vmatrix}
		y_1(x) & y_2(x) \\
		y_1'(x) & y_2'(x) \\
		\end{vmatrix}
			= \begin{vmatrix}
				e^x & xe^{-x} \\
				e^x & e^{-x}-xe^{-x} \\
			\end{vmatrix}
			 = 1-2x\] on \((-\infty, \infty).\)
\end{solution}
The next theorem provides a Wronskian test for linear independence of differentiable functions.

\begin{theorem}[Test for Linear Independence]\label{thm:wronskian-nonzero-1}
Let $y_1, \dots, y_k$ be \(k-1\) times differentiable functions on an interval $I$ with  \[W(y_1, \dots, y_k)(x)\ne 0 \mbox{ on } I.\] Then $y_1, \dots, y_k$ are linearly independent on $I.$
\end{theorem}	
\begin{proof} For simplicity, we treat the case $k=2;$ the general case is analogous.
	Suppose that $y_1$ and $y_2$ are differentiable functions on an interval $I$ such that $W(y_1, y_2)(x)\ne 0$ on $I.$ Let $c_1$ and $c_2$ be constants such that 
	\begin{equation}\label{eq:wronskian-zero-1}
		c_1 y_1(x) + c_2 y_2(x) = 0
	\end{equation} for all $x$ in $I.$ Differentiating gives
	\begin{equation}\label{eq:wronskian-zero-2}
		c_1 y_1'(x) + c_2 y_2'(x) = 0
	\end{equation} for all $x$ in $I.$ 
	Since \(W(y_1,y_2)(x)\neq 0\) for every $x$ in $I,$ the linear system consisting of the  \eqref{eq:wronskian-zero-1} and \eqref{eq:wronskian-zero-2} has the unique solution \(c_1=c_2=0\) for every $x$ in $I.$ Hence \(y_1\) and \(y_2\) are linearly independent on \(I\).
\end{proof}

\begin{remark}$\empty$
	\begin{enumerate}[label= (\roman*),noitemsep]
		\item 
		The conclusion of Theorem~\ref{thm:wronskian-nonzero-1} holds even if \(W(y_1,\dots,y_k)(x_0)\neq 0\) at just one point $x_0$ in $I$. The case \(k=2\) appears as Problem~2(i) in Exercises~\ref{EX31}.
		
		\item 	The converse of Theorem~\ref{thm:wronskian-nonzero-1} is not true. For example, take the differentiable functions  $y_1= x^2$ and $y_2 = x\abs{x}$ on $(-\infty, \infty).$ We know from Example that $y_1, y_2$ are linearly independent on $(-\infty, \infty);$  however, their Wronskian
		\[W(x^2, x\abs{x}) = \begin{vmatrix}
			x^2&x\abs{x}\\
			2x &2\abs{x}
		\end{vmatrix} =0\] 
		on $(-\infty, \infty).$
		\item We will see in Theorem~\ref{thm:wronskian-nonzero-2} for the case $k=2$ and in Theorem~\ref{thm:wronskian-nonzero-3} for the general case that the converse of  Theorem~\ref{thm:wronskian-nonzero-1} also holds if $y_1, \dots, y_k$ are solutions of a $k^\text{th}$ order homogeneous linear differential equation.
	\end{enumerate}
\end{remark}



\begin{Exercise}\label{EX31}
	\vspace{-\baselineskip}% <-- You don't need this line of code if there's some text here
	\Question\label{3-1-1}%Linear-depedence 		
	Determine whether each set of functions below is linearly independent on the specified interval using Definition~\ref{defn:linear-indepednence} or the discussion that follows it.
	\begin{tasks}(2)
		\task\(\{1, x, x^2\}\), \quad \((-\infty, \infty)\)
		\task\(\{1, x-1, 2x\}\), \quad \((-\infty, \infty)\)
		\task\(\{x, \abs{x}\}\), \quad \((0, \infty)\) 
		\task\(\{x, \abs{x}\}\), \quad \((-\infty,0)\) 
		\task\(\{x, \abs{x}\}\), \quad \((-\infty, \infty)\) 
		\task \(\{e^{2x}, e^{-2x}\}\), \quad \((-\infty, \infty)\)
		\task \(\{e^{2x}, xe^{2x}\}\), \quad \((-\infty, \infty)\)
		\task\(\{\sin x, \cos x\}\), \quad \((-\infty, \infty)\)
		\task\(\{1, \sin x, \cos x\}\), \quad \((-\infty, \infty)\)
		\task\(\{0, \sin x, \cos x\}\), \quad \((-\infty, \infty)\)
		\task\(\{\sin^2 x, \cos^2 x\}\), \quad \((-\infty, \infty)\)
		\task\(\{1, \sin^2 x, \cos^2 x\}\), \quad \((-\infty, \infty)\)
		\task \(\{\frac{1}{x}, x^3\}\), \quad \((0, \infty)\)
		\task \(\{\cos(\ln x), \sin(\ln x)\}\),  \quad \((0, \infty)\)
			
	
	
	\end{tasks}

	
	\Question\label{3-1-2} Suppose that $y_1,y_2$ are differentiable functions on an interval $I.$ 
	\begin{tasks}(1)
		\task	Let $x_0$ be point in $I$ such that 
		$W(y_1, y_2)(x_0)\ne 0.$ Show that \(y_1, y_2\) are linearly independent on $I.$
		\task Determine whether the functions $y_1 = e^x$ and $y_2 = xe^{-x}$ considered in Example~\ref{eg:wronskian-0} are linearly independent on $(-\infty, \infty).$
		\task Find examples of \(y_1, y_2\) that are linearly independent on $I$ and satisfy 
		$W(y_1, y_2)(x)= 0$ for all $x$ in $I.$
	\end{tasks}
\end{Exercise}

\setboolean{firstanswerofthechapter}{true}
\begin{multicols}{2}\scriptsize
	\begin{Answer}[ref={EX31}]
		\Question \label{3-1-1a}
		\begin{tasks}
			\task Linearly independent 
			\task Linearly dependent
			\task Linearly dependent 
			\task Linearly dependent 
			\task Linearly independent 
			\task Linearly independent 
			\task Linearly independent 
			\task Linearly independent 
			\task Linearly independent 
			\task Linearly dependent 
			\task Linearly independent 
			\task Linearly dependent
			\task Linearly independent 
			\task Linearly independent 
		\end{tasks} 
		
	\Question \label{3-1-2a}
	\begin{tasks}
		\task 	Suppose, for a contradiction, that $y_1,y_2$ are linearly dependent on $I$. Then there exist constants $c_1,c_2$, not both zero, such that
		\[
		c_1y_1(x)+c_2y_2(x)=0\qquad\text{for all \(x\) in \(I\)}.
		\]
		Differentiating gives
		\[
		c_1y_1'(x)+c_2y_2'(x)=0\qquad\text{for all \(x\) in \(I\)}.
		\]
		Evaluating these two identities at $x=x_0$ yields
		\[
		\begin{pmatrix}
			y_1(x_0) & y_2(x_0)\\[4pt]
			y_1'(x_0) & y_2'(x_0)
		\end{pmatrix}
		\begin{pmatrix} c_1\\ c_2\end{pmatrix}
		=\begin{pmatrix}0\\0\end{pmatrix}.
		\]
		The determinant of the coefficient matrix is $W(y_1,y_2)(x_0)\neq0$, so the matrix is invertible and the only solution is $c_1=c_2=0$, contradicting the assumption. Thus $y_1,y_2$ are linearly independent on $I$.
		\task It follows from part (i).
		\task Take $y_1(x) = x^2$ and $y_2(x) = x\abs{x}$ on \((-\infty, \infty).\) Then \(W(y_1, y_2)(x) = 0\) on \((-\infty, \infty).\)
	\end{tasks}	
	
	\end{Answer}
\end{multicols}
\setboolean{firstanswerofthechapter}{false}



%\subsection{The Wronskian Test for Linear Independence}
\section{Second Order Homogeneous Linear Equations}
 We now discuss the Wronskian of solutions of second order homogeneous  second order linear  equations. The same reasoning naturally extends  to higher order linear equations (see Section~\ref{sec:higher order DEs} for higher order  equations).
 
 Let \( y_1 \) and \( y_2 \) be solutions of the  homogeneous linear  differential equation  
 \[
 y'' + a_1(x)y' + a_0(x)y = 0,
 \]
 where \( a_1(x) \) and \( a_0(x) \) are continuous functions on an interval $I.$  
 The \textit{Wronskian} of \( y_1 \) and \( y_2 \), denoted by \( W(y_1, y_2)(x) \), \( W(y_1(x), y_2(x)) \), or simply \( W(x) \), is defined as  
 \[
 W(x) =
 \begin{vmatrix}
 	y_1(x) & y_2(x) \\
 	y_1'(x) & y_2'(x)
 \end{vmatrix}
 = y_1(x)y_2'(x) - y_1'(x)y_2(x).
 \]
Let is examine the derivative of the Wronskian $W(x).$  We have
\[
\begin{split}
	W'(x) &= y_1(x)y_2''(x)+y'_1(x)y'_2(x) - y_1'(x)y'_2(x)- y_2(x)y''_1(x)\\
	&=y_1(x)y_2''(x)-y_2(x)y''_1(x)\\
	&=y_1(x)\big(-a_1(x) y'_2(x)- a_0(x)y_2(x)\big)\\
	&\quad - y_2(x)\big(-a_1(x) y'_1(x)- a_0(x)y_1(x)\big)\\
	&=- a_1(x) W(x),
\end{split}\]
where we used \(y_1''(x) = -a_1(x) y'_1(x)- a_0(x)y_1(x)\) and \(y_2''(x) = -a_1(x) y'_2(x)- a_0(x)y_2(x).\)
It then follows that 
\begin{equation}\label{eq:Abel's-formula}
	W(x) =W(x_0)\; e^{-\int_{x_0}^x a_1(t)\, dt},
\end{equation}  where $x_0$ is a fixed number in $I.$ 
Formula \eqref{eq:Abel's-formula} is known as \textbf{Abel's formula}. It follows from this formula that \(W(x)\) is a constant  when $a_1(x) =0$  on $I.$ Also, if $W(x_0)= 0$, then $W(x) = 0$ for all $x$ in $I.$ On the other hand,  since \[e^{-\int_{x_0}^x a_1(t)\, dt} >0 \quad \mbox{ for all } x \mbox{ in } I,\]  it follows that $W(x) <0$ for all $x$ in $I$ if $W(x_0)<0,$ and $W(x) >0$ for all $x$ in $I$ if $W(x_0)>0.$   Thus, $W(x)$ is either identically zero or does not change its sign in $I.$

\begin{theorem}\label{thm:wronskian-nonzero-2}
	 Let \( y_1 \) and \( y_2 \) be solutions  of the second order  homogeneous linear  differential equation  
	
	\begin{equation}\label{eq:homogeneous-linear-second-order-de}
		y'' + a_1(x)y' + a_0(x)y = 0,
	\end{equation}
	where \( a_1(x) \) and \( a_0(x) \) are continuous functions on an interval $I.$  Then $y_1$ and $y_2$ are linearly independent on $I$ if and only if $W(y_1, y_2)(x)\ne 0$ for all $x$ in $I.$
\end{theorem}
\begin{proof}
Let \(y_1\) and \(y_2\) be linearly independent solutions of \eqref{eq:homogeneous-linear-second-order-de} on \(I\). Suppose, for a contradiction, that there is a point $x_0$ in $I$ such that
	\(W(y_1,y_2)(x_0)= 0.\) Let  $c_1$ and $c_2$ be constants. Then the linear system
	\begin{equation}
		\begin{cases}\label{eq:system}
		c_1 y_1(x_0) + c_2 y_2(x_0) = 0\\
		c_1 y_1'(x_0) + c_2 y_2'(x_0) = 0
	\end{cases}
	\end{equation}
	has a solution \(c_1, c_2,\) not both zero because the determinant of the coefficient matrix is zero. Fix $c_1, c_2$, not both zero,  satisfying \eqref{eq:system}, and define  the function \[\phi(x) = c_1 y_1(x) + c_2 y_2(x)\] on $I.$ By \eqref{eq:system} and the superposition principle,  \( y = \phi(x)\) is a  solution on $I$  of the initial value problem
	\[y''+a_1(x) y' + a_2(x) y = 0, \quad y(x_0) = 0,\; y'(x_0) =0.\] The uniqueness part of Theorem~\ref{thm:existence-theorem} implies  \(\phi(x) =0\) on $I,$ that is,
	\[c_1 y_1(x) + c_2 y_2(x) = 0\] on $I.$ Thus, $y_1$ and $y_2$ are linearly dependent on $I$, contradicting our assumption. Hence  \(W(y_1,y_2)(x)\neq 0\) for every $x$ in $I.$

	
	Suppose that $W(y_1, y_2)(x)\ne 0$ on $I.$ By Theorem~\ref{thm:wronskian-nonzero-1}, \(y_1\) and \(y_2\) are linearly independent on \(I\). Note that \(y_1\) and \(y_2\) need not be solutions \eqref{eq:homogeneous-linear-second-order-de} for this part of the theorem to hold.
	
\end{proof}
\subsection{Existence of Fundamental Sets of Solutions}
\begin{definition}
Consider the second order  homogeneous linear  differential equation  
\[
y'' + a_1(x)y' + a_0(x)y = 0,
\]
where \( a_1(x) \) and \( a_0(x) \) are continuous functions on an interval $I.$  Two solutions $y_1$ and $y_2$  that are linearly independent on $I$ are said to form a \textbf{\textit{fundamental set of solutions}} of the differential equation. Each of these solutions, $y_1$ and $y_2$, is referred to as a fundamental solution.
\end{definition}

An important question arises: does  a fundamental set of solutions always exist?   The following theorem ensures the existence of a fundamental set of solutions for  second order  homogeneous linear equations when  $a_1(x)$ and $a_0(x)$ are continuous. These concepts  extend naturally to higher order  equations.


\begin{theorem}[Existence of a Fundamental Set of Solutions]\label{thm:existence of a fundamental set}
	Consider the second order  homogeneous linear  differential equation  
	\begin{equation}\label{eq:second order-Linear_homo-de}
	y'' + a_1(x)y' + a_0(x)y = 0,
	\end{equation}
	where \( a_1(x) \) and \( a_0(x) \) are continuous functions on an interval $I.$ Then there exist two functions $y_1$ and $y_2$ on $I$ that form a fundamental set of solutions for  \eqref{eq:second order-Linear_homo-de}.
\end{theorem}


\begin{proof} By Theorem~\ref{thm:existence-theorem}, there exists a unique solutions of $y_1$  the initial value problem
	\[
	y'' + a_1(x)y' + a_0(x)y = 0, \quad y(x_0) = 1,\; y'(x_0) = 0,
	\]
	and  a unique  $y_2$ of the initial value problem \[
	y'' + a_1(x)y' + a_0(x)y = 0, \quad y(x_0) = 0,\; y'(x_0) = 1.
	\]
	The two solutions $y_1$ and $y_2$ are linearly independent on $I.$ In fact, let $c_1, c_2$ be constants such that 
	$c_1y_1(x) + c_2y_2( x) = 0$ for all $x$ in $I.$ Since $x_0$ is in $I,$ we have
	$c_1y_1(x_0) + c_2y_2( x_0) = 0,$ which implies $c_1 =0$. We then only have $c_2y_2(x) = 0$ for all $x$ in $I.$  Differentiating $ c_2y_2( x) = 0$ yields $c_2 y'_2(x) = 0$ for all $x$ in $I.$ Since $y'_2(x_0) = 1,$ we must have $c_2 = 0.$
	Thus, \(\{y_1, y_2\}\) forms a fundamental set of solutions for   \eqref{eq:second order-Linear_homo-de} on $I.$
\end{proof}

The next theorem provides a method for finding the general solutions of second order  homogeneous linear differential equations.

\begin{theorem}[General Solution]\label{thm:homogeneous-general-solution}
	Let \(\{y_1,y_2 \}\) form a fundamental set of  solutions  of the second order  homogeneous linear  differential equation  
	\begin{equation}\label{eq:homogeneous-3}
		y'' + a_1(x)y' + a_0(x)y = 0,
	\end{equation}
	where \( a_1(x) \) and \( a_0(x) \) are continuous functions on an interval $I.$ Then the functions of the form
	\begin{equation}\label{eq:homogeneous-general-solution}
		y = c_1 y_1(x) + c_2 y_2(x)
	\end{equation} are  solutions  on $I$ for  any constants $c_1$ and $c_2$.  Moreover, every solution  of \eqref{eq:homogeneous-3} is of  form \eqref{eq:homogeneous-general-solution} for some constants $c_1$ and $c_2$.
\end{theorem}

\begin{proof} For any $c_1$ and $c_2$  constants, the superposition principle (see p.\pageref{superposition}) implies that  
	\[y = c_1 y_1(x) + c_2 y_2(x)\]   is a solution  of \eqref{eq:homogeneous-3} on $I.$
	
	
	To show that every solution of \eqref{eq:homogeneous-3} is of this form \eqref{eq:homogeneous-general-solution}, let $\phi(x)$ be a solution of \eqref{eq:homogeneous-3}  on $I$ and let $x_0$ be in $I.$ Denote  by $a$ and $b$ the  constants 
	\(\phi(x_0) \) and  \(\phi'(x_0)\), respectively. This means that $y = \phi(x)$ solves the initial value problem
	\begin{equation}\label{eq:ivp-1}
		y'' + a_1(x)y' + a_0(x)y = 0, \quad y(x_0) = a, y'(x_0) =b.
	\end{equation}
	On the other hand, let $c_1$ and $c_2$ be constants to be determined so that $y(x) = c_1y_1(x)+ y_2(x)$ satisfies $y(x_0) = a, y'(x_0) = b.$ Then
	\[
	\begin{cases}
		c_1y_1(x_0) +c_2y_2(x_0) = a,\\
		c_1y'_1(x_0) +c_2y'_2(x_0) = b.\\
	\end{cases}
	\] Since $\{y_1, y_2\}$ is a linearly independent set of solutions, Theorem~\ref{thm:wronskian-nonzero-2} implies $W(y_1, y_2)(x_0)\ne 0,$ and therefore, by the Cramer's rule, we find the unique values of $c_1$ and $c_2$ given by
	\[
	c_1 = \frac{
		\begin{vmatrix}
			a & y_2(x_0) \\
			b& y_2'(x_0)
	\end{vmatrix}}{W(y_1, y_2)(x_0)}  \mbox{ and }
	c_2 = \frac{
		\begin{vmatrix}
			y_1(x_0)&a \\
			y_1'(x_0)&b
	\end{vmatrix}}{W(y_1, y_2)(x_0)}.
	\]
	With these values of $c_1$ and $c_2$, the function $y= c_1 y_1 + c_2 y_2$ solves the same initial value problem \ref{eq:ivp-1} as $\phi(x)$. By the uniqueness part of Theorem~\ref{thm:existence-theorem}, we conclude that $\phi(x) = c_1 y_1(x) + c_2 y_2(x)$ on \(I.\)
\end{proof}

\subsection{Constructing Fundamental Sets of Solutions}
	 We now discuss methods for determining fundamental sets of solutions for  second order  homogeneous linear differential equations with constant coefficients, specifically
	\begin{equation}\label{eq:homogeneous-constant-coefficients-1}
	ay''+by'+cy =0,
	\end{equation}
	where $a, b, c$ are constant coefficients and $a\ne 0.$ This equation can equivalently be written in the standard form 
		\begin{equation}\label{eq:homogeneous-constant-coefficients-2}
		y''+\frac{b}{a}y'+\frac{c}{a}y =0.
	\end{equation}	
	
Because the constant coefficient functions are defined on $(-\infty, \infty),$ the interval $I$ in the existence and uniqueness theorem (Theorem~\ref{thm:existence of a fundamental set}) for \eqref{eq:homogeneous-constant-coefficients-1} is the entire set of real numbers $(-\infty, \infty).$ Consequently, the solutions in any fundamental set that we construct  are defined on $(-\infty, \infty).$
	 	
Motivated from first order equations, we expect that solutions appear as exponential functions. So, we let $y = e^{\lambda x}$ be a \textit{trial} solution of \eqref{eq:homogeneous-constant-coefficients-1} for some number (possibly complex) $\lambda.$  Differentiating $y = e^{\lambda x}$  twice with respect to $x$  and substituting  $y, y',y''$ into \eqref{eq:homogeneous-constant-coefficients-1} yields
\[
	(a\lambda^2+ b\lambda +c)e^{\lambda x} = 0
\]
for all $x$ in $(-\infty, \infty).$ In particular, when $x=0,$ we obtain
\begin{equation}\label{eq:auxiliary-equation}
	a\lambda^2+ b\lambda +c =0,
\end{equation}
which is called the \textbf{\textit{auxiliary}} (or \textbf{\textit{characteristic}}\footnote[1]{The polynomial \(p(\lambda) = a\lambda^2+b\lambda+c\) is called the \textbf{\textit{characteristic polynomial}} for the differential equation \(ay''+by'+cy=0\).}) equation for \eqref{eq:homogeneous-constant-coefficients-1}.
The roots\footnote[2]{\textit{Solutions} would be a more precise term, but we use \textit{\textbf{roots}} here to avoid confusion with solutions of differential equations.} of \eqref{eq:auxiliary-equation} are given by 
\begin{equation}\label{eq:auxiliary-equation-solutions}
	\lambda=\frac{-b\pm\sqrt{b^2-4ac}}{2a},
\end{equation}
%The expression \(b^2-4ac\) is known as the \textbf{\textit{discriminant}} of the quadratic equation \eqref{eq:auxiliary-equation}.
which are  \textbf{\textit{distinct real}}, \textbf{\textit{complex}}, or \textbf{\textit{repeated}}  according as \(b^2-4ac>0\), \(b^2-4ac<0\), or \(b^2-4ac=0\). We now construct a fundamental set of solutions for each case.

\subsubsection{Case I. \mathversion{bold}\(b^2-4ac>0\) (Distinct Real Roots)}		 	
	When \(b^2-4ac>0\), the auxiliary equation \eqref{eq:auxiliary-equation}  possesses two distinct real roots $\lambda_1$ and $\lambda_2$ given by \eqref{eq:auxiliary-equation-solutions}.
Take 
\begin{equation}\label{eq:auxiliary-equation-solutions-real}
	\lambda_1= -\frac{b}{2a}+ \frac{\sqrt{b^2-4ac}}{2a}
	\quad\mbox{and}\quad 
	\lambda_2=-\frac{b}{2a}- \frac{\sqrt{b^2-4ac}}{2a}.
\end{equation}Since $\lambda_1\ne \lambda_2,$ it follows that \(y_1=e^{\lambda_1x}\) and \(y_2=e^{\lambda_2x}\) are linearly independent solutions of \eqref{eq:homogeneous-constant-coefficients-1}. We  confirm the linear independence of  \(y_1\) and \(y_2\)  by evaluating  their Wronskian: 
	 \[W(y_1,y_2)(x) = \begin{vmatrix}
	 	e^{\lambda_1x}&e^{\lambda_2x}\\
	 \lambda_1	e^{\lambda_1x}&\lambda_2e^{\lambda_2x}
	 \end{vmatrix} = (\lambda_2-\lambda_1)e^{\lambda_1x} e^{\lambda_2x} \ne 0\]
	 on \((-\infty, \infty)\). Therefore, \(\{y_1, y_2\}\) is a fundamental set of solutions.  By Theorem~\ref{thm:homogeneous-general-solution}, the general solution of \eqref{eq:homogeneous-constant-coefficients-1} is 
	 \begin{equation}\label{eq:real-distict-general-solution}
	 	\boxed{y = c_1e^{\lambda_1x}+c_2e^{\lambda_2x},}
	 \end{equation}
	 	where $c_1$ and $c_2$ are arbitrary constants.
	 	
 \subsubsection{Case II. \mathversion{bold}\(b^2-4ac<0\) (Complex Roots)}
 When \(b^2-4ac<0\), we have \(\sqrt{b^2-4ac}= i\sqrt{4ac-b^2},\) where $i = \sqrt{-1}$,  the imaginary unit in the complex number system. Then the auxiliary equation \eqref{eq:auxiliary-equation}  possesses two distinct complex roots $\lambda_1$ and $\lambda_2$  given by  \eqref{eq:auxiliary-equation-solutions}. Take 
 \begin{equation}\label{eq:auxiliary-equation-solutions-complex}
 	\lambda_1= -\frac{b}{2a}+ i\frac{\sqrt{4ac-b^2}}{2a}
 \quad\mbox{and}\quad 
 	\lambda_2=-\frac{b}{2a}- i\frac{\sqrt{4ac-b^2}}{2a}.
 \end{equation}
  Let\[\alpha = -\frac{b}{2a} \quad \mbox{and}\quad \beta =\frac{\sqrt{4ac-b^2}}{2a}.\] It is evident that $\beta \ne 0.$ Then the roots $\lambda_1=\alpha+ i\beta$ and $\lambda_2=\alpha - i \beta$ of the auxiliary equation \eqref{eq:auxiliary-equation}  give the complex-valued solutions \(z_1=e^{\lambda_1x}\) and \(z_2=e^{\lambda_2x}\) of \eqref{eq:homogeneous-constant-coefficients-1}. To extract real-valued solutions, we use first use the Euler formula 
  \[e^{i\theta} = \cos \theta + i \sin \theta, \quad \theta \mbox{ in radian,}\]
  and write
  \[\begin{split}
  	z_1&=e^{\lambda_1x} = e^{(\alpha+ i\beta)x} = e^{\alpha x} e^{i\beta x} =e^{\alpha x} \cos(\beta x)+ ie^{\alpha x}\sin(\beta x), \mbox{and }\\
	 z_2&=e^{\lambda_1x} = e^{(\alpha- i\beta)x} = e^{\alpha x} e^{-i\beta x} =e^{\alpha x} \cos(\beta x)- ie^{\alpha x}\sin(\beta x).
  \end{split}\]
  It is evident that 
  \[z_1 + z_2 =2e^{\alpha x} \cos(\beta x) \quad\mbox{and}\quad z_1 - z_2 =2ie^{\alpha x} \sin(\beta x),\]
  which imply
  \[
  \begin{split}
  	e^{\alpha x} \cos(\beta x) &=\frac{1}{2}\left(z_1+z_2\right)= \frac{1}{2}z_1 + \frac{1}{2}z_2\\
  	e^{\alpha x} \sin(\beta x)&=\frac{1}{2i}\left(z_1-z_2\right)=\frac{1}{2i}z_1 - \frac{1}{2i}z_2.
  \end{split}\]
  Since $z_1$ and $z_2$ are solutions of \eqref{eq:homogeneous-constant-coefficients-2} (and equivalently of \eqref{eq:homogeneous-constant-coefficients-1}), it follows from 
 the superposition principle (see p.\pageref{superposition}) that \(e^{\alpha x} \cos(\beta x)\) and \(e^{\alpha x} \sin(\beta x)\)
 are solutions of the homogeneous equation \eqref{eq:homogeneous-constant-coefficients-1}. 
 Let \(y_1=e^{\alpha x} \cos(\beta x)\) and \(y_2=e^{\alpha x} \sin(\beta x).\)
  Then  \(y_1\) and \(y_2\) are  linearly independent  solutions because their Wronskian satisfies
 \[\begin{split}
W(y_1,y_2)(x) &= \begin{vmatrix}
 	e^{\alpha x} \cos(\beta x)&e^{\alpha x} \sin(\beta x)\\
 	-\beta e^{\alpha x} \sin(\beta x) +\alpha e^{\alpha x} \cos(\beta x) &\beta e^{\alpha x} \cos(\beta x) +\alpha e^{\alpha x} \sin(\beta x)
 \end{vmatrix}\\
 	&=e^{2\alpha x} \beta\\
 	&\ne 0
 	 \end{split} \]
 for all \(x\) in \((-\infty, \infty)\). Thus, we identify \(\{y_1, y_2\}\) as a fundamental set of solutions.  By Theorem~\ref{thm:homogeneous-general-solution}, the general solution of \eqref{eq:homogeneous-constant-coefficients-1} is 
 \begin{equation}\label{eq:complex-general-solution}
 	\boxed{y = c_1e^{\alpha x} \cos(\beta x)+c_2e^{\alpha x} \sin(\beta x) = e^{\alpha x} \big(c_1\cos(\beta x)+c_2 \sin(\beta x)\big),}
 \end{equation}
 where $c_1$ and $c_2$ are arbitrary constants.
 
 \subsubsection{Case III. \mathversion{bold}\(b^2-4ac=0\) (Repeated Root)}
 
 In this case, the auxiliary equation \eqref{eq:auxiliary-equation} has a repeated real root 
 \[
 \lambda_1 = \lambda_2 = -\frac{b}{2a},
 \] 
 as given in \eqref{eq:auxiliary-equation-solutions}. Consequently, we obtain only a single solution:
 \[
 y_1 = e^{\lambda_1 x} = e^{-\frac{b}{2a}x}.
 \] 
 Since a fundamental set of solutions for \eqref{eq:homogeneous-constant-coefficients-1} requires two linearly independent solutions, we must determine a second solution \(y_2\) so that \(\{y_1, y_2\}\) forms a  fundamental set of solutions. 
 
We next discuss a method known as \textbf{\textit{reduction of order}},\label{page:reduction-of-order}  which allows us to find a second linearly independent solution of a second order  homogeneous linear differential equation, even when the equation has variable coefficients, as in \eqref{eq:second order-Linear_homo-de}, given that one nontrivial solution is already known.

%It sometimes occurs for  homogeneous linear equations with constant coefficients that the auxiliary equation has repeated roots, resulting in fewer independent solutions than the order of the equation. In such cases, we can obtain  additional  solutions needed for a fundamental set of solutions using the  method for reducing the order of the equations.

Let $y_1(x)$ be a known  solution of 	
\begin{equation}\label{eq:second order-Linear_homo-de-2}
	y'' + a_1(x)y' + a_0(x)y = 0,
\end{equation}
where \( a_0(x) \) and \( a_1(x) \) are continuous functions on an interval $I,$  satisfying either $y_1(x)>0$ for all $x$ in $I$ or $y_1(x) <0$ for all $x$ in $I$.  Theorem~\ref{thm:existence-theorem} guarantees the existence of such a nonzero solution on $I$ for the initial condition $y(x_0) = y_0 \neq 0.$  We assume $y_1(x) > 0$ on $I$; the other case can be treated similarly.


 Our goal is to find  a second solution $y_2$ such that $\{y_1, y_2\}$ is  linearly independent on $I.$ To this end, 
let $u(x)$ be a function on $I$ such that $y_2 (x) = u(x) y_1(x)$ is a solution of \eqref{eq:second order-Linear_homo-de}.  If we  find a non-constant function $u,$ then it will follow that $\{y_1, y_2\}$ is a fundamental set of solutions. Then,  suppressing $x$ from the arguments for simplicity,  we have
\[\begin{split}
	y'_2 &= u' y_1 + u y'_1\quad \mbox{ and }\\
	y''_2 &= u'' y_1 +2u' y'_1+ u y''_1.
\end{split}\]
Substituting these into \eqref{eq:second order-Linear_homo-de} gives
\[
	u'' y_1 +2u' y'_1+ u y''_1
	+a_1u' y_1 + a_1u y'_1
	+ a_0 u y_1=0.\,
\]
which becomes
\[
	u'' y_1 +u'\big(2 y'_1+ a_1y_1\big)
	+u\big( y''_1 + a_1 y'_1+ a_0 y_1\big) =0.
\]
Since $y_1$ is a solution of
 \eqref{eq:second order-Linear_homo-de}, we have \(y''_1 + a_1 y'_1 +a_0 y_1 = 0,\)  and therefore we obtain
 \[
 	u'' y_1 +u'\Big(2 y'_1 +a_1y_1\Big) =0.
\]
 We observe that there is no $u$ in this equation. Let $v = u'.$ Then the equation becomes
 \[
 v' y_1 +v\Big(2 y'_1 +a_1y_1\Big) =0,
 \]
 which is of the first order.
 Since $y_1$ never vanishes on $I$, we can rewrite the above equation as 
 \begin{equation}\label{eq:reduction-or-order}
 v'  +\left(\frac{2 y'_1 +a_1y_1}{y_1}\right) v =0.
 \end{equation}
 A solution of \eqref{eq:reduction-or-order} is given by
 \[
 v(x) = e^{-\int\left(2\frac{y'_1(x)}{y_1(x)}+a_1(x)\right)dx}=  e^{-2\ln\abs{y_1(x)}} \; e^{-\int a_1(x)\, dx}= \frac{e^{-\int a_1(x)\, dx}}{[y_1(x)]^2}.\]
 Since $v(x) = u'(x),$ we obtain 
 \[u(x) = \int \frac{e^{-\int a_1(x)\, dx}}{[y_1(x)]^2}\, dx\] on $I,$
and consequently, the solution $y_2(x)$ is given by 
 \begin{equation}\label{eq:reduction-of-order-formula}
 	\boxed{y_2(x)  = y_1(x) \int \frac{e^{-\int a_1(x)\, dx}}{[y_1(x)]^2}\, dx.}
 \end{equation} on $I.$\footnote[1]{This formula is due  the French mathematician Joseph Liouville}  Thus, $\{y_1, y_2\}$ is a fundamental set of solutions for \eqref{eq:second order-Linear_homo-de}. \\
 
We now return to finding a second linearly independent solution $y_2$ for \textbf{Case III}. Since
\[
y_1 = e^{\lambda_1 x} = e^{-\frac{b}{2a}x}
\]
and $a_1(x) = \frac{b}{a}$ is a constant, the reduction of the order formula \eqref{eq:reduction-of-order-formula} gives
\[
y_2(x)
= e^{-\frac{b}{2a}x}
\int \frac{e^{-\int (b/a)\, dx}}{e^{-\frac{b}{a}x}}\, dx
= e^{-\frac{b}{2a}x} \int 1\, dx
= x e^{-\frac{b}{2a}x}
\]
on $I$. Since we only need a second solution $y_2$, the constants of integration are set to zero for convenience. Thus, $\{y_1, y_2\}$ forms a fundamental set of solutions. By
Theorem~\ref{thm:homogeneous-general-solution}, the general solution of
\eqref{eq:homogeneous-constant-coefficients-1} is
\begin{equation}\label{eq:repeated-solution}
\boxed{
	y = c_1 e^{-\frac{b}{2a}x}
	+ c_2 x e^{-\frac{b}{2a}x}
	= (c_1 + c_2 x) e^{-\frac{b}{2a}x},}
\end{equation}
where $c_1$ and $c_2$ are arbitrary constants.

An alternative method  for finding a second linearly independent solution when the characteristic polynomial has a repeated root is  outlined in  Exercises~\ref{EX31}. 



\begin{remark}\label{rem:Abel-formula}$\empty$
	\begin{enumerate}[label =(\roman*), noitemsep]
		\item 	The method discussed above for finding $u(x)$ is known as the \textit{reduction of order} because the resulting differential equation \eqref{eq:reduction-or-order} for $v(x)$ is of first order; one order less than the orginianl equation \eqref{eq:second order-Linear_homo-de}.
			\item If the known solution \( y_1 \) vanishes at some points in the interval \( I \), we can first apply the reduction of the order  on a subinterval \( J\)  of \( I \) where \( y_1 \) does not change sign, and then extend the solution to all of $I$ using a theorem on  continuation of solutions (which is beyond the scope of this book).
		In practice, we may formally compute \( y_2(x) \) using  \eqref{eq:reduction-of-order-formula} and then verify that \( y_2 \) is actually well-defined on the entire interval \( I \). For this, see Example~\ref{eg:order-reduction}.
	\end{enumerate}
\end{remark}

\begin{example}\label{eg:distict-real-1}
	Find the general solution of $$y''-16y = 0.$$ 
\end{example}
\begin{solution}
		 Assume a solution of the form $y = e^{\lambda x}$. Then $y' = \lambda e^{\lambda x}$ and $y'' = \lambda^2 e^{\lambda x}$. Substituting into the differential equation gives
		\[
		\lambda^2 e^{\lambda x} - 16 e^{\lambda x} = 0,\] which gives \[(\lambda^2 - 16) e^{\lambda x} = 0.
		\]
		Since $e^{\lambda x} \neq 0$, the auxiliary equation is
		\[
		\lambda^2 - 16 = 0,
		\]
		whose distinct real roots are \[\lambda_1 = 4\quad\mbox{and}\quad \lambda_2 =-4.\]
		By \eqref{eq:real-distict-general-solution}, the general solution is
		\[
		y(x) = c_1 e^{4x} + c_2 e^{-4x},
		\]
		where $c_1$ and $c_2$ are arbitrary constants.
	\end{solution}
 
 \begin{example}\label{eg:distict-real-2}
 	Find the general solution of  $$y''-y'-6y = 0.$$
 \end{example}
 \begin{solution}
 		Assume a solution of the form $y = e^{\lambda x}$. Then $y' = \lambda e^{\lambda x}$ and $y'' = \lambda^2 e^{\lambda x}$. Substituting into the differential equation gives
 		\[
 		\lambda^2 e^{\lambda x} - \lambda e^{\lambda x} - 6 e^{\lambda x} = 0.
 		\]
 		Then the auxiliary equation is
 		\[
 		\lambda^2 - \lambda - 6 = 0,
 		\]
 		whose real and distinct root are
 		 are
 		\[
 		\lambda_1 = 3, \quad \lambda_2 = -2.
 		\]
 		By \eqref{eq:real-distict-general-solution}, the general solution is
 		\[
 		y(x) = c_1 e^{3x} + c_2 e^{-2x},
 		\]
 		where \(c_1\) and \(c_2\) are arbitrary constants.
 	\end{solution}
 \begin{example}\label{eg:complex}
 	Find the general solution of  $$3y''+2y'+y = 0.$$
 	\end{example}
 	\begin{solution}
 		Assume a solution of the form $y = e^{\lambda x}$. Then $y' = \lambda e^{\lambda x}$ and $y'' = \lambda^2 e^{\lambda x}$. Substituting into the differential equation gives
 		\[
 		(3\lambda^2 + 2\lambda + 1)e^{\lambda x} = 0.
 		\]
 		Then the auxiliary equation is
 		\[3\lambda^2 + 2\lambda + 1=0,\]
 		whose complex roots are
 		\[
 		\lambda = \frac{-2 \pm \sqrt{-8}}{6} = \frac{-1 \pm \sqrt{2} i}{3} = -\frac{1}{3} \pm i \frac{\sqrt{2}}{3}.
 		\]
 		Substituting \(\alpha = -\frac{1}{3}\) and \(\beta = \frac{\sqrt{2}}{3}\)  in \ref{eq:complex-general-solution}, the general solution is 
 		\[
 	y = e^{-x/3} \left( c_1 \cos \left(\frac{\sqrt{2}}{3} x\right) + c_2 \sin \left(\frac{\sqrt{2}}{3} x\right) \right),
 		\]	where \(c_1\) and \(c_2\) are arbitrary constants.
 	\end{solution}
 	
 
 \begin{example}\label{eg:repeated}
 Find the general solution of  $$y''+8y'+16y = 0.$$
 \end{example}
 \begin{solution}
 	Assume a solution of the form $y = e^{\lambda x}$. Then $y' = \lambda e^{\lambda x}$ and $y'' = \lambda^2 e^{\lambda x}$. Substituting into the differential equation gives
 	\[
 	(\lambda^2 + 8\lambda + 16)e^{\lambda x} = 0.
 	\]
 	Then the auxiliary equation is
 	\[\lambda^2 + 8\lambda + 16=0,\]
 	whose complex roots are
 	\[
 	\lambda = -4, -4 \quad \mbox{(repeated).}
 	\]
 	Because the auxiliary equation has a repeated root,  the general solution (from \eqref{eq:repeated-solution}) is
 	\[
 	y(x) = c_1 e^{-4x} + c_2 x e^{-4x},
 	\]  
 	where \(c_1\) and \(c_2\) are arbitrary constants.
 \end{solution}
 
 
 
  \begin{example}\label{eg:order-reduction}
Given that \(y_1 = \sin x\) solves \(y'' + y = 0\) on \((-\infty, \infty)\), find a second linearly independent solution using the formula \eqref{eq:reduction-of-order-formula} for the reduction of the order.
 \end{example}
 \begin{solution} For \(y'' + y = 0\), we have  $a_1 =0.$ Also, the given solution \(y_1 = \sin x\) vanishes at the points $x = n\pi,$ $ n= 0, \pm 1, \pm 2, \dots.$ Therefore, we  restrict $y_1$ to an interval \(J\) not containing these points. For example, take $J = (0, \pi).$
 	Applying the formula \eqref{eq:reduction-of-order-formula}, a second  solution $y_2$  is given by
 	\[y_2(x)  = \sin x \int \frac{1}{\sin^2x}\, dx  = \sin x \int \csc^2 x\, dx= -\sin x \cot x = -\cos x\] on \(J.\)
 	By rescaling, we redefine $y_2 = \cos x$ on \(J=(0, \pi).\) Since $y_2= \cos x$ can be extended by the same formula  to all of $(-\infty, \infty)$ and  \[W(\cos x, \sin x) = 1\ne 0\] for all $x$ in  $(-\infty, \infty),$ it follows Theorem~\ref{thm:wronskian-nonzero-2} that $y_1= \sin x$ and $y_2 = \cos x$ are linearly independent on  $(-\infty, \infty).$
 \end{solution}
 
 Techniques for finding fundamental sets of solutions for  homogeneous linear equations with variable coefficients are discussed in Section~\ref{sec:change-of-variables}.
 In this section, we will only  discuss the \textbf{Cauchy–Euler} equations of the form
 \begin{equation}\label{eq:Cauchy-Euler}
 	x^2y''+b xy' +cy =0,\quad  x>0,
 \end{equation}
 where $b$ and $c$ are real constants. The method uses  \(y = x^\lambda\) as a trial solution, with $\lambda$ to be determined from the auxiliary equation 
 \[\lambda^2 +(b-1)\lambda +c=0,\] whose roots are
  real and distinct, complex, or repeated  according as 
  \[(b-1)^2-4c>0,\;\; (b-1)^2-4c<0,\;\;  \mbox{ or }  (b-1)^2-4c=0. \]
  We next discuss three examples, one for each case.
 
 \begin{example}\label{eg:Cauchy-Euler}
 	Find the general solution of
 the Cauchy-Euler equation
 \begin{equation}
 	x^2 y''+4xy'+y = 0,\quad x>0.
 \end{equation}   
 \end{example}
 \begin{solution}
 	Take $y = x^\lambda$ as a trial solution, where $\lambda$ is a number to be determined. Differentiating $y = x^\lambda$ twice with respect to $x$ and substituting $y, y', y''$ into the differential equation, we obtain
 	%\[x^2 \lambda^2 x^{\lambda-2} +3x\lambda x^{\lambda-1} + x^\lambda = 0,\] which yields
 	\[(\lambda^2+ 3\lambda +1)x^\lambda =0\] for all $x$ in an interval $I,$ so the auxiliary equation
 	is \[\lambda^2+ 3\lambda +1=0\] with roots
 		\[
 			\lambda = \frac{-3\pm \sqrt{5}}{2}.
 		\]
 		Let \[\lambda_1=\frac{-3+ \sqrt{5}}{2}\quad\mbox{and}\quad\lambda_2=\frac{-3- \sqrt{5}}{2}.\] Obviously, $\lambda_1<0$ and $\lambda_2<0.$ From the two options \((-\infty,0)\) and \((0, \infty),\)  let us take $I = (0, \infty).$
 		Then  $y_1 = x^{\lambda_1}$ and $\lambda_2 = x^{\lambda_2}$ are solutions on $I,$ and  their Wronskian
 		\[W(y_1, y_2) = \begin{vmatrix}
 			x^{\lambda_1}&x^{\lambda_2}\\
 			\lambda_1x^{\lambda_1-1}&\lambda_2x^{\lambda_2-1}
 		\end{vmatrix} = (\lambda_2-\lambda_1) x^{(\lambda_1+\lambda_2-2)}=(\lambda_2-\lambda_1) x^{-5} \ne 0\]
 		 for all $x$ in $I$ since \(\lambda_1+\lambda_2 = -3\). Thus \(\{y_1, y_2\}\) is a fundamental set of solutions, and therefore the general solution of the Cauchy-Euler equation is 
 		 \[y = c_1x^{\lambda_1}+c_2x^{\lambda_2} = c_1x^{\frac{-3+ \sqrt{5}}{2}}+c_2x^{\frac{-3- \sqrt{5}}{2}}\]
 	on \(I,\) where $c_1$ and $c_2$ are arbitrary constants.
 \end{solution}
 
 \begin{example}
 	Solve the Cauchy–Euler equation
 	\[
 	x^2 y'' - y' + 5y = 0, \quad x>0.
 	\]
 \end{example}
 \begin{solution}
 	Let \(y = x^\lambda\) be a trial solution. Then
 	\[
 	y' = \lambda x^{\lambda-1}, \qquad y'' = \lambda(\lambda-1)x^{\lambda-2}.
 	\]
 	Substituting into the differential equation gives
 	 \[(\lambda^2 - 2\lambda + 5)x^\lambda = 0.\]
 	Then the auxiliary equation is
 	\[\lambda^2 - 2\lambda + 5=0\] whose solutions are
 
 	\[
 \lambda = \frac{2 \pm \sqrt{-16}}{2} = 1 \pm 2i.
 	\]
 	For complex numbers \(\lambda = 1 \pm 2i\), we have
 	\[x^{1 \pm 2i}= x\, x^{\pm 2i}=  x e^{\pm 2i\ln x} = 
 	x\big(\cos(2 \ln x) \pm i \sin(2\ln x)\big).\]
 	Let $y_1 = x\cos(2 \ln x)$ and $y_2 = x\sin(2 \ln x).$ It is routine to verify that $y_1$ and $y_2$ are linearly independent solutions on $(0, \infty).$ Then the
 	 general solution of the Cauchy–Euler equation is
 	\[
 	y(x) = x \big(c_1 \cos(2 \ln x) + c_2 \sin(2 \ln x)\big), \quad x>0, 
 	\]
 	where $c_1$ and $c_2$ are arbitrary constants.
 \end{solution}
 
 \begin{example} Solve the Cauchy–Euler equation
 	\[
 	x^2 y'' - y' + y = 0, \quad x>0.
 	\]
 \end{example}
 \begin{solution}
 Let \(y = x^\lambda\) be a trial solution. Then
 \[
 y' = \lambda x^{\lambda-1}, \qquad y'' = \lambda(\lambda-1)x^{\lambda-2}.
 \]
 Substituting into the differential equation yields
 \[(\lambda^2 - 2\lambda + 1)x^\lambda = 0.\] 
 Then the auxiliary equation is
 \[\lambda^2 - 2\lambda + 1=0,\quad  \mbox{i.e.},\;
 (\lambda-1)^2 = 0.\] Its solutions are
 \[
 \lambda =   1, 1 \; (\text{repeated})
 \]
 We thus have one solution as $y_1= x.$  By  the  reduction of the order formula \ref{eq:reduction-of-order-formula}, a second linearly independent solution \(y_2\) is given by
 \[\begin{split}
 	y_2&= y_1 \int \frac{e^{-\int a_1(x)\, dx}}{y_1^2}\, dx\quad \mbox{with } a_1(x) = -\frac{1}{x}\\
 	&= x \int \frac{e^{-\int \left(-\frac{1}{x}\right)\, dx}}{x^2}\, dx\\
 	&=x \int \frac{1}{x}\, dx\\
 	&= x \ln x.
 \end{split}
 \]
Hence the general solution of the Cauchy-Euler equation is 
\[y= c_1 x+c_2 x\ln x,\quad x>0,\]
where $c_1$ and $c_2$ are arbitrary constants.
 \end{solution}
 
 
 \begin{Exercise}\label{EX32}
 	\vspace{-\baselineskip}% <-- You don't need this line of code if there's some text here
% 		\Question\label{3-2-1}%Linear-depedence 		
% 		Determine whether each of the  sets of functions below is linearly independent on \((-\infty,\infty)\) using Definition~\ref{defn:linear-indepednence} or the discussion that follows it.
% 	\begin{tasks}(2)
% 		\task\(\{1, x, x^2\}\)
% 		\task \(\{e^{2x}, e^{-2x}\}\)
% 		\task \(\{e^{2x}, xe^{2x}\}\)
% 		\task\(\{x, \abs{x}\}\) 
% 		\task\(\{\sin x, \cos x\}\)
% 		\task\(\{\sin^2 x, \cos^2 x\}\)
% 		\task\(\{1, \sin x, \cos x\}\)
% 		\task\(\{1, \sin^2 x, \cos^2 x\}\)
% 		\task\(\{1, x-1, 2x\}\)
% 		\task\(\{0, e^x, e^{-x}\}\)
% 	\end{tasks}
 
 	\Question\label{3-2-1}
 		Verify that the indicated solution set for each  differential equation below  is linearly independent on the specified  interval $I$ by computing  Wronskian of the solutions (see Theorem~\ref{thm:wronskian-nonzero-2}). 
 		\begin{tasks}
 			\task\(y''=0\), \quad  \(\{7,\, x\}\),\quad  \(I=(-\infty,\infty)\)
 			\task \(y''+y=0\),\quad  \(\{\sin x,\, \cos x\}\),\quad \(I=(-\infty,\infty)\)
 			\task \(y''+y=0\),\quad  \(\{\sin x,\, \sin x+\cos x\}\),\quad \(I=(-\infty,\infty)\)
 			\task \(y''-4y=0\),\quad \(\{e^{2x},\, e^{-2x}\}\),\quad \(I=(-\infty,\infty)\)
 			%\task  \(y''-4y'+4y=0\), \quad \(\{e^{2x},\, 7e^{2x}\}\), \quad \(I=(-\infty,\infty)\)
 			\task \(y''-4y'+4y=0\),\quad \(\{e^{2x},\, xe^{2x}\}\),\quad \quad \(I=(-\infty,\infty)\)
 			\task  \(y''-4y'+4y=0\),\quad \(\{e^{2x},\, (x+1)e^{2x}\}\),\quad  \(I=(-\infty,\infty)\)
 			\task  \(y''-4y'+13y=0\),\quad \(\{e^{2x}\cos(3x),\, e^{2x}\sin(3x)\}\),\quad  \(I=(-\infty,\infty)\)
 			\task \(y''-\frac{2}{x}y'=0\), \quad \(\{1, x^3\}\), \quad \(I = (0, \infty)\)
 			\task \(x^2 y''+xy'+y = 0,\)\quad  \(\{\cos(\ln x),\, \sin(\ln x)\}\), \quad \(I = (0, \infty)\)
 			\task \(x^2 y''+4xy'+y = 0,\) \quad  \(\{x^{\frac{-3+ \sqrt{5}}{2}}, x^{\frac{-3- \sqrt{5}}{2}}\}\), \quad \(I = (0, \infty)\)
 		\end{tasks}
 	\Question\label{3-2-2}
 	Find the general solution of each differential equation below (see Theorem~\ref{thm:homogeneous-general-solution}).
 	\begin{tasks}(2)
 		\task \(y''-5y'+6y=0\)
 		\task \(y''+11y'+30y=0\)
 		\task \(2y''+5y'+3y=0\)
 		\task \(y''-7y'-30y=0\)
 		\task \(y''+4y'=0\)
 		\task \(y''+4y'+13y=0\)
 		\task \(y''-5y'+y=0\)
 		\task \(2y''+5y'+y=0\)
 	\end{tasks}
 	
 	
 	
\Question\label{3-2-3}
 	Find the general solution of each of the following Cauchy-Euler differential  equations (see Example~\ref{eg:Cauchy-Euler}):
 	
 	\begin{tasks}(2)
 		\task \(x^2y''+xy'-y = 0\)
 		\task \(x y''-2y'=0\)
 		\task \(x^2 y''-x y'-3y=0\)
 		\task \(x^2 y''-2y=0\)
 		\task \(x^2 y''+x y'+y=0\)
 		\task \(x^2y''+3xy'+5y = 0\)
 	\end{tasks}
 	
% 	
% 	\Question\label{3-1-4} Suppose that $y_1,y_2$ are differentiable functions on an interval $I.$ 
% 	\begin{tasks}(1)
% 		\task	Let $x_0$ be point in $I$ such that 
% 		$W(y_1, y_2)(x_0)\ne 0.$ Show that \(y_1, y_2\) are linearly independent on $I.$
% 		\task Find examples of \(y_1, y_2\) that are linearly independent on $I$ and satisfy 
% 		$W(y_1, y_2)(x)= 0$ for all $x$ in $I.$
% 	\end{tasks}

\Question\label{3-2-4}
Find the solution of each of the following initial value problems:
\begin{tasks}(1) 
	\task  \(y''-16y =0,\; y(0)= 0, \; y'(1) = 1\). See Example~\ref{eg:distict-real-1} for the general solution.
	\task \( y''-y'-6y = 0, \;y(0)= -1, \; y'(0) = 1.\) See Example~\ref{eg:distict-real-2} for the general solution.
	\task \(3y''+2y'+y=0,\; y(0)=1,\; y'(0)=0.\) See Example~\ref{eg:complex} for the general solution.
	\task \(y''+8y'+16y = 0,\; y(1)=1,\; y'(1)=-1.\) See Example~\ref{eg:repeated} for the general solution.
\end{tasks}
% 
 \Question\label{3-2-5} Solve the following initial value problems by using the general solutions of the related equations obtained in Problem~\ref{3-2-3} and Problem~\ref{3-2-4}.
 \begin{tasks}
 \task \(y''-5y'+6y=0, \quad y(0) = 0, \quad y'(0)=1\)
 \task \(y''+11y'+30y=0, \quad y(0) = 0, \quad y'(0)=1\)
 \task \(2y''+5y'+3y=0, \quad y(0) = 1, \quad y'(0)=0\)
 \task \(y''-7y'-30y=0, \quad y(0) = 1, \quad y'(0)=0\)
 \task \(y''+4y'=0, \quad y(0) = 1, \quad y'(0)=1\)
 \task \(y''+4y'+13y=0, \quad y(0) = 1, \quad y'(0)=0\)
 \task \(y''-5y'+y=0, \quad y(0) = 0, \quad y'(0)=1\)
 \task \(2y''+5y'+y=0, \quad y(0) = 0, \quad y'(0)=1\)
 \task \(x^2y''+xy'-y = 0, \quad y(1) = 1, \quad y'(1)=0\)
 \task \(x^2 y''+x y'+y=0, \quad y(1) = 1, \quad y'(1)=1\)
 \task \(x^2 y''-x y'-3y=0, \quad y(1) = 1, \quad y'(1)=0\)
 \task \(x^2y''+3xy'+5y = 0,\quad y(1) = 1, \quad y'(1)=0\)
 \end{tasks}
 	
 	\Question\label{3-2-6} Suppose that  the characteristic polynomial \(p(\lambda) = a\lambda^2 + b\lambda + c\) of
 the differential equation
 	\[
 	ay'' + by' + cy = 0, \quad a \ne 0,
 	\]
 	where $a, b, c$ are constants,
 	  has a repeated root \(\lambda_0.\) Then \(y_1 = e^{\lambda_0 x}\) is a solution of the differential equation. To find a second linearly independent solution, proceed as follows.
 	
 	\begin{enumerate}[label=(\alph*), noitemsep]
 		\item\label{item:3-1-5-a} Let \(y = e^{\lambda x}\) with parameter \(\lambda\). Show that
 		\[
 		a\frac{\partial^2}{\partial x^2} (e^{\lambda x}) + b\frac{\partial}{\partial x} (e^{\lambda x}) + c (e^{\lambda x}) = p(\lambda) e^{\lambda x}.
 		\]
 		
 		\item Differentiate both sides of the equation in part \ref{item:3-1-5-a} with respect to \(\lambda\) to obtain
 		\[
 		a\frac{\partial^2}{\partial x^2} (x e^{\lambda x}) + b\frac{\partial}{\partial x} (x e^{\lambda x}) + c (x e^{\lambda x}) = \big(p'(\lambda) + x\, p(\lambda)\big) e^{\lambda x}.
 		\]
 		
 		\item Using \(p(\lambda_0) = p'(\lambda_0) = 0\), conclude that \(y_2 = x e^{\lambda_0 x}\) is also a solution.
 		
 		\item Show that \(y_1 =e^{\lambda_0 x}\) and \(y_2=x e^{\lambda_0 x}\) are linearly independent on \((-\infty, \infty)\).
 	\end{enumerate}
 	
 \Question\label{3-2-7} Let $y_1(x)$ be a known  solution of 	
 \begin{equation}\label{eq:second order-Linear_homo-de-3}
 	y'' + a_1(x)y' + a_0(x)y = 0,
 \end{equation}
 where \( a_0(x) \) and \( a_1(x) \) are continuous functions on an interval $I,$  satisfying $y_1(x)>0$ for all $x$ in $I$. Use the Wronskian to derive Abel’s formula \eqref{eq:reduction-of-order-formula} as follows:
 \begin{enumerate}[label=(\alph*), noitemsep]
 	\item\label{item:wronskian-method} Suppose that $y_2$ is  a solution of \eqref{eq:second order-Linear_homo-de-3} such that $\{y_1, y_2\}$ is linearly independent on $I.$ Show that 
 	\[\left(\frac{y_2}{y_1}\right)' = \frac{W(x)}{y_1^2} \quad \mbox{on } I,\]  where $W(x)$ is the Wronskian of $y_1$ and $y_2$ given by $W(x) =y_1(x)y_2'(x) - y_1'(x)y_2(x).$
 	\item\label{item:Abel-2} Using part \ref{item:wronskian-method}, show  that 
 	\[y_2 = y_1(x) \int\frac{W(x)}{[y_1(x)]^2}\, dx.\]
 	\item Obtain  Abel's formula \[y_2 = y_1(x) \int\frac{e^{-\int a_1(x)\,dx}}{[y_1(x)]^2}\, dx\] from part \ref{item:Abel-2}.
 \end{enumerate}
 The  same steps  also apply when the  known solution $y_1$ satisfies $y_1(x) <0$ for all $x$ in $I$. For the general case, see (ii) of Remark~\ref{rem:Abel-formula}.
 \end{Exercise}
 
 \setboolean{firstanswerofthechapter}{true}
 \begin{multicols}{2}\scriptsize
 	\begin{Answer}[ref={EX32}]
 		
\Question \label{3-2-1a} 	
		\begin{tasks}
			\task \(W(x) =7 \mbox{ or } -7,\) linearly independent
			\task \(W(x) =-1 \mbox{ or } 1,\) linearly independent
			\task \(W(x) =-1 \mbox{ or } 1,\) linearly independent
			\task \(W(x) =-4 \mbox{ or } 4,\) linearly independent
			\task \(W(x) =e^{4x} \mbox{ or } -e^{4x},\) linearly independent
			
			\task \(W(x) =e^{4x},\mbox{ or } -e^{4x},\) linearly independent
			\task \(W(x) =3e^{4x}\mbox{ or } -3e^{4x},\) linearly independent
			\task \(W(x) =3x^2 \mbox{ or }-3x^2,\) linearly independent
			\task \(W(x) =1/x \mbox{ or } -1/x,\) linearly independent
			\task \(-\frac{\sqrt{5}}{x^2}  \mbox{ or } \frac{\sqrt{5}}{x^2},\)linearly independent on 
		\end{tasks}
\Question \label{3-2-2a} 	
\begin{tasks}
	\task \(y(x)=c_1 e^{2 x}+c_2 e^{3 x}\)
	\task \(y=c_1 e^{-6 x}+c_2 e^{-5 x}\)
	\task \(y=c_1 e^{-\frac{3 x}{2}}+c_2 e^{-x}\)
	\task \(y=c_1 e^{-3 x}+c_2 e^{10 x}\)
	\task \(y=c_1+ c_2 e^{-4 x}\)
	\task \(y=e^{-2 x} (c_1 \cos (3 x)+c_2  \sin (3 x))\)
	\task \(y=c_1 e^{\left(\frac{5}{2}-\frac{\sqrt{21}}{2}\right) x}+c_2 e^{\left(\frac{\sqrt{21}}{2}+\frac{5}{2}\right) x}\)
	\task \(y=c_1 e^{\left(-\frac{\sqrt{17}}{4}-\frac{5}{4}\right) x}+c_2 e^{\left(\frac{\sqrt{17}}{4}-\frac{5}{4}\right) x}\)
	
\end{tasks}

\Question \label{3-2-3a}
\begin{tasks}
	\task \(y = c_1 x+\frac{c_2}{x}\)
	\task \(y =c_1+c_2 x^3\)
	\task \(y = c_1 x^3+\frac{c_2}{x}\)
	\task \(y =c_1 x^2+\frac{c_2}{x}\)
	\task \(y= c_1 \cos (\ln (x))+c_2 \sin (\ln (x))\)
	\task \(y = c_1\frac{\cos (2 \ln (x))}{x}+c_2\frac{\sin (2 \ln (x))}{x} \)
\end{tasks}
\Question \label{3-2-4a}
\begin{tasks}
	\task \(y=\frac{e^{4 x}}{8}-\frac{1}{8} e^{-4 x}\)
	\task \(y =\frac{1}{5} (-4) e^{-2 x}-\frac{e^{3 x}}{5}\)
	\task \(y=\frac{1}{2} e^{-\frac{x}{3}} \Big(\sqrt{2} \sin \left(\frac{\sqrt{2} x}{3}\right)\\
	+2 \cos \left(\frac{\sqrt{2} x}{3}\right)\Big)\)
	\task \(y=e^{4-4 x} (3 x-2)\)
\end{tasks}

\Question \label{3-2-5a}
\begin{tasks}
	\task \(y=e^{3 x}-e^{2 x}\)
	\task \(y =e^{-5 x}-e^{-6 x}\)
	\task \(y=3 e^{-x}-2 e^{-\frac{3 x}{2}}\)
	\task \(y=\frac{10 e^{-3 x}}{13}+\frac{3 e^{10 x}}{13}\)
	\task \(y=\frac{5}{4}-\frac{e^{-4 x}}{4}\)
	
	\task \(y=\frac{2}{3} e^{-2 x} \sin (3 x)+e^{-2 x} \cos (3 x)\)
	\task \(y= \frac{e^{\left(\frac{\sqrt{21}}{2}+\frac{5}{2}\right) x}}{\sqrt{21}}-\frac{e^{\left(\frac{5}{2}-\frac{\sqrt{21}}{2}\right) x}}{\sqrt{21}}\)
	\task \(\frac{2 e^{\left(\frac{\sqrt{17}}{4}-\frac{5}{4}\right) x}}{\sqrt{17}}-\frac{2 e^{\left(-\frac{\sqrt{17}}{4}-\frac{5}{4}\right) x}}{\sqrt{17}}\)
	
	\task \(y= \frac{x}{2}+\frac{1}{2 x}\)
	\task \(y=\sin (\ln x )+\cos (\ln x)\)
	\task \(y =\frac{x^3}{4}+\frac{3}{4 x}\)
	\task \(y=\frac{\sin (2 \ln x)}{2 x}+\frac{\cos (2 \ln x)}{x}\)
\end{tasks}
\Question \label{3-2-6a} Follow the directions provided.
\Question \label{3-2-7a} Follow the directions provided.
 	\end{Answer}
 \end{multicols}
 \setboolean{firstanswerofthechapter}{false}
 
%\subsection{The Method of Undetermined Coefficients}

%\begin{Exercise}
%	\vspace{-\baselineskip}% <-- You don't need this line of code if there's some text here
%	
%	\Question\label{4-2-1}
%Solve the following differential equations using the method of undetermined coefficients to find  particular solutions
%	
%	\begin{tasks}(2)
%		\task \(y'' + 8 y' + 16 y = 4 x\)
%		
%	\end{tasks}
%	
%	
%	
%\end{Exercise}
%
%\setboolean{firstanswerofthechapter}{true}
%\begin{multicols}{2}\scriptsize
%	\begin{Answer}[ref={EX74}]
%		\Question \label{4-2-1a}
%		\begin{tasks}
%			\task \(y = c_1 e^{-4x} + c_2 x e^{-4x} + \frac{x}{4} - \frac{1}{8}\)
%		
%			
%		\end{tasks} 
%	\end{Answer}
%\end{multicols}
%\setboolean{firstanswerofthechapter}{false}
%
\section{Second Order Nonhomogeneous Linear  Equations}

In this section, we discuss  methods for solving second order nonhomogeneous linear differential equations.
% of the form
%\begin{equation} \label{eq:nonhomogeneous2}
%	y'' + a_1(x)y' + a_0(x)y = f(x),
%\end{equation}
%where  the coefficient functions \( a_0(x),  a_1(x) \) and  forcing term \( f(x) \) are continuous on some interval \( I .\) 
The same ideas extend naturally to higher order equation (see Section~\ref{sec:higher order DEs}). 

\begin{theorem}[General Solution]\label{thm:second-order-nonhomogeneous-general-solution}
	Consider the nonhomogeneous  equation 
	\begin{equation}\label{eq:second-order-nonhomogeneous}
		y'' + a_1(x)y' + a_0(x)y = f(x),
	\end{equation}
	where \( a_1 \), \( a_0 \) and \(f\) are continuous functions on an interval $I$  and the associated homogeneous equation  
	\begin{equation}\label{eq:second-order-homogeneous-associated-to-nonhomogeneous}
		y'' + a_1(x)y' + a_0(x)y = 0
	\end{equation}
	on $I$. Let $y_h$ be the general solution of \eqref{eq:second-order-homogeneous-associated-to-nonhomogeneous} and
	$y_p$ be a particular solution of the nonhomogeneous equation \eqref{eq:second-order-nonhomogeneous} on $I.$ Then the general solution of \eqref{eq:second-order-nonhomogeneous} is of the form
	\begin{equation}\label{eq:second-order-nonhomogeneous-general-solution}
		\boxed{y = y_h(x) +y_p(x)}
	\end{equation} on $I.$ 
\end{theorem}
\begin{proof}  Let $y_p$ be a particular solution of \eqref{eq:second-order-nonhomogeneous}. Let $y$ be any other solution of \eqref{eq:second-order-nonhomogeneous}. Then the difference
\begin{equation}\label{eq:general-solution-form}
		\phi(x)=y(x) -y_p(x)
\end{equation}
is a solution of the homogeneous equation \eqref{eq:second-order-homogeneous-associated-to-nonhomogeneous}. In fact, for all $x$ in $I$, we have
\[
\begin{split}
		&\phi''(x) + a_1(x)\phi'(x) + a_0(x)\phi(x)\\
		 &= \left(y''(x) -y_p''(x)\right)+a_1(x)\left(y'(x) -y_p'(x)\right)+a_0(x)\left(y(x) -y_p(x)\right)\\
		&= \left(y''(x) + a_1(x)y'(x) + a_0(x)y(x)\right)- \left(y_p''(x) + a_1(x)y_p'(x) + a_0(x)y_p(x)\right)\\
		&=f(x)-f(x)\\
		&=0. 
\end{split}
\] 
Therefore, $\phi(x)$ takes the form of $y_h(x)$ on $I.$ It follows from \eqref{eq:general-solution-form}  that \[y= y_h(x) + y_p(x)\quad \mbox{ on } I \] is the general solution of \eqref{eq:second-order-nonhomogeneous}.
\end{proof}

	\begin{remark}
		The function $y_h$ in \eqref{eq:second-order-nonhomogeneous-general-solution} is called the \textbf{\textit{complementary function}} of the general solution, and it is given by
	\[y_h(x)=c_1 y_1(x) + c_2 y_2(x),\] where $y_1$ and  $y_2$ form a fundamental set of solutions for  \eqref{eq:second-order-homogeneous-associated-to-nonhomogeneous} and $c_1$ and $c_2$ are arbitrary constants. 
\end{remark}

In practice, solving a nonhomogeneous equation begins with finding the complementary function by determining a fundamental set of solutions to the associated homogeneous equation. We then find a particular solution of the nonhomogeneous equation, and the sum of the two provides the general solution. Finding a particular solution is typically the more difficult step; we outline two procedures for finding it  in this section.

\subsection{Variation of Parameters}\label{subsec:variation-parameters}

The \textbf{\textit{variation of parameters}} is a systematic method for constructing a particular solution of a  nonhomogeneous linear differential equation of the form
\begin{equation}\label{eq:general-from}
	a_2(x) y'' + a_1(x) y' + a_0(x) y = g(x),
\end{equation}
where the forcing term $g$ and  coefficient functions $a_0, a_1,$ and $a_2$ are assumed to be continuous on an interval $I$ with $a_2(x)\ne 0$    on $I.$  We will be working with scenarios in which $a_2(x)$ is either only positive or only negative on  $I$. 

To formulate the method, we first divide both sides of \eqref{eq:general-from} by $a_2(x)$ to rewrite it in standard form
\begin{equation}\label{eq:nonhomogeneous1}
	y'' + a_1(x) y' + a_0(x) y = f(x),
\end{equation}
where 
\[
a_1(x) = \frac{a_1(x)}{a_2(x)}, 
\quad 
a_0(x) = \frac{a_0(x)}{a_2(x)}, 
\quad 
f(x) = \frac{g(x)}{a_2(x)},
\]
and each of these functions remains continuous on $I$.

Assume that $\{ y_1, y_2 \}$ is a fundamental set of solutions on $I$ for the  homogeneous equation
\begin{equation}\label{eq:homogeneous1}
	y'' + a_1(x) y' + a_0(x) y = 0
\end{equation} associated to the nonhomogeneous equation \eqref{eq:nonhomogeneous1}. Then the general solution to \eqref{eq:homogeneous1} is
\begin{equation}\label{VOP-4}
	y_h(x) = c_1 y_1(x) + c_2 y_2(x),
\end{equation}
for all $x$ in $I$,
where $c_1$ and $c_2$ are arbitrary constants

To find a particular solution of the nonhomogeneous equation \eqref{eq:nonhomogeneous1}, the method of variation of parameters assumes a solution of the form
\[
y_p(x) = u_1(x) y_1(x) + u_2(x) y_2(x),
\] for all $x$ in $I$,
where $u_1$ and $u_2$ are functions to be determined.   \textit{The choice of $y_p$ clarifies the name of the method:  variation of parameters. The constants $c_1$ and $c_2$ in the  solution $y_h$ of the homogeneous equation \eqref{eq:homogeneous1} are replaced by the functions $u_1(x)$ and $u_2(x)$, thereby \textit{varying} the parameters in order to obtain a particular solution
	to \eqref{eq:nonhomogeneous1}}. For the remainder of the discussion of the method, we suppress  $x$ in all functions  to minimize distractions. 

Differentiating $y_p = u_1y_1 +u_2y_2$, we have
\[
y_p' = u_1 y_1' +u_1' y_1  + u_2 y_2'+  u_2' y_2
=  u_1 y_1' + u_2 y_2'+(u_1' y_1 + u_2' y_2)
\]
and 
\[
\begin{aligned}
y_p'' &=  u_1 y_1'' +u_1' y_1' +u_2 y_2''+  u_2' y_2' + \big( u_1' y_1 + u_2' y_2 \big)'\\[4pt]
&= u_1 y_1''+u_2 y_2'' +u_1' y_1' +  u_2' y_2' + \big( u_1' y_1 + u_2' y_2 \big)'
\end{aligned}
\]
We impose the condition
\[
u_1' y_1 + u_2' y_2 = 0, 
\]
which is trivially satisfied when $u_1$ and $u_2$ are constant functions.  
The condition simplifies the expressions for $y_p'$ and $y_p''$ as follows:
\[\begin{aligned}
y_p' &=  u_1 y_1' + u_2 y_2', \text{ and }\\
y_p'' &= u_1 y_1''+u_2 y_2'' +u_1' y_1' +  u_2' y_2'.
\end{aligned}
\]
Substituting $y_p$, $y_p'$, and $y_p''$ for $y, y',$ and $y''$, respectively, into \eqref{eq:nonhomogeneous1} gives
\[
	u_1' y_1' + u_1 y_1'' + u_2' y_2' + u_2 y_2''+a_1u_1 y_1' + a_1u_2 y_2'
+a_0u_1 y_1 + a_0u_2 y_2 = f,
\]
which can be rewritten as
\[
u_1 (y_1''+a_1 y_1'+a_0 y_1)+ u_2 (y_2''+a_1 y_2'+a_0 y_2) +u_1' y_1' +  u_2' y_2' 
= f.
\]
Since $y_1$ and $y_2$ satisfy the homogeneous equation \eqref{eq:homogeneous1}, we obtain the system
\[
\begin{cases}
	y_1 u_1' + y_2 u_2' = 0,\\
	y_1' u_1' + y_2' u_2' = f.
\end{cases}
\]
This system can be solved for $u_1'$ and $u_2'$:
\[
u_1' = -\frac{y_2 f}{W}, 
\quad 
u_2' = \phantom{-}\frac{y_1 f}{W},\]
where 
\(W = y_1 y_2' - y_2 y_1'\)
 is the Wronskian of  \(y_1\)  and  \(y_2.\)
By integrating, we obtain
\begin{equation}\label{eq:VOP-integrals}
u_1 = -\int \frac{y_2 f}{W} \, dx, 
\quad 
u_2 = \int \frac{y_1 f}{W} \, dx.
\end{equation}
Therefore,  $y_p$ is  given by
\begin{equation}\label{eq:particular-final-form}
y_p = -y_1 \int \frac{y_2 f}{W} \, dx 
+ y_2 \int \frac{y_1 f}{W} \, dx.
\end{equation}
Consequently,  the general solution of \eqref{eq:nonhomogeneous1} is
\[
y = y_h + y_p 
= c_1 y_1 + c_2 y_2 
- y_1 \int \frac{y_2 f}{W} \, dx 
+ y_2 \int \frac{y_1 f}{W} \, dx.
\]

\begin{remark}\label{re:variational-over-undertermined-coefficients}
	The main advantage of the method of \textit{variation of parameters} over the method of \textit{undetermined coefficients} is that 
	it is applicable to all non   homogeneous linear differential equations, including the equations with variable coefficients. However, the evaluation of the  integrals in \eqref{eq:VOP-integrals} can become quite cumbersome   even when  $f(x)$ involves only elementary functions, such as the ones treated  with the method of undetermined coefficients.
\end{remark}
 We illustrate the method of variation of parameters with the following initial value problem which can been readily solved using the method of undetermined coefficients.

\begin{example}\label{ex:3-3-1}
Solve the initial value problem \[y'' + 8 y' + 16 y = 4 x,\quad  y(0) = 0,\; y'(0) = 0\] by variation of parameters.
\end{example}
\begin{solution}
		The auxiliary equation for the associated homogeneous equation is
		\(\lambda^2 + 8\lambda + 16 = 0\) which gives
		\(\lambda = -4, -4,\)
and so the functions
		\[
		y_1(x) = e^{-4x}, \quad y_2(x) = x e^{-4x}
		\]
		form a fundamental set solutions to \(y'' + 8 y' + 16 y=0.\)
		The Wronskian of \(y_1\) and \(y_2\) is
		\[
		W(x) = 
		\begin{vmatrix}
			e^{-4x} & x e^{-4x} \\ 
			-4 e^{-4x} & (1 - 4x) e^{-4x}
		\end{vmatrix}
		= e^{-8x} (1 - 4x + 4x) = e^{-8x}.
		\]
		We note that the given nonhomogeneous equation is in the standard form with \(f(x)  = 4x\)  and therefore a particular solution by the method of variation of parameters is given by
		\[
		y_p = -y_1 \int \frac{y_2 f}{W} dx + y_2 \int \frac{y_1 f}{W} dx.
		\]
		Then  the integration by parts with  the Kronecker method gives	
		
%%Open below locally
%\tikzset{Arrow Style/.style={text=black, font=\boldmath}}
%
%\newcommand{\tikzmark}[1]{%
	%	\tikz[overlay, remember picture, baseline] \node (#1) {};%
	%}
%
%\newcommand*{\XShift}{0.5em}
%\newcommand*{\YShift}{0.5ex}
%
%\NewDocumentCommand{\DrawArrow}{s O{} m m m}{%
	%	\begin{tikzpicture}[overlay,remember picture]
		%		\draw[->, thick, Arrow Style, #2] 
		%		($(#3.west)+(\XShift,\YShift)$) -- 
		%		($(#4.east)+(-\XShift,\YShift)$)
		%		node [midway,above] {#5};
		%	\end{tikzpicture}%
	%}
		
		
		
		\[
		\renewcommand{\arraystretch}{1.5}
		\begin{array}{c @{\hspace*{1.0cm}} c}\toprule
			\text{Derivative} &  \text{Integral} \\\cmidrule{1-2}
			4x^2\tikzmark{Left 1} & \tikzmark{Right 1}e^{4x} \\
			8x \tikzmark{Left 2} & \tikzmark{Right 2}\frac{1}{4} e^{4x} \\      
			8  \tikzmark{Left 3} & \tikzmark{Right 3}\frac{1}{16} e^{4x} \\      
			0  \tikzmark{Left 4} & \tikzmark{Right 4}\frac{1}{64} e^{4x} \\\bottomrule
		\end{array}
		\]
		%-----------------------------------------
		\DrawArrow[draw=red]{Left 1}{Right 2}{$+$}%
	\DrawArrow[draw=brown]{Left 2}{Right 3}{$-$}%
	\DrawArrow[draw=blue]{Left 3}{Right 4}{$+$}%	

\[
\int \frac{y_2 f}{W} dx= \int \frac{4x^2 e^{-4x}}{e^{-8x}}\, dx =x^2 e^{4x} - \frac{1}{2} x e^{4x} + \frac{1}{8} e^{4x}.\]
	In a similar manner, we also evaluate
		\[
		 \int \frac{y_2 f}{W} dx=\int \frac{4x e^{-4x}}{e^{-8x}}\, dx= \int 4x e^{4x} dx 
		= x e^{4x} - \frac{1}{4} e^{4x}.
		\]	
Then  \( y_p \) becomes
		\[\begin{aligned}
		y_p &= -e^{-4x} \left( x^2 e^{4x} - \frac{1}{2} x e^{4x} + \frac{1}{8} e^{4x} \right)
		+ x e^{-4x}\left( x e^{4x} - \frac{1}{4} e^{4x} \right)\\
		&=- x^2 + \frac{x}{2}  - \frac{1}{8} + x^2 - \frac{x}{4}\\
		 &=\frac{x}{4}  - \frac{1}{8}.
		\end{aligned}
		\]
The general solution of the nonhomogeneous differential equation  is
		\[
		y(x) = c_1 e^{-4x} + c_2 x e^{-4x} + \frac{x}{4} - \frac{1}{8},
		\]
	where $c_1$ and $c_2$ are arbitrary constants.	
	Since \(y(0) = 0,\) we get
		 \(c_1 = \frac{1}{8}.\)
	We find
		\[
		y'(x) = -4 c_1 e^{-4x} + c_2 e^{-4x} - 4 c_2 x e^{-4x} + \frac{1}{4}.\]
	Since
		\(y'(0) =0,\) we have\( -4 c_1 + c_2 + \frac{1}{4} = 0.\)
	Using $c_1= \frac{1}{8}$, we get
		\(
		-\frac{1}{2} + c_2 + \frac{1}{4} = 0\), so that 
	\(
		c_2 = \frac{1}{4}.
		\)
Hence the solution to the given initial value problem is 
		\[
	y(x) = \frac{1}{8} e^{-4x} + \frac{1}{4} x e^{-4x} + \frac{x}{4}  - \frac{1}{8} 
		\]
		 on $(-\infty, \infty).$
\end{solution}

\subsection{Undetermined Coefficients}\label{subsec:undertermined-coefficients}

We note that variation of parameters often leads to integrals that can be quite involved—and sometimes unexpectedly time-consuming—even for simple forcing terms such as \( f(x) = \sin x \). For instance, to find a particular solution of
\[
y'' - 9y = \sin x,
\]
one may instead \textit{guess} a solution of the form \( y_p(x) = A\sin x \) (or a slightly more general form, \( y_p(x) = A\sin x +B\cos x,\)  if needed). See Example~\ref{eg:Method-UC-1} and Example~\ref{eg:Method-UC-2} below.  This strategy of guessing a  trial function for $y_p$ based on type of \( f(x) \) and determining the unknown constant coefficients in the trial function is known as the method of  \textbf{\textit{undetermined coefficients}}.
However, a demerit of this method is that it becomes quite complicated when differential equations has variable coefficients.
 
% which is an alternative 
% procedure finding a particular solution of the nonhomogeneous linear equation 
%\begin{equation}\label{eq:nonhomogeneous-linear-method-of-undetermined-coefficient}
%	y'' + a_1(x) y' + a_0(x) y = f(x),
%\end{equation}
%where \(a_1(x)\),  \(a_0(x),\) and \(f(x)\) are continuous on an interval $I$.  


\begin{example}\label{eg:Method-UC-1}
	Find a particular solution of 
\begin{equation}\label{eq:Method UC-1}
	y'' - 9y = \sin x
\end{equation} and write down the general solution.
\end{example}
\begin{solution}
Let \[y_p = A\sin x\] be a particular solution of the equation, where $A$ is a constant to be determined. Differentiating yields
\[y_p' = A\cos x, \quad y_p''= -A\sin x.\] Substituting these into \eqref{eq:Method UC-1} gives
\[-A\sin x+ 9A\sin x = \sin x,\] i.e.,
\[ 8A\sin x = \sin x,\] an identity in $x.$ This implies $A = \frac{1}{8}.$ Therefore, \[y_p = \frac{1}{8}\sin x\] is a particular solution.

To determine the complementary function of the general solution,  the auxiliary equation for the associated homogeneous equation \[y'' - 9y =0\]
is \[\lambda^2 -9 =0,\] whose roots are $\lambda_1 = 3$ and $ \lambda_1 = -3.$ Then the complementary function $y_h$ is given by 
\[y_h = c_1 e^{3x} + c_2 e^{-3x},\] where $c_1$ and $c_2$ are arbitrary constants. Finally, the general solution of \eqref{eq:Method UC-1} is
\[y = y_h(x)+y_p(x) = c_1 e^{3x} + c_2 e^{-3x}+\frac{1}{8}\sin x,\] where $c_1$ and $c_2$ are arbitrary constants.
\end{solution}

\begin{remark}\label{rem:method of UCM}$\empty$
	\begin{enumerate}[label=(\roman*), noitemsep]
		\item 	A particular solution of a differential equation is just one  member of the family represented by its general solution, and thus infinitely many particular solutions  exist. 
		In Example~\ref{eg:Method-UC-1}, the guess for $y_p$  was relatively easy to make by observing that the equation has no first derivative term and its forcing function is $\sin x.$
		\item  When variation of parameters demands complicated integrals, the method of undetermined coefficients  might be a  quicker alternative technique for finding particular solutions of equations with constant coefficients and  simple combination of elementary functions for $f(x).$  This method, however,  requires an intelligent guessing for \(y_p\), which makes it challenging to apply even when \(f(x)\) is a common function, such as \(\sec x\) or \(\tan x\).
	\end{enumerate}

\end{remark}

In the next example, the  equation also has a first derivative term,  so the trial solution  used for $y_p$ in Example~\ref{eg:Method-UC-1} does not work.
\begin{example}\label{eg:Method-UC-2}
	Find a particular solution of 
	\begin{equation}\label{eq:Method UC-2}
	y'' -6y'+ 9y = \sin x.
	\end{equation}
\end{example}
\begin{solution} First, we use the same trial solution  as in Example~\ref{eg:Method-UC-1} and see what happens. Let \[y_p = A\sin x\] be a particular solution, where $A$ is a constant to be determined. Differentiating yields
	\[y_p' = A\cos x, \quad y_p''= -A\sin x.\] Substituting these into \eqref{eq:Method UC-2} would give
	\[-A\sin x-6A\cos x+ 9A\sin x = \sin x,\] i.e.,
	\[ (8A-1)\sin x -6A\cos x = 0,\]  which is not an identity in $x$ for any  value of  $A$ because $\sin x$ and $\cos x$ are linearly independent.  Thus,  no value of $A$ makes  $y_p = A\sin x$ a valid particular solution. This occurs because $\cos x$ is involved in the process.
	  Therefore, we modify our trial function for $y_p$ and take 
	\[ y_p = A\sin x + B\cos x\]  where  $A$ and $B$ are constants to be determined. Differentiating twice gives
	\[y_p' = A\cos x-B\sin x, \quad y_p''= -A\sin x - B\cos x.\] Substituting these into \eqref{eq:Method UC-2}  gives
	\[(-A\sin x - B\cos x)-6(A\cos x-B\sin x)+9(A\sin x + B\cos x) = \sin x,\] i.e.,
	\[(8A+6B) \sin x +(8B-6A)\cos x  =\sin x,\] an identity in $x.$ Since $\sin x$ and $\cos x$ are linearly independent,  we obtain
	\[\begin{cases}
		8A+6B=1,\\
		8B-6A=0.
	\end{cases}\]
	Solving these equations for $A$ and $B$ yields $A = \frac{2}{25}$ and $B= \frac{3}{50}.$ Thus, 
	\[y_p =\frac{2}{25}\sin x+\frac{3}{50}\cos x \] is  a particular solution.
	
	
	To determine the complementary function of the general solution of \eqref{eq:Method UC-2},  the auxiliary equation for the associated homogeneous equation \[y'' -6y'+ 9y =0\]
	is \[\lambda^2 -6\lambda +9 =0,\] which has a repeated root  $\lambda = 3.$  Then the complementary function $y_h$ is given by 
	\[y_h = c_1 e^{3x} + c_2 xe^{3x},\] where $c_1$ and $c_2$ are arbitrary constants. Finally, the general solution of \eqref{eq:Method UC-2} is
	\[y = y_h(x)+y_p(x) = c_1 e^{3x} + c_2 xe^{3x}+\frac{2}{25}\sin x+\frac{3}{50}\cos x,\] where $c_1$ and $c_2$ are arbitrary constants.
\end{solution}

	In each of the previous two examples, we chose trial function \(y_p\) before finding the complementary function without any difficulty. In general, however, a trial function may include terms that also appear in the complementary function, which are redundant (even useless) since they satisfy the associated homogeneous equation. Therefore, it is preferable to determine the complementary function first and then select a trial function for the particular solution. This will be the procedure we adopt henceforth.


\begin{example}\label{eg:Method-UC-3}
		Find a particular solution of 
		\begin{equation}\label{eq:Method UC-3}
			y'' +y = \cos x
		\end{equation} and write down the general solution.
	\end{example}
	\begin{solution}
	The auxiliary equation of the associated homogeneous equation \[y''+y=0\] is 
	\[\lambda^2 + 1=0,\] whose complex roots are 
	\[\lambda = \pm i.\] Therefore, the complementary function of the general solution of the nonhomogeneous equitation is 
	\[ y_h = c_1 \cos x+ c_2 \sin x,\] where $c_1$ and $c_2$ arbitrary constants.
	 If  $y_h$ was not known in advance, we might initially guess 
	 \[y_p = A\cos x + B\sin x,\] with $A$ and $B$ as undetermined coefficients. However, this form of $y_p$ duplicates $y_h$  and it cannot satisfy the nonhomogeneous equation. A point of action  is to modify  $y_p$ to 
	 \[y_p = x( A\cos x + B\sin x).\] This will work! Let us proceed to determine $A$ and $B.$ Differentiating $y_p$ gives
	 \[\begin{split}
	 	y_p'&=  A\cos x + B\sin x+ x( -A\sin x + B\cos x),\\
	 	y_p''&=  -2A\sin x + 2B\cos x+ x( -A\cos x - B\sin x). 
	 \end{split}\]
	 Substituting these into the nonhomogeneous equation \eqref{eq:Method UC-3} yields
	 \[-2A\sin x + 2B\cos x = \cos x.\] For this to be an identity in $x,$ we must have
	 \[ A= 0, \quad B= \frac{1}{2},\] so 
	 \[y_p = \frac{1}{2}x\sin x\] is a particular solution \eqref{eq:Method UC-3}. Finally, the general solution of \eqref{eq:Method UC-3} is 
	 \[y = c_1 \cos x + c_2 \sin x +\frac{1}{2}x\sin x,\] where $c_1$ and $c_2$ arbitrary constants.
	\end{solution}
	
 
 	When the forcing function can be decomposed into a sum of several forcing functions, then the sum of  particular solutions with these individual forcing functions yields a particular solution of the original equation. The following theorem addresses this property.
 \begin{theorem}\label{thm:superposition-particular-solutions}
 Let $a_0(x), a_1(x), \dots, a_{n-1}(x) $  and $f_i(x), \quad i = 1, \dots, k,$ be continuous on an interval $I.$ Suppose $y_{p_i},\; i = 1, \dots, k,$ is a solution of
 \begin{equation}\label{eq:nonhomogeneous-particular-1}
 	y^{(n)}+ a_{n-1}(x)y^{(n-1)}+ \cdots+ + a_1(x) y' + a_0(x) y = f_i(x),
 \end{equation}
 for each $i = 1, \dots, k.$ 
  Then the function  \[y_p= \sum_{j=1}^ky_{p_j}(x) \] is a solution on $I$ of
  \begin{equation}\label{eq:nonhomogeneous-particular-3}
  		y^{(n)}+ a_{n-1}(x)y^{(n-1)}+ \cdots+ + a_1(x) y' + a_0(x) y = \sum_{j=1}^kf_j(x).
  \end{equation}
 \end{theorem}
 \begin{proof}
 	 Define the linear differential operator $L$ on the class of $n$-times differentiable functions by
 	\[L[y](x) = y^{(n)}+ a_{n-1}(x)y^{(n-1)}+ \cdots+ + a_1(x) y' + a_0(x) y.\]
 	Since each $y_{p_j}$ satisfies \eqref{eq:nonhomogeneous-particular-1},
 	we have
 	\[L[y_{p_j}](x)= f_j(x) \quad \mbox{on } I,\quad j  = 1, \dots, k.\] Then the linearity of the differential operator $L$ implies
 	\[L\left[\sum_{j=1}^ky_{p_j}\right](x) =\sum_{j=1}^kL[y_{p_j}](x)= \sum_{j=1}^kf_j(x).\] This shows that the function \[y_p= \sum_{j=1}^ky_{p_j}\] is a solution of \eqref{eq:nonhomogeneous-particular-1} on $I.$
 \end{proof}
 \begin{example}\label{eg:superposition-particular-solutions}
 Find first a particular solution and then the general solution of 
 \begin{equation}\label{eq:superposition-particular-solutions}
 	y''+4y = 4\sin(2x)+8 \sin x+ 2x^2 -4x+ 3 e^x.
 \end{equation} 
 \end{example}
\begin{solution}
	The auxiliary equation of the associated homogeneous equation \[y''+4y=0\] is $\lambda^2 +4=0$ whose complex roots are $\pm 2i.$ Therefore, the complementary function of the general solution of \eqref{eq:superposition-particular-solutions} is
	\[y_h = c_1\cos x + c_2 \sin x,\] where $c_1$ and $c_2$ are arbitrary constants. To find a particular solution of \eqref{eq:superposition-particular-solutions}, we split the forcing  term into four simpler terms:
	\[f_1(x)=4\sin(2x), \quad f_2(x)=8\sin x,\quad f_3(x)=2x^2-4x,\quad f_4(x)=3e^x.\]
	In view of Theorem~\ref{thm:superposition-particular-solutions}, 
	we find particular solutions $y_{p_1},y_{p_2},y_{p_3},y_{p_4}$  and then add them together.
			
	
			
			Since $\sin(2x)$ in $f_1(x)= 4\sin(2x)$ is a solution of the homogeneous equation, we take 
			\[
			y_{p_1}(x) = x(A\cos(2x) + B\sin(2x)),
			\] and find  $A=-1$, $B=0$. Thus, 
			\[
			\boxed{y_{p_1}(x) = -x \cos(2x).}
			\]
			
			 For $f_2(x) =8 \sin x,$
		we take
			\[
			y_{p_2}(x) = C \cos x + D \sin x
			\] and find  $C=0$, $D=\frac{8}{3}.$ Thus,
			\[
			\boxed{y_{p_2}(x) = \frac{8}{3} \sin x.}
			\]
			
		 For $ f_3(x)= 2x^2-4x$, we  try the quadratic
			\[
			y_{p_3}(x) = E x^2 + F x + G.
			\]
		and find $E=\frac12$, $F=-1$, $G=-\frac14.$ Thus,
			\[
			\boxed{y_{p_3}(x) = \frac12 x^2 - x - \frac14.}
			\]
			
		 For $f_4(x)=3 e^x$, we take
			
			\[
			y_{p_4}(x) = H e^x.
			\]
		and find $H = \frac{3}{5}.$ Thus, 
			\[
			\boxed{y_{p_4}(x) = \frac{3}{5} e^x.}
			\]
			
		 By Theorem~\ref{thm:superposition-particular-solutions}, a particular solution of \eqref{eq:superposition-particular-solutions} is given by
			\begin{align*}
				y_p (x)&= y_{p_1}(x) + y_{p_2}(x) + y_{p_3}(x) + y_{p_4}(x) \\
				&= -x \cos(2x) + \frac{8}{3} \sin x + \frac12 x^2 - x - \frac14 + \frac{3}{5} e^x.
			\end{align*}
		Hence the general solution of \eqref{eq:superposition-particular-solutions}  is
			\begin{align*}
				y(x) &= y_h(x) + y_p(x) \\
				&= c_1 \cos(2x) + c_2 \sin(2x)
				- x \cos(2x) + \frac{8}{3} \sin x + \frac12 x^2 - x - \frac14 + \frac{3}{5} e^x,
			\end{align*}
			where $c_1$ and $c_2$ are arbitrary constants.
\end{solution}
%Some suggested trial functions for \(y_p\) are provided in the table below. If a term in a trial function for  \(y_p\) coincides with a term in the complementary solution \(y_h\), modify \(y_p\) by multiplying it by \(x\) repeatedly until no term of \(y_p\) duplicates  any term of \(y_h\).


		
			
			\subsubsection{Some Basic Trial Particular Solutions}
			In principle, the method of undetermined coefficients can always be employed; however, it requires an educated guess for particular solution \(y_p\). Table~\ref{table:particular solutions} summarizes the basic form of  a trial particular solution  $y_p(x)$ for a given forcing function $f(x)$ in the constant coefficient equation $ay'' + by' + cy = f(x)$, provided no term in $y_p(x)$ duplicates any term in the complementary function $y_h(x)$. If a term in the trial solution duplicates a solution to the associated homogeneous equation, multiply the entire basic trial particular solution $y_p$  by $x^k$, where $k$ is the smallest integer such that no term in the modified $y_p(x)$ is a solution to the homogeneous equation. These ideas also apply to higher order linear equations.
			
			\vspace{0.5cm}
			\begin{table}
				\centering
			\begin{tabular}{@{}ll@{}}
				\toprule
				\text{$g(x)$} & \text{Basic form of trial $y_p(x)$} \\
				\midrule
				$ p_n(x)= a_n x^n + \dots + a_0$ & $ q_n(x)= A_n x^n + \dots + A_0$ \\
				$Ce^{\alpha x}$ & $Ae^{\alpha x}$ \\
				$p_n(x) e^{\alpha x}$ & $q_n(x) e^{\alpha x}$ \\
				$C \cos(\beta x)$  & $A \cos(\beta x) + B \sin(\beta x)$ \\
				$C \sin(\beta x)$  & $A \cos(\beta x) + B \sin(\beta x)$ \\
				$p_n(x) \cos(\beta x)$ & $q_n(x) \cos(\beta x) + r_n(x) \sin(\beta x)$ \\
				 $p_n(x) \sin(\beta x)$ & $q_n(x) \cos(\beta x) + r_n(x) \sin(\beta x)$ \\
				$Ce^{\alpha x} \cos(\beta x)$  & $Ae^{\alpha x} \cos(\beta x) + Be^{\alpha x} \sin(\beta x)$ \\
				 $Ce^{\alpha x} \sin(\beta x)$ & $Ae^{\alpha x} \cos(\beta x) + Be^{\alpha x} \sin(\beta x)$ \\
				\bottomrule\\
			\end{tabular}
			\caption{Basic trial particular solutions}
			\label{table:particular solutions}
			\end{table}
			\noindent In Table~\ref{table:particular solutions}, the general polynomial $p_n(x)$ is  of degree $n$ with known coefficients and $q_n(x),$ $r_n(x)$ are general polynomials of degree $n$ with  undetermined coefficients. Also,  $C$, $\alpha$, $\beta$ are known constants, and $A$  and $B$ are undetermined constants.
			
		
	
	\subsection{Operator Factorization}\label{subsec:method of operators}
	In this subsection, we discuss yet another method for finding  particular solutions of a second order nonhomogeneous linear equations of the form
	\begin{equation}\label{eq:nonhomogeneous-operator-method}
		y'' + a_1(x) y' + a_0(x) y = f(x),
	\end{equation}
	where 
	\(a_1(x), a_0(x), f(x)\)
	are continuous on some interval $I$. 
	For simplicity, we will restrict our attention to equations with constant coefficients and include only a few examples involving variable coefficients. The method extends naturally to higher-order linear equations, which we will briefly discuss in Section~\ref{sec:higher order DEs}.
	
	We let \(D\)  denote the differential operator \(\frac{d}{dx}\) defined by 
	\[Dy = \frac{dy}{dx}.\]	Higher order differential operators are  denoted by integer powers of $D.$ For example,
	\[
		D^2y = \frac{d^2y}{dx^2},\quad 
		D^3y = \frac{d^3y}{dx^3},
	\] and so forth. The differential operator of order $0$  is the identity operator denoted by $D^0$ and defined by \[ D^0y=y.\]  With these notations,
the second order equation \eqref{eq:nonhomogeneous-operator-method} can be written as
\[D^2y + a_1(x) Dy+a_0(x)y= f(x),\] or more concisely, 
\[p(D)y= f(x),\] where \[p(D) = D^2 + a_1(x) D+a_0(x)\] denotes the corresponding second order differential  operator.

To introduce the operator method, we begin by considering the first order linear equation
\begin{equation}\label{eq:first-order-linear-operator-method}
	y'+a_1(x) y = f(x),
\end{equation} 
where  
$a_1(x)$ and $f(x)$ are continuous  on an interval $I.$ The equation in operator notation can be written as
\[\big(D+a_1(x)\big)y = f(x).\] Formally,  a  solution may be expressed as 
\[y = \frac{1}{D+a_1(x)}f(x).\]  Since a  solution of \eqref{eq:first-order-linear-operator-method} is given by 
\begin{equation}\label{eq:first-order-lienar-operator-method-formula-general-solution}
	\begin{split}
		y&=\frac{1}{D+a_1(x)}f(x)\\
		& = c_1e^{-\int a_1(x)\,dx} +e^{-\int a_1(x)\,dx} \int\left( f(x) \,e^{\int a_1(x)\,dx}\right)\, dx,
	\end{split}
\end{equation} where $c_1$ is an arbitrary constant.
A particular solution $y_p$ (with $c_1=0$) is given by
 \begin{equation}\label{eq:first-order-lienar-operator-method-formula-1}
 	\boxed{y= e^{-\int a_1(x)\,dx} \int f(x) \,e^{\int a_1(x)\,dx}\, dx,}
 \end{equation}
%The constants of integration chosen in \eqref{eq:first-order-lienar-operator-method-formula-1} are for simplicity, as we only need a particular solution.
In particular, if $a_1(x) =-k,$ a constant,  then the formula \eqref{eq:first-order-lienar-operator-method-formula-1} takes the form
\begin{equation}\label{eq:first-order-lienar-operator-method-formula-2}
	\boxed{y =\frac{1}{D-k}f(x)= e^{kx} \int f(x) e^{-kx}\, dx.}
\end{equation}

We next study \eqref{eq:nonhomogeneous-operator-method} in the case where $a_0$ and $a_1$ are constants. 
When  formally substituting $D$  with $\lambda$ in  \[p(D) = D^2 + a_1(x) D+a_0(x),\] we get  \[p(\lambda) = \lambda^2 +a_1\lambda +a_0,\] which is the characteristic polynomial for the  homogeneous equation
\[y''+a_1y+a_0y=0.\] 
If the polynomial $p(\lambda)$ factors as
\[p(\lambda)= (p-\lambda_1)(p-\lambda_2),\] where $\lambda_1, \lambda_2$ may be real or complex, then the operator $p(D)$ factors correspondingly as
\[p(D) = (D-\lambda_1)(D-\lambda_2).\] Thus, $p(D)$ can be written a composition of two first order differential operators. Moreover,   we find
\[
\begin{split}
	(D-\lambda_1)(D-\lambda_2)y& = (D-\lambda_1)(Dy-\lambda_2y)\\
	&=D(Dy-\lambda_2y)-\lambda_1(Dy-\lambda_2y)\\
	&=D^2y- \lambda_2Dy- \lambda_1Dy+ \lambda_1\lambda_2y\\
	&=D^2y- (\lambda_1+\lambda_2)Dy+ \lambda_1\lambda_2y.\
\end{split} 
\]
Since $\lambda_1+\lambda_2 = \lambda_2+\lambda_1$ and $\lambda_1\lambda_2= \lambda_2\lambda_1,$ it follows that 
\[p(D) =(D-\lambda_1)(D-\lambda_2)y= (D-\lambda_2)(D-\lambda_1)y,\] and hence the factors commute. 	
A particular solution of 
\[p(D)y = f(x)\] is expressed formally by
\[y_p = \frac{1}{p(D)}f(x) = \frac{1}{D-\lambda_1}\frac{1}{D-\lambda_2} f(x).\]
 In this situation, a solution of \[p(D)y = f(x)\] can be written formally as 
 \begin{equation}\label{eq:operator-form-formula-for-particular-solutions}
 \boxed{y = \frac{1}{p(D)}f(x) =\frac{1}{D-\lambda_1}\; \frac{1}{D-\lambda_2} f(x),}
 \end{equation} 
 and it  can be obtained by applying \eqref{eq:first-order-lienar-operator-method-formula-2} successively to these two first order operators. These techniques extend naturally to higher order linear equations with constant coefficients. 

The following example illustrates these techniques.
\begin{example}\label{eg:operator-method-constant-coefficients-1}
	Find a particular solution of 
	\[y''-5y'+6y= 4xe^{x}.\]
	\end{example}
\begin{solution}
	The equation in the operator form is 
\begin{equation*}
	p(D)y =4xe^x,
\end{equation*}where 
\[p(D)= D^2-5D+6=(D-2)(D-3).\] By \eqref{eq:operator-form-formula-for-particular-solutions}, a particular solution is formally given by 
\[y_p = \frac{1}{D-2}\frac{1}{D-3} 4xe^x.\]
 Using \eqref{eq:first-order-lienar-operator-method-formula-2} and integrating by parts, we get
 \[\frac{1}{D-3} xe^x = e^{3x}\int 4xe^x e^{-3x}\, dx = 4e^{3x}\int x e^{-2x}\, dx = -(2x+1)e^x.\]
 Consequently, again by \eqref{eq:first-order-lienar-operator-method-formula-2} and integrating by parts, we obtain
 \[\begin{split}
 	y_p& =-\frac{1}{D-2}(2x+1)e^x \\
 	&= -e^{2x}\int (2x+1)e^x e^{-2x}\, dx \\
 	&=e^{2x}\int (2x+1)e^{-x}\, dx \\
 	&=(2x+3)e^x.\qedhere
 \end{split}\]
\end{solution}
The formula \eqref{eq:operator-form-formula-for-particular-solutions} can sometimes be more conveniently used using partial fraction decomposition:
 \begin{equation}\label{eq:operator-form-formula-for-particular-solutions-partial-fraction-decomposition}
	\boxed{y = \frac{1}{p(D)}f(x) =\frac{1}{D-\lambda_1}\; \frac{1}{D-\lambda_2} f(x) = \left(\frac{A_1}{D-\lambda_1}+ \frac{A_2}{D-\lambda_2}\right)f(x),}
\end{equation}
with the constants $A_1$ and $A_2$ to be evaluated. We can then evaluate the individual terms by using \eqref{eq:first-order-lienar-operator-method-formula-2}. Let us revisit Example~\ref{eg:operator-method-constant-coefficients-1} with this  technique.
\begin{example}\label{eg:operator-method-constant-coefficients-1-revisited}
	Find a particular solution $y_p$ of 
	\[y''-5y'+6y= 4xe^{x}.\]
\end{example}
\begin{solution}
	In Example~\ref{eg:operator-method-constant-coefficients-1}, we have
	\[y_p = \frac{1}{D-2}\frac{1}{D-3} 4xe^x.\]
	Suppose $A_1$ and $A_2$ are constants for which \[\frac{1}{D-2}\frac{1}{D-3} = \frac{A_1}{D-2}+\frac{A_2}{D-3}.\]
	Then we find $A_1= -1$ and $A_2= 1.$  In view of \eqref{eq:operator-form-formula-for-particular-solutions-partial-fraction-decomposition}, we have
	\[y_p=-\frac{1}{D-2} 4xe^x+ \frac{1}{D-3}4xe^x,\] which, by \eqref{eq:first-order-lienar-operator-method-formula-2} and integration by parts, yields
	\[\begin{split}
		y_p &= -4e^{2x} \int xe^x e^{-2x}\, dx + 4e^{3x} \int xe^x e^{-3x}\, dx\\
		&= -4e^{2x} \int x e^{-x}\, dx + 4e^{3x} \int x e^{-2x}\, dx\\
		&=e^x(4x+4)+e^x(-2x-1)\\
		&=(2x+3)e^x.\qedhere
	\end{split}\]
\end{solution}
The partial fraction decomposition method employed in Example~\ref{eg:operator-method-constant-coefficients-1-revisited} works neatly for higher order equations as well.

The next example shows how we deal with partial fraction decomposition when certain factor $D-\lambda$ is repeated. 
\begin{example}\label{eg:operator-method-constant-coefficients-3}
	Find a particular solution $y_p$ of 
	\[y''-4y'+4y= 4e^{2x}.\]
\end{example}
\begin{solution}
	The equation in the operator form is 
\begin{equation*}
	p(D)y =4e^{2x},
\end{equation*}where 
\[p(D)= D^2-4D+4=(D-2)^2.\] By \eqref{eq:operator-form-formula-for-particular-solutions}, a particular solution is formally given by 
\[y_p = \frac{1}{D-2}\frac{1}{D-2} 4e^{2x}.\]
By \eqref{eq:first-order-lienar-operator-method-formula-2}, we have
\[\frac{1}{D-2} 4e^{2x} = e^{2x}\int4e^{2x}e^{-2x}\, dx=4xe^{2x}.\] Then, by  \eqref{eq:first-order-lienar-operator-method-formula-2}, we obtain
\[y_p = \frac{1}{D-2}4xe^{2x} =e^{2x}\int4xe^{2x}e^{-2x}\, dx = e^{2x}\int4x\, dx = 2x^2e^{2x}.\qedhere\]
\end{solution}

Whenever $f(x)$ contains a factor of the form $e^{\alpha x},$  a more convenient method exists.  To discuss this, suppose $f(x) = e^{\alpha x} g(x)$ for some continuous function $g(x).$ Suppose 
\[p(D) = (D-\lambda_1) (D-\lambda_2),\] $\lambda_1$ and $\lambda_2$ real or complex constants.
Then
\[\begin{split}
(D-\lambda_2)(e^{\alpha x} g(x)) & = e^{\alpha x} Dg(x)+\alpha e^{\alpha x} g(x)- \lambda_2 e^{\alpha x} g(x)\\
	&=e^{\alpha x}  (D+\alpha-\lambda_2)g(x).
\end{split}\]
and 
\[\begin{split}
	p(D)(e^{\alpha x} g(x))&=(D-\lambda_1)(D-\lambda_2) (e^{\alpha x} g(x))\\
	 &=(D-\lambda_1)\big(e^{\alpha x}  (D+\alpha-\lambda_2)g(x)\big)\\
	&=e^{\alpha x} (D+\alpha-\lambda_1)(D+\alpha-\lambda_2)g(x)\\
	&= e^{\alpha x} p(D+\alpha)g(x).
\end{split}\]
Thus, we have
\begin{equation}\label{eq:shifting-property-1}
	p(D)(e^{\alpha x} g(x)) =e^{\alpha x} p(D+\alpha)g(x).
\end{equation}
We also have a similar property of the inverse operator; namely,
\begin{equation}\label{eq:shifting-property-2}
	\frac{1}{p(D)} e^{\alpha x} g(x) = e^{\alpha x}\frac{1}{p(D+\alpha)}g(x).
\end{equation} To see this, using \eqref{eq:shifting-property-1}, we have 
\[p(D)\left(e^{\alpha x}\frac{1}{p(D+\alpha)}g(x)\right) =e^{\alpha x}p(D+\alpha) \frac{1}{p(D+\alpha)}g(x)= e^{\alpha x}g(x),\] which yields \eqref{eq:shifting-property-2}.

The formulas \eqref{eq:shifting-property-1} and \eqref{eq:shifting-property-2} are called the  \textit{\textbf{exponential shifting properties}} of $p(D)$ and its inverse operator, respectively.
The next example provides an  application of the exponential shifting properties.

\begin{example}\label{eg:shifting-properties}
	Find a particular solution $y_p$ of 
\[y''-5y'+6y= 4xe^{x}\] by using the exponential shifting properties.
\end{example}
\begin{solution}
As shown in Example~\ref{eg:operator-method-constant-coefficients-1-revisited}, we have
	\[y_p = \frac{1}{D^2-5D+6} 4xe^x = \frac{1}{(D-2)(D-3)} 4xe^x = \left(\frac{1}{D-3}-\frac{1}{D-2}\right) 4xe^x.\]
	Using the exponential shifting property \eqref{eq:shifting-property-2}, we obtain
		\[\begin{split}
			y_p& = 4e^x\left(\frac{1}{D+1-3}x-\frac{1}{D+1-2}x\right)\\
			&=4e^x\left(\frac{1}{D-2}x-\frac{1}{D-1}x\right)\\
			&=4e^x\left(e^{2x}\int xe^{-2x}\, dx - e^x\int xe^{-x}\,dx\right)\\
			&=4e^x\left(e^{2x}\left(-\frac{x}{2}e^{-2x} -\frac{1}{4}e^{-2x}\right) - e^x\left(-xe^{-x} -e^{-x}\right)\right)\\
			&=4e^x\left(-\frac{x}{2} -\frac{1}{4} +x +1\right)\\
			&=e^x(-2x-1+4x+4)\\
			&=(2x+3)e^x.\qedhere
		\end{split} 
		\]
\end{solution}



When the coefficient functions $a_1(x)$ and $a_0(x)$ are variable,   the factors of $p(D)$ generally do not commute, as illustrated in Example~\ref{eg:operator-method-example-variable-coefficients}.
\begin{example}\label{eg:operator-method-example-variable-coefficients}
	Find a particular solution of 
\[y''+(x+1)y'+xy=e^{-x^2/2}.\]
\end{example}	
\begin{solution}
	The equation in the operator form is 
	\begin{equation}\label{eq:operator-method}
		p(D)y =e^{-x^2/2},
	\end{equation}where 
	\[p(D)= D^2+(x+1)D+x.\] By inspection,  $p(D)$ may be factored  as \((D+x)(D+1)\) or \((D+1)(D+x),\) and so which one is it?  Direct computation shows 
	\[\begin{split}
		(D+x)(D+1)y &= (D+x)(Dy+y)\\
		&= D(Dy+y)+x(Dy+y)\\
		&= D^2y+Dy+xDy+xy\\
		&= (D^2+(x+1)D+x)y\\
		&=p(D)y,
	\end{split}\]
	and
	\[\begin{split}
		(D+1)(D+x)y &= (D+1)(Dy+xy)\\
		&= D(Dy+xy)+(Dy+xy)\\
		&= D^2y+xDy+y+Dy+xy\\
		&= (D^2+(x+1)D+(x+1))y\\
		&\ne p(D)y.
	\end{split}\]	
	Thus, we only have
	\[p(D)y= (D+x)(D+1)y.\] Then  \eqref{eq:operator-method} turns into
	\[(D+x)(D+1)y= e^{-\frac{1}{2}x^2}.\]
	Put $(D+1)y =z.$ Then we have 
	\[(D+x)z=e^{-x^2/2},\] which is a first order linear equation in $z.$ 
	In view of \eqref{eq:first-order-lienar-operator-method-formula-1}, a particular solution of this equation is given by
             	\[z_p=e^{-x^2/2}\int e^{-x^2/2} e^{x^2/2}\,dx= (x-1) e^{-x^2/2}.\]  The constant of integration is chosen so as to simplify the  integral below.  Using this $z_p$ for $z$ in $(D+1)y =z,$ we obtain
	\[(D+1)y = (x-1) e^{-x^2/2}.\] By \eqref{eq:first-order-lienar-operator-method-formula-2}, a particular solution of this equation is given by
	\[\begin{split}
		y_p &= e^{-x} \int (x-1)e^x e^{-x^2/2}\, dx\\
		& = e^{-x+\frac{1}{2}} \int (x-1)e^{-\frac{1}{2}(x-1)^2}\, dx\\
		&=e^{-x+\frac{1}{2}} \int e^{-t}\,dt\quad \left(\mbox{using } t= \frac{1}{2}(x-1)^2 \mbox{ so that }(x-1)dx =dt\right)\\
		&=-e^{-(x-\frac{1}{2})} e^{-\frac{1}{2}(x-1)^2}\\
		&=-e^{-\frac{1}{2}x^2}.\qedhere
	\end{split} \]
\end{solution}

 We note that a fundamental set of solutions for the equation in Example~\ref{eg:operator-method-example-variable-coefficients} involves an integral that cannot be evaluated in terms of elementary functions. We can construct the general solution by leaving these integrals unevaluated.   A sketch of this construction is outlined in Exercises~\ref{EX33}.



\begin{example} \label{eg:Cauchy-Euler-Operator-method}
	Find the general solution of the Cauchy-Euler equation
	\[x^2y''+4xy'+2y=e^x, \quad x>0,\] using the  method of operator factorization.
\end{example}	
\begin{solution}
	In the operator notation, the equation reads
	\[(x^2D^2+4xD+2)y = e^x.\] By inspection,  the differential operator can be factored as
	\[(x^2D^2+4xD+2)= (xD+1)(xD+2),\] and this product is commutative. For the general conditions under which the product of two binomial differential operators with variable coefficients commutes, see Exercises~\ref{EX33}.
	Then the differential equation reads
	\[(xD+1)(xD+2)y =e^x.\] Let $z=(xD+2)y.$ Then the equation in $z$ reads
		\[(xD+1)z=e^x,\] which is linear in $z$,  that is, 
		\[\left(D+\frac{1}{x}\right)z=\frac{e^x}{x}.\] The solution of this equation is  given formally by
		\[z = \frac{1}{D+\frac{1}{x}}\left(\frac{e^x}{x}\right).\]
		By \eqref{eq:first-order-lienar-operator-method-formula-general-solution}, we obtain
		
	\[\begin{split}
		z&=c_1 e^{-\int \frac{1}{x}\,dx}+e^{-\int \frac{1}{x}\,dx}\int\left(\frac{e^x}{x}e^{\int \frac{1}{x}\,dx}\right)\,dx\\
		&=\frac{c_1}{x}+\frac{1}{x}\int e^x\,dx\\
		&=\frac{c_1}{x}+\frac{e^x}{x},
	\end{split} \] where $c_1$ is an arbitrary constant. Since $z=(xD+2)y,$ we have
	\[(xD+2)y = \frac{c_1}{x}+\frac{e^x}{x},\] that is,
	\[\left(D+\frac{2}{x}\right)y = \frac{c_1}{x^2}+\frac{e^x}{x^2},\]
	 the solution of which is given formally by
		\[y = \frac{1}{D+\frac{2}{x}}\left(\frac{c_1}{x^2}+\frac{e^x}{x^2}\right).\]  	In view of \eqref{eq:first-order-lienar-operator-method-formula-general-solution}, the general solution is given by
		\[
		\begin{split}
			y& = c_2e^{-\int\frac{2}{x}\,dx}+e^{-\int\frac{2}{x}\,dx}\int\left(\left(\frac{c_1}{x^2}+\frac{e^x}{x^2}\right)e^{\int\frac{2}{x}\,dx}\right)\,dx\\
			&=\frac{c_2}{x^2}+\frac{1}{x^2}\int (c_1+e^x)\, dx\\
			&=\frac{c_2}{x^2}+\frac{1}{x^2}(c_1x+e^x)\\
			&=\frac{c_1}{x}+\frac{c_2}{x^2}+\frac{e^x}{x^2},
		\end{split}
		\] where $c_2$ is an arbitrary constant.
\end{solution}

\begin{Exercise}\label{EX33}
	%\vspace{-\baselineskip}% <-- You don't need this line of code if there's some text here
\hspace{-.75ex}\textbf{\ref{subsec:variation-parameters}} \textbf{Variation of Parameters}\\

	\Question\label{3-3-1}
	Solve the following differential equations using the method of variation of parameters
	to find  particular solutions.
	\begin{tasks}(2)
		\task \(y''+y = 2e^x\)
		\task \(y''-y =xe^{-2x}\)
		\task \(y''+y = \sin x\)
		\task \(y''+y = x\sin x\)
		\task \(y''+y = \sec x\)
		\task \(y''+y = \tan x\)
		\task \(y''+y = \sin^2 x\)
		\task \(y''+y = \sec x\tan x\)
		\task \(y''+y = \sec^2 x\)
		\task \(y''+4y'+4y = e^{-2x}\)
		\task \(y''+3y'+2y = \frac{1}{1+e^{x}}\)
		\task \(y''+3y'+2y = \cos(e^x)\)
		\task \(y''+4y'+4y = 2\cos(2x)\)
		\task \(y''+4y'+4y = 2e^{-2x}\cos(2x)\)
	\end{tasks}
	
	\Question\label{3-3-2}
	Find the general solution of each  differential (Cauchy-Euler) equation below using the variation of parameters to find  particular solutions.

	\begin{tasks}(1)
		\task \(x^2y''+xy'-y = 5\)
		\task \(x^2 y''+x y'-y=2 x^4\)
		\task \(x^2 y''-x y'+y=x\)
		\task \(x y''-y'=x\)
	\end{tasks}
\Question\label{3-3-3}
	Find the general solution of each differential equation below. One nontrivial solution to the associated homogeneous equation is provided.
	
	\begin{tasks}(1)
		\task \(x^2 y''+x y'-4 y=x; \quad u(x) = x^2\)
		\task \(x^2 y''+x y'-4  y=\dfrac{1}{x}+1; \quad u(x) = x^2\)
		\task \(x^2 y''+x y'+4  y=\dfrac{1}{x^2};\quad u(x) = x^2 \)
		\task \((x-1)y''-x y'+y=(x-1)^2; \quad u(x) = x, \)
		\task \((\tan^2x)y''-(2 \tan x) y'+ \left(\sec ^2x+1\right)y=\sin^2x \, \tan x; \quad u(x) = \sin x\)
			\end{tasks}
	
	\Question\label{3-3-4}
	Show that  $y_p$ in \eqref{eq:particular-final-form} may be written as 
	\[y_p(x) = \int_{x_0}^x G(x, t)  f(t) \, dt,\]
	where $x, x_0$ are in $I$ and 
	\[G(x, t) = \frac{y_1(t) y_2(x) - y_1(x) y_2(t)}{W(t)}=\frac{\begin{vmatrix}
			y_1(t)&y_2(t)\\
			y_1(x)&y_2(x)
	\end{vmatrix}}{\begin{vmatrix}
	y_1(t)&y_2(t)\\
	y'_1(t)&y'_2(t)
\end{vmatrix}}\] with $W(t) = W(y_1(t), y_2(t))$, the Wronskian of $y_1$ and $y_2$. The function $G(x, t)$ is called the \textbf{Green's function}  for  \eqref{eq:homogeneous1}.


\Question\label{3-3-5}  Use Problem~\ref{3-3-4} to express a particular solution $y_p(x)$ of $y''+k^2 y = f(x),$ where $k>0$ is in the form
\[y_p(x) = \frac{1}{k}\int_{x_0}^x f(t) \sin(k(x-t))\, dt. \]
The integral on the right with $x_0=0$ is  called the \textit{convolution} of $f(x)$ and $\sin (kx),$ in the context of the Laplace transform (see Definition~\ref{def:convolution} in Chapter~\ref{ch:Laplace Transform}).

\Question\label{3-3-6} 
Show that the solution of \eqref{eq:nonhomogeneous1} with the initial conditions $y(x_0) = 0$ and $y'(x_0) = 0$ is 
		\[y(x) = \int_{x_0}^x G(x, t)  f(t) \, dt.\] \textit{Hint:} Use the following rule for  differentiating under the integral sign: 
		if \[F(x) = \int_{x_0}^x g(x, t)\, dt,\] then 
		\[F'(x) = g(x,x) + \int_{x_0}^x \frac{\partial g}{\partial x}(x, t) \, dt.\]

\Question\label{3-3-7}  Use the result in Problem~\ref{3-3-6} to obtain the solution of the differential equation in Example~\ref{ex:3-3-1} satisfying $y(0) = 0, y'(0) =0.$\\


\hspace{-3.5ex}\textbf{\ref{subsec:undertermined-coefficients}} \textbf{Undetermined Coefficients}\\

\Question\label{3-3-8} Find the general solution of each of the following nonhomogeneous linear differential equations.
\begin{tasks}(2)
	\task \(y''-5y'+4y= 5x+1\)
	\task \(y''-y'+2y= x^2+x+1\)
	\task \(y''-5y'+4y= 5e^{2x}\)
	\task \(y''+4y= 5\cos x\)
	\task \(y''+4y= 5\sin(2x)\)
	\task \(y''+4y'+5y = 2e^{-x}\sin x\)
	\task \(y''+4y'+4y = 3e^{2x}\)
	\task \(y''-4y'+4y = 3xe^{2x}\)
	\task \(y''- 4y'+5y = 2e^{2x}\sin x\)
	\task \(y''+16y = 2\sin(4x)+3\cos(4x)\)
	\task \(y''-5y'+4y= 5e^x+3e^{4x}\)
	\task \(y''+16y = x^3+1+2e^{4x}+ \sin x\)
	\task \(y''-y'=1+7e^x\)
	\task \(y''+y' =2+\sin x -7e^{-x}\)
\end{tasks}
%\Question\label{3-3-9} Let $\omega$ and $\gamma$ be positive constants. Find the general solution of 
%\[y''+\omega^2 y= \sin(\gamma x)\] when 
%\begin{tasks}
%\task \(\omega\ne \gamma \)
%\task \(\omega = \gamma\)
%\end{tasks}
\Question\label{3-3-9} Let $\omega$ be a fixed positive constant. Find the  solution of the initial value problem
\[y''+\omega^2 y= 5\cos(\omega x),\quad  y(0)= 0, y'(0)=0\] by following the  steps below.
\begin{tasks}
	\task\label{item:nonresonance} Find the  solution of the initial value problem 
	\[y''+\omega^2 y= 5\cos(\gamma x),\quad  y(0)= 0, y'(0)=0\] for 
	\(\gamma\ne \omega.\)
	\task Denote by $y_\gamma(x)$  the  solution obtained in part~\ref{item:nonresonance}. For each $x$ in $(-\infty, \infty),$ evaluate the limit \[\lim\limits_{\gamma\to\omega}y_\gamma(x).\] 
	\task\label{item:resonance-problem-solution} Verify that the function defined by \[y = \lim\limits_{\gamma\to\omega}y_\gamma(x)\] is the  solution of \[y''+\omega^2 y= 5\cos(\omega x), \; y(0) =0, y'(0)=0.\] 
	\task Obtain the solution found in part \ref{item:resonance-problem-solution} by  using the method of undetermined coefficients to find a particular solution.
\end{tasks}

\Question\label{3-3-10} Solve the following initial value problems and plot the corresponding solutions  using a graphing utility. Describe any  differences you observe in the behavior of the two solutions.
\begin{tasks}
	\task \(y''+y=  \sin (2x), \quad y(0)=1, y'(0)=0\)
	\task \(y''+y=  \sin x, \quad y(0)=1, y'(0)=0\)\\
\end{tasks}


\hspace{-.75cm}\textbf{\ref{subsec:undertermined-coefficients}} \textbf{Operator Factorization}\\

\Question\label{3-3-11} Find a particular solution of 
\[y''-9y= e^{3x}\] by using the method used in Example~\ref{eg:operator-method-constant-coefficients-1}.

\Question\label{3-3-12}Find a particular solution of 
\[y''-9y= e^{-3x}\] by using the method used in Example~\ref{eg:operator-method-constant-coefficients-1-revisited}.

\Question\label{3-3-13} 
Find a particular solution of each of the following equations by using the method used in Example~\ref{eg:shifting-properties} (the exponential shifting properties).
\begin{tasks}
	\task \(y''-6y'+ 9y= xe^{3x}\)
	\task \(y''-2y'-3y = 16 e^{5x}\)
\end{tasks}
 \Question\label{3-3-14} Use the exponential shifting properties to show that the general solution of the homogeneous equation
 \[(D-\lambda)^2y = 0,\] where $\lambda$ a  real constant, has
 \[y = (c_1+c_2x)e^{\lambda x},\] with $c_1$ and $c_2$  arbitrary constants.
\Question\label{3-3-15} 
Find a particular solution of 
\[y''+(x+1)y'+(x+1)y= e^{-x}, \quad x>0,\] by using the method used in Example~\ref{eg:operator-method-example-variable-coefficients}.
% \(y= e^{-x^2/2}\) is a solution to homogeneous equaiton.
\Question\label{3-3-16} Determine a fundamental set of solutions for
\begin{equation}\label{ex:3-3-fundamental-set-of-solutions}
	y''+(x+1)y'+xy=0 
\end{equation} by carrying out the following steps.
\begin{enumerate}[label=(\roman*),noitemsep]
	\item Express  \eqref{ex:3-3-fundamental-set-of-solutions}  in  operator from as\[(D+x)(D+1)y=0.\] 
	\item Solve $(D+1)y=0$  to obtain a nonzero solution $y_1,$ and verify that \(y_1\) is indeed a solution of \eqref{ex:3-3-fundamental-set-of-solutions}.
	\item   Find a  function \(y_2\) by using the Liouville  reduction formula ~\eqref{eq:reduction-of-order-formula} so that $\{y_1, y_2\}$ forms a fundamental set of solutions for \eqref{ex:3-3-fundamental-set-of-solutions}.
	
\end{enumerate}
\Question\label{3-3-17} Let $a_0(x), a_1(x), b_0(x), b_1(x)$ be differentiable functions on an interval $I.$ Prove that 
\[(a_1 D+a_0)(b_1D+b_0) =(b_1D+b_0)(a_1 D+a_0)\] if and only if \[a_1b_1'= a_1'b_1\quad \mbox{ and }\quad a_1b_0'= a_0'b_1.\]
\Question\label{3-3-18} Let $a_1(x)$ be a continuous function on some interval $I.$ Let $\alpha(x)$ and $g(x)$ be  differentiable functions on $I.$   Prove the following general exponential shifting properties.
\begin{enumerate}[label=(\roman*), noitemsep]
	\item \((D+a_1(x))(e^{\alpha (x)} g(x)) = e^{\alpha(x)} (D+\alpha'(x)+a_1(x)) g(x).\)
	\item \(\displaystyle\frac{1}{D+a_1(x)}(e^{\alpha (x)} g(x)) = e^{\alpha(x)} \frac{1}{D+\alpha'(x)+a_1(x)} g(x).\)
\end{enumerate}
\Question\label{3-3-19}  Use the shifting properties established in  Problem~\ref{3-3-18} to find a particular solution of each of the following equations.
\begin{enumerate}[label=(\roman*), noitemsep]
	\item \(y''+(x+1)y'+(x+1)y= e^{-x}, \quad x>0,\)
	\item \(y''+(x+1)y'+xy= e^{-x^2/2}, \quad x>0,\)
\end{enumerate}
\end{Exercise}

\setboolean{firstanswerofthechapter}{true}
\begin{multicols}{2}\scriptsize
	\begin{Answer}[ref={EX33}]\\
\ref{subsec:variation-parameters} \textbf{Method of Variation of Parameters}
		\Question \label{3-3-1a}
		\begin{tasks}
			\task \(y= c_1 \cos x+c_2 \sin x +e^x\)
			\task \(y=c_1 e^x+c_2 e^{-x}+\frac{1}{9} e^{-2 x} (3 x+4)\)
			\task \(y=c_1\cos x+c_2 \sin x-\frac{x}{2} \cos x \)
			\task \(y= c_1\cos x+c_2\sin x\\+ \frac{1}{8} \left(-2 x^2+1\right) \cos x
			+\frac{1}{4} (x+4) \sin x\)
			\task \(y=c_1 \cos x+c_2 \sin x\\+x \sin x+\cos x \ln \abs{\cos x}\)
			\task \(y=c_1 \cos x+c_2 \sin x\\-\cos x \ln\abs{\sec x + \tan x}\)
			\task \(y=c_1 \cos x+ c_2 \sin (x)\\+\frac{1}{6} \cos (2 x)+\frac12\)
			\task \(y= c_1 \cos x+c_2\sin x\\ +x \cos x - \sin x \ln\abs{\cos x}\)
			\task \(y=c_1 \cos x+c_2 \sin x\\+\sin x\, \ln\!\Big|\tan\!\big(\frac{\pi}{4} + \frac{x}{2}\big)\Big|- 1,\)
			\task \(y=c_2 e^{-2 x} x+c_1 e^{-2 x}+\frac{1}{2} e^{-2 x} x^2\)
			\task 	\(y=c_1 e^{-2 x}+c_2 e^{-x}\\+ e^{-x} \ln \left(e^x+1\right)+e^{-2 x}\ln\left(e^x+1\right)\)
			\task \(y=c_1 e^{-2 x}+c_2 e^{-x}-e^{-2 x} \cos \left(e^x\right)\)
			\task \(y=c_1 e^{-2 x}+c_2x e^{-2 x}+\frac{1}{4} \sin (2 x) \)
			\task \(y=c_1 e^{-2 x}+c_2 e^{-2 x} x-\frac{1}{2} e^{-2 x} \cos (2 x)\)
		\end{tasks} 
		
		
	\Question \label{3-3-2a}
	\begin{tasks}	
		\task \(y=\frac{c_1}{x}+c_2 x-5\)
		\task \(y=c_2 x+\frac{c_1}{x}+\frac{2 x^4}{15}\)
		\task \(y= c_1 x+c_2 x \ln\abs{x}+ \frac{1}{2} x (\ln\abs{x})^2\)
		\task \(y=c_1+c_2 x^2+\frac{1}{2} x^2 \ln\abs{x}\)
	\end{tasks}
	
		\Question \label{3-3-3a}
	\begin{tasks}	
		\task \(y=c_1 x^2+\frac{c_2}{x^2}-\frac{x}{3}\)
		\task \(y=c_1 x^2+\frac{c_2}{x^2}-\frac{1}{3 x}-\frac{1}{4}\)
		\task \(y=\frac{-4 \ln x -1}{16 x^2}+c_1 x^2+\frac{c_2}{x^2}\)
		\task \(y= c_1 x+c_2 e^x-1-x-x^2\)
		\task \( y = c_1 \sin x+c_2 x \sin x-\sin x \cos x\)
	\end{tasks}
	
	
		\Question \label{3-3-4a}
		It follows from the relation \eqref{eq:particular-final-form} that 
		\(y_p(x)= -y_1(x) \int_{x_0}^x \frac{y_2(t) f(t)}{W(t)} \, dt 
		+ y_2(x) \int_{x_0}^x \frac{y_1(t) f(t)}{W(t)} \, dt\\
		=\int_{x_0}^{x} f(t)\frac{y_1(t) y_2(x) - y_1(x) y_2(t)}{W(t)}\,dt\)
	
	\Question \label{3-3-5a} Use $y_1(x)= \cos(kx)$ and $y_2(x)= \sin(kx)$ in Problem~\ref{3-3-4}. 
	\Question \label{3-3-6a} Follow the hint provided in the problem.
	\Question \label{3-3-7a} Follow the direction provided for the problem.\\
	
	\hspace{-4ex}\ref{subsec:undertermined-coefficients} \textbf{Method of Undetermined Coefficients}
	\Question \label{3-3-8a}
	\begin{tasks}	
		\task \(y= \frac{1}{16} (20 x+29)+c_1 e^x+c_2 e^{4 x}\)
		\task \(y=\frac{1}{2} \left(x^2+2 x+1\right)+c_1 e^{x/2} \cos \left(\frac{\sqrt{7} x}{2}\right)+c_2 e^{x/2} \sin \left(\frac{\sqrt{7} x}{2}\right)\)
		\task \(y=c_1 e^x+c_2 e^{4 x}-\frac{5 e^{2 x}}{2}\)
		\task \(y=c_1 \cos (2 x)+c_2 \sin (2 x)+\frac{5 \cos (x)}{3}\)
		\task \(y=-\frac{5}{16} (4 x \cos (2 x)-\sin (2 x))+c_1 \cos (2 x)+c_2 \sin (2 x)\)
		
		\task \(y=c_1 e^{-2 x} \cos x+c_2 e^{-2 x} \sin x-\frac{2}{5} e^{-x} (2 \cos x-\sin x)\)
		\task \(y=c_1 e^{-2 x}+c_2 xe^{-2 x} +\frac{3 e^{2 x}}{16} \)
		\task \(y=c_1 e^{2 x} x+c_2 e^{2 x}+\frac{1}{2} e^{2 x} x^3\)
		\task \(y=c_2 e^{2 x} \cos x+c_1 e^{2 x} \sin x\\-\frac{1}{2} e^{2 x} (2 x \cos x-\sin x)\)
		\task \(y = c_1\cos(4x) +c_2\sin(4x)\\ +\left(-\frac{x}{4}+\frac{3}{32}\right) \cos (4 x)+\frac{3}{8} x \sin (4 x)\)
		
		\task \(y=c_1 e^x+c_2 e^{4 x}\\+\frac{1}{9} e^x \left(9x e^{3 x} -3 e^{3 x}-15 x-5\right)\)		
		\task \(y= c_1 \cos (4 x)+c_2 \sin (4 x)+\frac{1}{128} \left(8 x^3-3 x+8\right) +\frac{1}{6}e^{4 x}+\frac{\sin x}{15}\)
		\task \(y=c_1+c_2e^x-x+ 7 (x-1)e^x\)
		\task \(y= c_1 +c_2 e^{-x}+2x -\frac{1}{2}(\sin x+\cos x)\\+7 (x+1)e^{-x}\)
	\end{tasks}
	\Question \label{3-3-9a}
	\begin{tasks}	
		\task \(y_\gamma(x)= -\frac{5 (\cos (\gamma  x)-\cos (x \omega ))}{\gamma ^2-\omega ^2}\)
		\task \(y(x) = \lim\limits_{\gamma\to\omega}y_\gamma(x)\\ = -5\lim\limits_{\gamma\to\omega}\frac{\cos (\gamma  x)-\cos (x \omega )}{\gamma ^2-\omega ^2}\\=\frac{5 x \sin (x \omega )}{2 \omega }\)
		\task It is a routine matter to verify \(y(x) =\frac{5 x \sin (x \omega )}{2 \omega }\) satisfies the initial value problem \(y''+\omega^2 y = 5 \cos(\omega x), \; y(0) = 0, y'(0) = 0.\)
		\task  Follow the method of undetermined coefficients
		to find \\\(y_p(x) = \frac{5  x   \sin (x \omega )}{2 \omega}.\)\\ Also, \(y_h(x) =c_1 \cos (x \omega )+c_2 \sin (x \omega ) \), so that the general solution of the differential equation is\\
		\(y = c_1 \cos (x \omega )+c_2 \sin (x \omega )+\frac{5  x   \sin (x \omega )}{2 \omega}.\) Using $y(0) = 0, y'(0) = 0,$ we find $c_1=0$ and $c_2=0$. Therefore, \(y=\frac{5  x   \sin (x \omega )}{2 \omega}\) solves the initial value problem.
	\end{tasks}
	
	\Question \label{3-3-10a}
	\begin{tasks}	
		\task \(y=\frac{10}{3} (2 \sin (x)-\sin (2 x))\)
		\task \(y=5 \sin (x)-5 x \cos (x)\)\\
		\includegraphics[width=1\linewidth]{chap03/mathematica/3-3-10a}
		The solution \(y=\frac{10}{3} (2 \sin (x)-\sin (2 x))\) is a bounded sinusoidal response, but the solution \(y=5 \sin (x)-5 x \cos (x)\) becomes unbounded because of the term that contains $x.$
	\end{tasks}
\hspace{-4ex}\ref{subsec:method of operators} \textbf{Method of Operator Factorization}\\
\Question \label{3-3-11a} \(y_p=\frac{1}{6}x e^{3 x}\)
\Question \label{3-3-12a} \(y_p = -\frac{1}{6}x e^{-3 x} \)
\Question \label{3-3-13a} 
\begin{tasks}
	\task 	\(y_p = \frac{1}{6}x^3 e^{3 x} \)
	\task 	\(y_p=\frac{4 }{3}e^{5 x}\)
\end{tasks}
\Question \label{3-3-14a} Follow the direction provided.
\Question \label{3-3-15a} \(y_p= e^{-x}\); 
\((D+1)(D+x)y = e^{-x}\) gives \((D+x)y =\frac{1}{D+1}e^{-x} = e^{-x}\frac{1}{D}1 = (x-1)e^{-x};\) note the choice of \(x-1\) instead of just \(x\). Then \(y =\frac{1}{D+x}xe^{-x} = e^{-x}\frac{1}{D+(x-1)}(x-1)\), which implies
\(y_p=e^{-x}\left[e^{-(x-1)^2/2}\int(x-1)e^{(x-1)^2/2}\, dx\right] = e^{-x}.\)
\Question \label{3-3-16a}
\begin{tasks}
	\task See Example~\ref{eg:operator-method-example-variable-coefficients}.
	\task \((D+1)y=0\) gives \(y = e^{-x}\) which also satisfies the main equation.
    \task Take \(y_1= e^{-x}\),\quad\(y_2= e^{-x}\int\frac{\int e^{-\frac{1}{2}(x+1)^2}}{e^{-2x}}\,dx=e^{-x}\int e^{-\frac{1}{2}(x-1)^2}\,dx\)
\end{tasks}
\Question \label{3-3-17a} Let $A = a_1D+a_0$ and $B=b_1D+b_0$, and compute \(AB = \bigl(a_1 b_1\bigr) D^{2}
+
\Big(a_1(b_1' + b_0) + a_0 b_1\Big) D
+
\big(a_1 b_0' + a_0 b_0\big)\) and a similar formula for \(BA.\) Show that $[AB-BA]y = (a_1b_1'-a_1'b_1)y' + (a_1b_0'-a_0'b_1)y.$ This, $AB=BA$ if and only if \(a_1b_1'=a_1'b_1\) and \(a_1b_0'=a_0'b_1.\)
\Question \label{3-3-18a}\begin{tasks}
	\task The product rule of differential yields he result.
	\task Apply \(D+a_1(x)\) to the right side and use the result in the first part.
\end{tasks}
\Question \label{3-3-19a}
\begin{tasks}
	\task \((D+1)(D+x)y= e^{-x}\) implies\\ \((D+x)y = \frac{1}{D+1}e^{-x} = e^{-x} \frac{1}{D}1\\= (x-1)e^{-x}\) so that 
	\(y_p(x) =\frac{1}{D+x}(x-1)e^{-x} = e^{-x}\frac{1}{D+(x-1)}(x-1)\\ = e^{-x}e^{-(x-1)^2/2}\\ \int (x-1)e^{(x-1)^2/2}\, dx =e^{-x}.\)
	\task \((D+x)(D+1)y= e^{-x}\) implies\\ \((D+1)y =\frac{1}{D+x}e^{-x^2/2}\\ = e^{-x^2/2}\frac{1}{D-x+x}1=e^{-x^2/2}\frac{1}{D}1= (x-1)e^{-x^2/2}\); note the choice of $x-1$  and not just $x.$
	Then \(y_p= \frac{1}{D+1}(x-1)e^{-x^2/2} = e^{-x^2/2}\frac{1}{D-(x-1)}(x-1)\\=e^{-x^2/2}e^{(x-1)^2/2}\\\int(x-1)e^{-(x-1)^2/2}\, dx= -e^{-x^2/2}.\)
	
\end{tasks}

	\end{Answer}
\end{multicols}
\setboolean{firstanswerofthechapter}{false}




\section{Higher Order Linear Differential Equations}\label{sec:higher order DEs}

In this section, we discuss  methods for solving the $n^\text{th}$ order linear differential equations of the form
	\begin{equation} \label{eq:nonhomogeneous2}
	y^{(n)} + a_{n-1}(x)y^{(n-1)} + \cdots + a_1(x)y' + a_0(x)y = f(x),
\end{equation}
where  the coefficient functions \( a_0(x), \dots, a_{n-1}(x) \) and  forcing term \( f(x) \) are continuous on some interval \( I .\)

To find the general solution of \eqref{eq:nonhomogeneous2}, we first find the general solution $y_h$ of the associated homogeneous equation
	\begin{equation} \label{eq:homogeneous2}
	y^{(n)} + a_{n-1}(x)y^{(n-1)} + \cdots + a_1(x)y' + a_0(x)y = 0
\end{equation}
and a particular solution $y_p$ of \eqref{eq:nonhomogeneous2}. 
	Then the general solution  of the nonhomogeneous equation \eqref{eq:nonhomogeneous2} is given by
\begin{equation} \label{eq:general-solution}
	\boxed{	y = y_h(x)+y_p(x)
		= c_1 y_1(x) + c_2 y_2(x) + \cdots + c_n y_n(x) + y_p(x),}
\end{equation}
on $I,$ where   \( \{y_1, y_2, \dots, y_n\} \) is   a fundamental set of solutions of \eqref{eq:homogeneous2} and $c_1, \dots, c_n$ are arbitrary constants. The function $y_h(x),$  also denoted by $y_c(x),$ is  called the \textbf{\textit{complimentary function}} of the general solution of \eqref{eq:homogeneous2}.
	

The linear independence of solutions \(y_1, y_2, \dots, y_n\)  of \ref{eq:homogeneous2} can be determined using their Wronskian, as stated in the following theorem.

\begin{theorem}\label{thm:wronskian-nonzero-3}
	Let \( y_1, y_2, \dots, y_n \) be solutions  of \eqref{eq:homogeneous2} on $I.$
	 Then they are linearly independent on $I$ if and only if 
	 \[W(y_1, \dots, y_n)(x)\ne 0 \mbox{ for all } x \mbox{ in } I.\]
\end{theorem}
\begin{proof}
	The proof is similar to that of Theorem~\ref{thm:wronskian-nonzero-2} and is omitted.
\end{proof}



Methods for constructing a fundamental set of solutions for  \eqref{eq:homogeneous2} vary. When we have means to find a single nonzero solution,  the reduction of the order (see page \pageref{page:reduction-of-order}) can be employed to obtain additional linearly independent solutions.  This method applies not only to second order  homogeneous linear equations but also extends to higher order equations, as discussed in the next theorem.

\begin{theorem}[Reduction of the Order]
	Let $y_1$ be a  solution of \eqref{eq:homogeneous2} such that \(y_1(x)\ne 0\) on $I,$  and let $y = u y_1$ be a solution of \eqref{eq:homogeneous2} for a nonconstant function $u$  on $I.$ If \[\{v_1, \dots, v_{n-1}\}\] is a fundamental set of solutions for  the equation  of  order $n-1$ in $v=u'$ obtained by substituting $ y= u y_1$ into \eqref{eq:homogeneous2} and if $u'_k= v_k$ for $k = 1, \dots, n-1,$  then 
	\[\{y_1, u_1y_1, u_2y_1, \dots, u_{n-1}y_1\}\] forms a fundamental set of solutions for  \eqref{eq:homogeneous2} on $I.$
\end{theorem}

\begin{proof}
	We present a proof for the case $n=3.$ The general case proceeds in a similar way, and a careful reader is encouraged to verify it, noting that it involves the use of binomial coefficients that are produced for higher order derivatives of the product function $ y= u y_1.$ 
	
	For $y= uy_1$ to be a solution of \eqref{eq:homogeneous2}, we must have
	\begin{equation*}
		\begin{split}
		&u'''y_1+ 3u''y'_1+ 3u'y''_1 + uy'''_1\\
		&+a_2 u''y_1+2a_2 u'y'_1+a_2uy''_1\\
		&+a_1u'y_1+a_1uy'_1\\
		&+a_0uy_1 =0,
		\end{split}
	\end{equation*}
	which gives
	\[\begin{split}
		&u'''y_1+ (3y'_1+a_2y_1) u''+ (3y''_1+2a_2y'_1+a_1y_1)u'\\
		&+ (y'''_1+a_2y''_1+a_1y'_1+a_0y_1)u=0
	\end{split}\]
	Since $y_1$ is a solution of  \eqref{eq:homogeneous2},  we have \(y'''_1+a_2y''_1+a_1y'_1+a_0y_1=0.\) Consequently, we obtain
	\[u'''y_1+ (3y'_1+a_2y_1) u''+ (3y''_1+2a_2y'_1+a_1y_1)u'=0.\]
	Letting  $v= u'$ gives
		\[v''y_1+ (3y'_1+a_2y_1) v'+ (3y''_1+2a_2y'_1+a_1y_1)v=0,\]
which is a second order  homogeneous linear equation in $v.$ Suppose that $\{v_1, v_2\}$ is a fundamental set of solution of the second order equation and let $u_k$ be functions satisfying $u'_k = v_k,\; k = 1, 2.$
 To show
\[\{y_1, u_1y_1, u_2y_1 \}\] is a fundamental set of solutions for  \eqref{eq:homogeneous2}, 
let $c_0, c_1, c_2$ be real numbers such that 
\[c_0y_1(x) +c_1u_1(x) y_1(x) + c_2 u_2(x) y_1(x) = 0\] for all $x$ in $I.$ Since $y_1(x) \ne 0$ on $I,$ we must have
\[c_0 +c_1u_1(x) + c_2 u_2(x)  = 0\] for all $x$ in $I.$ Differentiating yields
\[c_1u'_1(x) + c_2u'_2(x)  = 0\] for all $x$ in $I,$ 
that is,
\[c_1v_1(x) + c_2v_2(x)  = 0\] for all $x$ in $I.$  
The linear independence of $\{v_1, v_2\}$ on $I$ implies that $c_1 = c_2 = 0.$ 
Consequently, $c_0 = 0$ as well. Therefore,
\[
\{y_1,\, u_1 y_1,\, u_2 y_1\}
\]
forms a linearly independent set of solutions of~\eqref{eq:homogeneous2} for $n = 3,$ 
and hence constitutes a fundamental set of solutions.
\end{proof}

\begin{example}
	Find a fundamental set of solutions for  
	\begin{equation}\label{eg:reduction of order}
		y^{(4)}-4y'''+6y''-4y'+y=0, 
	\end{equation} given that $y= e^x$ is one of its  solutions. Also, write down the general solution.
\end{example}
\begin{solution}
	Let $y= ue^x$ a solution of \eqref{eg:reduction of order}, where $u$ is a nonconstant function on $(-\infty, \infty).$ Differentiating four times gives
	\[
		\begin{split}
			&u^{(4)}e^x+ 4u'''e^x+ 6u''e^x + 4u'e^x+ ue^x\\
			&-4u'''e^x -12u''e^x-12u'e^x -4 ue^x\\
			&+6u''e^x +12u'e^x+6 ue^x\\
			&-4u'e^x -4ue^x \\
			&+ue^x =0.
		\end{split}\]
		After canceling terms, we obtain \(u^{(4)}e^x=0,\) which yields \(u^{(4)}(x)=0\) in $(-\infty, \infty).$  Let $v= u'.$ Then we find
		\(v'''(x) = 0\) on $I.$
		Integrating, we obtain
		\[v(x) = Ax^2+ Bx + C,\] where $A, B, C$ are arbitrary constants.
		
		 Take $v_1(x) = 1, v_2(x)= 2x,$ and  $v_3(x)= 3x^2.$ Then $\{v_1, v_2, v_3\}$ forms  a fundamental set of solutions for  \(v'''(x) = 0\) on $I.$  Since $u'=v,$ we take $u_1(x) = x, u_2(x) = x^2,$ and  $u_3(x) =x^3,$ so that 
		\[\{e^x, xe^x, x^2e^x, x^3 e^x\}\] forms a fundamental set of solutions for  \eqref{eg:reduction of order}. The general solution of the equation is
		\[y= c_1e^x+c_2xe^x+c_3x^2e^x+c_4x^3 e^x = e^x(c_1+c_2x+c_3x^2+c_4x^3), \] where $c_1, c_2, c_3, c_4$ are arbitrary constants.
\end{solution}
	

	

	
	To find a particular solution \( y_p(x) \) of \eqref{eq:nonhomogeneous2} by using the method of variation of parameters, we assume that it has the form
	\begin{equation} \label{eq:particular}
		y_p(x) = u_1(x)y_1(x) + u_2(x)y_2(x) + \cdots + u_n(x)y_n(x),
	\end{equation}
	where the functions \( u_1(x), u_2(x), \dots, u_n(x) \) are to be determined.
	
	Differentiating \eqref{eq:particular} successively, and to simplify calculations, imposing the conditions
	\begin{align*}
		u_1'y_1 + u_2'y_2 + \cdots + u_n'y_n &= 0, \\
		u_1'y_1' + u_2'y_2' + \cdots + u_n'y_n' &= 0, \\
		&\vdots \\
		u_1'y_1^{(n-2)} + u_2'y_2^{(n-2)} + \cdots + u_n'y_n^{(n-2)} &= 0,
	\end{align*}
	we obtain
	\[u_1'y_1^{(n-1)} + u_2'y_2^{(n-1)} + \cdots + u_n'y_n^{(n-1)} = f(x).\]
	These \( n \) equations can be solved for \( u_1', u_2', \dots, u_n'\) by using  Cramer's rule as follows.
	We first write down the system  of \( n \) equations  in the matrix form 
	\begin{equation}\label{eq:matrixform-variation-of-parameters}
	\begin{bmatrix}
		y_1 & y_2 & \cdots & y_n \\
		y_1' & y_2' & \cdots & y_n' \\
		\vdots & \vdots & \ddots & \vdots \\
		y_1^{(n-1)} & y_2^{(n-1)} & \cdots & y_n^{(n-1)}
	\end{bmatrix}
	\begin{bmatrix}
		u_1' \\
		u_2' \\
		\vdots \\
		u_n'
	\end{bmatrix}
	=
	\begin{bmatrix}
		0 \\
		0 \\
		\vdots \\
		f(x)
	\end{bmatrix},
	\end{equation}
which  can  be solved for \( u_1', u_2', \dots, u_n'\) by using   \textbf{Cramer's rule}. In fact, we have
\[u'_j= \frac{W_j}{W},\] where $W$ is the Wronskian of $y_1, \dots, y_n$ and $W_k$ is the determinant obtained by replacing the $j-$th column of $W$ by the column on the right-hand side of \eqref{eq:matrixform-variation-of-parameters}. We  find \[u_j(x) =\int \frac{W_j(x)}{W(x)}\, dx \] for  each \( j = 1, \dots, n\)   and hence $y_p(x)$ is then fully determined by \eqref{eq:particular}.


	As discussed in Remark~\ref{re:variational-over-undertermined-coefficients},  the method of variation of parameters is quite general and works for any linear nonhomogeneous differential equation (even with variable coefficients), as long as the required integrals for \( u_1, u_2, \dots, u_n \) exist.
	
\begin{example}
 Find the general solution of 
		\[y''' + y' = \tan x\] by using the variation of parameters.
\end{example}

\begin{solution}
		The homogeneous equation associated to the nonhomogeneous  equation \(y''' + y' = \tan x\) is
		\[
		y''' + y' = 0
		\]
		with the auxiliary equation 
		\[
		r^3 + r = 0 \quad \text{ so that } \quad r(r^2 + 1) = 0.
		\]
	Then we have
		\[
		r = 0, \quad r = i, \quad r = -i,
		\]
	and therefore the functions
		\[
		y_1(x) = 1, \quad
		y_2(x) = \sin x, \quad
		y_3(x) = \cos x
		\]
		form a fundamental set of solutions for  the homogeneous equation. 
The general solution to $y'''+y= \tan x$ is then given by
		\[
		y(x) = c_1 + c_2 \sin x + c_3 \cos x + y_p(x),
		\]
where $y_p(x)$ is a particular solution of the nonhomogeneous equation.

By the method of variation of parameters, we assume that $y_p(x)$ is of the form
		\[
		y_p(x) = u_1(x) + u_2(x) \sin x + u_3(x) \cos x,
		\]
where the functions $u_1, u_2, u_3$ are to be determined so that $u_1', u_2', u_3'$ satisfy
		
		\[
		\begin{cases}
			u_1' + u_2' \sin x + u_3' \cos x = 0, \\[6pt]
			u_2' \cos x - u_3' \sin x = 0, \\[6pt]
			u_2' \sin x - u_3' \cos x= \tan x.
		\end{cases}
		\]
	The system in the matrix form is
		\[
		\begin{bmatrix}
			1 & \sin x & \cos x \\
			0 & \cos x & -\sin x \\
			0 & -\sin x & -\cos x
		\end{bmatrix}
		\begin{bmatrix}
			u_1' \\ u_2' \\ u_3'
		\end{bmatrix}
		=
		\begin{bmatrix}
			0 \\ 0 \\ \tan x
		\end{bmatrix}.
		\]
	The Wronskian $W(x)$ of $1, \sin x, \cos x$ is
	\[
	W(x) = 
	\begin{vmatrix}
		1 & \sin x & \cos x\\[6pt]
		0 & \cos x & -\sin x \\[6pt]
		0 & -\sin x & -\cos x
	\end{vmatrix}.
	\]
Using the co-factor expansion along the first column, we have
	\[
	W(x) = 
	\begin{vmatrix}
		\cos x & -\sin x \\[6pt]
		-\sin x& -\cos x
	\end{vmatrix}\\
	= -\cos^2 x - \sin^2 x\\
	 = -1.
	\]
We compute
	\[
	W_1(x) = 
	\begin{vmatrix}
		0 & \sin x & \cos x\\[6pt]
		0 & \cos x& -\sin x \\[6pt]
		\tan x & -\sin x & -\cos x
	\end{vmatrix}
	\]
by expanding along the first column and obtain
	\[
	W_1(x) =  \tan x
	\begin{vmatrix}
		\sin x & \cos x \\[6pt]
		\cos x & -\sin x
	\end{vmatrix} 
	= \tan x( -\sin^2 x- \cos^2 x ) = -\tan x.
	\] 
Then we have
	\[
	u_1' = \frac{W_1}{W} = \frac{ -\tan x }{ -1 } = \tan x,
	\]
	and consequently, we can take
	\[u_1(x) = -\ln\abs{\cos x}.\]
We next compute
	\[
	W_2(x) = 
	\begin{vmatrix}
		1 & 0 & \cos x \\[6pt]
		0 & 0 & -\sin x\\[6pt]
		0 & \tan x& -\cos x
	\end{vmatrix}
	\]
by expanding along the first column and obtain
	\[
	W_2(x) = 
	\begin{vmatrix}
		0 & -\sin x \\[6pt]
		\tan x & -\cos x
	\end{vmatrix}
	=   0 - (-\sin x) \tan x 
	= \sin x \tan x.
	\]
Then we have
	\[
	u_2' = \frac{W_2}{W} = \frac{ \sin x \tan x }{ -1 } = -\sin x \tan x =-\frac{\sin^2x}{\cos x}= \cos x- \sec x,
	\]
	and so  \[u_2(x) = \int \big(\cos x- \sec x\big)\, dx.\] We can take \[u_2(x)= \sin x- \ln\abs{\sec x+\tan x}.\]
We also compute
	\[
	W_3(x) =
	\begin{vmatrix}
		1 & \sin x & 0 \\[6pt]
		0 & \cos x & 0 \\[6pt]
		0 & -\sin x & \tan x
	\end{vmatrix}
	\]
by expanding along the first column and obtain
	\[
	W_3(x) = 
	\begin{vmatrix}
		\cos x & 0 \\[6pt]
		-\sin x & \tan x
	\end{vmatrix}
= \cos x \tan x = \sin x,
	\]
and so
	\[
	u_3' = \frac{W_3}{W} = \frac{ \cos x \tan x }{ -1 } = - \sin x.
	\]
	We can take \[u_3(x) = \cos x.\]
Consequently, $y_p$ is given by
	\[
	\begin{aligned}
			y_p(x) &= u_1(x)  + u_2(x) \sin x + u_3(x) \cos x\\
			&= \ln\abs{\sec x}+ \sin x\big(\sin x-\ln\abs{\sec x+\tan x}\big)+\cos^2(x)\\
			&=1+\ln\abs{\sec x}-\sin x\ln\abs{\sec x+\tan x}.
	\end{aligned}
	\]
Hence the general solution to $y'''+y' = \tan x$ is
	\[
		y(x) = c_1 + c_2 \sin x + c_3 \cos x +\ln\abs{\sec x}-\sin x\ln\abs{\sec x+\tan x},\]
		where $c_1$, $c_2$ and $c_3$ are arbitrary constants.
\end{solution}



\begin{Exercise}\label{EX35}
	\vspace{-\baselineskip}% <-- You don't need this line of code if there's some text here
	
	\Question\label{4-4-1}
Find the general solution of each differential equation below by using variation of parameters to find a particular solution.
	\begin{tasks}(2)
	\task \(y'''+y' = x+\sin x\)
	\task \(y'''+4y' = \tan(2x)\)
	\task \(y'''+y' = \cot x\)
	\task \(y'''+9y' = \csc(3x)\)

	\end{tasks}
	
	\Question\label{4-4-2} Solve each differential equation below by using the provided fundamental set of solutions for the associated homogeneous equation to find a particular solution  by variation of parameters.
	\begin{tasks}(1)
		\task \(x^3 y'''-x^2 y''+2 x y'+2y =-\frac{1}{x},\;\quad \{x, x^2, x \, \ln x\}\)
		\task \(x^3 y'''+x^2 y''-2 x y'+2 y=-\frac{1}{x^2},\;\quad \{x, x^2, 1/x\}\)
		\task \(xy'''-y''-xy'+y = 4x^2 e^x,\; \quad \{x, e^x, e^{-x}\}\)
		\task \(x^3 y'''-(x+3) x^2 y''+2 (x+3) x y'-2 (x+3) y=x^5,\;\quad \{x, x^2, x \, \ln x\}\)
		\task \(x^4 y^{(4)}+4 x^3 y'''+x^2 y''+x y'-y=\ln x,\; \quad \{\frac{1}{x}, x, x \ln x, x (\ln x)^2\}\)
		
	\end{tasks}
	
		\Question\label{4-4-3} Find a fundamental set of solutions for each differential equation below by using the given solution for the reduction of the order.
	\begin{tasks}(1)
		\task \( y^{(4)}-6 y'''+13 y''-12 y'+4y=0,\; \quad  y_1(x) =e^x \) for $x>0$
		\task \(x^3 y'''-x^2 y''+2 x y'+2y =0,\;\quad y_1(x) =x \) for $x>0$
		\task \(x^3 y'''+x^2 y''-2 x y'+2 y=0,\;\quad y_1(x)=x\) for $x>0$
		\task \(xy'''-y''-xy'+y = 0,\; \quad y_1(x)=x\) for $x>0$
	\end{tasks}
	
\end{Exercise}

\setboolean{firstanswerofthechapter}{true}
\begin{multicols}{2}\scriptsize
	\begin{Answer}[ref={EX35}]
		\Question \label{4-4-1a}
		\begin{tasks}
			\task \(y= c_1-c_2 \cos (x)+c_3 \sin (x)\\+ \frac{x^2}{2}-\frac{1}{2} x \sin x-\frac{\cos x}{2}\)
			\task \(y(x) = c_1 + c_2 \cos(2x)\\ + c_3 \sin(2x)
			- \frac{1}{8} \ln\abs{\cos(2x)}\\
			- \frac{1}{8} \sin(2x) \ln\abs{\sec(2x) + \tan(2x)}.
			 \)
			 \task \(y=c_1 + c_2 \cos x\ + c_3 \sin x+ \ln\abs{\sin x}+\cos x\ln\abs{\csc x+ \cot x}\)
			 \task \(y= c_1 + c_2 \cos (3x) + c_3 \sin (3x)-\frac19\big(\cos x + \ln\abs{\csc x + \cot x} + x \sin(3x) \\+ \cos(3x) \ln\abs{\sin x}\big)\)
		\end{tasks} 
		
		\Question \label{4-4-2a}
		\begin{tasks}
			\task \(y= c_1 x  + c_2x^2  + c_3x \ln x + \frac{1}{12x}\)
			\task \(y=  \frac{c_1}{x} + c_2x  + c_3x^2+ \frac{1}{12 x^2}\)
			\task \(y=c_1x+ c_2e^x+c_3e^{-x}+ x(x-1)e^x \)
			\task \(y =  c_1 x+c_2 e^x x-c_3 x (x+1)\\-\frac{1}{6} (x^4+3 x^3+6 x^2+6 x)\)
			\task \(y = \frac{c_1}{x}+c_2 x+c_3 x\ln x+c_4 x (\ln x)^2\\- \ln x-2\)
		\end{tasks} 
		\Question\label{4-4-3a} 
		\begin{tasks}(1)
			\task \( \{e^x, xe^x, e^{2x}, xe^{2x}\}\)
			\task \(\{x, x^2, x \ln x\}\)
			\task \(\{x, x^2, 1/x\}\)
			\task \(\{x, e^x, e^{-x}\}\)
		\end{tasks}
		
	\end{Answer}
\end{multicols}
\setboolean{firstanswerofthechapter}{false}


%\begin{hardsec}
\section{Applications} 


\subsection{Spring-Mass Systems}\label{subsection:spring-mass}\index{spring-mass systems}
A common example of a second order linear differential equation with constant coefficients arises in modeling the motion of a mass  attached to a spring whose other end is fixed to a support.  Let us consider a spring suspended vertically from a fixed ceiling (rigid support) with the mass attached to its lower end, as illustrated in Figure~\ref{fig:mass-spring}. 

%\textcolor{red}{open the figure below}

%open this figure
\begin{figure}[h]
	\begin{tikzpicture}[scale=1.2,>=stealth]
		\tikzset{
			spring/.style={thick,decorate,decoration={coil,aspect=0.4,segment length=4pt,amplitude=3pt}},
			mass/.style={draw,fill=pink!60,minimum width=1cm,minimum height=0.7cm},
			dashedline/.style={dashed,gray},
			labeltext/.style={font=\small}
		}
		%--------------------------------
		% (a) Unstretched spring
		%--------------------------------
		\begin{scope}[xshift=0cm]
			\node[below] at (0,-2.6) {(a)};
			% Ceiling
			\draw[thick,fill=gray!40] (-0.8,0) rectangle (0.8,0.2);
			\node[above] at (0,0.2) {\small Rigid Support};
			% Spring (natural length l)
		\draw[decoration={aspect=0.4, segment length=2mm, amplitude=3mm,coil},decorate] (0,0) -- (0,-2.2); 
			%\draw[spring, aspect=0.7] (0,0) -- (0,-2.2);
			\draw[<->] (0.4,0) -- (0.4,-2.2);
			\node[right,labeltext] at (0.41,-1) {$\ell$};
			% Dashed reference line
			\draw[dashedline] (-0.8,-2.2) -- (0.8,-2.2);
			\node[below,labeltext] at (0,-2.4) {unstretched spring};
		\end{scope}
		
		%--------------------------------
		% (b) Equilibrium position
		%--------------------------------
		\begin{scope}[xshift=1.75cm]
			\node[below] at (0,-4.1) {(b)};
			% Ceiling
			\draw[thick,fill=gray!40] (-0.8,0) rectangle (0.8,0.2);
			%\node[above] at (0,0.2) {\small Rigid Support};
			% Spring
			\draw[decoration={aspect=0.4, segment length=3mm, amplitude=3mm,coil},decorate] (0,0) -- (0,-3.1);
			%\draw[spring] (0,0) -- (0,-3.1);
			\node[mass] (m1) at (0,-3.4) {$m$};
			% Labels
			\draw[<->] (0.45,0) -- (0.45,-2.2);
			\node[right,labeltext] at (0.45,-1) {$\ell$};
			%\node[right,labeltext] at (0,-2.4) {$s$};
			\draw[dashedline] (-0.8,-3.1) -- (5,-3.1); %the line corresponding to the origin
			\draw[dashedline] (-0.8,-2.2) -- (0.8,-2.2);
			% Bracket for s
			\draw[<->] (0.45,-2.2) -- (0.45,-3.1);
			\node[right,labeltext] at (0.45,-2.7) {$s$};
			\node[below,labeltext] at (0,-3.8) {mass in equilibrium};
		\end{scope}
		
		%--------------------------------
		% (c) Motion (displacement x or y)
		%--------------------------------
		\begin{scope}[xshift=4.25cm]
			\node[below] at (0,-5.5) {(c)};
			% Ceiling
			\draw[thick,fill=gray!40] (-0.8,0) rectangle (0.8,0.2);
			%\node[above] at (0,0.2) {\small Rigid Support};
			% Spring
			\draw[decoration={aspect=0.4, segment length=4mm, amplitude=3mm,coil},decorate] (0,0) -- (0,-4.1);
			%\draw[spring] (0,0) -- (0,-3);
			\node[mass] (m2) at (0,-4.4) {$m$};
			% Labels
%			\node[right,labeltext] at (0,-1.3) {$\ell$};
%			\node[right,labeltext] at (0,-2.7) {$s$};
			
			% Bracket for total extension
			\draw[<->] (0.7,0) -- (0.7,-3.1);
			\node[right,labeltext] at (0.7,-1.5) {$\ell + s$};
	
			\draw[<->] (0.7,-4.1) -- (0.7,-3.1);
			\node[right,labeltext] at (0.7,-3.6) {$y(t)$};
			% Dashed lines
			%\draw[dashedline] (-0.8,-2.2) -- (0.8,-2.2);
			\draw[dashedline] (-0.8,-4.1) -- (0.85,-4.1);
			%\draw[dashedline] (-0.8,-4.7) -- (0.8,-4.7);
			\node[below,labeltext] at (0,-5.2) {mass in motion};
			
			% Thick vertical wall
			\draw[thick] (-0.4,-4.8) -- (-0.4,0);
			% Friction hatch marks (diagonal lines)
			\foreach \y in {-4.8,-4.6,...,-0.2}
			\draw[thick] (-0.4,\y) -- (-0.5,\y+0.2);
			
			
			% Optional label
			\node[rotate=90,left] at (-0.8,-1) {\text{wall/friction}};
			
%			% Example mass block next to wall
%			\draw[fill=gray!30] (0.5,1) rectangle (1.5,2);
			
			% Optional arrow to indicate external force
			\draw[->,blue,very thick] (0.35,-4.1) -- (0.35,-3.7) node[right] {};
			\draw[->,blue,very thick] (0.35,-4.7) -- (0.35,-5.1) node[right] {$F(t)$};
		\end{scope}
			%--------------------------------
		% (d) Coordinate system
		%--------------------------------
		\begin{scope}[xshift=3.5cm]
				\node[below] at (3.1,-5.5) {(d)};
			
			% Bracket for total extension
			\draw[->,very thick] (3,0.25) -- (3,-5); % the coordinate axis line
		
	
			\node[right,labeltext] at (3.1,-1.9) {$y<0$};
			
			\filldraw[black] (3,-3.1) circle (2pt); 
			\node[right,labeltext] at (3.1,-3.1) {$y=0$\;(origin)};
			\node[right,labeltext] at (3.1,-4.5) {$y>0$};
			% Dashed lines
			%\draw[dashedline] (-0.8,-2.2) -- (0.8,-2.2);
%			\draw[dashedline] (-0.8,-4.3) -- (0.8,-4.3);
%			\draw[dashedline] (-0.8,-4.7) -- (0.8,-4.7);
			\node[below,labeltext] at (3.1,-5.2) {the coordinate axis};
			%\node[below,labeltext] at (3.1,-5.1)  {positive downward};
	
		\end{scope}

	\end{tikzpicture}
\caption{}
\label{fig:mass-spring}
\end{figure}


Let \(\ell\) denote the natural (unstretched) length of the spring, and let \(s\) be the amount by which the spring stretches when the mass $m$ is attached and allowed to hang in equilibrium under gravity. We assume that the motion of the mass is restricted to the vertical line  through the point of the equilibrium  of the mass. To take this vertical line (see Figure~\ref{fig:mass-spring} (d)) as the coordinate axis, we take the equilibrium point as the origin and assign the vertically downward direction as the positive direction.



Denote by \(y(t)\) the displacement of the mass at time \(t\) from its equilibrium position (the origin). 
In addition to the restoring force \(F_{\text{s}}(t)\) of the spring and gravitational force $mg$, assume that the mass is acted upon by a damping force \(F_{\text{d}}(t)\) as well as an external force \(F(t).\) See  Figure~\ref{fig:mass-spring} (c) for wall friction as an example of a damping force.  Thus, we have
\[
\text{the net force on the mass} = F_{\text{s}}(t) +mg+  F_{\text{d}}(t) + F(t).
\]
According to Newton’s second law, we then have
\begin{equation}\label{eq:mass-spring1}
m y''(t) = F_{\text{s}}(t) + mg+ F_{\text{d}}(t) + F(t).
\end{equation}


 \noindent$\bullet$ How do we model the spring force $F_{\text{s}}(t)?$  
 
 Since a spring that is stretched or compressed tends to return to its natural length, the spring force  wants to pull the mass toward its equilibrium position. For an ideal spring—one made of a homogeneous material with uniform coiling—Robert Hooke, a 17th-century English physicist, established that the restoring force required to stretch or compress the spring (without damaging it) is directly proportional to the resulting displacement.\footnote[1]{ For a spring made up of a heterogeneous material and with nonuniform coiling, the restoring force may be modeled by a nonlinear function of  the displacement. For example, \( F_{\text{s}}(t) = - k\,\abs{y(t)}^{p-2}y(t)\) with $p$ a fixed number in \((1, \infty)\) and $k>0$  a fixed real number.} 
This means that 
\begin{equation}\label{eq:mass-spring2}
	 F_{\text{s}}(t) = - k\,(s+y(t)),
	 \end{equation}
where $k$ is a fixed positive constant, called the spring constant which measures the stiffness of the spring. The negative $(-)$ means that the restoring force is directed toward the origin. 

One may ask:  {\textit{how do we practically calculate $k$ of a spring?}} The simplest way to calculate $k$ is to simply suspend the spring from a support with the mass attached to its other end and observe the stretch $s$ the spring experiences for its equilibrium under gravity. Since the mass is at rest, the net force acting on the mass is zero, that is, 
\[\text{the net force on the mass}=0 = - ks +mg,\] which yields
\begin{equation}\label{eq:mass-spring3}
	mg = ks,
	\end{equation} and so
\[k = \frac{mg}{s}.\] 

 \noindent$\bullet$ How do we model $F_{\text{d}}(t)?$
 
It is intuitive that the damping force (or drag) increases as the speed of the mass increases. Hence, the simplest way to model the damping force is to assume that it is directly proportional to the velocity.\footnote[2]{For a medium that resists the motion of the mass  depending on a power of the speed,  the damping force may be modeled by a nonlinear function of  the velocity. For example, \( F_{\text{s}}(t) = - b\,\abs{y'(t)}^{q-2}y'(t)\) with $q$ a fixed number in \((1, \infty)\) and $b>0$ a fixed real number.}
This means that 
\begin{equation}\label{eq:mass-spring4}
F_{\text{d}}(t)=  - b\, y'(t), \end{equation}
where 
$b$ is the constant of proportionality, called the damping coefficient of the medium offering the resistance to the motion of the mass.

Using \eqref{eq:mass-spring2} and \eqref{eq:mass-spring4} in \eqref{eq:mass-spring1}, we obtain

\[
	m y''(t) = -ks -k y(t) + mg- b y'(t) + F(t),
\] which, in view of \eqref{eq:mass-spring3}, yields
\begin{equation}\label{eq:mass-spring5}
	my''(t) + by'(t) + k y(t) = F(t).
\end{equation}
This is a second order linear differential equation with constant coefficient, and it can be solved using the methods from the previous sections. For example, the method of undetermined coefficients or variation of parameters can be used.

We now examine several interesting and special cases of \eqref{eq:mass-spring5} before  discussing the general form. 
 

\subsubsection{I. Free Undamped Motion ($b=0,\;F(t) =0$)}	
  When $b= 0$ (no damping) and $F(t) = 0$ (no external forces), the mass-spring system is said to have  a free undamped motion and the system  is called a \textit{simple harmonic oscillator}.   The resulting differential equation for the displacement of the mass from its equilibrium position is 
	\[	my''(t)  + k y(t) = 0,\] which, with $w^2 = k/m$,  becomes
	\begin{equation}\label{eq:mass-spring6}
		y''(t) + \omega^2 y(t) = 0,
		\end{equation}
	also referred to as a simple harmonic oscillator.
	Then the general solution of the simple harmonic oscillator is 
	\begin{equation}\label{eq:mass-spring7}
		y(t) = c_1 \cos(\omega t) + c_2 \sin(\omega t),
		\end{equation} where $c_1$ and $c_2$ are arbitrary constants  which can be determined when the initial conditions $y(t_0) = y_0,  y'(t_0) = y_1$ are  provided.
	
		The displacement $y(t)$ described by \eqref{eq:mass-spring7} can  interpreted most effectively when it is written in the form of a single sine or cosine function. To express it in the form of a single sine function,
	we seek to find two constants $A$ and $\phi$ so that 
	\begin{equation}\label{eq:mass-spring8}
	y(t) = A \sin(\omega t -\phi).
	\end{equation}
	Then 
		\begin{equation}\label{eq:mass-spring9}
	y(t) = A \sin(\omega t) \cos(\phi) - A\cos(\omega t) \sin (\phi).\
\end{equation}
	Since the terms in \eqref{eq:mass-spring7} and \eqref{eq:mass-spring9} must match, we obtain
	\[\begin{cases}
		A\sin(\phi) = -c_1,\\
		A\cos(\phi) = c_2.
	\end{cases}
	\]
	Squaring and adding these equations gives $A^2 = c_1^2 + c_2^2.$  We take 
	$A =\sqrt{c_1^2 + c_2^2}.$ It is possible that $A =0$, in which case, we have $y(t) = 0$ for all $t$, the trivial solution, and this means that the mass is at rest in the equilibrium position for all $t.$ Therefore the only interesting case is when $A>0.$  
	If $c_2 \ne 0,$ then \[\arctan(\frac{-c_1}{c_2}) \text{ lies in }\left(-\frac{\pi}{2}, \frac{\pi}{2}\right). \]
 We have
	\[\phi = \begin{cases}
		\arctan(\frac{-c_1}{c_2}) &\text{if } c_2>0,\\[1em]
		\arctan(\frac{-c_1}{c_2})+\pi&\text{if } c_2<0,\\
		\frac{\pi}{2}&\text{if }c_1<0 \text{ and } c_2=0,\\
		-\frac{\pi}{2}&\text{if } c_1>0 \text{ and } c_2=0.
		\end{cases}
			\]
Also, note that \eqref{eq:mass-spring8} can be written as
\begin{equation}
	y(t) = A \cos(\omega t -\phi -\frac{\pi}{2}).
\end{equation}		
It is now clear from \eqref{eq:mass-spring8} that the mass oscillates about the origin with the amplitude  $A,$ which is the maximum displacement from the equilibrium position. The  time period for one complete oscillation is $ T= 2\pi / \omega,$ and therefore the number $\omega $ is  the circular frequency of the oscillations, that is, the number of cycles per unit $2\pi$ units of time.
	

\begin{example}
A 2-pound mass stretches a spring 6 inches. At time \( t = 0 \), the mass is released from a position 8 inches below equilibrium with an initial upward velocity of \( \tfrac{4}{3} \) ft/s. 
\begin{enumerate}[label =(\roman*), noitemsep]
	\item\label{item:1} Find the equation of motion in both the forms \eqref{eq:mass-spring7} and \eqref{eq:mass-spring8}.
	\item\label{item:2} Graph the velocity versus the displacement.
\end{enumerate}


\end{example}		

\begin{solution} 
	\ref{item:1} Let $y(t)$ denote in feet the displacement of the mass from the equilibrium position measured positively in the vertically downward direction. We have 
\(mg = 2~\text{lb}, 
	\text{spring stretch}, s = 6~\text{in} = 0.5~\text{ft}.
\)
We have
\(
m = \frac{mg}{g} = \frac{2}{32} = \frac{1}{16}~\text{slug},\) and 
the spring constant
\(k = \frac{mg}{s} = \frac{2}{0.5} = 4~\text{lb/ft}.
\)
The differential equation of motion is
\[
m y'' + k y = 0,\] which becomes
\[\frac{1}{16} y'' + 4y = 0,\] and therefore
\[y'' + 64y = 0.\]

with the initial condition $y(0) =  -\tfrac{2}{3}, 
y'(0) = \tfrac{4}{3}.$
The general solution of the differential equation is
\[y(t) = c_1 \cos(8t) + c_2 \sin(8t).
\]

Applying the initial conditions gives
\[
y(0)=  -\frac{2}{3} = c_1.\]
Also, since
\[y'(t) = -8c_1 \sin(8t) + 8c_2 \cos(8t)\]  and
\(y'(0) =\frac{4}{3},\)
we obtain 
\(8c_2 = \frac{4}{3},\) that is, 
\(
 c_2 = \frac{1}{6}.
\)
Hence the equation of the motion is
\[
	y(t) = -\frac{2}{3} \cos(8t) + \frac{1}{6} \sin(8t),
\]
which is of the form \eqref{eq:mass-spring7}.
Using the method discussed for \eqref{eq:mass-spring8} to find
 the equation of the motion in the amplitude-phase, we compute
 \[A= \sqrt{c_1^2+c_2^2}=\sqrt{\frac{4}{9}+\frac{1}{36}}= \frac{\sqrt{17}}{6} \]
  and
  \[ \phi = \arctan(\frac{-c_1}{c_2}) = \arctan(\frac{2/3}{1/6}) =\arctan(4)\approx 1.3258~\text{rad},\] and therefore
\[
y(t) = \frac{\sqrt{17}}{6} 
\sin\!\left(8t - \varphi\right)
\approx  \frac{\sqrt{17}}{6} 
\sin\!\left(8t - 1.3258\right).
\] Its graph is shown in Figure~\ref{graph:simple-harmonic-oscillator}.


%\textcolor{red}{open the figure below}
\begin{figure}[h]
	\centering
	\begin{tikzpicture}[scale=.8]
		\begin{axis}[
			width=10cm,
			height=7cm,
			axis lines=middle,
			xlabel={$t$},
			ylabel={$y(t)$},
			ylabel style={at={(axis description cs:0, -0.02)}, anchor =north},
			samples=300,
			domain=0:1,
			xtick={0,0.1,0.2,...,1.0},
			ytick={-1,-0.5,0,0.5,1},
			smooth,
			thick,
			y dir=reverse, 
			grid=both,
			minor tick num=1,
			]
			\addplot[blue, thick] {sqrt(17)/6 * sin(deg(8*x - 1.3258))};
			\addplot[black, thick, dashed] {sqrt(17)/6 * sin(deg(8*x))};
			\node[anchor=west, blue] at (axis cs:0.37,.5) 
			{$y(t)=\dfrac{\sqrt{17}}{6}\sin(8t - \varphi)$};
				\node[anchor=west, black] at (axis cs:0.15,-0.3) 
			{$y(t)=\dfrac{\sqrt{17}}{6}\sin(8t)$};
		\end{axis}
	\end{tikzpicture}\\[1em]
	\caption{The graph of \(y(t) 
		\approx  \frac{\sqrt{17}}{6} 
		\sin\!\left(8t - 1.3258\right)\)}
	\label{graph:simple-harmonic-oscillator}
	\end{figure}
	
 \ref{item:2} We find \[y'(t) =  8\frac{\sqrt{17}}{6}
 \cos\!\left(8t - \varphi\right).\] Then
 \[\left(\frac{y(t)}{\frac{\sqrt{17}}{6} }\right)^2+\left(\frac{y'(t)}{\frac{8\sqrt{17}}{6}}\right)^2 = 1\] for all $t$. This takes the form 
 \[272 (y(t))^2 + 17(y'(t))^2= 272 \] for all $t$ and is an ellipse in the $y'y$-plane as shown in Figure~\ref{fig:phase-space-orbit}.  The graph is called the orbit of the simple harmonic oscillator and visualizes the oscillatory motion.
 
	

%\textcolor{red}{open the figure below}
\begin{figure}[h]
\centering	
			\begin{tikzpicture}[scale=.8]
				\begin{axis}[
					width=12cm,
					height=8cm,
					axis lines=middle,
					xlabel={$y'(t)$},
					ylabel style={rotate=-90},
					%ylabel={$y(t)$},
					%ylabel style={at={(axis description cs:.0, 1)}, anchor =north},
					grid=none,
					minor tick num=1,
					samples=400,
					domain=0:2*pi,
					axis equal image,  % keeps scaling equal after manual adjustment
					thick,
					y dir=reverse,     % make y(t) axis point downward
					x dir=normal,
					xscale=1,
					yscale=3,
					xmin=-6.5, xmax=6.5,               % explicit limits prevent huge dimensions
					ymin=-1.1, ymax=1.1,     % scale horizontal axis by 1/8 (since y' = 8A cos)
					]
					
					% Parameters: A = sqrt(17)/6, omega = 8, phi = 1.3258
					
				
					\addplot[
					blue,
					thick,
					smooth,
					postaction={decorate, decoration={markings,
							mark=between positions 0.05 and 0.95 step 0.1 with {\arrow{stealth};}
					}}
					] (
					{8*sqrt(17)/6 * cos(deg(8*x - 1.3258))},   % x = y'(t)
					{sqrt(17)/6 * sin(deg(8*x - 1.3258))}      % y = y(t)
					);
					
				% Initial point (inside axis limits)
				\addplot[only marks, mark=*, mark options={fill=red, scale=.7}] coordinates {(4/3,-2/3)};
				\node[above right, red] at (axis cs:.1,-0.7) {$(y'(0),y(0)) = (4/3, -2/3)$};
				
				% Key points (placed within axis limits)
				\addplot[only marks, mark=*, mark options={fill=black, scale=.7}] coordinates {(0,2/3) (0,-2/3) (16/3+.19,0) (-16/3-.19,0)};
				\node[right]       at (axis cs: 0, 1)  {{$y(t)$}};

		
				\end{axis}
			\end{tikzpicture}\\[1em]
			\caption{The orbit of  $y''+64y = 0, y'(0)= -2/3, y'(0) = 4/3.$}
			\label{fig:phase-space-orbit}
\end{figure}
\end{solution}		

	
\subsubsection{II. Free Damped Motion
	($b>0, \;F(t) = 0)$} 
	When $b> 0$ (with damping) and $F(t) = 0$ (no external forces), the mass-spring system is said to have a \textit{free damped motion}. The resulting differential equation for the motion of the mass is 
	\[	my''(t)+by'(t) + k y(t) = 0,\] which, with $2\beta = \frac{b}{m}$ and $w^2 = \frac{k}{m}$,  becomes
	\begin{equation}\label{eq:mass-spring10}
		y'' + 2\beta y'+ \omega^2 y = 0,
	\end{equation}
	and 
the differential equation \eqref{eq:mass-spring10} is known as a \textit{free damped spring–mass system.} Its general solution depends on the relative strengths of the damping and spring forces. When the spring force dominates, the motion is oscillatory with amplitude decaying to zero  as \( t \to \infty \) and when the damping force dominates, the motion is non-oscillatory. There exists a threshold value of the damping coefficient—called the critical damping—above which no oscillations occurs, and below which the system exhibits damped oscillations whose amplitude decays to zero as \( t \to \infty \). To analyze this, we start with the auxiliary equation for \eqref{eq:mass-spring10}; namely,
\[r^2 + 2\beta r + \omega ^2 =0,\] with $y(t)=e^{rt}$ is a trial solution. It then follows that 
\begin{equation}\label{eq:mass-spring11}
	r = \frac{-2\beta\pm\sqrt{4\beta^2 - 4 \omega^2}}{2} = -\beta\pm\sqrt{\beta^2 -  \omega^2}. \end{equation}
	
There are three significantly different cases of a free damped motion. 
\begin{enumerate}[leftmargin=1.5em,label = (\arabic*)]
	\item\label{item:damping2}\textbf{\textit{Underdamped Motion}:}  When $\beta <\omega,$  \eqref{eq:mass-spring11} yields  \[r = - \beta \pm i \sqrt{\omega^2- \beta^2},\] so that the general solution of \eqref{eq:mass-spring10} is
	\[y(t) =  e^{-\beta t}\left(c_1\cos(\sqrt{\omega^2- \beta^2}\; t)+ c_2 \sin(\sqrt{\omega^2- \beta^2}\; t)\right),\] where $c_1$ and $c_2$ are arbitrary constants are determined when the initial conditions $y(t_0) = y_0,  y'(t_0) = y_1$ are provided.  
	The displacement $y(t)$ is periodic with the period $ 2\pi/\sqrt{\omega^2- \beta^2}.$  Moreover, 
	since $\beta >0,$ we observe that  \[\lim\limits_{t\to\infty} e^{-\beta t}\cos(\sqrt{\omega^2- \beta^2}\; t)=0 \quad \mbox{and} \quad \lim\limits_{t\to\infty}e^{-\beta t}\sin(\sqrt{\omega^2- \beta^2}\; t)=0.\]
	It then follows that 
	\[\lim\limits_{t\to\infty} y(t)=0.\] 
	
	
	With regard to the velocity of the mass, we find
		\[\begin{split}
	y'(t) &= -\beta e^{-\beta t}\left(c_1\cos(\sqrt{\omega^2- \beta^2}\; t)+ c_2 \sin(\sqrt{\omega^2- \beta^2}\; t)\right)\\
	& + \sqrt{\omega^2- \beta^2}e^{-\beta t}\left(-c_1\sin (\sqrt{\omega^2- \beta^2}\; t)+ c_2 \sin(\sqrt{\omega^2- \beta^2}\; t)\right).
	\end{split}
	\]
	Again, we observe that
	\[\lim\limits_{t\to\infty} y'(t)=0.\]  Hence, the mass approaches the equilibrium position while oscillating with the circular frequency \(\sqrt{\omega^2- \beta^2}\) and the velocity approaches zero as $t$ goes to $\infty.$
	
	Since $\beta = b/(2m)$ and $\omega^2 = k/m,$ $\beta < \omega$ is equivalent to $b< 2\sqrt{mk}.$ 
	We conclude that when the damping force is too weak to counteract a strong spring force, the mass undergoes damped oscillations: the amplitude  gradually decreases to zero. In this case, the mass-spring system is said to be \textbf{\textit{underdamped}}.\index{spring-mass systems!underdamped}
	
	
	\item\label{item:damping3}\textbf{\textit{Overdamped Motion}:}  
	When $\beta >\omega,$  \eqref{eq:mass-spring11} yields  $r = - \beta \pm  \sqrt{\beta^2- \omega^2},$ so that the general solution of \eqref{eq:mass-spring10} is
	\[y(t) =  c_1e^{(-\beta + \sqrt{\beta^2- \omega^2})t}+ c_2e^{(-\beta - \sqrt{\beta^2- \omega^2})t},\] where $c_1$ and $c_2$ are arbitrary constants are determined when the initial conditions $y(t_0) = y_0,  y'(t_0) = y_1$ are provided.  
	Therefore $y(t)$ is not periodic   Moreover, 
	since $\beta >0$ and \(0<\sqrt{\beta^2- \omega^2} < \beta\),  it follows that both 
	\(-\beta + \sqrt{\beta^2- \omega^2}\) and 	\(-\beta - \sqrt{\beta^2- \omega^2}\) are negative.
	Consequently,
	  we have  \[\lim\limits_{t\to\infty} e^{(-\beta + \sqrt{\beta^2- \omega^2})t}=0 \quad \mbox{and} \quad \lim\limits_{t\to\infty}e^{(-\beta - \sqrt{\beta^2- \omega^2})t}=0.\]
	It then follows that \[\lim\limits_{t\to\infty} y(t)=0.\]  We can show that the mass passes through the equilibrium position at most once by setting $y(t) = 0$ and finding the time $t$ at which this happens.

Regarding the velocity of the mass, we find
	\[\begin{split}
		y'(t) &=  c_1 (-\beta + \sqrt{\beta^2- \omega^2})e^{(-\beta + \sqrt{\beta^2- \omega^2})t}\\
		&+ c_2(-\beta - \sqrt{\beta^2- \omega^2}) e^{(-\beta - \sqrt{\beta^2- \omega^2})t},
	\end{split}
	\]

	Again, we observe that
	\[\lim\limits_{t\to\infty} y'(t)=0.\]  
   We can also show that the mass reaches  an extreme position  at most once by setting $y'(t) = 0$ and finding the time $t$ at which this happens.\\
   
	Since $\beta = b/(2m)$ and $\omega^2 = k/m,$ $\beta > \omega$ is equivalent to $b> 2\sqrt{mk}.$ 
	We conclude that when the spring force is too weak to counteract a strong damping damping force, the mass  approaches the equilibrium position with no oscillation and its velocity approaches zero, by changing its sign at most once, as $t$ goes to $\infty$.  In this case, the mass-spring system is said to be \textbf{\textit{overdamped.}}\index{spring-mass systems!overdamped}
	
	
\item\label{item:damping1} \textbf{\textit{Critically Damped Motion}:} 
	When $\beta = \omega$,
	 \eqref{eq:mass-spring11} yields  $r = - \beta, -\beta,$ so that the general solution of \eqref{eq:mass-spring10} is
	\[y(t) = c_1 e^{-\beta t}+ c_2 te^{-\beta t} = (c_1+ c_2t) e^{-\beta t},\] where $c_1$ and $c_2$ are arbitrary constants  which can be determined when the initial conditions $y(t_0) = y_0,  y'(t_0) = y_1$ are known.  The velocity of the mass is given by 
	\[y'(t) = -\beta (c_1 + tc_2) e^{-\beta t} + c_2 e^{-\beta t} = (c_2 - \beta c_1- \beta c_2 t) e^{-\beta t}.\]  Since $\beta >0,$ we find $e^{-\beta t}\to 0$ and $t e^{-\beta t}\to 0$ as $t\to\infty.$  These imply that 
	\[\lim\limits_{t\to\infty} y(t) = 0 \qquad \mbox{ and }\qquad \lim\limits_{t\to\infty} y'(t) = 0.\]
As note earlier in the case of an overdamped motion, the mass in this case can also pass thorough the equilibrium at most once and can reach an extreme position at most once while its velocity changes its sign.  As $t$ goes to $\infty,$ the mass approaches its equilibrium position and its velocity also tends to zero.

\medskip Since $\beta = b/(2m)$ and $\omega^2 = k/m,$ $\beta = \omega$ is equivalent to  $b= 2\sqrt{mk}.$ It follows
from the case of underdamped motion  that when the damping coefficient $b$ is even slightly less than  $2\sqrt{mk},$ the mass undergoes  an oscillatory motion. The spring-mass system with the smallest damping  coefficient  $b$ that maintains a nonoscillatory motion is  to be \textbf{\textit{critically damped}}.\index{spring-mass systems!critically damped}
\end{enumerate}	

\subsubsection{III. Undamped and Forced Motion ($b=0, \; F(t) \ne 0$)}
When $b=0$ (no damping) and $F(t) \ne 0$ (with external force), the spring-mass system is governed by the differential equation
\begin{equation*}
	my''(t)  + ky(t) = F(t).
\end{equation*}
 which, with $w^2 = k/m$ and $f(t) = F(t)/m,$  becomes
 \begin{equation}\label{eq:mass-spring-undamped-forced-1}
 	y''(t)  + \omega^2y(t) = f(t).
 \end{equation}
 
The nonhomogeneous equation \eqref{eq:mass-spring-undamped-forced-1} can be solved by using the methods of previous sections. The general solution  is of the form
\begin{equation}\label{eq:mass-spring13}
	y(t) = y_h(t) + y_p(t) = c_1 \cos(\omega t) + c_2 \sin(\omega t) +y_p(t),
\end{equation}
where $c_1$ and $c_2$ are arbitrary constants and $y_p(t)$ is a particular solution of \eqref{eq:mass-spring-undamped-forced-1}. Recall that 
 \[y_h(t) = c_1 \cos(\omega t) + c_2 \sin(\omega t) \] is the general solution of the  simple harmonic oscillator \[ y''(t)  + \omega^2 y(t)=0,\]  associated with  \eqref{eq:mass-spring-undamped-forced-1}. 

\begin{example}[\(b=0, F(t)= F_0 \sin(\gamma t)\)]\label{eg:undamped-forced-system}
	A \(10\)-kilogram mass attached to the  end of a vertically hanging spring causes the spring to stretch \(39.2\) centimeters. At time \( t = 0 \), the mass–spring system is set into motion by an external force \( F(t) = 180\sin(10t) \), where \( F(t) \) is measured in newtons and is positive in the downward direction, and time is measured in seconds. Determine the equation of motion and estimate the maximum excursion of the mass from its equilibrium position.  
\end{example}

 \begin{solution}
 We have
 \(
 m = 10~\text{kg},  
 s = 39.2~\text{cm} = 0.392~\text{m},  
 F(t) = 180\sin(10t)~\text{N}.
 \)
 For the mass in equilibrium, we have \( mg = ks \), and so
 \[
 k = \frac{mg}{s} = \frac{10 \times 9.8}{0.392} = 250~\text{N/m}.
 \]
 The natural  frequency of the system is
 \[
 \omega = \sqrt{\frac{k}{m}} = 5 \text{ cycles in $2\pi$ seconds} = 5 \text{ rad/s}
 \]
 The differential equation of motion is therefore
 \[
 10y'' + 250y' = 180\sin(10t),
 \]
 or equivalently,
 \begin{equation}\label{eq:mass-spring-150}
 	y'' + 25y = 18\sin(5t). 
 \end{equation}
 The general solution of \( y'' + 25y = 0 \) is
 \[
 y_h(t) = c_1\cos(5t) + c_2\sin(5t),
 \]
 where $c_1$ and $c_2$ are arbitrary constants.
 We find a particular solution $y_p(t)$ of \eqref{eq:mass-spring-150} by variation of parameters.
 Let
 \[
 y_p(t) = A\cos(10t) + B\sin(10t)
 \]
be a particular solution of \ref{eq:mass-spring-150}, where $A$ and $B$ are to be determined.
We find that $y''_p(t) = - 100 y_p(t).$ Substituting $y_p$  for $y$ and $y''_p$ for $y''$  into \eqref{eq:mass-spring-15}   yields
\[- 100 y_p(t)+25y_p(t) = 18 \sin(10t),\]
that is,
\[-75A \cos(5t) - 75 B\sin(5t) =18 \sin(10t).\]
Equating the coefficients of the like terms, we have
\(-75 A =0\) and \(-75 B = 18\), that is, $A=0$ and $B = -6/25.$
Thus, we have 
\[ y_p(t) = -\frac{6}{25} \sin(5t).\]
 The general solution of \ref{eq:mass-spring-15} is 
 \[
 y(t) = c_1\cos(5t) + c_2\sin(5t) - \frac{6}{25}\sin(10t),
 \]
 where $c_1$ and $c_2$ are arbitrary constants. The initial conditions for $y(t)$ are  $y(0) = 0, y'(0) = 0.$
 Substituting these gives \( c_1 = 0 \) and \( 5c_2 - \frac{12}{5} = 0\), i.e.,  \(c_1=0\) and \(c_2 = \frac{12}{25} \)
 Thus, we have
 \[
 y(t) = \frac{12}{25}\sin(5t) - \frac{6}{25} \sin(10t).
 \] The graph of the solution is shown in Figure~\ref{fig:undamped-forced-system}.
 Using the critical numbers and second derivative test, we can verify that the maximum occurs at $t = \frac{2\pi}{15}.$ The maximum value is 
 \[y\left(\frac{2\pi}{15}\right) = \frac{9\sqrt{3}}{25} \approx 0.6235.\qedhere\]
 
 %\textcolor{red}{open the figure below}
\begin{figure}[H]
	\centering
	\includegraphics[width=0.5\linewidth]{chap03/mathematica/undamped-forced-system}
	\caption{The graph of \(y(t) = \frac{12}{25} \sin(5t)-\frac{6}{25}\sin(10t)\)}
	\label{fig:undamped-forced-system}
\end{figure}
 \end{solution}
In Example~\ref{eg:undamped-forced-system},  the frequency of the external source is $\gamma =10$ which is not equal to  or near the natural frequency $\omega = 5.$  

\textbf{Resonance:}\index{resonance} An important aspect of analyzing undamped  forced spring-mass systems is understanding how large (in absolute values) the values of $y_p(t)$ can become  when $F(t)$ is  sinusoidal. If the amplitude of $y_p(t)$ grows without bound,  the system is said to be in \textbf{\textit{resonance}}. In practical applications,  resonance may be undesirable because it can lead to excessive energy buildup and potential system failure. Since the system \eqref{eq:mass-spring-undamped-forced-1} is undamped, the only way energy can dissipate from the system is by the nature of the source term $F(t).$ However, it is possible that the energy  buildup can happen in the system even when $F(t)$ is sign-changing. The following example demonstrates how resonance occurs.


 
\begin{example}[Resonance: $b=0, F(t)= F_0 \sin(\omega t)$]\label{eg:resonance-mass-spring-1}
With the external force in Example~\ref{eg:undamped-forced-system} replaced with  \( F(t) = 180\sin(5t)\),  determine the equation of motion, analyze the amplitude of the resulting motion, and verify that the system exhibits resonance.
\end{example}
\begin{solution} Similarly to the solution in Example~\ref{eg:undamped-forced-system}, the initial value problem for the current example is
	\begin{equation} \label{eq:mass-spring-15}
		y'' + 25y = 18\sin(5t), \quad y(0) =0,\; y'(0) = 0.
	\end{equation}
	Recall from Example~\ref{eg:undamped-forced-system} that the natural  frequency of the system is
	\(
	\omega  = 10.
	\)
	We note here that the frequency of the external force $\gamma =10$ matches the natural frequency $\omega = 10.$
	The general solution of \( y'' + 25y = 0 \) is
	\[
	y_h(t) = c_1\cos(5t) + c_2\sin(5t),
	\]
	where $c_1$ and $c_2$ are arbitrary constants.
	We find a particular solution $y_p(t)$ of \eqref{eq:mass-spring-15} by variation of parameters.
	Let
	\[
	y_p(t) = u_1(t)\cos(5t) + u_2(t)\sin(5t),
	\]
	where \( u_1'(t) \) and \( u_2'(t) \) satisfy
	\[
	\begin{cases}
		u_1'(t)\cos(5t) + u_2'(t)\sin(5t) = 0, \\[4pt]
		-5\,u_1'(t)\sin(5t) + 5u_2'(t)\cos(5t) = 18\sin(5t).
	\end{cases}
	\]
	Solving this system gives
	\[
	u_1'(t) = -\frac{18}{5}\sin^2(5t) =-\frac{9}{5}(1-\cos(10t)),\] and  
	\[u_2'(t) = \frac{18}{5}\sin(5t)\cos(5t) =\frac{9}{5}\sin(10t).
	\]
	Integrating yields
	\[
	u_1(t) = -\frac{9}{5}t + \frac{9}{50} \sin(10t), \qquad 
	u_2(t) = -\frac{9}{50} \cos(10t).
	\]
	Hence,
	\[
	\begin{aligned}
		y_p(t) &= u_1\cos(5t) + u_2\sin(5t) \\[2pt]
		&= -\frac{9}{5}t \cos(5t)  + \frac{9}{50}\sin(10t) \cos(5t) 
		-\frac{9}{50}\cos(10t)\sin(5t)\\[2pt]
		&=-\frac{9}{5}t \cos(5t)  + \frac{9}{50}\sin(5t). \\[2pt]
	\end{aligned}
	\]
%	The dominant (secular) term responsible for resonance is
%	\[
%	y_p^{(\text{secular})} = -\frac{9}{10} t\cos(10t).
%	\]
%	
%	---
The general solution  is 
	\[
	y(t) = c_1\cos(5t) + c_2\sin(5t) - \frac{9}{5}t\cos(10t),
	\]
	$c_1$ and $c_2$ are constants to be determined so that  $y(0) = 0, y'(0) = 0.$
	Substituting these gives \( c_1 = 0 \) and \( 5c_2 - \frac{9}{5} = 0\), i.e.,  \(c_2 = \frac{9}{25} \)
Thus, we have
	\[
	y(t) = -\frac{9}{5}t\cos(5t) + \frac{9}{25}\sin(5t).
	\]
	It is evident  that the amplitude $a(t)$ of \(y(t)\) is given by \[a(t)=\frac{ 9}{5}\sqrt{t^2+\frac{1}{25}}\] which shows  that $\abs{y(t)}$ grows without bound as $t\to\infty.$ Hence,  the system exhibits \textbf{resonance} in this case, where the forcing frequency coincides the natural frequency--illustrating the classical example of resonance in an undamped forced spring--mass system. The plots of the solution $y(t)$ and amplitude $a(t)$ are shown in Figure~\ref{fig:resonance-1}.
	
%	\textcolor{red}{open the figure below}
\begin{figure}[h]
	\centering
	\includegraphics[width=0.9\linewidth]{chap03/mathematica/resonance-1}
	\caption{}
	%The graphs of $y(t) = -\frac{9}{10}t\cos(10t) + \frac{9}{100}\sin(10t)$\\ and its amplitude $a(t)=\frac{ 9}{10}\sqrt{t^2+\frac{1}{100}}$. 
	\label{fig:resonance-1}
\end{figure}
\end{solution}




\subsubsection{IV. Damped and Forced Motion ($b>0,\; F(t) \ne 0$)}
 When $b>0$ and $F(t) \ne 0$, the spring-mass system  is governed by the differential equation \eqref{eq:mass-spring6}; namely,
\begin{equation}\label{eq:mass-spring16}
	my''(t) + by'(t) + k y(t) = F(t).
\end{equation}
The nonhomogeneous equation \eqref{eq:mass-spring16} can be solved by using the methods of previous sections. The general solution is of the form
\begin{equation}\label{eq:mass-spring17}
	y(t) = y_h(t) + y_p(t),
\end{equation}
where $y_h(t)$ is the general solution of the  homogeneous equation \[ my''(t) + by'(t) + k y(t)=0,\] which describes the free damped motion corresponding to $F(t) = 0$ and $y_p(t)$ is a particular solution of \eqref{eq:mass-spring16}.  We recall from all three cases (underdamped,  critically damped, and overdamped) of a free damped motion that
\begin{equation}\label{eq:mass-spring18}
	\lim\limits_{t\to\infty}y_h(t) =0,
\end{equation}
and therefore $y_h(t)$ is  a \textbf{\textit{transient term}} in \eqref{eq:mass-spring17} and it is either oscillatory or nonoscillatory according as $b<2\sqrt{mk}$ or $b \ge \sqrt{mk}.$ We conclude from \eqref{eq:mass-spring17} and \eqref{eq:mass-spring18}  that 
\[y(t)\approx  y_p(t)\quad \mbox{ for large } t.\]
The particular solution $y_p(t)$ is called the \textbf{\textit{steady-state}} solution of \eqref{eq:mass-spring16}. For short times,  $y_h(t)$ must be included in the solution  $y(t) = y_h(t) + y_p(t),$ which then becomes  the \textit{\textbf{transient solution.}}


%\textcolor{red}{Even for a non-sinusoidal $F(t)$ in an overdamped system, we like to avoid situations when $y_p(t)$ assumes unexpectedly large values, as illustrated in an example to follow.}


\begin{example}[From Resonance to a Simple Harmonic Oscillator]\label{eg:resonance-mass-spring-4}
	For the resonant system  in Example~\ref{eg:resonance-mass-spring-1}, determine the  damping coefficient that should be introduced at \(t = 3\pi \) so that the system transitions immediately into  a simple harmonic oscillator with the amplitude $27\pi/5.$ 
\end{example}
\begin{solution}
	We recall that the solution of the resonant system in Example~\ref{eg:resonance-mass-spring-1}
		\begin{equation}\label{eq:resonance-mass-spring-2}
	\begin{cases}
		y'' +25 y = 18 \sin(5 t),\\
	y(0) =0, \quad y'(0) = 0,
	\end{cases}
	\end{equation}
  is
		\[
	y(t) = -\frac{9}{5}t\cos(5t) + \frac{9}{25}\sin(5t). 
	\]
	We observe that  $y(3\pi) = {27\pi}/{5}$ and $y'(3\pi)=0.$
	
	We now proceed to find the damping coefficient $b>0$ required to turn the resonant system ~\ref{eq:resonance-mass-spring-2}  into a simple harmonic oscillator starting at $t=3\pi$  with the  frequency of oscillation same as the natural frequency of resonant system and the amplitude $27\pi/5.$
We consider the initial value problem
	\begin{equation}\label{eq:resonance-mass-spring-6}
	\begin{cases}
		10y''+by' +250 y = 180 \sin(5t),\\
		y(3\pi)  ={27\pi}/{5}, \quad y'(3\pi)  = 0.
	\end{cases}
	\end{equation}
It is evident that for small $b>0$ the complementary function should be oscillatory. Suppose then that $b^2< 40000$ and put 
\[d = \frac{1}{20}\sqrt{10000-b^2}.\]
Then the general solution of the differential equation  in \eqref{eq:resonance-mass-spring-6}   can be written in the form
\[ y(t) =e^{-\frac{b}{20}(t-3\pi)} \Big(c_1 \cos\big(d(t-3\pi)\big)+c_2 \sin\big(d(t-3\pi)\big)\Big)-\frac{36}{b} \cos (5 t)\]	
where $c_1$ and $c_2$ are arbitrary constants.  The only way this represents the solution to a harmonic oscillator is when its transient term is zero which happens only when $c_1=c_2= 0$ because of the linear independence of $\cos\big(d (t-3\pi)\big)$ and $\sin\big(d (t-3\pi)\big).$ 
Since $y(3\pi) = 27\pi/5,$ we have
\[\frac{27\pi}{5}  = c_1 +\frac{36}{b}.\] In order that $c_1=0,$ we must have \[b= \frac{20}{3\pi}.\]
The condition $y'(3\pi) = 0$ ensures that $c_2=0$ because
\[0 =y'(3\pi)= -\frac{b}{20} c_1 +dc_2,\] which gives
\[c_2 = \frac{b}{20d} c_1 =0.\]
Thus, the damped  system introduced at $t  = 3\pi$ is 
\begin{equation}\label{eq:resonance-mass-spring-7}
	\begin{cases}
		10y''+\frac{20}{3\pi}y' +250 y = 180 \sin(5 t),\\
		y(3\pi)  ={27\pi}/{5}, \quad y'(3\pi)  = 0,
	\end{cases}
\end{equation}
with the solution 
\[z(t) = -\frac{27\pi}{5} \cos (5 t)\] for all $t\ge 3\pi.$ The problem \eqref{eq:resonance-mass-spring-7} is equivalent to the simple harmonic oscillator described by 
\begin{equation}\label{eq:resonance-mass-spring-8}
		\begin{cases}
			y''+25 y = 0,\\
			y(3\pi)  ={27\pi}/{5}, \quad y'(3\pi)  = 0.
		\end{cases}
	\end{equation}
This harmonic oscillator has  frequency same  as the natural frequency of the resonant system \eqref{eq:resonance-mass-spring-2} and  amplitude equal to $27\pi/5.$ The solution of the resonant system for $0\le t\le 3\pi$ and  its continuous transition to a damped forced system that exhibits the simple harmonic oscillator \eqref{eq:resonance-mass-spring-8}   for $t\ge 3\pi$ is 
\[y(t) = \begin{cases}
	-\frac{9}{5}t\cos(5t) + \frac{9}{25}\sin(5t) &\text{ for } 0
	\le t\le3\pi\\
	\frac{-27\pi }{5}  \cos (5 t) &\text{ for }  t\ge 3\pi.
\end{cases}
\]
The graph of this solution is shown in  Figure~\ref{fig:resonance-to-sho}.

%\textcolor{red}{open the figure below}
\begin{figure}[h]
	\centering
	\includegraphics[width=0.5\linewidth]{chap03/mathematica/resonance-to-sho}
	\caption{}
	\label{fig:resonance-to-sho}.
\end{figure}
\end{solution}

When \(b = 0\) and the natural frequency of the system matches that of an external sinusoidal force, as illustrated in Example~\ref{eg:resonance-mass-spring-1}, the system exhibits resonance. In contrast, we  recall that a free damped system  approaches its equilibrium as \(t \to \infty\). In most practical applications, the damping constant \(b\) is relatively small. Interestingly, when \(b > 0\) but  small, although the amplitude of the particular solution $y_p$ is not unbounded as $t\to \infty$,  the system displays a behavior approaching resonance because the damping is insufficient to suppress  large oscillations. Therefore $\gamma = \omega$ should be avoided even for small damping. The following example illustrates this phenomenon.


\begin{example}[$b\approx 0, F(t)= F_0 \sin(\omega t)$]\label{eg:resonance-mass-spring-5}
	In the resonant system discussed in Example~\ref{eg:resonance-mass-spring-1}, 
 suppose that the surrounding medium offers a damping force that is numerically equal to 0.01 times the instantaneous velocity. Find the equation of the motion and analyze the danger of having a low value of the damping coefficient.
\end{example}
\begin{solution}
	The initial value problem is
	\begin{equation}\label{eq:small-damping-1}
		\begin{cases}
			y''+0.001y' +25 y = 18 \sin(5 t),\\
			y(0)  =0, \quad y'(0)  = 0.
		\end{cases}
		\end{equation}
		The auxiliary equation for \(y''+0.001y' +25 y=0\) is
		\[r^2 +0.001 r + 25 =0 \] with solutions $r = -a\pm ib$, where 
		$a = 0.0005$ and $b = \frac{1}{2}\sqrt{100-10^{-6}}.$
Then the general solution of the homogeneous differential equation 
\[y_h(t) = e^{-at}\left(c_1 \cos(bt) + c_2 \sin(bt)\right),\] where $c_1$ and $c_2$ are arbitrary constants. Let $y_p(t) = A\cos(5t) + B\sin(5t)$ be  particular solution of the nonhomogeneous equation in \eqref{eq:small-damping-1}. Then
\[y'_p(t) = -5A\sin(5t) + 5B\cos(5t)\] and \[y''_p(t) = -25A\cos(5t) - 25B\sin(5t) = -25 y_p(t).\] Substituting $y_p$ for $y$ in the nonhomogeneous equation yields
\[-25 y_p(t)-0.005A \sin(5t)+0.005B\cos(5t) + 25y_p(t) =18 \sin(5 t), \] which gives $A = -3600$ and $B=0.$ Then $y_p(t) = -3600 \cos(5t)$, and therefore the general solution of the nonhomogeneous equation is
\[y(t) =  e^{-at}\left(c_1 \cos(bt) + c_2 \sin(bt)\right)- 3600 \cos(5t).\]
Since $y(0) = 0$, we have $0 = c_1 - 3600,$ i.e., $c_1 = 3600.$ Also, since $y'(0) = 0,$ we have 
$0 = -a c_1 +bc_2,$ which gives $c_2 = ac_1/b = 3600a/b.$ Thus, the solution to the initial value problem is 
\[y(t) = 3600 e^{-at}\left( \cos(bt) + \frac{a}{b} \sin(bt)\right)- 3600 \cos(5t).\] The graph of the solution is shown in Figure~\ref{fig:small-damping-0} over $[0, 20]$  shows that the system appears to have resonance-like behavior as in Figure~\ref{fig:resonance-1}. However,
it is evident that $y(t) \approx - 3600 \cos(5t)$ for large values of $t$. In particular, $y\left(7001\pi\right) \approx 3600.$

%\textcolor{red}{open the figure below}
\begin{figure}[h]
	\centering
	\includegraphics[width=0.5\linewidth]{chap03/mathematica/small-damping}
	\caption{The graph of \(y(t) =  3600 e^{-at}\left(\cos(bt) + \frac{3600a}{b} \sin(bt)\right)- 3600  \cos(5t).\) }
	\label{fig:small-damping-0}
\end{figure}
\end{solution}

In the following example, we examine an undamped forced system in which  the frequency of the external force is  slightly different from the natural frequency.


\begin{example}[$b=0, F(t)= F_0 \cos(\gamma t)$ with $\gamma \approx \omega$]\label{eg:resonance-mass-spring-6}
	In the setting of the resonant problem in Example~\ref{eg:resonance-mass-spring-1}, 
let us take the external force  $F(t) = 180 \cos(5t+2\epsilon t)$, where $\epsilon$ is very small.  Find the equation of the motion.
\end{example}

\begin{solution}
	The initial value problem  is
	\begin{equation}\label{eq:small-change-frequency-2}
		\begin{cases}
			y'' +25 y = 18 \cos(\gamma t),\\
			y(0)  =0, \quad y'(0)  = 0,
		\end{cases}
	\end{equation}
	where $\gamma =5+2\epsilon$ with $\epsilon$ a small nonzero number. We have thus considered the frequency of the external force close to the natural frequency. The general solution of $y''+25y =0$ is 
	\[y_h(t) = c_1 \cos(5t) + c_2 \sin(5t),\] where $c_1$ and $c_2$ are arbitrary constants.
	
	 Let $y_p(t) = A\cos(\gamma t) + B\sin(\gamma t)$ be a particular solution of the nonhomogeneous equation in \eqref{eq:small-damping-1}. Substituting $y_p$ for $y$ and $y''_p$ for $y''$ into the nonhomogeneous equation in \eqref{eq:small-change-frequency-2} gives
	\[(25-\gamma^2) A \cos(\gamma t) + (25-\gamma^2) B \sin(\gamma t) 
	= 18 \cos(\gamma t).\] This  gives
	\[ A = -\frac{18}{\gamma^2 -25} \quad\mbox{and}\quad B=0.\] Thus,
	\[y(t) = -\frac{18}{\gamma ^2-25}\cos (\gamma  t),\] and therefore the general solution of the nonhomogeneous equation is
	\[y_p(t)=c_1 \cos ( 5 t )+c_2 \sin (5 t )-\frac{18}{\gamma ^2-25} \cos (\gamma  t).\]
	Since \(y(0) = 0\) and $y'(0) = 0,$ we have
	\[0 = c_1 -\frac{18}{\gamma ^2-25} \quad\mbox{and}\quad 0=5c_2,\] that is,
	\[c_1=\frac{18}{\gamma ^2-25}\quad \mbox{and}\quad c_2=0.\]
	Thus, the solution of the initial value problem is 
	\[\begin{split}
		y(t) &= \frac{18}{\gamma ^2-25} (\cos (5 t)-\cos (\gamma t))\\
	&=-\frac{36}{(\gamma+5)(\gamma-5)} \sin\left(\frac12(\gamma-5)t\right)\; \sin\left(\frac12(\gamma+5)t\right)  \\
	&=-\frac{9}{(5+\epsilon)\epsilon} \sin(\epsilon t) \sin\left((5+\epsilon)t\right)\\
	&= A(t)\sin\left((5+\epsilon)t\right),
	\end{split}\]
	where \[A(t) = -\frac{9}{(5+\epsilon)\epsilon} \sin(\epsilon t).\] 
	We observe that 
	\[\abs{y(t)}\le \abs{A(t)}\] for all $t$. This means that the graph of $y(t)$ will be enveloped in the graphs of \(\pm A(t)\), bounding the amplitude of \(\sin\left((5+\epsilon)t\right)\) as shown in Figure~\ref{fig:beats} for $\varepsilon =0.5$. In music, when both waves are heard simultaneously, the  sound pattern is known as \textbf{\textit{beats}}\index{beats} and the sound system is said to behave  \textbf{\textit{near resonance}}\index{near resonance}.	

%\textcolor{red}{open the figure below}
\begin{figure}[h]
	\centering
	\includegraphics[width=.6\linewidth]{chap03/mathematica/beats}
	\caption{``beats''-- near resonance}
	\label{fig:beats}
\end{figure}

Using a graphing utility, the reader can plot \( y(t) \) for a smaller value of \(\varepsilon \) (e.g., \( \varepsilon = 0.02 \)) and for a larger value (e.g., \( \varepsilon = 5 \)) to observe how the motion changes when the frequency of the external force is close to the natural frequency and when it is far from it.


\end{solution}
In Example~\ref{eg:resonance-mass-spring-6}, we could have used $F(t) = 180 \sin(5t+2\epsilon t)$  with reference to Example~\ref{eg:resonance-mass-spring-1}.  The choice of the cosine function is only for  convenience.



%\begin{hardsubsec}
	\subsection{Electrical Circuits}\label{subsec:electric-circuits}
%\end{hardsubsec}


In this subsection, we first study  ordinary differential equations  as initial value problems  in electric circuits  that consist of a voltage source  of the electromotive force $e(t)$  driving electric charges  and producing a current $i(t)$ (e.g., a battery  for direct voltage or a generator for alternating voltage), a resistor (e.g., appliances) of resistance $R$, and an inductor of inductance $L$ in series as depicted in Figure~\ref{fig:LRC}.
% Set the inductor style globally

%\textcolor{red}{open the figure below}
\begin{figure}[h]
	\centering
	% First LRC Circuit
	\begin{subfigure}[b]{0.45\textwidth}
			\centering
			\begin{circuitikz}[scale=.5]
					%\ctikzset{bipoles/vsourceam/inner plus={\tiny $+$}}
					%\ctikzset{bipoles/vsourceam/inner minus={\tiny $-$}}
					\draw
					%(0,0) to[battery1, l=$V(t)$] ++(0,3)
					(0,0) to[battery2, invert,  v_<=$e(t)$,i>_=${}$] ++(0,4) %voltage can be replaced with V
					 to[cute inductor, l=$L$, i>_=${}$] (4,4)
					 to[american resistor, i_=${}$, l=$R$] (8,4)
					 to[capacitor, l_=$C$, i_=$i(t)$] (8,0) %capacitor can be replaced with C
					-- (0,0);
				\end{circuitikz}
			\subcaption{Direct voltage source}
		\end{subfigure}
	\hfill
	% Second LRC Circuit
	\begin{subfigure}[b]{0.45\textwidth}
			\centering
			\begin{circuitikz}[scale=.5]
					\draw
					(0,0) to[sV, invert,  l_=$e(t)$] ++(0,4)
						to[L, l=$L$] (4,4)
					to[R,  l=$R$] (8,4)
					to[C, l_=$C$] (8,0)
					-- (0,0);
				\end{circuitikz}
			\subcaption{Alternating voltage source}
		\end{subfigure}
	\caption{Series \(LRC\)  circuits}
	\label{fig:LRC}
\end{figure}
Let $q(t)$ denote the electric charge  on the capacitor at time $t$.
According to Kirchhoff's second law, \textit{the voltage impressed on a closed loop equals the sum of the voltage drops across the components in the loop.} The voltage drops across a resistor, inductor and capacitor are modeled as being proportional to \(i(t), i'(t), q(t),\) respectively.  These voltage drops are modeled as follows:
\begin{enumerate}[label=(\roman*),noitemsep]
	\item the voltage drop across a inductor \(= L \;i'(t),\)
	\item the voltage drop across a resistor \(= R \;i(t),\) 
	\item  the voltage drop across a capacitor \(= \displaystyle\frac{1}{C}\; q(t),\)
\end{enumerate}
where 
the  constants $L$, $R$,  and $C$ are called the inductance, resistance,  and capacitance, respectively, and these quantities are typically measured in henry (h), ohm ($\Omega$), and farad (f), respectively. The corresponding  units for  $q$,  $E$, and  $I$ are coulomb (C), volt (V), and ampere (A).

We recall that the current \(i(t)\) flowing through the capacitor is given by
\[i(t) = q'(t),\quad  \text{ so that }\quad  i'(t) = q''(t).\] and therefore Kirchhoff's law\index{Kirchhoff's law} yields
\begin{equation}\label{eq:LRC}
	Lq''(t)+R  q'(t)+ \frac{1}{C} q(t) = e(t), 
\end{equation} which is a second-order linear differential equation in $q(t)$ with constant coefficients.

For analogy, we can interchange the spring-mass system parameters in \[my''(t)+ by'(t) + k y(t) = F(t)\] (see \eqref{eq:mass-spring5})
and  series \(LRC\) circuit parameters in \eqref{eq:LRC} as follows:
\begin{center}
	\begin{tabular}{ccc}
		\hline
		Spring-Mass  &     & Series \(LRC\)   \\ \hline
		\(m\) &\(\longleftrightarrow\)       & $L$ \\
		\(b\)  &\(\longleftrightarrow\)         & $R$ \\
		\(k\)    &\(\longleftrightarrow\)      & $\displaystyle\frac{1}{C}$ \\[2ex]
		\(\omega =\sqrt{k/m}\)&\(\longleftrightarrow\)  &\(\omega =\sqrt{1/LC}\)  \\[2ex]
		\hline
	\end{tabular}
\end{center}
In addition, the  three significantly different cases of the spring-mass system apply to the \(LRC\) circuit  as follows:
\begin{center}
	\begin{tabular}{ccc}
		\hline
		Damping Type &  Spring-Mass     & Series \(LRC\)   \\ 
		\hline\\ [-1ex]
		Underdamped &\(b^2< 4mk\)       & \(\displaystyle R^2<\frac{4L}{C}\) \\[2ex]
		Critically damped&\(b^2= 4mk\)         & \(\displaystyle R^2=\frac{4L}{C}\) \\[2ex]
		Overdamped  &\(b^2> 4mk\)      & \(\displaystyle R^2>\frac{4L}{C}\) \\[2ex]
		\hline
	\end{tabular}
\end{center}

We now consider the parallel $LRC$  circuit shown in Figure \ref{fig:parallelLRC}, where a current source supplies the input current $i(t)$. The diagram also shows that the  current $i(t)$ splits into three branch currents $i_1(t)$, $i_2(t)$, and $i_3(t)$ flowing through a capacitor, resister and inductor, respectively. We develop a differential equation for the impressed voltage  $e(t)$ across the components produced from the source current \(i(t).\)

%\textcolor{red}{open the figure below}
\begin{figure}[h]
	\begin{circuitikz}[american]
		% Current source from bottom to top
		\draw
		(0,0) to[I={$i(t)$}] (0,4)
		to[short] (2.5,4); % top horizontal wire
		\node[circle, fill=white, inner sep=1pt, label=above left: Current source] at (-0.4,1.75) {};
		% Split point at (2,4), three parallel branches
		\draw
		(2,4) 
		to[short] (2.5,4)
		to[short] (3,4) % upward for labeling if needed
		(3,4.01) -- (3,3.5); % central node
		\node[circle, fill=black, inner sep=1pt, label=above right:] at (3,3.5) {};
		% Resistor branch (left)
		\draw
		(3.5,3.5) -- (1.5,3.5)
		to[C=$C$, i>_=${i_1(t)}$] (1.5,1)
		-- (1.5,0.5); % back to bottom of source
		% Capacitor branch (middle)
		\draw
		(3,3.5) to[R=$R$, i>_=${i_2(t)}$] (3,0.5)
		-- (3,0.5);
		% Inductor branch (right)
		\draw
		(3.5,3.5) -- (4.5,3.5)
		to[cute inductor, l=$L$, i>_=${i_3(t)}$] (4.5,0.5)
		-- (4.5,0.5);
		% Merge point at (0,4), three parallel branches
		\node[circle, fill=black, inner sep=1pt, label=above right:] at (3,0.5) {};
		\draw
		(4.5,0.5) 
		to[short] (1.5,0.5)
		to[short] (3,0.5) 
		to[short] (3,0) 
		-- (0,0); % central node
	\end{circuitikz}
	\\
	\caption{Parallel $LRC$ circuit}
	\label{fig:parallelLRC}
\end{figure}

By Kirchhoff's second law which states that  \textit{the current flowing to a point in a circuit must equal the current flowing away from the point}, we have
\begin{equation}\label{eq:LRC-parallel-1}
	i_1(t) + i_2(t) +i_3(t) = i(t).
\end{equation}
Since
\[i_1(t) = C\;e'(t), \quad i_2(t) =  \frac{e(t)}{R},  \quad \text{and } \quad i_3(t) = \frac{1}{L}\int e(t)\, dt,\] \eqref{eq:LRC-parallel-1} reduces to
\begin{equation*}
	C\;e'(t)+\frac{e(t)}{R}+\frac{1}{L}\int e(t)\, dt = i(t).
\end{equation*}
Differentiating with respect to $t$ yields
\begin{equation}\label{eq:LRC-parallel-2}
	C\; e''(t)+\frac{1}{R}\,e'(t)+\frac{1}{L}\,e(t) = i'(t).
\end{equation}
For analogy, we can interchange the spring-mass system parameters, series \(LRC\)  circuit parameters, and   parallel \(LRC\)  circuit parameters in \eqref{eq:LRC-parallel-2} as follows:
\begin{center}
	\begin{tabular}{cccc}
		\hline
		Spring-Mass  &    Series \(LRC\)  &  Parallel \(LRC\)   \\ \hline
		\(m\)  &\(L\) &       $C$ \\
		\(b\)     & \(R\)&        $\displaystyle\frac{1}{R}$ \\[2ex]
		\(k\)     & \(\displaystyle\frac{1}{C}\)&       $\displaystyle\frac{1}{L}$ \\[2ex]
		\(\omega =\sqrt{k/m}\) &\(\omega =\sqrt{1/LC}\)    &\(\omega =\sqrt{1/LC}\)  \\[2ex]
		\hline
	\end{tabular}
\end{center}

In addition, the  three significantly different cases of the spring-mass system apply to the \(LRC\)   circuits  as follows:
\begin{center}
	\begin{tabular}{cccc}
		\hline
		Damping Type &  Spring-Mass     & Series \(LRC\)  & Parallel \(LRC\)   \\ 
		\hline\\ [-1ex]
		Underdamped &\(b^2< 4mk\)       &\(\displaystyle R^2<\frac{4L}{C}\)& \(\displaystyle R^2>\frac{L}{4C}\) \\[2ex]
		Critically damped&\(b^2= 4mk\)   &\(\displaystyle R^2=\frac{4L}{C}\)& \(\displaystyle R^2=\frac{L}{4C}\) \\[2ex]
		Overdamped  &\(b^2> 4mk\)      &\(\displaystyle R^2>\frac{4L}{C}\)& \(\displaystyle R^2<\frac{L}{4C}\) \\[2ex]
		\hline
	\end{tabular}
\end{center}

\medskip
\begin{example}
	% ---------------- Problem ----------------
	Consider a series $LRC$–circuit with inductance $L=1\,\text{h}$, resistance 
	$R=12\,\Omega$, and capacitance $C=\dfrac{1}{40}\,\text{f}$. 
	The circuit is driven by an external voltage
	\[
	e(t) = 10\sin(2t) \qquad t>0.
	\] measure in volts (V).
	Let $q(t)$ denote the charge (in coulombs) on the capacitor, and let 
	$i(t)=q'(t)$ be the current (in amperes). The initial charge and current are
	\[
	q(0)=0, \qquad q'(0)=0.
	\]
	\begin{enumerate}[label= (\alph*), noitemsep]
		\item\label{eg-item:LRC-1} Derive the differential equation satisfied by $q(t)$ and solve it.
		\item\label{eg-item:LRC-2} Find the \emph{transient part} and 
		\emph{steady-state} solution.
		\item \label{eg-item:LRC-3} Find the critical resistance at which the charge on the capacitor changes from oscillatory to nonoscillatory.
	\end{enumerate}
\end{example}
	
	
	% ---------------- Solution ----------------
\begin{solution}
	\ref{eg-item:LRC-1} 
	For a series $LRC$–circuit, Kirchhoff's voltage law gives
	\[
	Lq''(t) + Rq'(t) + \frac{1}{C}q(t) = e(t).
	\]
	With $L=1$, $R=12$, $C=\dfrac{1}{40}$, and $e(t)=10\sin(2t)$, we obtain the differential equation
	\[
	q'' + 12q' + 40q = 10\sin(2t).
	\]
	The corresponding homogeneous equation is
	\[
	q'' + 12q' + 40q = 0.
	\]
	The roots of the characteristic equation
	\[
	\lambda^2 + 12\lambda + 40 = 0
	\]
	are
	\[
 \frac{-12 \pm \sqrt{-16}}{2}
	= -6 \pm 2i.
	\]
	Thus the general  solution of the homogeneous equation is
	\[
	q_h(t) = e^{-6t}\bigl(c_1\cos(2t) + c_2\sin(2t)\bigr)
	\]
	where $c_1$ and $c_2$ are arbitrary constants.

For a particular solution $q_p(t),$ suppose that  $A$ and $B$ are constants to be determined, so that
	\[
	q_p(t) = A\cos(2t) + B\sin(2t),
	\] is a particular solution of the nonhomogeneous equation.
	Then
	\[
	q_p'(t) = -2A\sin(2t) + 2B\cos(2t),\qquad
	q_p''(t) = -4A\cos(2t) - 4B\sin(2t).
	\]
	Substituting into the equation $q''+12q'+40q=10\sin(2t)$ yields
	\begin{align*}
		&(-4A\cos 2t - 4B\sin 2t)
		+ 12(-2A\sin 2t + 2B\cos 2t)
		+ 40(A\cos 2t + B\sin 2t) \\
		&= \bigl(36A + 24B\bigr)\cos(2t)
		+ \bigl(-24A + 36B\bigr)\sin(2t)
		= 10\sin(2t).
	\end{align*}
	Thus we must have
	\[
	36A + 24B = 0, \qquad
	-24A + 36B = 10.
	\]
	Solving for $A$ and $B$ gives
	\[A= -\frac{5}{39} \mbox{ and } B = \frac{5}{26},\] and so 
	\[
	q_p(t) = -\frac{5}{39}\cos(2t) + \frac{5}{26}\sin(2t).
	\]
	
	The general solution is
	\[
	q(t) = q_h(t) + q_p(t)
	= e^{-6t}\bigl(c_1\cos(2t) + c_2\sin(2t)\bigr)
	-\frac{5}{39}\cos(2t) + \frac{5}{26}\sin(2t),
	\]
	where $c_1$ and $c_2$ are arbitrary constants.
	Using $q(0)=0$ gives
	\[
	 c_1 - \frac{5}{39} = 0, \quad \mbox{ which implies }\quad 
	c_1 = \frac{5}{39}.
	\]
	To use $q'(0) = 0,$
	we first compute $q'(t)$:
	\[
	\begin{aligned}
		q'(t) 
		&= e^{-6t}\Bigl[-6(c_1\cos(2t) + c_2\sin(2t))
		+(-2c_1\sin(2t) + 2c_2\cos(2t))\Bigr] \\
		&\qquad\quad + \Bigl(\frac{10}{39}\sin(2t) + \frac{5}{13}\cos(2t)\Bigr).
	\end{aligned}
	\]
	Then, using \(q'(0)=0\), we get
	\[
	-6c_1 + 2c_2 + \frac{5}{13}=0.
	\]
Since \(	c_1 = \frac{5}{39},\) we have
	\[
	-6\cdot\frac{5}{39} + 2c_2 + \frac{5}{13} = 0,
	\] which gives
	\[c_2 = \frac{5}{26}.\]
	Thus, the  solution for $q(t)$  is
	\[
		q(t)
		= e^{-6t}\!\left(\frac{5}{39}\cos(2t) + \frac{5}{26}\sin(2t)\right)
		-\frac{5}{39}\cos(2t) + \frac{5}{26}\sin(2t).
	\]
	
	\medskip
	
	\noindent\ref{eg-item:LRC-2}  The \textbf{\textit{transient part}} of the solution is the part that \emph{decays} to zero as $t\to\infty$
	and depends on the initial conditions. Since
	\[
	\lim\limits_{t\to\infty}
	e^{-6t}\!\left(\frac{5}{39}\cos(2t) + \frac{5}{26}\sin(2t)\right)=0,
	\] the transient part of the solution for $q(t)$ is
	\[e^{-6t}\!\left(\frac{5}{39}\cos(2t) + \frac{5}{26}\sin(2t)\right)\] 


	The \textbf{\textit{steady-state solution}} is the part of the solution $q(t)$ that \emph{remains} after the transient part
	has decayed to zero. Therefore the steady-state solution for $q(t)$ is  given by
	\[
	\lim_{t\to\infty} q(t) = -\frac{5}{39}\cos(2t) + \frac{5}{26}\sin(2t).
	\]
	
\noindent\ref{eg-item:LRC-3}  The series $LRC$-circuit is  critically damped for
\[R = 2 \sqrt{\frac{L}{C}} =  4\sqrt{10}\; \Omega.\]
\end{solution}




\begin{Exercise}[title={Mixed Spring Mass and Electrical Network Systems}]\label{EX36}
	
	%\textbf{\ref{subsection:spring-mass} Spring-Mass Systems}
%	\vspace{-\baselineskip}% <-- You don't need this line of code if there's some text here
\noindent\textbf{\ref{subsection:spring-mass}} \textbf{\large Spring-Mass Systems}
	\Question\label{3-6-1}
An $8$-kg mass attached to the lower end of a spring that is suspended vertically from a ceiling stretches the spring $5$ cm from its natural length until the mass comes to rest. The mass is then pulled downward causing an additional stretch of $3$ cm of the spring  and then it is released. Assume that there is no air resistance and use the acceleration due to gravity  \(g = 980\) cm/s\(^2\). Let \(y(t)\) be the displacement (in centimeters) of the mass from its equilibrium position measured  positive in the downward direction.
\begin{enumerate}[label =\textbf{\roman*.}, noitemsep]
	\item\label{item:sho-1} Determine a differential equation in $y(t)$ with  initial conditions that describe the motion of the mass.
	\item Solve for the initial value problem in part~\ref{item:sho-1}.
	\item Find the amplitude, period and circular frequency of the oscillatory motion.
\end{enumerate}
%	\begin{tasks}(1)
	%		\task 
	%		\task 
	%		\task 
	%		\task 
	%		\task 
	%	\end{tasks}
	
	\Question\label{3-6-2} Suppose a vertical spring with spring constant \( 8 \ \mathrm{N/m} \) is suspended from a ceiling, and a \( 2 \ \mathrm{kg} \) mass is attached to its lower end. The mass slides along a vertical wall that exerts a damping force proportional to its velocity, with damping constant \( 3 \ \mathrm{Ns/m} \).
	Let \( y(t) \) denote the displacement (in meters) of the mass from its equilibrium position, where positive values of \( y(t) \) correspond to positions below the equilibrium position.
	\begin{enumerate}[label =\textbf{\roman*.}, noitemsep]
		\item\label{item:sho-2} Derive a differential equation governing the motion of the system in terms of \( y(t) \).
		\item Find the general solution of the differential equation obtained in part~\ref{item:sho-2}.
		\item Determine whether the system is underdamped, critically damped, or overdamped.
		\item  If the system is not critically damped, find the value of the damping constant that would make the system critically damped.
	\end{enumerate}
	\Question\label{3-6-3} A \( 10 \)-kg mass suspended from the lower 
	end of a vertically hanging spring stretches the spring
	\( 9.8 \) centimeters.  At time \( t = 0 \) when the mass is at rest in the equilibrium position, the 
	 mass-spring system is acted by the force \( F(t) = 360\cos(8t) \),
 measured positive (in newtons) in the downward direction. Suppose that time \(t\) is measured in 
	seconds. Let $y(t)$ denote the displacement (in centimeters) of the mass from its equilibrium position, measured positive in the downward direction.
	\begin{enumerate}[label =\textbf{\roman*.}, noitemsep]
	\item Determine the spring constant $k$.
	\item\label{item:forced-1} Develop an initial value problem for $y(t)$.
	\item Solve the initial value problem developed in part~\ref{item:forced-1}.
	\item Using a graphing utility, plot the solution $y(t)$ and determine whether the motion exhibits a ``beat''.
	\item Find the maximum positive displacement of the mass from its equilibrium position.
	\end{enumerate}
	
	\Question\label{3-6-4}
	A 8-pound weight attached to a spring of natural length 7 feet stretches the spring to a total length of 8.6 feet.  The entire system is placed in a medium that exerts a damping force numerically equal to 4 times the instantaneous velocity.
	
	\begin{enumerate}[label =\textbf{\roman*.}, noitemsep]
		\item Write down an initial value problem describing the motion.
		\item Determine the equation of motion if the mass is released from rest at a point  $\tfrac{1}{2}$ ft below the equilibrium position with a downward velocity of 1 ft/s.
		\item Express the  equation of motion in terms of a single sine function.
		\item Find the times at which the mass passes through the equilibrium position moving downward.
		\item Plot the equation of motion.
	\end{enumerate}
	\medskip

	%\textbf{\ref{subsection:electrical-network} Electrical Networks}
\setlength{\parindent}{-4ex}\textbf{\ref{subsec:electric-circuits}} \textbf{\large Electrical Circuits}
	%\setcounter{Question}{0}
	\Question\label{3-6-5} A series $LRC$–circuit contains an inductor with inductance $L=1\,\text{h}$,
	a resistor with resistance $R=10\,\Omega$, and a capacitor with capacitance
	$C=0.02\,\text{f}$. The circuit is driven by a constant external voltage
	$e(t)=5\,\text{V}$. Let $q(t)$ denote the charge (in coulombs)
	on the capacitor, and let $i(t)=q'(t)$ be the current (in amperes).
	Given the initial conditions $q(0)=4$~C and $i(0)=q'(0)=0$~A, 
	\begin{tasks}
		\task 	determine the charge $q(t)$ for $t>0$; and
		\task find  the transient current and steady-state current.
	\end{tasks}
%	
%	\textbf{Solution.}
%	We solve the differential equation
%	\[
%	q'' + 10q' + 50q = 5,
%	\qquad q(0)=4,\quad q'(0)=0.
%	\]
%	
%	\medskip
%	\textbf{1. Homogeneous solution.}
%	The associated homogeneous equation
%	\[
%	q'' + 10q' + 50q = 0
%	\]
%	has characteristic equation
%	\[
%	r^2 + 10r + 50 = 0,
%	\]
%	whose roots are
%	\[
%	r = -5 \pm 5i.
%	\]
%	Thus
%	\[
%	q_h(t) = e^{-5t}\bigl(C_1 \cos 5t + C_2 \sin 5t\bigr).
%	\]
%	
%	\medskip
%	\textbf{2. Particular solution.}
%	Because the forcing term is constant, try $q_p(t)=A$. Substituting,
%	\[
%	50A = 5 \quad\Rightarrow\quad A=\frac{1}{10}.
%	\]
%	Hence the general solution is
%	\[
%	q(t) = e^{-5t}\bigl(C_1 \cos 5t + C_2 \sin 5t\bigr) + \frac{1}{10}.
%	\]
%	
%	\medskip
%	\textbf{3. Apply initial conditions.}  
%	From $q(0)=4$:
%	\[
%	C_1 + \frac{1}{10} = 4
%	\quad\Longrightarrow\quad
%	C_1 = \frac{39}{10}.
%	\]
%	
%	To apply $q'(0)=0$, first compute
%	\[
%	q'(t)
%	= e^{-5t}\Bigl[-5(C_1 \cos 5t + C_2 \sin 5t)
%	+ (-5C_1 \sin 5t + 5C_2 \cos 5t)\Bigr].
%	\]
%	Then
%	\[
%	q'(0) = -5C_1 + 5C_2 = 0
%	\quad\Longrightarrow\quad
%	C_2 = C_1 = \frac{39}{10}.
%	\]
%	
%	\medskip
%	\textbf{4. Final charge.}
%	\[
%	\boxed{
%		q(t)
%		= \frac{39}{10}\, e^{-5t}\bigl(\cos 5t + \sin 5t\bigr)
%		+ \frac{1}{10}.
%	}
%	\]
%	
%	\medskip
%	\textbf{5. Current in the circuit.}
%	Since $i(t)=q'(t)$, substituting $C_1=C_2=\frac{39}{10}$ yields
%	\[
%	\boxed{
%		i(t) = -39\, e^{-5t}\sin 5t.
%	}
%	\]
	
 

	\Question\label{3-6-6} Consider a series $LRC$–circuit with inductance $L=1\ \text{h}$, capacitance 
	$C=16\ \text{f}$, resistance $R$, and no external voltage source.  
	Let $q(t)$ denote the charge (in coulombs) on the capacitor at time $t$ (in seconds).
	\begin{tasks}(1)
		\task Find $q(t)$ when $R=4\ \Omega$, $q(0)=5$ C, and $q'(0)=0$ C/s.
		\task Find the first time when the charge on the capacitor is equal to zero.
		\task Find the critical resistance at which the charge on the capacitor  changes from
		oscillatory to nonoscillatory behavior.
	\end{tasks}
	
%	\textbf{Solution.}
%	For an $LRC$–circuit with no external source, the charge $q(t)$ satisfies
%	\[
%	Lq'' + Rq' + \frac{1}{C}q = 0.
%	\]
%	With $L=1$ and $C=0.05$, we have $\tfrac{1}{C}=20$, so
%	\[
%	q'' + Rq' + 20q = 0.
%	\]
%	
%	\begin{enumerate}
%		%------------------------------------------------------------
%		\item[(a)] \textbf{Find $q(t)$ when $R=4$, $q(0)=5$, $q'(0)=0$.}
%		
%		For $R=4$,
%		\[
%		q'' + 4q' + 20q = 0.
%		\]
%		The characteristic equation
%		\[
%		r^2 + 4r + 20 = 0
%		\]
%		yields
%		\[
%		r = -2 \pm 4i.
%		\]
%		Thus
%		\[
%		q(t) = e^{-2t}\bigl(A\cos(4t) + B\sin(4t)\bigr).
%		\]
%		
%		Using $q(0)=5$ gives $A=5$.  
%		Using $q'(0)=0$ gives
%		\[
%		-2A + 4B = 0 \quad\Rightarrow\quad -10 + 4B = 0 \Rightarrow B=\frac{5}{2}.
%		\]
%		
%		Hence
%		\[
%		\boxed{
%			q(t) = 5 e^{-2t}\!\left(\cos 4t + \tfrac12 \sin 4t\right).
%		}
%		\]
%		
%		%------------------------------------------------------------
%		\item[(b)] \textbf{Find the first time when $q(t)=0$.}
%		
%		Since $e^{-2t}\neq 0$, we solve
%		\[
%		\cos(4t) + \tfrac12 \sin(4t) = 0.
%		\]
%		Thus
%		\[
%		\tan(4t) = -2.
%		\]
%		The smallest positive solution occurs when
%		\[
%		4t = \pi - \arctan(2),
%		\]
%		hence
%		\[
%		\boxed{
%			t_1 = \frac{\pi - \arctan(2)}{4} \approx 0.51\ \text{s}.
%		}
%		\]
%		
%		%------------------------------------------------------------
%		\item[(c)] \textbf{Determine the critical resistance.}
%		
%		For the general equation
%		\[
%		q'' + Rq' + 20q = 0,
%		\]
%		the discriminant of the characteristic equation is
%		\[
%		\Delta = R^2 - 80.
%		\]
%		
%		Critical damping occurs when $\Delta = 0$, so
%		\[
%		R^2 - 80 = 0
%		\quad\Rightarrow\quad
%		R_{\text{critical}} = \sqrt{80} = 4\sqrt{5}.
%		\]
%		
%		\[
%		\boxed{
%			R_{\text{critical}} = 4\sqrt{5}\ \Omega \approx 8.94\ \Omega.
%		}
%		\]
%	\end{enumerate}
	
	\Question\label{3-6-7} Consider a parallel $LRC$–circuit with inductance $L=\frac{4}{5}\ \text{h}$, capacitance 
	$C=\frac{1}{16}\ \text{f}$, resistance $R$, and a constant external  current source.  
	Let $q(t)$ denote the charge (in coulombs) on the capacitor at time $t$ (in seconds).
	\begin{tasks}(1)
		\task Find $q(t)$ when $R=4\ \Omega$, $q(0)=5$ C, and $q'(0)=0$ C/s.
		\task Find the first time when the charge on the capacitor is equal to zero.
		\task Find the critical resistance at which the charge on the capacitor  changes from
		oscillatory to nonoscillatory behavior.
	\end{tasks}
	
%	\textbf{Solution.}
%	For a parallel $LRC$–circuit driven by a (constant) current source $I_0$, the
%	node equation (KCL) is
%	\[
%	I_0 = i_C + i_R + i_L.
%	\]
%	Let $v(t)$ be the common voltage across the elements. Then
%	\[
%	i_C = C v'(t), \qquad i_R = \frac{v(t)}{R}, \qquad
%	v(t) = L\,i_L'(t).
%	\]
%	We are given that $q(t)$ is the charge on the capacitor, so
%	\[
%	q(t) = C v(t), \qquad v(t) = \frac{q(t)}{C}, \qquad q'(t) = i_C(t).
%	\]
%	Rewrite KCL in terms of $q$:
%	\[
%	I_0 = q' + \frac{v}{R} + i_L
%	= q' + \frac{1}{R}\frac{q}{C} + i_L
%	= q' + \frac{q}{RC} + i_L.
%	\]
%	Differentiate this equation:
%	\[
%	0 = q'' + \frac{1}{RC} q' + i_L'.
%	\]
%	Using $v = L i_L'$ and $v = q/C$, we have
%	\[
%	i_L' = \frac{v}{L} = \frac{q}{CL}.
%	\]
%	Thus
%	\[
%	0 = q'' + \frac{1}{RC} q' + \frac{1}{CL}q.
%	\]
%	Hence $q(t)$ satisfies the homogeneous second–order ODE
%	\[
%	q'' + \frac{1}{RC} q' + \frac{1}{LC} q = 0.
%	\]
%	
%	With $L = \dfrac{4}{5}$ and $C = \dfrac{1}{16}$, we compute
%	\[
%	LC = \frac{4}{5}\cdot \frac{1}{16} = \frac{1}{20},
%	\qquad
%	\frac{1}{LC} = 20,
%	\qquad
%	\frac{1}{RC} = \frac{16}{R}.
%	\]
%	Therefore the differential equation is
%	\[
%	q'' + \frac{16}{R}q' + 20q = 0.
%	\]
%	
%	\begin{enumerate}
%		%------------------------------------------------------------
%		\item[(a)] \textbf{Find $q(t)$ when $R=4,\ q(0)=5,\ q'(0)=0$.}
%		
%		For $R=4$, the equation becomes
%		\[
%		q'' + 4q' + 20q = 0.
%		\]
%		The characteristic equation
%		\[
%		r^2 + 4r + 20 = 0
%		\]
%		has roots
%		\[
%		r = \frac{-4 \pm \sqrt{16 - 80}}{2}
%		= -2 \pm 4i.
%		\]
%		Thus the general solution is
%		\[
%		q(t) = e^{-2t}\bigl(A\cos(4t) + B\sin(4t)\bigr).
%		\]
%		
%		Using $q(0)=5$ gives
%		\[
%		q(0) = A = 5 \quad\Rightarrow\quad A = 5.
%		\]
%		Compute $q'(t)$:
%		\begin{align*}
%			q'(t)
%			&= e^{-2t}\Bigl[(-2)(A\cos4t + B\sin4t)
%			+ (-4A\sin4t + 4B\cos4t)\Bigr].
%		\end{align*}
%		Then
%		\[
%		q'(0) = -2A + 4B.
%		\]
%		The condition $q'(0)=0$ implies
%		\[
%		-2A + 4B = 0 \quad\Rightarrow\quad
%		4B = 2A = 10 \Rightarrow B = \frac{5}{2}.
%		\]
%		
%		Hence
%		\[
%		\boxed{
%			q(t) = e^{-2t}\Bigl(5\cos(4t) + \tfrac{5}{2}\sin(4t)\Bigr)
%			= 5e^{-2t}\Bigl(\cos(4t) + \tfrac12\sin(4t)\Bigr).
%		}
%		\]
%		
%		%------------------------------------------------------------
%		\item[(b)] \textbf{Find the first time when $q(t)=0$.}
%		
%		We seek the smallest $t>0$ such that $q(t)=0$. Since $e^{-2t}\neq 0$,
%		this reduces to
%		\[
%		\cos(4t) + \tfrac12\sin(4t) = 0.
%		\]
%		Thus
%		\[
%		\tan(4t) = -2.
%		\]
%		In general,
%		\[
%		4t = \arctan(-2) + k\pi = -\arctan(2) + k\pi,
%		\qquad k\in\mathbb{Z}.
%		\]
%		The smallest positive solution occurs for $k=1$:
%		\[
%		4t = \pi - \arctan(2),
%		\]
%		hence
%		\[
%		\boxed{
%			t_1 = \frac{\pi - \arctan(2)}{4}
%			\approx 0.51\ \text{s}.
%		}
%		\]
%		
%		%------------------------------------------------------------
%		\item[(c)] \textbf{Find the critical resistance.}
%		
%		For the general equation
%		\[
%		q'' + \frac{16}{R}q' + 20q = 0,
%		\]
%		the characteristic equation is
%		\[
%		r^2 + \frac{16}{R} r + 20 = 0,
%		\]
%		with discriminant
%		\[
%		\Delta = \left(\frac{16}{R}\right)^2 - 4\cdot 20
%		= \frac{256}{R^2} - 80.
%		\]
%		Critical damping occurs when $\Delta = 0$, so
%		\[
%		\frac{256}{R^2} - 80 = 0
%		\quad\Rightarrow\quad
%		\frac{256}{R^2} = 80
%		\quad\Rightarrow\quad
%		R^2 = \frac{256}{80} = \frac{16}{5}.
%		\]
%		Thus the critical resistance is
%		\[
%		\boxed{
%			R_{\text{critical}} = \sqrt{\frac{16}{5}} = \frac{4}{\sqrt{5}}\ \Omega
%			\approx 1.79\ \Omega.
%		}
%		\]
%		
%	\end{enumerate}
	
	
\end{Exercise}

\setboolean{firstanswerofthechapter}{true}
\begin{multicols}{2}\scriptsize
	\begin{Answer}[ref={EX36}]
		\Question \label{3-6-1a}
		\begin{tasks}
			\task \(y''+196y=0, \; y(0) = 3,\; y'(0) = 0\)
			\task \(y(t) = 3 \cos(14t)\)
			\task \(3,\; \pi/7,\; 14\)
		\end{tasks} 
	
		\Question \label{3-6-2a}	
			\begin{tasks}
			\task \(2y''+3y'+8y=0\)
			\task \(y(t) =e^{-3 t/4} \Big( c_1 \cos \left(\frac{\sqrt{55} t}{4}\right)\\+c_2  \sin \left(\frac{\sqrt{55} t}{4}\right)\Big)\)
			\task  underdamped
			\task \(8\)
		\end{tasks}
		
		\Question \label{3-6-3a}
		\begin{tasks}
			\task \(k=1000\)
			\task \( y''+100 y = 36\cos(8t)\)
			\task \(y(t)= \cos (8 t)-\cos (10 t)\)
			\task\includegraphics[width=0.9\linewidth]{chap03/mathematica/exercises-beats}
			\task \(2 \) at \(t= \pi/2\)
	\end{tasks}
	
		\Question \label{3-6-4a}
	\begin{tasks}
		\task \(y''+4y'+20y 0; \; y(0) = 1/2, \; y'(0) = 1.\)
		\task \(y(t)= \frac{1}{2} e^{-2 t} \left(\cos(4t) + \sin (4t)\right)\)
		\task \(y(t)= \frac{\sqrt{2}}{2} e^{-2 t} \sin\left(4t+\frac{\pi}{4}\right)\)
		\task \((4n-1)\frac{\pi}{16}\) with $n = 2, 4, 6, \dots$
		\task 	\includegraphics[width=0.9\linewidth]{chap03/mathematica/solution-through-equilibrium}
	\end{tasks}
	
	
	
	\hspace{-4ex}\ref{subsec:electric-circuits} \textbf{Electrical Circuits}\\
	\Question\label{3-6-5a}
	\begin{tasks}
		\task \(q(t) = \frac{39}{10} e^{-5t}\left(\cos(5t)+ \sin(5t)\right) + \frac{1}{10}\)
		\task transient current:\\ \(i(t) = -39 e^{-5t} \sin(5t)\)\\
		steady-state current: \(0\)
	\end{tasks}
	\Question\label{3-6-6a}
	\begin{tasks}
		\task  \(q(t) = 5 e^{-2t}\!\left(\cos 4t + \tfrac12 \sin 4t\right)\)
		\task Approximately \(0.51\) s.
		\task \(4\sqrt{5}\; \Omega\)
	\end{tasks}
		\Question\label{3-6-7a}
	\begin{tasks}
		\task  \(q(t) = 5 e^{-2t}\!\left(\cos 4t + \tfrac12 \sin 4t\right)\)
		\task Approximately \(0.51\) s.
		\task \(\frac{4}{\sqrt{5}}\; \Omega\)
	\end{tasks}
	
	\end{Answer}



\end{multicols}
\setboolean{firstanswerofthechapter}{false}


%\begin{hardsec}
	\section{Change of  Variables} \label{sec:change-of-variables}
%\end{hardsec}



We begin this section by exploring how a change of the dependent variable can be used to reduce a second order linear differential equation to a  form  can  be solved using methods developed earlier. We then examine how a change of the independent variable can be used to develop  solution techniques applicable to a variety of differential equations.

\subsection{Change of the Dependent Variable}
Let \(u(x)\) be a function to be chosen later and let $w(x)$ be a function such that \( y(x) = u(x) w(x)\) is a solution of
\begin{equation}\label{eq:4-4-1}
	y''+ a_1(x) y' + a_0(x)y = f(x)
\end{equation}
 on some interval $I,$ where  $a_1, a_0, f$ are continuous. Upon differentiation, we get
\[
\begin{split}
	y' &= u'w+uw'\\
	y''&=u''w+2u'w'+uw''.
\end{split}
\]
Substituting these into the differential equation, we get
\begin{equation}\label{eq:4-4-2}
	(u''+a_1u'+a_0u)w + (2u'+a_1u)w'+ uw'' = f.
	\end{equation}
Our goal  is to find $w$ by choosing $u$ in a way that simplifies \eqref{eq:4-4-2}. A choice of \( u \) can be made so that it satisfies one of the following two conditions:

\noindent \textbf{Condition A:} \(u''+ a_1 u' + a_0u=0,\) i.e., \(u\) is a solution of \begin{equation}\label{eq:4-4-3}
		y''+ a_1(x) y' + a_0(x)y = 0.
	\end{equation}
\noindent \textbf{Condition B:} \(2u'+a_1(x)u=0\) and  $a_1$ is differentiable.

Suppose first that \(u\) satisfies \textbf{Condition A}. Then \eqref{eq:4-4-2} becomes
\begin{equation}\label{eq:4-4-4}
	(2u'+a_1u)w'+ uw'' = f,
	\end{equation} which, being a linear equation in \(w',\) can be solved first for $w'$ and then for $w.$ Thus, we have obtained a particular solution $y_p(x) = u(x) w(x)$ of \eqref{eq:4-4-1}.

To find the general solution of \eqref{eq:4-4-1}, we need one another solution $v(x)$ \eqref{eq:4-4-3}, such that $u(x)$ and  $v(x)$ are linearly independent. Let $v(x) = u(x) z(x)$
for some function $z(x)$ such that $v$ satisfies \eqref{eq:4-4-3}. Running the same calculation as above with $z$ in place of $w,$ but with $f=0,$ we get 
\begin{equation}\label{eq:4-4-5}
	(2u'+a_1u)z'+ uz'' = 0,
	\end{equation} which solves for a nonzero function $z(x).$ Consequently, the general solution to \eqref{eq:4-4-1} is given by 
\[\begin{split}
	y &= c_1u(x)+ c_2v(x) + u(x)w(x)\\
	&=c_1u(x)+ c_2u(x) z(x) + u(x)w(x)\\
	&= u(x) (c_1 + c_2 z(x) + w(x)),
\end{split}\] where $c_1$ and $c_2$ are arbitrary constants.

Let us work through an example to show how the above procedure  works.
\begin{example}
	Find the general solution of 
	\[(x-1)y''-x y'+y =1\] using the fact that $u(x) = e^x$ is a solution to the associate homogeneous equation.
\end{example}

\begin{solution}
First, we note that \(a_1(x) = -\frac{x}{x-1}\) and \(f(x) = \frac{1}{x-1}\). We compute \(w\) and \(z\) satisfying \eqref{eq:4-4-4} and \eqref{eq:4-4-5}, respectively. Using \eqref{eq:4-4-4}, we get
\[\left(2e^x-\frac{x}{x-1} e^x\right)w' + e^x w'' = \frac{1}{x-1}\] which gives
\[w''+\left(1-\frac{1}{x-1}\right)w' = \frac{e^{-x}}{x-1}.\] The integrating factor is
\[\frac{e^x}{x-1}.\] Then
\[\left(\frac{e^x}{x-1} w'\right)'= \frac{1}{(x-1)^2},\] which yields
\[\frac{e^x}{x-1} w' = -\frac{1}{x-1}+A,\] so that 
\[w' = -e^{-x} +A (x-1)e^{-x}.\] Integrating yields
\[w = e^{-x} -Axe^{-x} + B,\] where $A$ and $B$ are arbitrary constants. We simply take $w(x) = e^{-x},$ so that $y_p(x) = u(x)w(x) = e^x e^{-x} = 1.$ Moreover,  it is  clear that \(z(x)=-Axe^{-x} + B\) solves \eqref{eq:4-4-5} for all $A$ and $B$. Since we only need one nonzero function  $z(x),$ we can simply take $z(x) = xe^{-x}.$ Therefore $v(x) = e^x z(x) = x.$
Hence the general solution of the given nonhomogeneous equation is 
\[y = c_1e^x + c_2 x +1,\] where $c_1$ and $c_2$ are arbitrary constants.
\end{solution}

Suppose now that \(u\) satisfies \textbf{Condition B}. Then \(2u'+a_1(x)u=0\) and so
\[ u' = -\frac{a_1}{2}u\quad \text{ so that }\quad  u(x) = e^{-\frac{1}{2}\int a_1(x) \,dx}.\]
 Moreover, 
 \begin{equation}\label{eq:4-4-6}
 	2u''= - a_1 u' - a'_1 u =\left(\frac{a_1^2}{2}-a'_1\right)u.
\end{equation}
Then 
\[\begin{split}
	u''+a_1 u'+a_0 u &=\left(\frac{a_1^2}{4}-\frac{1}{2}a'_1\right)u - \frac{a_1^2}{2}u +a_0u\\
	&=\left(a_0-\frac{a_1^2}{4}-\frac{a'_1}{2}\right)u.
\end{split} \]
Substituting these into \eqref{eq:4-4-2} gives
\begin{equation}\label{eq:4-4-7}
w''+\left(a_0-\frac{a_1^2}{4}-\frac{a'_1}{2}\right)w = g(x), 
\end{equation}
where \(g(x) = \frac{f(x)}{u(x)}.\)

The differential equation \eqref{eq:4-4-7} is said to be the \textbf{\textit{normal form}} of \eqref{eq:4-4-1}.
The coefficient of \(w\) in \eqref{eq:4-4-7} is called the \textbf{\textit{invariant}} of \eqref{eq:4-4-1} and is denoted by \(I_y(x).\) That is,
\begin{equation}
	I_y(x) = a_0(x)-\frac{a_1^2(x)}{4}-\frac{a'_1(x)}{2}.
	\end{equation} 
 We observe that \eqref{eq:4-4-7} turns into a differential equation with constant coefficients if and only if $I_y(x)$ is constant. Consequently, \eqref{eq:4-4-7} can be readily solved  in this case. When $I_w(x)$  is a constant function,  \[y = u(x) w(x)= w(x)\;e^{-\frac{1}{2}\int a_1(x)\, dx} \] is the general solution of \eqref{eq:4-4-1} if and only if \(w\) is the general solution of \eqref{eq:4-4-7}. 

We discuss some examples below to show how the above procedure works.

\begin{example}
	Find the general solution of the differential equation
 \[y''+4y'+4y = \sin x\]   by using its normal form.
\end{example}
\begin{solution} Comparing the differential  equation with \eqref{eq:4-4-1}, we have $a_0= a_1=4.$
	So the invariant of the differential equation is
	\[I_y(x) = 4-\frac{4^2}{4}-0 = 0.\]
	With the change of variable $y = u(x) w(x)$ with $u(x) = e^{-\frac{1}{2}\int 4\,dx} = e^{-2x},$ the normal form of the equation is
	 \[w''(x) = \frac{\sin x}{u(x)}=  e^{2x}\sin x.\] Upon integration, we get
	 \[w(x) = c_1+c_2 x-\frac{1}{25} e^{2 x} (4 \cos x-3 \sin x).
	 \]
	 Therefore the general solution  is
	 \[y = e^{-2x}w(x) = c_1e^{-2x}+c_2 x e^{-2x}-\frac{1}{25} (4 \cos x-3 \sin x),\]
	 which is the same result we would have  obtained by  the method of undermined coefficients.
\end{solution}

The above procedure is particularly helpful for solving linear equations with variable coefficients, as the following examples illustrate.

\begin{example}
Find the general solution of 
\[x^2y'' -2xy' + (k^2x^2+2)y = x^3\sin x,\] where $k\ne 1$ is a positive constant, by using its normal form.
\end{example}
\begin{solution}
	 Comparing the given  equation with \eqref{eq:4-4-1}, we have 
	 \[ a_0(x) = k^2+\frac{2}{x^2} \quad \text{ and } a_1(x) = - \frac{2}{x}.\]
	 So the invariant of the differential equation is
	 \[I_y(x) = a_0(x)-\frac{a_1^2(x)}{4}-\frac{a'_1(x)}{2} =k^2+\frac{2}{x^2}-\frac{1}{x^2}- \frac{1}{x^2} = k^2.\]
	 Since \[u(x) = e^{-\frac{1}{2}\int \left(-\frac{2}{x}\right)\, dx} = e^{\ln\abs{x}} = \abs{x}, \]
	 we  take $u(x) = x.$ With the change of variable \(y = u(x) w(x),\) the differential equation is transformed into its  normal form 
	 \[ w'' + k^2 w = \frac{x\sin x}{x}  = \sin x,\] whose general solution, by the method of undetermined coefficients or variation of parameters, is
	 \[w=c_1 \cos (k x)+c_2 \sin (k x)+\frac{\sin x}{k^2-1},\]
	 where $c_1$ and $c_2$ are arbitrary constants. Hence the general solution of the given differential equation is
	 \[y = x w(x) = c_1 x \cos (k x)+c_2 x \sin (k x)+\frac{x\sin x}{k^2-1},\]
	 where $c_1$ and $c_2$ are arbitrary constants.
\end{solution}

\subsection{Change of the Independent Variable}\label{subsec:4-5-1}

By changing the independent variable in the  equation 
\begin{equation}\label{eq:4-5-8}
	y''+ a_1(x) y' + a_0(x)y = f(x)
\end{equation}from $x$ to $t$ via a one-to-one function $t= h(x)$ which is differentiable with the differentiable inverse $x= h^{-1}(t),$ we may be able to transform \eqref{eq:4-5-8} into a differential equation which may be solved using the methods that are previously discussed. To see how such a change of the independent variable works, we note
\[\frac{dt}{dx} = h'(x)\]  and compute, by the chain rule, the derivatives
\begin{equation}\label{eq:4-5-9}
	\frac{dy}{dx} = \frac{dy}{dt}\frac{dt}{dx} = \frac{dy}{dt}h'(x)
\end{equation}
and 
\begin{equation}\label{eq:4-5-10}
	\frac{d^2y}{dx^2} = \frac{d}{dt}\left(\frac{dy}{dt}h'(x)\right)\frac{dt}{dx}   =\frac{d^2y}{dt^2} \left(h'(x)\right)^2+ h''(x) \frac{dy}{dt}.
	\end{equation}
Substituting these into \eqref{eq:4-5-8} gives
\begin{equation}\label{eq4-5-11}
	\frac{d^2y}{dt^2} \left(h'(x)\right)^2+ \left(h''(x) +a_1(x) h'(x)\right)\frac{dy}{dt}+ a_0(x)y= f(x),
\end{equation}
 where \(x = h^{-1}(t).\)  
  The substitution $t = h(x)$ is beneficial only if it transforms \eqref{eq4-5-11} into an equation solvable by a known method.
 
 To illustrate the effectiveness of changing the independent variable, we present two examples. The first involves a second order \textbf{Cauchy–Euler} equation, and the second is the \textbf{Legendre equation of order $n.$}
 Here, the term \textit{order} refers to something different from the order of the differential equation itself. This distinction in the use of the same terminology is important to keep in mind.


\begin{example}[Cauchy-Euler]
	Find the general solution of
the Cauchy-Euler equation
\begin{equation}\label{eq:4-5-11}
	x^2 y''+3xy'+y = x
\end{equation}    by solving the transformed equation that is obtained from  the change of variable $t= \ln x.$
\end{example}
\begin{solution}
	Using $t = h(x) = \ln x$ in \eqref{eq:4-5-9} and \eqref{eq:4-5-10}, we find
	\[\frac{dy}{dx} = \frac{dy}{dt}h'(x) =\frac{1}{x}\frac{dy}{dt}\]
	 and 
	 \[	\frac{d^2y}{dx^2}  =\frac{d^2y}{dt^2} \left(h'(x)\right)^2+ h''(x) \frac{dy}{dt} = \frac{1}{x^2} \left(\frac{d^2y}{dt^2}-\frac{dy}{dt} \right).\]
	 Substituting these into \eqref{eq:4-5-11} gives
	 \[\frac{d^2y}{dt^2}+2\frac{dy}{dt}+  y= e^t.\]
	 The general solution of this equation is
	 \[y= c_1 e^{-t} + c_2 t e^{-t} + \frac{1}{4} e^t,\] where $c_1$ and $c_2$ are arbitrary constants.
	 Since $t=\ln x,$  it follows that the general solution of \eqref{eq:4-5-11} is
	  \[y= \frac{c_1}{x} + \frac{c_2 e^x}{x} + \frac{x}{4},\]
	  where $c_1$ and $c_2$ are arbitrary constants.
\end{solution}

\begin{example}[Legendre equation]
Transform the Legendre equation 
\begin{equation}\label{eq:4-5-12}
y''+ (\cot x) \;y' +n(n+1)y=0
%(1-x^2)y''-2xy'+n(n+1)y = 0,
\end{equation}
of order $n$ (a real number)   using the change of variable \(t = \cos x.\)
\end{example}
\begin{solution}
	Using $t = h(x) = \cos x$ in \eqref{eq:4-5-9} and \eqref{eq:4-5-10}, we find
	\[\frac{dy}{dx} = \frac{dy}{dt}h'(x) =-\sin x\frac{dy}{dt}\]
	and 
	\[	\frac{d^2y}{dx^2}  =\left(h'(x)\right)^2\frac{d^2y}{dt^2} + h''(x) \frac{dy}{dt} = \sin^2x \frac{d^2y}{dt^2} -\cos x\frac{dy}{dt} .\]
	Substituting these into \eqref{eq:4-5-11} gives
	\[\sin^2 x\frac{d^2y}{dt^2}-\cos x\frac{dy}{dt} +\cot x (-\sin x)\frac{dy}{dt}+  n (n+1)y= 0,\] which simplifies to 
		\[\sin^2 x\frac{d^2y}{dt^2}-2\cos x\frac{dy}{dt} +n (n+1)y= 0.\]  Using $t= \cos x$ gives the standard form of the Legendre equation
		\begin{equation*}\label{eq:4-5-13}
			(1-t^2)\frac{d^2y}{dt^2}- 2t\frac{dy}{dt}+ n(n+1)y = 0.\qedhere
		\end{equation*}
\end{solution}

\begin{remark}\(\empty\)
	\begin{enumerate}[label=(\roman*),noitemsep]
		\item Since the invariant
		\[ I_y(t) =\frac{n(n+1)}{1-t^2} - \frac{1}{4}\left(\frac{-2t}{1-t^2}\right)^2 - \frac{1}{2}\left(\frac{-2t}{1-t^2}\right) =\frac{n(n+1)}{1-t^2}+\frac{1}{(1-t^2)^2}\] of \eqref{eq:4-5-12} is not a constant,  the transformed equation in $w$ by the change of dependent variable 
		\(y(t) = u(t)w(t)\) with 
		\[u(t) = e^{ -\frac{1}{2}\int \left(\frac{-2t}{1-t^2}\right)\, dt}\]   does not have  constant coefficients.  Likewise,  the invariant $I_y(x)$ of \eqref{eq:4-5-11} for the change of variable $y(x) = u(x) w(x)$ with
		\[u(x) = e^{ -\frac{1}{2}\int \cot x\, dx}\] is
		\[ I_y(x) =n(n+1) - \frac{1}{4}\cot^2x + \frac{1}{2}\csc x\cot x \]
		which is not a constant, and therefore the transformed equation in $w$ does not have constant coefficients.  A standard approach for solving the Legendre equation is the  method of power series solutions, which will be discussed in Section~\ref{sec:power-series}.

	\item The Legendre equation \eqref{eq:4-5-13} lies in an important class known as \textbf{Sturm-Liouville problems}\index{Sturm-Liouville problem}, which are parameter dependent problems of the form
	\begin{equation}\label{eq:Sturm–Liouville}
		\big( a_1(x)\, y'\big)'+ \big( \lambda\, w(x) - a_0(x) \big) y = 0,
	\end{equation} $\lambda$ being a parameter (called an eigenvalue) and $p, q, r$  known functions of $x.$  In fact, \eqref{eq:4-5-13} can be written as
	\[\big((1-x^2)y'\big)'+n(n+1)y=0,\] which is of the form \eqref{eq:Sturm–Liouville} with \(a_1(x)= 1-x^2,\) \(w(x) =1,\) \(a_0(x)= 0,\) and \(\lambda = n(n+1).\) 
	A detailed discussion of Sturm–Liouville problems, however, lies beyond the scope of this book.
	\end{enumerate}
	\end{remark}

We now discuss a useful change of dependent variable for the equation 
\begin{equation}\label{eq:4-5-80}
	y''+ a_1(x) y' + a_0(x)y = f(x)
\end{equation}
given by \[t = h(x) = \int\sqrt{a_0(x)}\, dx.\] It is clear that 
\[h'(x) = \sqrt{a_0(x)} \quad \text{ and }\quad h''(x) = \frac{a'_0(x)}{2\sqrt{a_0(x)}}.\] Using this change of variable in \eqref{eq4-5-11},  we get
\[a_0(x)\frac{d^2y}{dt^2}+\left(\frac{a'_0(x)+2 a_0(x) a_1(x)}{2\sqrt{a_0(x)}}\right)\frac{dy}{dt} + a_0(x) y=0,\] where $x= h^{-1}(t).$ The above equation can be rewritten as 
\begin{equation}\label{eq:4-5-14}
	\frac{d^2y}{dt^2}+\left(\frac{a'_0(x)+2 a_0(x) a_1(x)}{2{\big(a_0(x)\big)}^{3/2}}\right)\frac{dy}{dt}  +  y=0.
\end{equation} The  equation \eqref{eq:4-5-14} can be readily solved when 
\[\frac{a'_0(x)+2 a_0(x) a_1(x)}{2\big(a_0(x)\big)^{3/2}} =\text{ a constant}.\]
 The arguments made above make sense only for those $a_0(x)$ in \eqref{eq:4-5-80} for which these considerations are meaningful.


\begin{Exercise}\label{EX37}
	\vspace{-\baselineskip}% <-- You don't need this line of code if there's some text here
	
	\Question\label{3-7-1}
Find the general solution of each differential equation below. A solution to the associated homogeneous equation is provided.
	
	\begin{tasks}(1)
		\task \(x^2 y''+x y'-4 y=x; \quad u(x) = x^2\)
		\task \(x^3 y''+x^2 y'-4 x y=1+x; \quad u(x) = x^2\)
		\task \(x^4 y''+x^3 y'-4 x^2 y=1;\quad u(x) = x^2 \)
		\task \((x^2-1)y''-x y'+y=1; \quad u(x) = x\)
		\task \((\tan^2x)y''-(2 \tan x) y'+ \left(\sec ^2x+1\right)y=\sin^2x \, \tan x; \quad u(x) = \sin x\)
		
		
	\end{tasks}
	
	\Question\label{3-7-2}
		Using the techniques discussed in the this section, find the general solution of each differential equation below.
		\begin{tasks}(1)
		\task \(y''+6y'+9y = e^{-3x}\)
		\task \(y''+6y'+9y = xe^{-3x}\)
		\task \(y''+6 y'+9 y=x^2 e^{-3x}\)
		\end{tasks}
		
	\Question\label{3-7-3} Find the general solution of 
	\[xy''-2(x-1)y'+2(x-1)y =e^x\]
by using its normal form. 

	\Question\label{3-7-4}
Find the normal form of each differential equation below.
\begin{tasks}(1)
	\task \(y''-2xy'+2ny=0,\) \quad $n$ being a constant
	\task \(xy''+(1-c)y'+y=0,\) \quad$c$ being a constant
	\task \(y''+xy'+(x^2-n^2)y=0,\) \quad$n$ being a constant
\end{tasks}

\Question\label{3-7-5}
Find the general solution of each equation below by using the change of variable $t= h(x) = \int \sqrt{a_0(x)}\, dx$ that results  in \eqref{eq4-5-11}.
\begin{tasks}(1)
	\task \(y''+ (\tan x) y'+(\cos^2x)y =0\)
	
	\task \(x^4y''+x^2(2x-3)y'+2u =0\) 
	\task \(2xy''+(5x^2-2)y'+2x^3y=0\) 
\end{tasks}

\end{Exercise}

\setboolean{firstanswerofthechapter}{true}
\begin{multicols}{2}\scriptsize
	\begin{Answer}[ref={EX37}]
		\Question \label{3-7-1a}
		\begin{tasks}
			\task \(y= c_1 x^2+\frac{c_2}{x^2}-\frac{x}{3}\)
			\task \(y =c_1 x^2+\frac{c_2}{x^2}-\frac{1}{3 x}-\frac{1}{4}\)
			\task \(y= c_1 x^2+\frac{c_2}{x^2}-\frac{\ln\abs{x}}{4 x^2}\)
			\task \(y= 1+ c_1x\\+c_2 \left(-\sqrt{x^2-1}-x \ln \left|\sqrt{x^2-1}-x\right|\right)\)
		\end{tasks} 
		
		\Question \label{3-7-2a}	
		\begin{tasks}
			\task \(y=c_1 e^{-3 x}+c_2x e^{-3 x} +\frac{1}{2} e^{-3 x} x^2\)
				\task \(y=c_1 e^{-3 x}+c_2x e^{-3 x} +\frac{1}{6} e^{-3 x} x^3\)
				\task  \(y=c_1 e^{-3 x}+c_2 e^{-3 x} x+\frac{1}{12} e^{-3 x} x^4\)
				
		\end{tasks}
	\end{Answer}
\end{multicols}
\setboolean{firstanswerofthechapter}{false}


				
%\include{chap04/Systems of Differential Equations}             %%% Open this one 
%
\setcounter{chapter}{4}
\chapter{The Laplace Transform and Applications}\label{ch:Laplace Transform}
	

\mtcsetoffset{minitoc}{-1em}
% \mtcsetpagenumbers{minitoc}{off}

\minitoc %This TOC should only show the two subsubsection below!

\section{The Laplace Transform}\label{chap7:section1}



Consider a forced oscillator, as discussed in Section~\ref{subsection:spring-mass} of Chapter~\ref{chap:higher-order}, described by the equation  
\begin{equation}\label{chap7:01}
	y''(t) + \omega^2 y(t) = f(t); \quad  y(0) = 0,\; y'(0)= 0,
\end{equation}  
where \( f(t) \) is an impulsive  force. At \(t =0,\) the mass is at rest at its equilibrium position. At time \( t = a \), the mass experiences a sharp, brief downward impact, such as a hammer strike, modeled by an the impulsive force \( f(t) .\) 
An impulsive force of this nature is commonly modeled by using an idealized mathematical object, called the   \textbf{\textit{Dirac delta}} ``function'' denoted by $\delta (t-a),$ which is characterized by the following  properties:

\begin{enumerate}[label= (\roman*)]
	\item \( \delta(t - a) = 0 \) for all \( t \ne a \); and
%	\item \( \int_{-\infty}^\infty \delta(t - a)\, dt = 1 \); and
	\item \( \int_{-\infty}^\infty \delta(t - a)\, g(t)\, dt = g(a) \) for any function \( g \) continuous on an open interval containing \( a \).
\end{enumerate}
We note  that when $g(t) =1$ in (ii), we obtain 
\( \int_{-\infty}^\infty \delta(t - a)\, dt = 1 \).
Based on the  properties (i) and (ii) characterizing \( \delta(t - a) \), it is important to recognize that \( \delta(t - a) \) is not a  function in the usual sense; rather it is a typical example of \textit{generalized functions} or \textit{distributions} which were first developed rigorously  by Sergei Sobolev in 1936 and later in 1950s by Laurent Schwartz who is known to have given the most definitive and systematic development of the theory of distributions. For an elegant introduction to  distribution theory with application, the reader is referred to  Zemanian\footnote[1]{A.~H. Zemanian, {\it Distribution theory and transform analysis}, second edition, Dover, New York, 1987; MR0918977}.

 The Dirac delta function is particularly useful for modeling  phenomena involving instantaneous or localized effects, such as a lightning strike, a point heat explosion, or the sudden burst of dye at a specific location. Because of the properties (i) and (ii), the Dirac delta function \( \delta(t - a) \) is said to be concentrated at $t=a.$ 


Mathematically,  \( \delta(t - a) \) can be interpreted as the limit  of a sequence of approximating impulse functions.  One such approximation is given by  
 \[
 d_{a, \varepsilon}(t) = \begin{cases}
	 \dfrac{1}{\varepsilon} &\text{if } a \le t \le a + \varepsilon \\
	0 &\text{otherwise},
	 \end{cases}
 \]
 which represents a  pulse of height \( 1/\varepsilon \) and width \( \varepsilon>0 \), centered at \( t = a \). 
 
% \textcolor{red}{Open this figure later}
 \begin{figure}[h]
 	\centering
 	\begin{tikzpicture}[scale=1.5]
 		% Axes
 		\draw[->] (-0.5,0) -- (4,0) node[right] {$t$};
 		\draw[->] (0,-0.2) -- (0,2) node[above] {$d_{a,\varepsilon}(t)$};
 		
 		% Parameters
 		\def\a{1}
 		\def\eps{0.5}
 		\def\height{1.5} % 1 / eps
 		
 		% Rectangle function
 		\draw[thick, blue] (0,0) -- (\a,0)
 		-- (\a,\height) -- ({\a+\eps},\height)
 		-- ({\a+\eps},0) -- (4,0);
 		
 		% Labels
 		\draw[dashed] (\a,0) -- (\a,\height);
 		\draw[dashed] ({\a+\eps},0) -- ({\a+\eps},\height);
 		
 		\node[below] at (\a,0) {$a$};
 		\node[below] at ({\a+\eps},0) {$a+\varepsilon$};
 		\node[left] at (0,\height) {$\frac{1}{\varepsilon}$};
 		
 	\end{tikzpicture}
 \end{figure}
 

 
 As \(\varepsilon \to 0 \), this pulse becomes increasingly narrow and tall, concentrating its effect at \( t = a \), i.e.,  
 \begin{equation}\label{chap7:02}
 	\lim\limits_{\varepsilon\to 0} d_{a, \varepsilon}(t) = \begin{cases}
 	\infty &\mbox{if } t=a\\
 	0&\text{otherwise},
 \end{cases}\end{equation}
 while maintaining a total area of 1—mirroring the property (ii) of \( \delta(t - a) \), i.e.,
 \[\int_{-\infty}^\infty d_{a, \varepsilon}(t) \, dt = \frac{1}{\varepsilon} \varepsilon=1,\]
 so that 
 \begin{equation}\label{chap7:03}
 	\lim\limits_{\varepsilon\to 0} \int_{-\infty}^\infty d_{a, \varepsilon}(t) \, dt =1.
 	\end{equation}
 Furthermore, if $g$ is continuous from right at $t=a$,  then
  \begin{equation}\label{chap7:04}
 	\lim\limits_{\varepsilon\to 0} \int_{-\infty}^\infty d_{a, \varepsilon}(t) \, g(t) = 	\lim\limits_{\varepsilon\to 0}\frac{1}{\varepsilon} \int_a^{a+\varepsilon}  g(t) \, dt = \lim\limits_{\varepsilon\to 0} g(a+\varepsilon) = g(a)
 \end{equation}
by using the l'H\"opital's rule and the continuity of $g$ at $t=a$ from right. 

 By formally interchanging the limit and the integral in both (\ref{chap7:03}) and (\ref{chap7:04}), an operation which needs justification  in integration theories, we obtain  
\begin{equation}\label{chap7:05}
	\int_{-\infty}^\infty\lim\limits_{\varepsilon\to 0} d_{a, \varepsilon}(t)\, dt = 1
\end{equation}  
and 
\begin{equation}\label{chap7:06}
	 \int_{-\infty}^\infty \lim\limits_{\varepsilon\to 0} d_{a, \varepsilon}(t) \, g(t)\, dt  = g(a).
	\end{equation}
 Thus, we observe in (\ref{chap7:02}) and (\ref{chap7:06})  that  
 \( \lim\limits_{\varepsilon \to 0} d_{a, \varepsilon}(t) \) exhibits the same  properties (i) and (ii) that characterize \( \delta(t - a) \), and therefore \( \delta(t - a) \) can be interpreted as the limit of a sequence of functions that approximate an impulsive behavior.   
We recognize that 
\(
\lim\limits_{\varepsilon \to 0} d_{a, \varepsilon}(t)
\)
also does not define a function on $(-\infty, \infty)$ in the classical sense because its value at $t=a$ is $\infty.$ Instead, the limit is understood as a generalized function, or a distribution, because of the way it operates primarily through its action on other functions within integrals, as described by the property (ii) and motivated by (\ref{chap7:06}).
 


   By using the property (ii), we  find that for all $s$ in $(-\infty,  \infty),$ we have 
\begin{equation}\label{chap7:07}
	\int_{-\infty}^\infty \delta(t - a)\, e^{-st}\, dt = e^{-as} \end{equation}
for all real numbers $s.$
From this point onward, we will focus on the case where \( a \ge 0 \) and \( t \ge 0 \). Under this assumption,  (\ref{chap7:07}) reduces to  
 \begin{equation}\label{chap7:08}
	 \int_0^\infty \delta(t - a)\, e^{-st}\, dt = e^{-as}
	 \end{equation}
 for all real values of \( s \).  
 This integral is known as the \textbf{\textit{Laplace transform}} (see Definition~\ref{chap7:Laplace})of the Dirac delta function \( \delta(t - a) \),  denoted by  
 \[
 \mathscr{L}\{\delta(t - a)\} = e^{-as}.
 \]  
 In particular, when $a=0$ we have 
 \[
 \mathscr{L}\{\delta(t)\} = 1.
 \]
 
 

 The main objective of this chapter is to develop a method for solving initial value problems such as (\ref{chap7:01} in which $f(t)$ is either discontinuous or involves the Dirac delta functions.The methods of undetermined coefficients and  variation of parameters are not relevant to (\ref{chap7:01}) with $f(t) = \delta (t-a)$  because the calculus of distributions is beyond the scope of this book.   In particular,  we do not have a meaning of $y''$ in this case.
 
 Being motivated from  (\ref{chap7:08}) with the Dirac delta function \(\delta(t - a)\) in the integrand, we first define the Laplace transforms of classical functions defined on $[0, \infty).$

\begin{definition}[Laplace Transform]\label{chap7:Laplace}
Let $f:[0, \infty)\to\mathbb R$ be a function. The Laplace transform  of $f$  is a function $F(s)$, denoted commonly by $\mathscr{L}\{f(t)\}$, defined by
 \[F(s):=\int_0^\infty f(t)\, e^{-st} \, dt\]  for those $s$ for which the integral converges.
\end{definition}






The symbol $\mathscr{L}$ represents a transformation that maps an input function $f$ of the variable $t$ (usually time) to an output function $F$ of the variable $s$ (usually involving frequency when 
$s$ is a complex number). This action of the transformation is depicted in the figure below.


	
%	\begin{tikzpicture}[node distance=3cm, auto]
%		% Nodes
%		\node (input) [draw, rectangle, minimum height=1cm, minimum width=2.5cm] {\(f(t)\)};
%		\node (L) [draw, circle, right of=input, node distance=3.5cm] {\(\mathscr{L}\)};
%		\node (output) [draw, rectangle, right of=L, node distance=3.5cm, minimum height=1cm, minimum width=2.5cm] {\(F(s)\)};
%		
%		% Arrows
%		\draw[->, thick] (input) -- (L);
%		\draw[->, thick] (L) -- (output);
%		
%		% Labels
%		\node[below of=input, node distance=.75cm] {$t$ domain};
%		\node[below of=output, node distance=.75cm] {$s$ domain};
%	\end{tikzpicture}
	
	% \textcolor{red}{Open this figure later}
	\begin{center}
	\begin{tikzpicture}[node distance=2.5cm, thick]
		% Draw the box
		\node[draw, minimum width=2cm, minimum height=1cm, align=center] (laplace) {$\mathscr{L}$};
		
		% Input arrow and label
		\draw[->] ([xshift=-2cm]laplace.west) -- (laplace.west) node[midway, above] {$f(t)$};
		
		% Output arrow and label
		\draw[->] (laplace.east) -- ([xshift=2cm]laplace.east) node[midway, above] {$F(s)$};
	\end{tikzpicture}
\end{center}


The variable \( s \) in the definition of the Laplace transform can, in general, be a complex variable; however, in the current chapter,  we will restrict  \( s \) to a real variable.


 Before we begin using Laplace transforms to solve initial value problems, we  compute the Laplace transforms of several common functions that appeared in the previous chapters.

\begin{example}
Compute $\mathscr{L}\{e^{at}\}$ for $a\ne 0.$
\end{example}
\begin{solution}
	By Definition~\ref{chap7:Laplace}, we have
\begin{equation*}
	\begin{split}
		\mathscr{L} \bigl\{e^{at}\bigr\}
		=\int_0^\infty e^{at} \,e^{-st}  \, dt
		&= \lim_{h\to\infty}\int_0^h e^{(a-s)t} \, dt\\
	 	&=\lim_{h\to\infty}\left[ \frac{e^{(a-s)t}}{a-s} \right]_0^h\\
	 	&=\lim_{h\to\infty}\frac{e^{(a-s)h}}{a-s}- \frac{1}{a-s}\\
		& = \frac{1}{s-a}
		\end{split}
\end{equation*} for all $s>a.$ The improper integral converges only for $s>a.$ Thus, 
\[\boxed{\mathscr{L} \bigl\{e^{at}\bigr\}= \frac{1}{s-a}\quad \text{ for all } s>a.}\qedhere\]
\end{solution}


\begin{example}
	Compute $\mathscr{L}\{1\}.$ 
\end{example}
\begin{solution}
	By Definition~\ref{chap7:Laplace}, we have
	\begin{equation*}
		\mathscr{L} \{1\} = \int_0^\infty e^{-st} \, dt
		=
		\lim_{h\to\infty}
		\left[ \frac{e^{-st}}{-s} \right]_0^h
		=
		\lim_{h\to\infty}
		\left( \frac{e^{-sh}}{-s} - \frac{1}{-s} \right)
		= \frac{1}{s} 
	\end{equation*}
	for $s>0.$ The improper integral converges only for $s>0.$ 
	Thus, 
	\[\boxed{\mathscr{L} \{1\} = \frac{1}{s}\quad \text{ for all } s>0.}\qedhere\]
\end{solution}
In the sequel, the expression \(
\left[ g(t)\right]_0^\infty\) will be tacitly used for
\(\lim\limits_{h\to\infty}
\left[ g(t)\right]_{0}^h.\)
 
 
 
 \begin{example}
 Compute $\mathscr{L}\{t\} .$
 \end{example}
 \begin{solution}
 
 	By Definition~\ref{chap7:Laplace}, we have
\begin{equation*}
	\begin{split}
		\mathscr{L} \{t\}
		& = \int_0^\infty t\,e^{-st}  \, dt \\
		& =
		\left[ \frac{-te^{-st}}{s} \right]_0^\infty
		+
		\frac{1}{s}
		\int_0^\infty e^{-st} \,dt \qquad \text{(integration  by parts)} \\
		& =
		0
		+
		\frac{1}{s}
		\left[ \frac{e^{-st}}{-s} \right]_0^\infty \\
		& =
		\frac{1}{s^2}
	\end{split}
\end{equation*}
 for $s>0.$ The improper integral converges only for $s>0.$ Thus, 
 \[\boxed{\mathscr{L} \{t\} = \frac{1}{s^2}\quad \text{ for all } s>0.}\qedhere\]
\end{solution}


\begin{example}
	Compute $\mathscr{L}\{t^2\} .$
\end{example}
\begin{solution}
	By Definition~\ref{chap7:Laplace}, we have
		\[
	\mathscr{L}\{t^2\} = \int_0^\infty e^{-st} t^2 \, dt.
	\] 
To use integration by parts, put \(
	u = t^2\)  and \(dv = e^{-st} dt\). Then \( du = 2t\,dt\) and 
	\(v = -\frac{1}{s}e^{-st}\). Then
\begin{equation*}
	\begin{split}
	\mathscr{L}\{t^2\} = \int_0^\infty t^2 e^{-st} dt
						&= \left[ -\frac{1}{s} t^2 e^{-st} \right]_0^\infty - \int_0^\infty \left(-\frac{1}{s}e^{-st}\right) 2t dt\\
						&=-\frac{1}{s^2}\lim_{t\to\infty}t^2 e^{-st} + 0+ \frac{2}{s}\int_0^\infty t e^{-st}\, dt\\
						&=\frac{2}{s} \mathscr{L}\{t\}\\
						&= \frac{2}{s^3}
	\end{split}
\end{equation*}
for $s>0$. Here, we used the limit
\[\lim_{t\to\infty}t^2 e^{-st} =0\] which can be evaluated by using the l'H\"opital's rule. Also, the improper integral defining \(\mathscr{L}\{t^2\}\) converges only for $s>0.$ Thus, 
\[\boxed{\mathscr{L} \{t\} = \frac{2}{s^3}\quad \text{ for all } s>0.}\qedhere\]
\end{solution}


\begin{example}
	Compute $\mathscr{L}\{t^3\} .$
\end{example}
\begin{solution}
By repeatedly using integration by parts and the l'H\"opital's rule, we obtain
\[\boxed{\mathscr{L} \{t^3\} = \frac{3!}{s^4}\quad \text{ for all } s>0.}\qedhere\]
\end{solution}

The computation of \(\mathscr{L}\{t^n\}\) for any positive integer $n$ follows the same procedure as shown in the next example.

\begin{example}
	Compute $\mathscr{L}\{t^n\} $ for any positive integer $n.$
\end{example}
\begin{solution}
	By integration by parts and the l'H\"opital's rule, we obtain
	\[\mathscr{L} \{t^n\} = \frac{n}{s}{\mathscr{L} \{t^{n-1}\}}\quad \text{ for all } s>0.\]
	Using this recursive relation, we obtain
	\[\begin{split}
		\mathscr{L} \{t^n\} &= \frac{n}{s}\left(\frac{n-1}{s}\right){\mathscr{L} \{t^{n-2}\}}\\
		&= \frac{n}{s}\left(\frac{n-1}{s}\right)\left(\frac{n-2}{s}\right){\mathscr{L} \{t^{n-3}\}}\\
		&\quad \vdots\\
			&= \frac{n}{s}\left(\frac{n-1}{s}\right)\left(\frac{n-2}{s}\right)\cdots \frac{3}{s}\; \frac{2}{s}{\mathscr{L} \{t\}}\\
			&=\frac{n}{s}\left(\frac{n-1}{s}\right)\left(\frac{n-2}{s}\right)\cdots \frac{3}{s}\; \frac{2}{s}\;\frac{1}{s^2}\\
			&=\frac{n!}{s^{n+1}}
	\end{split}
	\]
	for $s>0$.  As in the previous cases, the improper integral defining \(\mathscr{L}\{t^n\}\) converges only for $s>0.$ Thus, 
	\[\boxed{\mathscr{L} \{t^n\} = \frac{n!}{s^{n+1}}\quad \text{ for all } s>0.}\qedhere\]
\end{solution}

\begin{example}
	Compute $\mathscr{L}\{\sin(kt)\}$ for any fixed constant $k.$
\end{example}
\begin{solution}

	By the definition of  the Laplace transform of \( \sin(kt) \), where \( k \) is a constant, we have
	
	\[
	\mathscr{L}\{\sin(kt)\} = \int_0^\infty e^{-st} \sin(kt) \, dt.
	\]
By using integration by parts,  we have 
\[
\int e^{-st} \sin(kt) \, dt = -\frac{e^{-st}}{s^2 + k^2}\left(s \sin(kt) + k \cos(kt)\right)   + C,
\]
where $C$ is an arbitrary constant.
Then 

\[
\int_0^\infty e^{-st} \sin(kt) \, dt =-\left[\frac{e^{-st}}{s^2 + k^2}\left(s \sin(kt) + k \cos(kt)\right) \right]_0^\infty.
\]
Since $e^{-st} \to 0$  as $t\to\infty$ and $s \sin(kt) + k \cos(kt)$ is bounded, we have
\[
\lim_{t \to \infty} e^{-st}\left( s \sin(kt) + k \cos(kt) \right)  = 0\] for $s>0.$ The improper integral converges only for $s>0$ when $k\ne 0.$
Therefore, we have
\[
\boxed{\mathscr{L}\{\sin(kt)\} = \frac{k}{s^2 + k^2} \quad \text{ for all } s>0.}\qedhere
\]
\end{solution}


\begin{example}
	Compute $\mathscr{L}\{\cos(kt)\}$ for any fixed constant $k.$
\end{example}
\begin{solution}
	By using integration by parts,  we have 
	\[
\int e^{-st} \cos(kt) \, dt = \frac{e^{-st}}{s^2 + k^2} \left( -s \cos(kt) + k \sin(kt) \right) + C,
	\]
	where $C$ is an arbitrary constant.
	Then 
	
	\[
	\int_0^\infty e^{-st} \cos(kt) \, dt =\left[\frac{e^{-st}}{s^2 + k^2}\left(-s \cos(kt) + k \sin(kt)\right) \right]_0^\infty.
	\]
	Since $e^{-st} \to 0$  as $t\to\infty$ and $-s \cos(kt) + k \sin(kt)$ is bounded, we have
	\[
	\lim_{t \to \infty} e^{-st}\left( -s \cos(kt) + k \sin(kt) \right)  = 0\] for $s>0.$ The improper integral converges only for $s>0.$ 
	Therefore, 
	we obtain
	\[
	\boxed{\mathscr{L}\{\cos(kt)\} = \frac{s}{s^2 + k^2}} \quad \text{ for all } s>0.\qedhere
	\]
\end{solution}
A simple but useful example of a discontinuous function is
the \textbf{\textit{unit step function}} $\mathcal{U}$, also  known as the \textbf{\textit{Heaviside}} function (named in honor of the English mathematician, engineer, and physicist Oliver Heaviside (1850--1925)). It is defined on $(-\infty, \infty)$ by
\[
\mathcal{U}(t) =
\begin{cases}
	0 & \text{if } t < 0, \\
	1 & \text{if } t \ge 0.
\end{cases}
\]

%\textcolor{red}{Open this figure later}
\begin{figure}[h]
	\centering
	\begin{tikzpicture}[scale=.75]
		\tkzInit[xmin=-1,xmax=4,ymin=-0.5,ymax=1.5]
		\tkzSetUpAxis[line width=1pt,tickwd=0pt,ticka=0pt,tickb=0pt]
		\tkzDrawXY
		
		% Draw U(t) = 0 for t < 0
		\draw[thick, blue] (-1,0) -- (0,0);
		\filldraw[blue,thick] (0,0) circle (2pt); 
		\filldraw[white] (0,0) circle (1.5pt); % open circle at t = 0
		
		% Draw U(t) = 1 for t >= 0
		\draw[thick, blue] (0,1) -- (4.5,1);
		\filldraw[blue] (0,1) circle (2pt); % filled circle at t = 0
		
		% Label the step
		\node[above right] at (0.1,1) {\(\mathcal{U}(t)\)};
	\end{tikzpicture}
	\caption{The unit step function \(\mathcal{U}(t)\)}
\end{figure}
The unit step function with the jump discontinuity at $t= a$ is given by 
\[
\mathcal{U}(t-a) =
\begin{cases}
	0 & \text{if } t < a, \\
	1 & \text{if } t \ge a,
\end{cases}
\]
and its graph is shown below.
%\textcolor{red}{Open this figure later}
\begin{figure}[h]
	\centering
	\begin{tikzpicture}[scale=.9]
		\tkzInit[xmin=-1,xmax=4,ymin=-0.5,ymax=1.5]
		  \tkzSetUpAxis[line width=1pt,tickwd=0pt,ticka=0pt,tickb=0pt]
		\tkzDrawXY
		
		% Draw U(t) = 0 for t < 0
		\draw[thick, blue] (-1,0) -- (2,0);
		\filldraw[blue,thick] (2,0) circle (2pt); 
		\filldraw[white] (2,0) circle (1.5pt); % open circle at t = 0
		
		% Draw U(t) = 1 for t >= 0
		\draw[thick, blue] (2,1) -- (4.5,1);
		\filldraw[blue] (2,1) circle (2pt); % filled circle at t = 0
		
		% Label the step
		\node[above right] at (2.1,1) {\(\mathcal{U}(t-a)\)};
		\node[above right] at (1.7,-0.6) {\(a\)};
	\end{tikzpicture}
	\caption{The unit step function \(\mathcal{U}(t-a)\)}
\end{figure}


\begin{example}\label{chap7:example-heaviside}
	Compute $\mathscr{L}\{\mathcal{U}(t-a)\}$ for  $a\ge 0.$
\end{example}

\begin{solution}
	By Definition~\ref{chap7:Laplace}, we have
	\begin{equation*}
		\mathscr{L} \bigl\{ \mathcal{U}(t-a) \bigr\}
		=
		\int_0^{\infty} e^{-st} \mathcal{U}(t-a) \, dt
		=
		\int_a^{\infty} e^{-st} \, dt
		=
		\left[ \frac{e^{-st}}{-s} \right]_a^\infty \\
		=
		\frac{e^{-as}}{s} 
	\end{equation*}
	 for $s>0.$   Also, the improper integral defining \(\mathscr{L}\{\mathcal{U}(t-a)\}\) converges only for $s>0.$ Thus, 
	 \[\boxed{\mathscr{L} \{\mathcal{U}(t-a)\} = \frac{e^{-as}}{s} \quad \text{ for all } s>0.}\qedhere\]
\end{solution}


To model situations in applications where an external force or source begins acting only after time \( t = a \), we analyze the effect of shifting the source function along the \( t \)-axis.



Suppose the function \( f(t) \) is intended to take effect only after time \( t = a \). In that case, we define a new function \( g(t) \) as  
\[
g(t) =\begin{cases}
	f(t-a) & \text{if } t \ge a,\\
	0 & \text{if } t < a,
\end{cases}
\]
whose graph is shown below in comparison to the graph of $f(t).$

%\textcolor{red}{Open this figure}
\begin{center}
	\begin{tikzpicture}[scale=.9]
		\begin{axis}[
			axis lines = middle,
			xlabel = $t$, ylabel = $y$,
			xmin = -1, xmax = 17,
			ymin = -0.5, ymax = 5,
			samples = 200,
			xtick = \empty,
			ytick = \empty,
			legend style={at={(0.5,-0.15)}, anchor=north},
			grid = both
			]
			% Original function f(t)
			\addplot[blue, thick, domain=0:10] {1+sin(deg(x))+0.25*x};
			%\addlegendentry{$y= f(t)$}
		% Draw U(t) = 0 for t < 0
			
	
			\filldraw[red,thick] (2,0) circle (2pt); 
			\filldraw[white] (2,0) circle (1.5pt);
			
			%\draw[thick, red] (-3,0) -- (0,0);
			\filldraw[blue,thick] (0,1) circle (2pt); 
			\filldraw[red,thick] (2,1) circle (2pt); 
			%\filldraw[white] (2,1) circle (1.5pt);
			
			% Shifted function g(t) = f(t-a) for t >= a, else 0
			\addplot[red, thick, domain=0:2] {0};
			\addplot[red, thick, domain=2:10] {1+sin(deg(x - 2))+0.25*(x-2)};
			\node[above right] at (4.7,3) {\(f(t)\)};
			\node[above right] at (6.4,.8) {\(g(t) =\mathcal{U}(t-a)\,f(t-a)\)};
			\node[above right] at (1.7,-0.6) {\(a\)};

			%\addlegendentry{$g(t) = f(t - a)$ for $t \ge a$}
		\end{axis}
	\end{tikzpicture}
\end{center}
It is clear that $g(t) = \mathcal{U}(t-a) f(t-a)$ for $t\ge 0.$



\begin{example}[Shifting on the $t-$axis]\label{shiting-on-t-axis}
	 Suppose \( F(s) \) is the Laplace transform of \( f(t) \). Compute  \(\mathscr{L} \{\mathcal{U}(t - a)\, f(t - a) \}\) where \( a \ge 0 \).
\end{example}
\begin{solution}
By the definition of the Laplace transform, we have
	\[
	\mathscr{L}\{\mathcal{U}(t - a)\, f(t - a)\} = \int_0^\infty e^{-st} \mathcal{U}(t - a)\, f(t - a)\, dt.
	\]
	Since \( \mathcal{U}(t - a) = 0 \) for \( t < a \), the lower limit of the integral can be shifted to \( a \), i.e., 
	\[
	\mathscr{L}\{\mathcal{U}(t - a)\, f(t - a)\} = \int_a^\infty e^{-st} f(t - a)\, dt.
	\]
  Put \(t - a =u\). Then  \(t = a+u\) and \(dt = du.\)
	When \( t = a  \), \( u = 0 \), and as $t\to \infty$, $u\to\infty$, so the integral becomes
	\[\int_0^\infty e^{-s(a+u)} f(u)\, du = e^{-as} \int_0^\infty e^{-su} f(u)\, du
	\]
	
	\[
	= e^{-as} F(s).
	\]
	Thus,
	\[
	\boxed{\mathscr{L}\{\mathcal{U}(t - a)\, f(t - a)\} = e^{-as} F(s).}
	\]
	The shifted function $f(t-a)$ is technically written $f(t)\vert_{t\to t-a}.$  Thus, we can write
	\[
	\boxed{
		\mathscr{L}\{\mathcal{U}(t)\, f(t)\vert_{t\to t-a}\} = e^{-as} F(s).
	}\qedhere
	\]
\end{solution}

 The linearity  of the Riemann integration yields the linearity of the Laplace transform operator 
\(\mathscr{L}\), as stated in the following theorem.

\begin{theorem}[Linearity of \(\mathscr{L}\)]
	Let $f$ and $g$ be functions defined on $[0, \infty)$ whose   Laplace transforms exist for all $s>s_0$, where $s_0$ is some real number. Then, for any constants $a$ and $b$, the function $h$ defined by $h(t) = af(t) + bg(t)$ also has Laplace transform $H(s)$ for all $s> s_0,$ and  we have
	\[\mathscr{L}\left\{ {af\left( t \right) + bg\left( t \right)} \right\} = a\,\mathscr{L}\left\{f (t)\right\} + b\,\mathscr{L}\left\{g (t)\right\},\] i.e.,
	\[H(s) = aF(s) + bG(s), \quad s>s_0.\]
\end{theorem} 

 \subsection{When does the Laplace transform exist?}
 An immediate question one might ask is: what sort of functions $f(t)$ actually have Laplace transforms according to Definition~\ref{chap7:Laplace}?
To answer this in the context of the Riemann integration, we note that $F(s) =\mathscr{L}\{f(t)\}$ is the limit of the definite integrals 
 	\[\int_0^b  e^{-st} f(t)\, dt\] 
 	as $b\to\infty$ whenever it exists,  and therefore
  $f(t)$ cannot have a vertical asymptote at any point $t =b$. To avoid any chances for vertical asymptotes, we  take $f(t)$ to be \textbf{\textit{piecewise continuous}} on $[0, \infty).$ A piecewise continuous function on \([0,\infty)\)  is one which can have at most a finite number of jump discontinuities on each closed subinterval \([a, b]\) of \([0,\infty)\).  An example of a piecewise continuous function is depicted in the figure below. \\
  
 %\textcolor{red}{Open this figure later}
  	\begin{figure}[h]
  \centering
  	\begin{tikzpicture}[scale=0.7]
  		\tkzInit[xmin=0,xmax=10,ymin=-1.5,ymax=4]
  		\tkzDrawXY
  		
  		% First piece: f(t) = t on [0,2)
  		\draw[domain=0:2, thick, blue] plot (\x,\x);
  		
  		% Open circle at (2,2) to indicate discontinuity
  		\filldraw[blue, thick] (0,0) circle (2pt);
  		\filldraw[white] (2,2) circle (2pt);
  		\draw[blue, thick] (2,2) circle (2pt);
  		
  		% Second piece: f(t) = 3 on [2,4)
  		\draw[thick, blue] (2,3) -- (4,3);
  		\filldraw[white] (4,3) circle (2pt);
  		\draw[blue, thick] (4,3) circle (2pt);
  		\filldraw[blue,thick] (2,3) circle (2pt);
  		
  		% Third piece: f(t) = sin(t) on [4,6]
  		\draw[domain=4:6, samples=100, thick, blue] plot (\x,{sin(deg(\x))});
  		\filldraw[blue,thick] (4,{sin(deg(4))}) circle (2pt);
  		\filldraw[white] (6,{sin(deg(6))}) circle (2pt);
  		\draw[blue, thick] (6,{sin(deg(6))}) circle (2pt);
  		
  	
  		% Fourth piece: f(t) = 0 on (6,10]
  		\filldraw[thick, blue] (6,0) -- (8,0);
  		\filldraw[blue,thick] (6,0) circle (2pt);
  		
  		\filldraw[white] (8,0) circle (2pt);
  		\draw[blue, thick] (8,0) circle (2pt);
  		
  		\draw[domain=8:10, samples=100, thick, blue,->] plot (\x,{\x-6});
  		\filldraw[blue, thick] (8,2) circle (2pt);
  	\end{tikzpicture}
  		\caption{An example a piecewise continuous function}
  	\end{figure}
  	
  \noindent However,  the condition that $f$ be piecewise continuous on $[0, \infty)$ would not be sufficient for the existence of its  Laplace transform. For instance,  the function $f(t)= e^{t^2}$ does not have its Laplace transform. In fact, for each fixed number $c$, there exists $t_0$ such that $t^2> ct$ for  all  $t\ge t_0,$ and this implies that  
  \[ \int_{t_0}^b  e^{-st} e^{t^2}  \, dt = \int_{t_0}^b  e^{t^2-st}  \, dt >  \int_{t_0}^b  e^{(c-s)t}  \, dt.\] This results in
   \[\int_0^\infty  e^{-st} e^{t^2}  \, dt\ge \int_{t_0}^\infty  e^{-st} e^{t^2}  \, dt \ge  \int_{t_0}^\infty  e^{(c-s)t}  \, dt = \infty\] for all $s\le c.$ This holds for every number $c$, and therefore $f(t) = e^{t^2}$ does have its Laplace transform. From this example, we observe that $e^{-st}$ must control the growth property of $f(t)$ as $t\to\infty$ for the existence its Laplace transform. In particular, in addition to the piecewise continuity, 
 if $f(t)$ is of an \textbf{\textit{exponential order}} as $t\to\infty,$ i.e., if  there exist constants $\alpha$, $M>0,$  and $t_0>0$ such that 
  \[ \abs{f(t)} \le M e^{\alpha t} \quad \mbox{ for all } t\ge t_0,\]
  then, for all $b\ge t_0$, we find
   
  \[\abs{\int_0^b  e^{-st} f(t) \, dt} \le \int_0^b  e^{-st} \abs{f(t)} \, dt\le \int_0^b  e^{-st} Me^{\alpha t} \, dt =M\int_0^b e^{-(s-\alpha)t}\, dt.\]
Consequently, 
  \[F(s) =\int_0^\infty  e^{-st} f(t) \, dt\] exists for $s>\alpha.$ 
We formalize all these ideas together in the following existence theorem.
 
 \begin{theorem}[Existence of the Laplace Transform]
 	Let $f$ be a piecewise continuous function on $[0,\infty)$ and of an exponential order as $t\to\infty$.  Then $F(s)=\mathscr{L}\{f(t)\}$ exists.
 \end{theorem}
% We remark that there exist continuous functions that are not of exponential order at infinity and  have no Laplace transform. For example, the function  $f(t)= e^{t^2}$ on $[0, \infty)$ does not have Laplace transform. In fact, for any real number $s$, we have
% \[e^{t^2} e^{-st}  = e^{t(t-s)}\ge 1\] whenever $t\ge s.$ This implies 
% \[\int_0^be^{t^2} e^{-st}\, dt = \int_0^b e^{t(t-s)}\, dt \ge  b.\] 
% Letting $b\to \infty,$ we see that \[\int_0^\infty e^{t^2} e^{-st}\, dt\] diverges, that is, $f$ does not have its Laplace transform.
% 
% \begin{remark}$\empty$
% 	\begin{enumerate}[label={(\roman*)}, noitemsep]
% 		\item  The conditions that $f$ be piecewise continuous on $[0, \infty)$ and of exponential order as $t \to \infty$ are sufficient, but not necessary, for the existence of the Laplace transform. For instance, consider the function 
% 		\[f(t) = \begin{cases}
% 			t^{-1/2} & \text{if } t>0,\\
% 			0&\text{if } t=0.
% 		\end{cases}\]
% 	Since $f$ is not continuous on any interval that contains $0,$ it follows that $f$ is not piecewise continuous on $[0, \infty),$  but $f$ is of exponential order as $t\to \infty.$
% 	
% 	
% 	
% 		
% 		
% 		\item There exist continuous functions that are not of exponential order at infinity and  have no Laplace transform.
% 	\end{enumerate}
% \end{remark}

 
 
We are interested in whether two different functions can have the same Laplace transform.  Trivial examples of functions that have the same Laplace transform are $g(t) =\mathcal U(t-a)$ with  $a>0$ and $h(t)$ defined by
 \[
 h(t) =\begin{cases}
 	1& \text{if } t > a,\\
 	0 & \text{if } t \le a.
 \end{cases}
 \]
 Note that $g(a) = 1$ and $h(a)= 0,$ and  $g(t) = h(t)$ for all $t\ne a.$ Since $g$ and $h$ differ only at one single point in $[0, \infty),$  it follows, by the definition of the Laplace transform, that \[\ds G(s) = H(s) = \frac{e^{-s}}{s}\] for $s>0.$ This implies that \(\mathscr{L}\) is not a one-to-one function from functions \(f(t)\) of the variable \(t\) to functions \(F(s)\) of the variable \(s\).
 
 \subsection{Inverse Laplace Transforms}\label{chap7:subsection1}

 As mentioned at the end of Section~\ref{chap7:section1}, the forced oscillator (\ref{chap7:01}) cannot be solved using the methods from previous chapters when \( f \) is a Dirac delta. Even if \( f \) is only piecewise continuous, the earlier methods can become quite tedious to apply. The Laplace transform method provides a more efficient approach, and this method involves taking the Laplace transform of a linear differential equation and converting it into an algebraic equation. After solving the algebraic equation in the $s$-domain for the Laplace transform of the solution function, we then apply the one-to-one property of the Laplace transform to return to the solution in the $t-$domain. Evidently, for the method to work, we need to know when \(\mathscr{L}\) is one-to-one, so that the inverse $\mathscr{L}^{-1}$ makes sense.


For continuous functions, the following theorem establishes the invertibility of the Laplace transform. 
\begin{theorem}\label{chap7:uniquenss}
Let $f$ and $g$ be two piecewise continuous functions on \([0, \infty)\) such that $F(s)$ and $G(s)$ exist for all $s>c,$ where $s_0$ is a positive number. 
%and of exponential order as $t\to\infty$ 
Suppose $F(s) = G(s)$ for all $s>s_0.$
Then $f(t) = g(t)$ for all \(t\in [0,\infty)\) at which both  functions are continuous. 
\end{theorem}
This theorem can be easily proved by using the continuity of \( f - g \). We omit its proof here.

By Theorem~\ref{chap7:uniquenss}, for continuous functions $f$ and $g$ satisfying $F(s) = G(s)$  for all $s>s_0$, where $s_0$ is a positive number,  we have $f(t) = g(t)$ for all $t$ in $[0, \infty).$ Consequently, $\mathscr{L}$ is\textbf{\textit{ one-to-one on the class of continuous functions whose Laplace transforms exist}}.
Thus, for every continuous function $f$ on $[0, \infty)$ with the Laplace transform of $F(s),$
we have \[f(t) =\mathscr{L}^{-1}\left\{F(s)\right\}\] for all $t\ge 0.$  Here, $\mathscr{L}^{-1}\{F(s)\}$ is called  the \textbf{\textit{inverse  Laplace transform}} $F(s).$ 


We recall from Example~\ref{chap7:example-heaviside} that 
\[\mathscr{L}\{ \mathcal{U}(t-a)\}= \frac{e^{-as}}{s}\]  for $a\ge 0.$ Therefore, when we speak of
\[\mathscr{L}^{-1}\left\{\frac{e^{-as}}{s}\right\},\] we mean a function of $t$ that agrees with $\mathcal{U}(t-a)$ for $t\ne a$, and it can have any arbitrary value at $t=a.$ In practice, we choose the function that best fits the problem at hand, but in this book, we will use  
\[
\mathscr{L}^{-1}\left\{\frac{e^{-as}}{s}\right\} = \mathcal{U}(t - a)
\]  
unless this choice is clearly inappropriate. It is understood that any other  function that differs from \(\mathcal{U}(t - a)\)  only at \(t = a\) may also be used. These considerations will also be made at the points of jump discontinuity when we take the inverse Laplace transform of the Laplace transform of a piecewise continuous functions of exponential order as $t\to\infty.$


 In view of Theorem~\ref{chap7:uniquenss},  $\mathscr{L}^{-1}\{F(s)\}$ may represent multiple functions when $f(t)$ has jump discontinuities on $[0,\infty)$. For most functions discussed in this chapter, it won't be  much of an issue since we will be working with either  continuous functions or, at worse,  functions that have finite number of discontinuity, with the exception of a Dirac delta distributions which has its Laplace transform  encoded in the definition itself as
 \[ \mathscr{L}\{\delta(t - a\} =e^{-as}. 
\]
Since $\delta(t-a)$ is discontinuous only at $t=a,$ we will adopt 
\[
\mathscr{L}^{-1}\{e^{-as}\} =\delta(t - a).\]
 More rigorous arguments about such ideas pertaining to Dirac delta functions exist in an area called ``distribution theory,'' which is beyond the scope of this book.
 
 It  follows that the inverse Laplace transform  operator \(\mathscr{L}^{-1}\) is linear, and we formalize this linearity property of \(\mathscr{L}^{-1}\) in the following theorem.
 
 \begin{theorem}[Linearity of \(\mathscr{L}^{-1}\)]
 	The inverse Laplace transform operator \(\mathscr{L}^{-1}\) is linear, that is, 
 	\[\mathscr{L}^{-1}\{c_1F_1(s)+ c_2 F_2(s)\}= c_1\mathscr{L}^{-1}\{F_1(s)\} +c_2 \mathscr{L}^{-1}\{F_2(s)\},\] where $c_1$ and $c_2$ are constants and $F_1$ and $F_2$ are  Laplace transforms of functions that are either piecewise continuous and of exponential order at infinity or delta functions. 
 \end{theorem}
 
 
 %Section7.1
 
\setcounter{Exercise}{0}
\begin{Exercise}\label{EX71}
	\vspace{-\baselineskip}% <-- You don't need this line of code if there's some text here
	
		\Question Find the Laplace transforms of the following functions on $[0, \infty)$ by using formulas derived in Section~\ref{chap7:section1}.
	\begin{tasks}(1)[resume=false]
		\task\label{EX71-1-i} $4+t^5+\sin(5t)-7\cos (3t)$     
		\task $e^{-11t}+ e^t\sin(3t)+ e^{t-3}\;\mathcal{U}(t-3)$ 
		\task\label{EX71-1-iii} $e^{\alpha t} \big(A\cos(\beta t)+B\sin(\beta t)\big)$ \quad ($A$, $B$, $\alpha$ and $\beta$ are fixed real numbers.)
		\task  $ e^{7t}\delta(t-3)$ 
		\task $ e^{21}\delta(t-3)$ 
		\task $\sum_{k=0}^\infty \delta(t-7k)$  (Assume that the integration and summation can be interchanged.)
	\end{tasks}

	\Question 
	Find the Laplace transforms of the following piecewise continuous functions either by using the definition or by first expressing the functions as a linear combination of unit step functions and then using formulas derived in Section~\ref{chap7:section1}.
	\begin{tasks}[resume=false](2)
		\task\label{EX71-2-iv} $f(t) = \begin{cases}
			3&\text{if } t\le 2\\
			0&\text{if } t>2
		\end{cases}$
		\task $f(t) = \begin{cases}
			0&\text{if } t\le 2\\
			3&\text{if } t>2
		\end{cases}$
		\task $f(t) = \begin{cases}
			e^{2t}&\text{if } t\le 2\\
			0&\text{if } t>2
		\end{cases}$
		\task $f(t) = \begin{cases}
			e^{2t}\sin t&\text{if } t\le \pi\\
			0&\text{if } t>\pi
		\end{cases}$
		\task $f(t) = \begin{cases}
			e^{2t}&\text{if } 1\le t\le 4\\
			0&\text{otherwise}
		\end{cases}$
		\task $f(t) = \begin{cases}
			0&\text{if } 0\le t< 2\\
			2&\text{if } 2\le t< 4\\
			0&\text{if } 4\le t< 6\\
			4&\text{if } 6\le t
		\end{cases}$
		\task $f(t) = \begin{cases}
			e^t&\text{if } 0\le t< \pi\\
			\sin t&\text{if } t\ge \pi
		\end{cases}$
	\end{tasks}
	
	\Question Find the Laplace transform of each of the following functions by using formulas derived in Section~\ref{chap7:section1}.
	 
	 \begin{tasks}(3)
	 	\task $(t-4)^2$
	 	\task $e^{3t-7}$
	 	\task $\cos^2(2t)$
	 	\task $\sin^2(2t)$
	 	\task $\cosh(t)$
	 	\task  $\sinh(t)$
	 \end{tasks}
	 
\Question Find the inverse Laplace transform of each of the following Laplace transforms.
\begin{tasks}(3)
	\task $\ds\frac{1}{s(s+1)}$
	\task $\ds \frac{s}{(s-1)^3}$
	\task $\ds\frac{1}{(s+2)(s-1)}$
	\task 	$\ds\frac{2s}{(s-1)^2 (s+1)}$
	\task $\ds\frac{4}{s^2-2s+5}$
	\task $\ds\frac{e^{-s}}{s+1}$
	\task  $\ds\frac{2}{(s^2+4)(s^2-4)}$
	\task $5$
	\task $e^{-7s}$
\end{tasks}
	
\end{Exercise}
\setboolean{firstanswerofthechapter}{true}
\begin{multicols}{2}\scriptsize
	\begin{Answer}[ref={EX71}]
		\Question 
		\begin{tasks}
			\task $\frac{120}{s^6}-\frac{7 s}{s^2+9}+\frac{5}{s^2+25}+\frac{4}{s}$  
			\task $\frac{e^{-3 s}}{s-1}+\frac{1}{s+11}+\frac{3}{(s-1)^2+9}$
			\task  $\frac{A (s-\alpha )}{(s-\alpha )^2 +\beta ^2}+\frac{\beta  B}{(s-\alpha )^2+\beta ^2}$
			\task $e^{21-3 s}$
			\task $e^{21-3 s}$
			\task $\frac{e^{7 s}}{e^{7 s}-1}$
		\end{tasks} 
		\Question 
		\begin{tasks}[resume=false]
			\task $\frac{3-3 e^{-2 s}}{s}$
			\task $\frac{3 e^{-2 s}}{s}$
			\task $\frac{e^{-2 s} \left(e^{2 s}-e^4\right)}{s-2}$
			\task $\frac{e^{-\pi  (s-2)} \left(e^{\pi  (s-2)}+1\right)}{s^2-4 s+5}$
			\task $\frac{e^{2-s}}{s-2}-\frac{e^{8-4 s}}{s-2}$
			\task $\frac{4 e^{-6 s}}{s}+\frac{4 e^{-3 s} \sinh (s)}{s}$
			\task $\frac{e^{-\pi  s} \left(e^{\pi  s} \left(s^2+1\right)-e^{\pi } \left(s^2+1\right)-s+1\right)}{(s-1) \left(s^2+1\right)}$
		\end{tasks} 
		
		\Question 
	\begin{tasks}[resume=false]
		\task $1-e^{-t}$
		\task $\frac{1}{2}t^2e^t +te^t $
		\task $\frac{e^t}{3}-\frac{1}{3} e^{-2 t}$
		\task $\frac{1}{2} e^t (2 t+1)-\frac{e^{-t}}{2}$
		\task $2 e^t \sin (2t)$
		\task $e^{1-t}\; \mathcal{U}(t-1)$
		\task $-\frac{e^{-2 t}}{16}+\frac{e^{2 t}}{16}-\frac{1}{8} \sin (2 t)$
		\task $5 \delta(t)$
		\task $\delta(t-7)$
	\end{tasks} 	
	\end{Answer}
\end{multicols}
\setboolean{firstanswerofthechapter}{false}




\section{Applications to Initial Value Problems}
As point out earlier in Subsection~\ref{chap7:subsection1}, we will  need to convert a linear initial value problem into an algebraic equation using the Laplace transform. After solving the equation in the $s-$domain for the Laplace transform of the solution function, we then apply the inverse Laplace transform operator $\mathscr{L}^{-1}$ to return to the $t-$domain and obtain the solution of the initial value problem. For these steps, we will need the Laplace transform of derivatives.

Suppose that $y(t)$ is continuous on $[0,\infty)$ and of exponential order as $t\to\infty.$  Then there exists constants $\alpha$, $M>0$ and $t_0>0$ such that 
\[|y(t)| \le M e^{\alpha t}\quad \text{for all } t\ge t_0.\] The number $\alpha$ is referred to as an \textit{\textbf{exponential order}} of $y$. Then, for $s> \alpha,$
we have 
\[|y(t) e^{-st}|\le M e^{-(s-\alpha)t} \quad \text{for all }t\ge t_0,\] and consequently,
\[\lim\limits_{t\to\infty}y(t) e^{-st} =0\quad \text{for }s>\alpha.\]
Suppose, further, that $y'$ is piecewise continuous on $[0,\infty).$  Then, using
 integration by parts, we have 
\begin{equation*}
	\begin{split}
\mathscr{L}\{y'(t)\} &= \int_0^{\infty} e^{-st} y'(t) \, dt \\
&= \left[ e^{-st} y(t) \right]_0^{\infty} + s \int_0^{\infty} e^{-st} y(t) \, dt\\
&= -y(0) + s\mathscr{L}\{y(t)\}\\
&=sY(s) - y(0)
\end{split}
\end{equation*}
 for $s>\alpha.$ Thus, 
\begin{equation}\label{chap7:laplace-derivative1}
	\boxed{
\mathscr{L}\{y'(t)\} = sY(s) - y(0).}
\end{equation}

We extend the above idea to higher order derivatives. For instance,
 suppose that $y(t)$ and $y'(t)$ are  continuous and of exponential order. If $y''(t)$ exists and is piecewise continuous on $[0, \infty),$ then applying the formula (\ref{chap7:laplace-derivative1}) first to $z'(t)$ with $z(t) = y'(t)$ and then to $y'(t)$, we obtain
 \[\mathscr{L}\{y''(t)\} =\mathscr{L}\{z'(t)\} = sZ(s) - z(0) = s\mathscr{L}\{y'(t)\}- y'(0) =s^2Y(s) - sy(0)-y'(0), \] 
 i.e.,
 \begin{equation}\label{chap7:laplace-derivative2}
 	\boxed{
 		\mathscr{L}\{y''(t)\} = s^2Y(s) - sy(0)-y'(0).}
 \end{equation}
 
Applying these ideas repeatedly to  higher order derivatives, we obtain the following theorem whose proof is omitted.
\begin{theorem}[Laplace Transforms of Derivatives]\label{chap7:laplace-derivatives}
If \( y, y', \ldots, y^{(n-1)} \) are continuous on \([0, \infty)\) and  of exponential order as $t\to\infty$ and if 
\( y^{(n)}(t) \) is piecewise continuous on \([0, \infty)\), then
\[
\mathscr{L}\{y^{(n)}(t)\} = s^n Y(s) - s^{n-1}y(0) - s^{n-2}y'(0) - \cdots - y^{(n-1)}(0),
\]
where \( Y(s) = \mathscr{L}\{y(t)\} \).
\end{theorem}

Theorem~\ref{chap7:laplace-derivatives} presents how \( y(0),y'(0), \dots, y^{(n-1)}(0) \) naturally arise, reflecting the initial conditions of an \( n \)-th order linear initial value problem and indicating the potential use of the Laplace transform to covert a linear differential equation with constant coefficients to an algebraic equation with no derivatives of the Laplace transform of the solution variable. 
Specifically, consider the linear initial value problem of the form:
\begin{equation}\label{chap7:IVP}
	\begin{cases}
	&a_n y^{(n)} + a_{n-1}y^{(n-1)}+ \cdots+a_1y'+a_0 y = f(t),\\
	&y(0) = y_0, \dots, y^{(n-1)}(0)= y_{n-1}.
\end{cases}
\end{equation}
Applying Theorem~\ref{chap7:laplace-derivatives} in conjunction with the linearity of $\mathscr{L}$, the differential equation in (\ref{chap7:IVP}) becomes

\begin{equation}\label{chap7:IVP-laplace1}
	\begin{split}
	&a_n\big(s^n Y(s) - s^{n-1}y(0) - s^{n-2}y'(0) - \cdots - y^{(n-1)}(0)\big)\\
	&+a_{n-1}\big(s^{n-1} Y(s) - s^{n-2}y(0) - s^{n-3}y'(0) - \cdots - y^{(n-2)}(0)\big)\\
	&\vdots\\
	&+a_1\big(sY(s)- y(0)\big) + a_0Y(s)= F(s),
	\end{split}
\end{equation}
which takes the form
\begin{equation*}\label{chap7:IVP-laplace2}
	p(s)Y(s)= q(s) +F(s),
\end{equation*}
where  $p(s) = a_ns^n+a_{n-1}s^{n-1} + \cdots+ a_1s+a_0$ and $q(s)$ is a polynomial in $s$ obtained from  all the remaining terms on the left side of the equation (\ref{chap7:IVP-laplace1}).
It then follows that 
\begin{equation*}\label{chap7:IVP-laplace2}
Y(s)= \frac{q(s)}{p(s)} +\frac{F(s)}{p(s)}.
\end{equation*}
The solution $y(t)$ to the initial value problem (\ref{chap7:IVP}) is given by 
\begin{equation}\label{chap7:IVP-solution}
	y(t) = \mathscr{L}^{-1}\{Y(s)\} =\mathscr{L}^{-1}\left\{\frac{q(s)}{p(s)}\right\} +\mathscr{L}^{-1}\left\{\frac{F(s)}{p(s)}\right\}.
	\end{equation}
	To find \[\mathscr{L}^{-1}\left\{\frac{q(s)}{p(s)}\right\} \quad\text{and}\quad\mathscr{L}^{-1}\left\{\frac{F(s)}{p(s)}\right\},\]   we typically use partial fraction decomposition of  ${q(s)}/{p(s)}$ and ${F(s)}/{p(s)}$  and then use appropriate inverse Laplace transforms term by term. For more details on the techniques of partial fraction decomposition, see Appendix~\ref{partial-fraction-decomposition}.
	

In principle, one should verify that equation~(\ref{chap7:IVP-solution}) satisfies the initial value problem~(\ref{chap7:IVP}) by verifying that \( y(t) \) solves the differential equation and satisfies the initial conditions, especially important when the differential equation is defined also for \( t < 0 \) since the Laplace transform only makes sense to functions on \([0, \infty)\). However, this verification is unnecessary when \(f(t)\) is continuous on an open interval containing \( t = 0 \), as the initial value problem~(\ref{chap7:IVP}) is guaranteed to have a unique local solution.

In the following examples, we apply the Laplace transform method to solve initial value problems.

\begin{example}  
	Solve the initial value problem using the Laplace transform:
	\[y'-y= 2\cos(3t), \quad y(0) = 1.\]
	
	\begin{solution} We proceed with the solution  the Laplace transform method through the following steps.
		\smallskip
		
	\noindent\textbf{Step 1: Take the Laplace transform of the differential equation}
	
		Applying the Laplace transform $\mathscr{L}$ to both sides of the differential equation yields
		\[
		sY(s) - y(0) - Y(s) = \frac{2s}{s^2 + 9}.
		\]
	We recall that
		\[
		\begin{split}
			\mathscr{L}\{y'(t)\} &= s Y(s) - y(0), \\ 
			\mathscr{L}\{y(t)\} &= Y(s), \\ 
			\mathscr{L}\{\cos(3t)\} &= \frac{s}{s^2 + 9}.
		\end{split}
		\]	
		Substituting these along with the initial conditions 
		\(
		y(0) = 1
		\) into the initial value problem gives
		\[
		(s - 1)Y(s) = \frac{2s}{s^2 + 9} + 1.
		\]
		
	\noindent	\textbf{Step 2: Solve for \( Y(s) \).}
		\[
		Y(s) = \frac{2s}{(s - 1)(s^2 + 9)} + \frac{1}{s - 1}.
		\]
		
		\noindent\textbf{Step 3: Perform partial fraction decomposition.}
		
		We write
		\[
		\frac{2s}{(s - 1)(s^2 + 9)} = \frac{A}{s - 1} + \frac{Bs + C}{s^2 + 9},
		\] where $A, B, C$ are constants to be determined.
		Multiplying both sides by \( (s - 1)(s^2 + 9) \) gives
		\[
		2s = A(s^2 + 9) + (Bs + C)(s - 1).
		\]
		Expanding the right-hand side and collecting like terms yields
		\[
		2s 
		= (A + B)s^2 + (C - B)s + (9A - C).
		\]
		Matching coefficients, we get
		\[
		\begin{cases}
			A + B = 0, \\
			C - B = 2, \\
			9A - C = 0.
		\end{cases}
		\]
	We find
		\[
		A = \frac{1}{5}, \quad B = -\frac{1}{5}, \quad C = \frac{9}{5}.
		\]
	We then have
		\[
		\frac{2s}{(s - 1)(s^2 + 9)} = \frac{1}{5} \left( \frac{1}{s - 1}\right) - \frac{1}{5} \left( \frac{s}{s^2 + 9}\right) + \frac{9}{5} \left( \frac{1}{s^2 + 9}\right), 
		\]
	and therefore
		\[
		Y(s) = \frac{6}{5}\left( \frac{1}{s - 1}\right) - \frac{1}{5} \left( \frac{s}{s^2 + 9}\right) + \frac{9}{5} \left( \frac{1}{s^2 + 9}\right).
		\]
		
	\noindent\textbf{Step 4: Take the inverse Laplace transform.}
		
		Using
		\[
		\mathscr{L}^{-1}\left\{ \frac{1}{s - a} \right\} = e^{at}, \quad
		\mathscr{L}^{-1}\left\{ \frac{s}{s^2 + a^2} \right\} = \cos(at), \quad
		\mathscr{L}^{-1}\left\{ \frac{a}{s^2 + a^2} \right\} = \sin(at).
		\]
		we obtain
		\[
		\boxed{
		y(t) = \frac{6}{5} e^t - \frac{1}{5} \cos(3t) + \frac{3}{5} \sin(3t),}
		\]
which can be verified to be the solution of the given initial value problem  on $(-\infty, \infty).$
	\end{solution}
\end{example}


\begin{example}  
	Solve the initial value problem using the Laplace transform:
	\[y''-5y'+4y= t, \quad y(0) = 1, y'(0) = 0.\]
	
	\begin{solution}
		Applying the Laplace transform $\mathscr{L}$ to both sides of the differential equation yields
		\[
		\mathscr{L}\{y''\} - 5\mathscr{L}\{y'\} + 4\mathscr{L}\{y\} = \mathscr{L}\{t\}
		\]
We recall that
		\[
		\begin{split}
		\mathscr{L}\{y''(t)\} &= s^2 Y(s) - s y(0) - y'(0),\\
		\mathscr{L}\{y'(t)\} &= s Y(s) - y(0), \\ 
		\mathscr{L}\{y(t)\} &= Y(s), \\ 
		\mathscr{L}\{t\} &= \frac{1}{s^2}.
		\end{split}
		\]	
		Substituting these along with the initial conditions 
		\(
		y(0) = 1, \; y'(0) = 0
		\) into the initial value problem
	gives
		\begin{align*}
			s^2 Y(s) - s - 5(s Y(s) - 1) + 4 Y(s) &= \frac{1}{s^2} \\
			(s^2 - 5s + 4)Y(s) - s + 5 &= \frac{1}{s^2} \\
			(s^2 - 5s + 4)Y(s) &= \frac{1}{s^2} + s - 5
		\end{align*}
	Using \(
		s^2 - 5s + 4 = (s - 4)(s - 1)
		\) and solving for $Y(s)$ gives
		\[
		Y(s) = \frac{1}{(s - 4)(s - 1)} \left( \frac{1}{s^2} + s - 5 \right)
		\]
		
		Combining into a single rational function, we have
		\[
		\frac{1}{s^2} + s - 5 = \frac{1 + s^2(s - 5)}{s^2} = \frac{s^3 - 5s^2 + 1}{s^2}
		\]
	and therefore we obtain
		\[
		Y(s) = \frac{s^3 - 5s^2 + 1}{s^2(s - 4)(s - 1)}.
		\]
		
		Performing the partial fraction decomposition:
		\[
		\frac{s^3 - 5s^2 + 1}{s^2(s - 4)(s - 1)} = \frac{A}{s} + \frac{B}{s^2} + \frac{C}{s - 1} + \frac{D}{s - 4},
		\]
		where $A,B, C, D$ are constants to be determined,  we get
		\[
		A = \frac{5}{16}, \quad B = \frac{1}{4}, \quad C = 1, \quad D = -\frac{5}{16}
		\]
		Therefore
		\[
		Y(s) = \frac{5}{16s}  +   \frac{1}{4s^2} + \frac{1}{s - 1} - \frac{5}{16(s-4)}.
		\]	
	We now apply the inverse Laplace transform $\mathscr{L}^{-1}$ and obtain
	\[
	\boxed{
		y(t) = \frac{5}{16} + \frac{1}{4} t + e^t - \frac{5}{16} e^{4t},
		}
	\]
which can be verified to the solution of the given initial value problem 
		 on $(-\infty, \infty).$
		\end{solution}
\end{example}
The general strategy for solving initial value problems by the Laplace transform method remains the same as demonstrated in the previous examples. The main difference lies in evaluating the Laplace transforms of more complicated functions and finding inverse transforms of more complicated expressions. 
In the followings sections, we continue  exploring additional methods for computing Laplace transforms and solve initial value problems that involve more complicated functions.



\begin{Exercise}\label{EX72}
	\vspace{-\baselineskip}% <-- You don't need this line of code if there's some text here
	\Question
 Solve the following initial value problems using the Laplace transform method.
	\begin{tasks}(1)
		\task $y'-y=1, \quad y(0) = 0$  
		\task $y' -6y = e^{2t}, \quad y(0) = 1$ 
		\task $y'' +5y'+6y = e^{2t}, \quad y(0) = 1, y'(0)=0$ 
		\task $y'' +5y'+6y = e^{2t}, \quad y(0) = 0, y'(0)=1$ 
		\task $y'' +9y = e^{2t}, \quad y(0) = 0, y'(0)=1$ 
\end{tasks}
	
	
	\Question Solve the following initial value problems using the Laplace transform method.
	\begin{tasks}[resume=false](1)
		\task $y'+y = e^{3t}\sin(2t), \quad y(0) = 0$
		\task $y''-4y'+5y = 0, \quad y(0) = 0, y'(0) =1$
		\task $y''-y'= e^t \sin t\quad y(0) =0, y'(0) =0$
		\task $y'''+2y''-y'-2y = \sin t, \quad y(0)= 0, y'(0)= 0, y''(0) =1$
	\end{tasks}
	
		\Question Solve the following initial value problems using the Laplace transform method.
	\begin{tasks}[resume=false](1)
		\task $y'+4y = f(t), \quad y(0) = 0,$ where $f(t) =\begin{cases}
			1 &\text{ if } 0\le t <1\\
			-1&\text{ if } t\ge 1
		\end{cases}$
		\task $y'+4y = f(t), \quad y(0) = 0,$ where $f(t) =\begin{cases}
			\sin t&\text{ if } 0\le t <1\\
			0&\text{ if } t\ge 1
		\end{cases}$
		\task $y''-y'= \cos t\; \mathcal{U}(t-2\pi)\quad y(0) =1, y'(0) =1$
		\task $y''+5y'+6y = \mathcal{U}(t-2\pi), \quad y(0)= 0, y'(0)= 1$
		\task $y''+4y'+3y = 1- \mathcal{U}(t-2) - \mathcal{U}(t-4) +\mathcal{U}(t-6), \quad y(0) = 0, y'(0) = 0$
	\end{tasks}
	
	
\end{Exercise}
\setboolean{firstanswerofthechapter}{true}
\begin{multicols}{2}
	\scriptsize
	\begin{Answer}[ref={EX72}]
		\Question 
		\begin{tasks}
			
			\task $y(t) = e^t-1$ 
			\task $y(t)= \frac{5 e^{6 t}}{4}-\frac{e^{2 t}}{4}$
			\task $y(t) =-\frac{9}{5} e^{-3 t}+\frac{11 e^{-2 t}}{4}+\frac{e^{2 t}}{20}$
			\task $y(t) =-\frac{4}{5} e^{-3 t}+\frac{3 e^{-2 t}}{4}+\frac{e^{2 t}}{20}$
		 	\task \(y(t) = \frac{1}{39} \big(3 e^{2 t}
		 					+ 11 \sin (3 t)-3 \cos (3 t)\big)\)
		
	
		
		\end{tasks} 
		\Question 
		\begin{tasks}[resume=false]
			\task $y(t)=\frac{e^{-t}}{10}+\frac{1}{5} e^{3 t} \sin (2 t)\\-\frac{1}{10} e^{3 t} \cos (2 t)$
			\task $y(t)= e^{2 t} \cos t-e^{2 t} \sin t$
			\task $e^t-\frac{1}{2} e^t \sin t-\frac{1}{2} e^t \cos t-\frac{1}{2}$
			\task \(y(t)=\frac{1}{20} \big(8 e^{-2 t}
			-15 e^{-t}+5 e^t\)\\\(-4 \sin t+2 \cos t\big)\)
		\end{tasks} 
		
		\Question 
		\begin{tasks}[resume=false]
		\task $ y(t)=
			\frac{1}{4}-\frac{e^{-4 t}}{4}$ for $ 0\leq t< 1$ and 
			$-\frac{1}{4} e^{-4 t} \left(1-2 e^4+e^{4 t}\right)$ for $t\ge 1$
		\task $y(t)=\frac{1}{17} \left(-\cos t+e^{-4 t}+4 \sin t\right)$\\ for $0\le t \le 1$\\ and $\frac{1}{17} e^{-4 t} \left(1-e^4 (\cos 1-4 \sin 1)\right)$ for   $t>1.$ 
		\task \(y(t) =e^t\\+\frac{1}{2} \mathcal{U}(t-2 \pi )\) \(\left(e^{t-2 \pi }-\sin t-\cos t\right)\)
		\task \(y(t)=e^t-1\\+\frac{1}{2} \mathcal{U}(t-2 \pi)\) \(\left(e^{t-2 \pi }-\sin t-\cos t\right)\)
		\task\(y(t) = \frac{1}{2} e^{-2 (t-6)} \left(e^{t-6}-1\right)^2 \mathcal{U}(t-6)\\-\frac{1}{2} e^{-2 (t-4)} \left(e^{t-4}-1\right)^2 \mathcal{U}(t-4)\\-\frac{1}{2} e^{-2 (t-2)} \left(e^{t-2}-1\right)^2 \mathcal{U} (t-2)\\+\frac{1}{2} e^{-2 t} \left(e^t-1\right)^2\)
	
%		\task \(y(t) = \begin{cases}
%			\begin{array}{cc}
%				\frac{1}{2} e^{-2 t} \left(-1+e^t\right)^2 & t\leq 2 \\
%				\frac{1}{2} e^{-2 t} \left(-1+e^2\right) \left(-1-e^2+2 e^t\right) & 2<t\leq 4 \\
%				\frac{1}{2} e^{-2 t} \left(-1+e^2\right)^2 \left(1+e^2\right) \left(1+e^2+e^4+e^6-2 e^t\right) & t>6 \\
%				-\frac{1}{2} e^{-2 t} \left(-1+e^4+e^8+2 e^t+e^{2 t}-2 e^{t+2}-2 e^{t+4}\right) & \text{True} \\
%			\end{array}
%		\end{cases}
%		\)
		\end{tasks} 
		
	\end{Answer}
\end{multicols}
\setboolean{firstanswerofthechapter}{false}


%\subsection{Laplace Transforms of ${t^n f(t)}$ and $f(t)/t$}
\section{Derivatives and Integrals of Laplace Transforms}

The evaluation of Laplace transforms and inverse Laplace transforms can often be carried out efficiently by applying differentiation or integration techniques to certain Laplace transforms.


Let $f$ be a piecewise continuous function on $[0, \infty)$ and of exponential order as $t\to\infty$.
Using  Leibniz rule from Theorem~\ref{thm:Leibnitz Improper Integrals} in  Appendix~\ref{Leibnitz Rule-Improper Integrals} for differentiating under the integral sign, we have
\[
\begin{split}
	F'(s) &= \frac{d}{ds}\int_0^\infty e^{-st} f(t)\, dt\\
	&= \int_0^\infty (-t) e^{-st} f(t)\, dt\\
		&= -\int_0^\infty  e^{-st} \big[tf(t)\big]\, dt\\
		&=-\mathscr{L}\{tf(t)\}.
\end{split}
\]
Thus, we have
\[\mathscr{L}\{tf(t)\} =- F'(s).\]
In a similar fashion, we obtain

\[
\mathscr{L}\{t^2f(t)\}=	(-1)^2F''(s).
\]
Repeating this procedure inductively we obtain

\begin{equation}\label{laplace-derivative}
\boxed{
\mathscr{L}\{t^nf(t)\} =(-1)^nF^{(n)}(s)}
\end{equation}
 for all positive integers $n$.

\begin{example}
Calculate $\L\{t \sin(3t)\}.$
\end{example}

\begin{solution}

By \ref{laplace-derivative}, we have
 \[
 \L\{t f(t)\} = -\frac{d}{ds} \left[ \L\{f(t)\}  = -F'(s)\right].
 \]
We know that  
 \[
 \L\{\sin(3t)\} = \frac{3}{s^2 + 9}.
 \]
 Therefore,
 \[
 \L\{t \sin(3t)\} = -\frac{d}{ds} \left( \frac{3}{s^2 + 9} \right)
 = -\frac{d}{ds} \left( 3(s^2 + 9)^{-1} \right)
 = \frac{6s}{(s^2 + 9)^2}.\qedhere
 \]
\end{solution}

\begin{example}
	Calculate $\L\{t^2 \sin(3t)\}.$
\end{example}

\begin{solution}
By \ref{laplace-derivative}, we have
\[
\L\{t^2 f(t)\} = (-1)^2\frac{d}{ds} \left[ \L\{f(t)\}  = F''(s)\right].
\]
Since
\[
\quad \L\{\sin(3t)\} = \frac{3}{s^2 + 9}
\]
we have
\[
\L\{t^2 \sin(3t)\} 
= \frac{d^2}{ds^2} \left( \frac{3}{s^2 + 9}\right) = \frac{-6s}{(s^2 + 9)^2}.\qedhere
\]
\end{solution}

Let us discuss about what happens when you integrate a Laplace transform. Let $f$ be a piecewise continuous function of exponential order as $t\to\infty$. Suppose $s_0$ is a real number such that $F(s)$ exists for all $s>s_0.$ To compute $\mathscr{L}\{f(t)/t\}$, let 
$g(t) = f(t)/t$ so that $t g(t) =f(t).$  Then, in view of (\ref{laplace-derivative}), we obtain
\[ G'(s) = -F(s).\]
Consequently, we have
\[G(s) = -\int_s^{\infty}  G'(p)\; dp = \int_s^\infty F(p)\, dp.\]
Here, we have considered only those $f$ for which the Laplace transform $G(s)$  of $g$ satisfies the limiting property that $G(s) \to 0$ as $s\to \infty.$
%Then, in view of Fubini's theorem for improper integrals, we have
%\[
%\begin{split}
%	\int_s^\infty F(p)\, dp &= \int_s^\infty\left( \int_0^\infty e^{-pt} \, f(t) \; dt\right)dp\\
%							& =\int_0^\infty \left(\int_s^\infty e^{-pt} \, f(t) \; dp\right) dt\\
%							&= \int_0^\infty f(t)\left[\frac{e^{-pt}}{-t}\right]_s^\infty\, dt\\
%							&= \int_0^\infty \left(\frac{f(t)}{t}\right)e^{-st}\; dt,
%\end{split}\]
%provided that all the  improper integrals exists.
Thus, we have
\begin{equation}\label{laplace-integral}
	\boxed{\mathscr{L}\left\{\frac{f(t)}{t}\right\}(s) =	\int_s^\infty F(p)\, dp.} 
\end{equation}
Moreover, we can use (\ref{laplace-integral}) to determine the existence of the Laplace transform of  $f(t).$  See the problem  \ref{chap7-3-3} in Exercises~\thesection.

We note that when the improper integral $\int_0^\infty f(t)\, dt$ converges, we have $F(0) =\int_0^\infty f(t)\, dt.$  
\begin{example}
Evaluate $\displaystyle \int_0^\infty \frac{\sin t}{t}\, dt.$

\begin{solution} We recall that \[\mathscr{L}\{\sin t\}(p) = \frac{1}{p^2+1}\] for all $p>0.$
Then, using (\ref{laplace-integral}) with $s=0,$
	\[\int_0^\infty \frac{\sin t}{t}\, dt =\mathscr{L}\left\{\frac{\sin t}{t}\right\}(0)= \int_0^\infty \frac{1}{p^2+1}\;dp = \left[\arctan(p)\right]_0^\infty = \pi/2.\qedhere\]
	\end{solution}
\end{example}

\begin{example}
	Compute $\ds\mathscr{L}\left\{\frac{e^{t}-1}{t}\right\}.$
	
	\begin{solution} We recall that $\ds \mathscr{L}\{e^{t}-1\} = \frac{1}{s-1}- \frac{1}{s}$ for $s>1$.
		 Using (\ref{laplace-integral}) with $f(t) = e^t-1,$ we find 
		\[\begin{split}
		\mathscr{L}\left\{\frac{e^{t}-1}{t}\right\} &= \int_s^\infty \left[\frac{1}{p-1}- \frac{1}{p}\right]dp\\
		&= \left[\ln(p-1) - \ln(p)\right]_s^\infty\\
		& = \left[\ln\left(\frac{p-1}{p}\right)\right]_s^\infty\\
		&=\ln\left(\frac{s }{s-1}\right).\qedhere
		\end{split} \]
	\end{solution}
\end{example}


%%%Section2



\begin{Exercise}\label{EX73}
	
	\vspace{-\baselineskip}% <-- You don't need this line of code if there's some text here
\Question\label{chap7-3-1}	Compute the Laplace transform of each of the following functions by using the formula (\ref{laplace-derivative}).

	\begin{tasks}(2)
		\task $t e^{-7t}$
		\task $t^2 e^{-7t}$
		\task $t\cos(2t)$
		\task $t^2\cos(2t)$
		\task $te^{-3t}\cos(2t)$
		\task $t^2e^{-3t}\cos(2t)$
		\task $t e^{3t}\sin(5t)$
	\end{tasks}
	
	\Question\label{chap7-3-2} Compute  $\ds\L\left\{\frac{e^{-t}-1}{t}\right\}.$
	\Question\label{chap7-3-3} Determine whether  $\ds \frac{e^t-2}{t}$ has its Laplace transform.\newline Hint: (\ref{laplace-integral})

	\Question  Solve the following initial value problems using the Laplace transform method.
	\begin{tasks}[resume=false](1)
			\task $y'+y= t\cos t, \quad y(0) =0$
			\task $y'-y= t e^t\cos t, \quad y(0) =0$
			\task $y''+4y= \cos (2t), \quad y(0) =2, y'(0)= 1$
			\task $y''+4y= f(t),\; y(0) =2, y'(0)= 1,$ with $f(t) =\begin{cases}
				\cos(2t) &\text{ if } t< 2\pi,\\
				0&\text{ if } t\ge 2\pi.
			\end{cases}$
			 
		\end{tasks}
\end{Exercise}


\setboolean{firstanswerofthechapter}{true}
\begin{multicols}{2}\scriptsize
	\begin{Answer}[ref={EX73}]
		\Question 
		\begin{tasks}
			\task $\frac{1}{(s+7)^2}$
			\task $\frac{2}{(s+7)^3}$
			\task $\frac{s^2-4}{\left(s^2+4\right)^2}$
			\task $\frac{2 s \left(s^2-12\right)}{\left(s^2+4\right)^3}$
			\task $\frac{s^2+6 s+5}{\left(s^2+6 s+13\right)^2}$
			\task $\frac{2 (s+3) \left(s^2+6 s-3\right)}{\left(s^2+6 s+13\right)^3}$
			\task $\frac{10 (s-3)}{\left(s^2-6 s+34\right)^2}$
		\end{tasks} 
		
		\Question   $\ln \left(\frac{s}{s+1}\right)$ 
	
		\Question Since  $\int_s^\infty \left(\frac{1 }{s+1}- \frac{2}{s}\right)\, ds$ diverges, the~Laplace transform does not exist.
		
		\Question 
		\begin{tasks}[resume=false]
			\task $y(t)=\frac{1}{2} \big((t-1) \sin t+t \cos t\big)$
			\task $y(t)=e^t \big(t \sin t+\cos t-1\big)$
			\task $y(t)=\frac{1}{4} (t+2) \sin (2 t)+2 \cos (2 t)$
			\task $y(t)=\begin{cases}
				2 \cos (2 t)+\frac{1}{4} (t+2) \sin (2 t)\\
				\text {for }   0\le t<2 \pi,\\
					2 \cos (2 t)+(1+\pi ) \cos t \sin t\\
				\text{for } t\ge 2 \pi
			\end{cases}$ 
		\end{tasks}
%		
		
	\end{Answer}
\end{multicols}
\setboolean{firstanswerofthechapter}{false}

%
%\begin{Exercise}\label{EX12}
%	Another exercise. 
%	\Question If you don't need a horizontal list, you can simply use \verb|\Question|
%\end{Exercise}
%\begin{multicols}{2}
%	\begin{Answer}[ref={EX12}]
%		\Question This is a solution of Ex 1
%	\end{Answer}
%\end{multicols}

%\setcounter{Exercise}{0}
%\begin{ExerciseList} 
%	\Exercise[label=chap7-1] Compute the Laplace transform of  $\ds\mathscr{L}\left\{\frac{e^{-t}-1}{t}\right\}.$
%	\Exercise[label=chap7-2] Determine whether  $\ds \frac{e^t-2}{t}$ has its Laplace transform.\newline Hint: (\ref{laplace-integral}) may be useful.
%	
%\end{ExerciseList}



\section{Laplace Transforms of Periodic Functions}
In Section~\ref{chap7:section1}, we computed the Laplace transforms of the periodic functions $\sin(kt)$ and $\cos(kt)$ that are continuous. In applications, we also have to address periodic functions that are periodic but piecewise continuous on $[0, \infty).$
A function $f$ is said to be \textit{periodic} on $(-\infty, \infty)$  if there exists a number $T>0,$ called a \textit{period} of $f,$ such that 
\[ f(t) = f(t+T)\quad\text{ for all } t\in (-\infty, \infty).\]
Graphically, a periodic function $f$ repeats the portion of its graph restricted to $[0, T]$ across subsequent intervals of the same length $T$. 

% \textcolor{red}{Open this figure later}
\begin{figure}[htbp]
	\centering
	\begin{tikzpicture}[scale=.9]
		\begin{axis}[
			axis lines=middle,
			xmin=-3.2, xmax=3.2,
			ymin=-1.1, ymax=1.5,
			xtick={-3, -2,-1,0,1,2,3},
			ytick={-1,0,1},
			xlabel={$t$},
			ylabel={$f(t)$},
			grid=both,
			width=13cm,
			height=5cm,
			samples=200,
			domain=0:6,
			]
			
			% Semicircular arcs: f(x) = sqrt(1 - (x - shift)^2)
			\addplot[blue, thick, domain=-3:-2]   {-0.5};
			\addplot[blue, thick, domain=-2:-1]   {sqrt(1 - (x +1)^2)};
			\addplot[blue, thick, domain=-1:0]   {-0.5};
			\addplot[blue, thick, domain=0:1]   {sqrt(1 - (x - 1)^2)};
			\addplot[blue, thick, domain=1:2]   {-0.5};
			\addplot[blue, thick, domain=2:3]   {sqrt(1 - (x - 3)^2)};
		
			%\node at (axis cs:1,1.1) {$f(x)$};
			
			% Dots to indicate discontinuities
			\foreach \x in {-2,0,2,4}
			\addplot[only marks, mark=*, blue] coordinates {(\x,0)};
			
			\foreach \x in {-2,0,2,4}
			\addplot[only marks, mark=*, fill=white, draw=blue]  coordinates {(\x,-0.5)};
			
				\foreach \x in {-3,-1,1,3}
			\addplot[only marks, mark=*, blue] coordinates {(\x,-0.5)};
			
			\foreach \x in {-1,1,3}
			\addplot[only marks, mark=*, fill=white, draw=blue]  coordinates {(\x,1)};
			
		\end{axis}
	\end{tikzpicture}
	\caption{A piecewise continuous periodic function $f(t)$ with period $T = 2$.}
	\label{fig:piecewise-periodic}
	\end{figure}
A piecewise continuous periodic function $f(t)$ with a period $T$  has the following property:
\[f(t) = f(t+T)  = f(t+2T) = \cdots = f(t+nT)\] 
%and
%\[f(t) = f(t-T)  = f(t-2T) = \cdots = f(t-nT),\] 
 for all positive integers $n.$  This property is visualized in Figure~\ref{fig:piecewise-periodic}. 
 To compute $\mathscr{L}\{f(t)\},$ we express it as an infinite series of integrals as follows:
 \[\begin{split}
 	\mathscr{L}\{f(t)\}& =\int_0^\infty f(t) e^{-st}\, dt\\
 	&= \int_0^T f(t) e^{-st}\, dt+ \int_T^{2T} f(t) e^{-st}\, dt+ \int_{2T}^{3T} f(t) e^{-st}\, dt +\dots\\
 	&=\int_0^T f(t) e^{-st}\, dt + \int_0^T f(t+T) e^{-s(t+T)}\, dt\\
 	& \quad + \int_0^T f(t+2T) e^{-s(t+2T)}\, dt+ \dots+ \int_0^T f(t+nT) e^{-s(t+nT)}\, dt+\cdots\\
 	&=\int_0^T f(t) e^{-st}\, dt\;\Big(1+e^{-sT}+ e^{-2sT} + \dots+ e^{-nsT}+\cdots \Big)\\
 	&=\int_0^T f(t) e^{-st}\, dt\;\; \sum_{n=0}^\infty e^{-nsT}.
 \end{split}
 \]
 Since the geometric series \(\ds \sum_{n=0}^\infty e^{-nsT}\) converges for $s>0$ to \(\ds \frac{1}{1- e^{-sT}},\)
  we obtain
  \begin{equation}\label{Laplace-periodic-function}
  	\boxed{\mathscr{L}\{f(t)\} = \frac{1}{1- e^{-sT}}\int_0^T f(t) e^{-st}\, dt.}
  \end{equation}
 
\begin{example}
	Let us revisit $\mathscr{L}\{\sin(kt)\}.$ Using (\ref{Laplace-periodic-function}) with $T= 2\pi/k$, we obtain
\[	\mathscr{L}\{\sin(kt)\} = \frac{1}{1- e^{-2\pi s/k}}\int_0^{2\pi/k} \sin(kt) e^{-st}\, dt.\]
We compute 
\[\int_0^{2\pi/k} \sin(kt) e^{-st}\, dt=\frac{k}{s^2+k^2}\left( 1- e^{-2\pi s/k}\right).\]
Hence \[ \mathscr{L}\{\sin(kt)\} =\frac{k}{s^2+k^2},\]
as expected.
\end{example}
The example above suggests that  the direct computation of the Laplace transform of a periodic function may  be more convenient.

The example below concerns a periodic function acting as the source term of an initial value problem.

\begin{example}
	Consider the initial value problem $y'+ y = f(t), \; y(0) = 0,$ where $f(t)$ is given by
	\[f(t) = \begin{cases}
		1 &\text{if } 0\le  t\le 1,\\
		0 &\text{if } 1< t < 2,
	\end{cases}\]
	 and $f(t) = f(t+2)$ for all $t$ in $(-\infty, \infty).$ 
	 The graph of $f$ on $[0, \infty)$ is shown below.
	 
%\textcolor{ red}{Open this later}
	 \begin{center}
	 	\begin{tikzpicture}[scale=.8]
	 			\begin{axis}[
	 					axis lines=middle,
	 					xmin=0, xmax=6.2,
	 					ymin=-0.2, ymax=1.5,
	 					xtick={0,1,2,3,4,5,6},
	 					ytick={0,1},
	 					xlabel={$t$},
	 					ylabel={$f(t)$},
	 					grid=both,
	 					width=13cm,
	 					height=5cm,
	 					samples=100,
	 					domain=0:6,
	 					]
	 					
	 					% Square wave segments (period 2: [0,1)=1, [1,2)=-1, repeat)
	 					\addplot[blue, thick, domain=0:1]   {1};
	 					\addplot[blue, thick, domain=1:2]   {0};
	 					\addplot[blue, thick, domain=2:3]   {1};
	 					\addplot[blue, thick, domain=3:4]   {0};
	 					\addplot[blue, thick, domain=4:5]   {1};
	 					\addplot[blue, thick, domain=5:6]   {0};
	 					\addplot[blue, thick, domain=6:6.1] {1};
	 					
	 					% Dots to indicate discontinuities
	 					\foreach \x in {0,1,2,3,4,5,6}
	 					\addplot[only marks, mark=*, blue] coordinates {(\x,1)};
	 					\foreach \x in {1,2,3,4, 5,6}
	 					\addplot[only marks, mark=*, fill=white, draw=blue] coordinates {(\x,0)};
	 				\end{axis}
	 		\end{tikzpicture}
	 \end{center}
	 Then 
	 \[\begin{split}
	 	F(s) &= \frac{1}{1-e^{-2s}}\int_0^2 f(t) e^{-st}\, dt\\
	 	 &= \frac{1}{1-e^{-2s}}\int_0^1 e^{-st}\, dt\\
	 	 &  = \frac{1}{1-e^{-2s}}\left(\frac{e^{-s}-1}{-s}\right)\\
	 	 &= \frac{1}{s(1+e^{-s})}.
	 	\end{split}
	 	\]
	 	Taking the Laplace transform of the differential equation, we get
	 	\[sY(s) - y(0) + Y(s) = F(s).\] With $y(0) = 0,$ we have
	 	\[Y(s) = \frac{1}{s(s+1)(1+e^{-s})}.\]
	 	Since \[\ds \frac{1}{s(s+1)} = \frac{1}{s}- \frac{1}{s+1}\] and \[\ds\frac{1}{1+e^{-s}} = \sum_{n=0}^\infty (-1)^ne^{-ns},\] it follows that 
	 	\[Y(s) = \left(\frac{1}{s}- \frac{1}{s+1}\right)\sum_{n=0}^\infty (-1)^ne^{-ns}.\]
	In view of the shifting on the $t-$axis property discussed in Example~\ref{shiting-on-t-axis},  the solution $y(t)$ of the initial value problem is
	 	\[y(t)  =  \sum_{n=0}^{\infty}(-1)^n \mathcal{U}(t-n)\; (1- e^{-(t-n)}).\]
	% \textcolor{red}{Open this figure later}
	\begin{figure}[!htb]
		\centering
	\begin{tikzpicture}[scale=0.8]
		\begin{axis}[
			axis lines=middle,
			xmin=0, xmax=10.2,
			ymin=0, ymax=0.75,
			samples=500,
			domain=0:10,
			%grid=both,
			width=14cm,
			height=6cm,
			xlabel={$t$},
			ylabel={$y(t)$}
			]
			
		 \addplot[blue, domain=0:1]     {1 - exp(-x)};
		 \addplot[blue, domain=1:2]     {1 - exp(-x) - (1 - exp(-(x - 1)))};
		 \addplot[blue, domain=2:3]     {1 - exp(-x) - (1 - exp(-(x - 1))) + (1 - exp(-(x - 2)))};
		 \addplot[blue, domain=3:4]     {1 - exp(-x) - (1 - exp(-(x - 1))) + (1 - exp(-(x - 2))) - (1 - exp(-(x - 3)))};
		 \addplot[blue, domain=4:5]     {1 - exp(-x) - (1 - exp(-(x - 1))) + (1 - exp(-(x - 2))) - (1 - exp(-(x - 3))) + (1 - exp(-(x - 4)))};
		 \addplot[blue, domain=5:6]     {1 - exp(-x) - (1 - exp(-(x - 1))) + (1 - exp(-(x - 2))) - (1 - exp(-(x - 3))) + (1 - exp(-(x - 4))) - (1 - exp(-(x - 5)))};
		 \addplot[blue, domain=6:7]     {1 - exp(-x) - (1 - exp(-(x - 1))) + (1 - exp(-(x - 2))) - (1 - exp(-(x - 3))) + (1 - exp(-(x - 4))) - (1 - exp(-(x - 5))) + (1 - exp(-(x - 6)))};
		 \addplot[blue, domain=7:8]     {1 - exp(-x) - (1 - exp(-(x - 1))) + (1 - exp(-(x - 2))) - (1 - exp(-(x - 3))) + (1 - exp(-(x - 4))) - (1 - exp(-(x - 5))) + (1 - exp(-(x - 6))) - (1 - exp(-(x - 7)))};
		 \addplot[blue, domain=8:9]     {1 - exp(-x) - (1 - exp(-(x - 1))) + (1 - exp(-(x - 2))) - (1 - exp(-(x - 3))) + (1 - exp(-(x - 4))) - (1 - exp(-(x - 5))) + (1 - exp(-(x - 6))) - (1 - exp(-(x - 7))) + (1 - exp(-(x - 8)))};
		 \addplot[blue, domain=9:10]    {1 - exp(-x) - (1 - exp(-(x - 1))) + (1 - exp(-(x - 2))) - (1 - exp(-(x - 3))) + (1 - exp(-(x - 4))) - (1 - exp(-(x - 5))) + (1 - exp(-(x - 6))) - (1 - exp(-(x - 7))) + (1 - exp(-(x - 8))) - (1 - exp(-(x - 9)))};
  \end{axis}
\end{tikzpicture}
\caption{Graph of $\ds f(t) = \sum_{n=0}^{10} (-1)^n \mathcal{U}(t - n)\; (1 - e^{n - t}) $}
\end{figure}


%Do not open the following figure unless reviewed
%	\begin{figure}
%		\centering
%		\includegraphics[width=0.7\linewidth]{chap05/figures/periodic-source}
%		\caption{}
%		\label{fig:periodic-source}
%	\end{figure}
	 	
\end{example} 


\begin{Exercise}\label{EX74}
	\vspace{-\baselineskip}% <-- You don't need this line of code if there's some text here
	\Question\label{6-4-1} Sketch the graph of three periods of each function below and compute its Laplace transform. The interval for the first period is provided.
	\begin{tasks}[resume=false](2)
		\task \label{6-4-1(i)} $f(t) = \cos t, \quad 0\le t < \pi$
		\task\label{6-4-1(ii)} $f(t) =  t, \quad 0\le t < 3$
		\task\label{6-4-1(iii)} $f(t) =t-2, \quad 0\le t  < 4$
		\task\label{6-4-1(iv)} $f(t) = \begin{cases}
			1	& \text{if } 0\le t< 1\\
			-1 		& \text{if } 1\le  t< 2\\
		\end{cases}$
		
		\task\label{6-4-1(v)} $f(t) = \begin{cases}
			\sin t 	& \text{if } 0\le t< \pi\\
			0 		& \text{if } \pi\le  t< 2\pi\\
		\end{cases}$
		\task\label{6-4-1(vi)} \(f(t) = \begin{cases}
		e^{-t}\sin t 	& \text{if } 0\le t< \pi\\
		0 		& \text{if } \pi\le  t< 2\pi\\
	\end{cases}\)
	\end{tasks}
	
	
	\Question Solve the following initial value problems using the Laplace transform method. You will need $\frac{1}{1-x} = \sum_{n=0}^\infty (-1)^n x^n$ and $\frac{1}{1+x} = \sum_{n=0}^\infty  x^n$  when $|x| <1.$
	\begin{tasks}[resume=false](1)
		\task $y''+4y'+3y = f(t), \quad y(0) = 0, y'(0) = 0$ where \(f(t) =\begin{cases}
			0 &\text{if } 0\le  t\le 1\\
			1 &\text{if } 1< t < 2 
		\end{cases}\)
		 and $f(t) = f(t+2)$ for all $t$ in $(-\infty, \infty).$ 
	\task $y''+4y'+4y = f(t),\quad y(0) = 0, y'(0) = 0,$ where 
	\(f(t)\) is the periodic function in part \ref{6-4-1(iv)} of Problem~\ref{6-4-1}.
	\task $y''+3y'+2y = f(t),\quad y(0) = 0, y'(0) = 0,$ where 
	\(f(t)\) is the periodic function in part \ref{6-4-1(v)} of Problem~\ref{6-4-1}.
		\task $y''+4y = f(t),\quad y(0) = 0, y'(0) = 0,$ where 
	\(f(t)\) is the periodic function in part \ref{6-4-1(ii)} of Problem~\ref{6-4-1}.	
	\end{tasks}
	
\end{Exercise}

\setboolean{firstanswerofthechapter}{true}
\begin{multicols}{2} \scriptsize
	\begin{Answer}[ref={EX74}]
		\Question 
		\begin{tasks} 
			\task 	\includegraphics[width=0.9\linewidth]{chap05/Mathematica/7-4-1}\newline
			$\frac{s\left(1+e^{-\pi  s}\right)}{\left(s^2+1\right)\left(1-e^{-\pi  s}\right) }$
			\task \includegraphics[width=0.9\linewidth]{chap05/Mathematica/7-4-2}\newline
			$\frac{1-e^{-3 s} (3 s+1)}{\left(1-e^{-3 s}\right) s^2}$
			\task \includegraphics[width=0.9\linewidth]{chap05/Mathematica/7-4-3}\newline
			$\frac{1-2 s-e^{-4 s} (2 s+1)}{\left(1-e^{-4 s}\right) s^2}$
			\task \includegraphics[width=0.9\linewidth]{chap05/Mathematica/7-4-4}\newline
			$\frac{1-e^{-s}}{\left(1-e^{-2 s}\right) s}$
			\task \includegraphics[width=0.9\linewidth]{chap05/Mathematica/7-4-5}\newline
			$\frac{e^{-\pi  s}+1}{\left(1-e^{-2 \pi  s}\right) \left(s^2+1\right)}$
			\task \includegraphics[width=0.9\linewidth]{chap05/Mathematica/7-4-6}\newline
			$\frac{e^{-\pi  s}+1}{\left(1-e^{-2 \pi  s}\right) \left(s^2+1\right)}$
		\end{tasks} 
		\Question 
		\begin{tasks}[resume=false]
			\task $F(s) = \frac{e^{-s}}{s(1+e^{-s})} = -\frac{1}{s}+\frac{1}{s(1-e^{-s})}$,\\ $Y(s) = \frac{F(s)}{(s+1)(s+3)}\\= \left(\frac{1}{3 s}-\frac{1}{2 (s+1)}+\frac{1}{6 (s+3)}\right)\ds\sum_{n=1}^\infty (-1)^{n} e^{-ns}$;
			$y(t) =\sum_{n=1}^\infty (-1)^{n} \mathcal{U}(t-n) \\\left(\frac13- \frac{1}{2}e^{-(t-n)} +\frac{1}{6}e^{-3(t-n)}\right).$
			\task $F(s) = -\frac{1}{s}+\frac{2}{s(1+e^{-s})}$,\\ $Y(s) = \frac{F(s)}{(s+2)^2}\\= -\frac{1}{s(s+2)^2}+ \frac{2}{s(s+2)^2}\sum_{n=0}^\infty (-1)^{n} e^{-ns}$\\
			$= \frac{1}{4 (s+2)}+\frac{1}{2 (s+2)^2}-\frac{1}{4 s}\\+\left(\frac{1}{2 s}-\frac{1}{2 (s+2)}-\frac{1}{(s+2)^2}\right)\ds\sum_{n=0}^\infty (-1)^{n} e^{-ns}$;\\
			$y(t) = \frac{1}{2} e^{-2 t} t+\frac{e^{-2 t}}{4}-\frac{1}{4}\\ + \ds\sum_{n=0}^\infty (-1)^{n} \mathcal{U}(t-n) \big(\frac{1}{2}-\frac{e^{-2 (t-n)}}{2}\\-e^{-2 (t-n)} (t-n)\big)$
			
			\task $F(s) = \frac{1}{(1-e^{-\pi  s})(s^2+1)}$; \\$Y(s) = \frac{F(s)}{(s+1)(s+2)}\\ =\frac{1}{(1-e^{-\pi  s})(s^2+1)(s+1)(s+2)} $\\$=\ds\frac{1}{(s^2+1)(s+1)(s+2)}\sum_{n=0}^\infty e^{-n\pi s} $;
			$y(t) = \frac{1}{10}\sum_{n=0}^\infty \mathcal{U}(t- n\pi)\\ \big( 5e^{-(t-n\pi)}- 2 e^{-2(t-n\pi)}\\
			-3 \cos(t-n\pi)+\sin (t-n\pi)\big)$
			\task $F(s) = \frac{1}{s^2}+\frac{3}{s}-\frac{3}{s(1-e^{-3s})}$;  $Y(s)  = \frac{1}{s^2(s^2+4)}+\frac{3}{s(s^2+4)}-\frac{3}{s (s^2+4)(1-e^{-3s})} =\frac{1}{s^2(s^2+4)} -\frac{3}{s(s^2+4)}\ds\sum_{n=1}^\infty  e^{-3ns}\\
			=\frac{1}{4}\left(\frac{1}{s^2} - \frac{1}{s^2+4}\right)\\ -\frac{3}{4}\left(\frac{1}{s}-\frac{s}{s^2+4}\right)\ds\sum_{n=1}^\infty  e^{-3ns}$;\\
			$ y(t) = \frac{1}{4} ( t- \sin t \cos t)\\- \frac{3}{2} \ds\sum_{n=1}^\infty \mathcal{U}(t-3n) \sin^2(t-3n)$
		\end{tasks} 
	\end{Answer}
\end{multicols}
\setboolean{firstanswerofthechapter}{false}

\section{A Convolution Theorem}
A crucial step in solving a linear initial value problem using Laplace transforms is determining the inverse Laplace transform of the solution's Laplace transform. This step can become challenging when the expression, whose inverse Laplace transform is desired, is complicated. In this section, we discuss a procedure with which we can determine the inverse Laplace transform of an expression that is  a product of  known Laplace  transforms. For example, we recall that \[\mathscr{L}^{-1}\left\{\ds\frac{1}{s^2}\right\} = t \quad \text{ and }\quad  \mathscr{L}^{-1}\left\{\ds\frac{2}{s^3}\right\} = t^2.\]
However, if we were to compute   
\[ \mathscr{L}^{-1}\left\{\ds\left(\frac{1}{s^2}\right) \left(\frac{2}{s^3}\right)\right\},\] a direct way is to see this problem as
\[ \mathscr{L}^{-1}\left\{\frac{2}{s^5}\right\} = \frac{2}{4!}\; t^4 = \frac{t^4}{12}.\]
This leads to a natural question: in what way can we combine $t$ and $t^2$ to produce $t^4/12$? The  operation for combining inverse Laplace transforms—specifically, to determine the inverse of a product—is known as \textit{convolution}, which is formalized in the following definition.

\begin{definition}\label{def:convolution}
	Let $f$ and $g$ be functions on $[0, \infty).$ The \textbf{\textit{convolution}} of $f$ and $g$, denoted by $f\ast g$, is the function defined by
	\[(f\ast g)(t) = \int_0^t f(u) g(t-u)\, du\] whenever the integral exists for a given $t$ in $[0, \infty).$
\end{definition}
For computational purposes, $(f * g)(t)$ may also be written as $f(t) * g(t)$.
It can be proved that $f\ast g = g\ast f;$ see, for example, \ref{7-5-2}\;\ref{7-5-2(i)} in  \textbf{Exercises~\ref{EX75}}. The convolution $1\ast f$ is particularly useful, and it  is given by
\begin{equation}\label{special-convolution}
	(1\ast f)(t) = \int_0^t f(u) \, du.
	\end{equation}
	See, also,  \ref{7-5-2}\;\ref{7-5-2(v)} in \textbf{Exercises~\ref{EX75}}.
\begin{example}
Let us return to the example discussed before Definition~\ref{def:convolution}  and calculate the convolution of $f(t) =t$ and $g(t) =t^2$. We have
	\[
	\begin{split}
		(f\ast g)(t)& = \int_0^t u (t-u)^2\, du\\
		&= \int_0^t (t^2u-2tu^2 +u^3)\, du\\
		&= \left[t^2\frac{u^2}{2}-2t\frac{u^3}{3}+\frac{u^4}{4}\right]_0^t\\
		&=\frac{t^4}{2}-\frac{2t^4}{3}+\frac{t^4}{4}\\
		&=\frac{t^4}{12},
	\end{split}
	\]
as expected.\hfill$\clubsuit$
\end{example}
We now state the most important theorem of this section. 
\begin{theorem}[Convolution Theorem]\label{thm:convolution-theorem}
	Let $f$ and $g$ be piecewise continuous functions on $[0,\infty)$ and of the exponential order $\alpha$ as $t\to\infty$.
	Then
	\[ \mathscr{L}\{(f\ast g)(t)\}= F(s) \,G(s);\]
	or equivalently,
	\[\mathscr{L}^{-1}\left\{ F(s) G(s)\right\} =\mathscr{L}^{-1}\{F(s)\}\ast \mathscr{L}^{-1}\{G(s)\}\] for $s>\alpha.$
\end{theorem}

To see how this theorem holds, we need to use a method for changing the order of an iterated integral. 
%See  Appendix~\ref{Leibniz-Rule} for further details. 
We include a proof of the Convolution Theorem below for the sake of completeness; however, not to be distracted from the proof, the reader may choose to skip it for now and proceed directly to apply the theorem.

\begin{proof}[Proof of Theorem~\ref{thm:convolution-theorem}]
We recall that 
\[
	F(s)  = \int_0^\infty e^{-s\eta} f(\eta) \, d\eta\quad \text{ and }\quad 
	G(s) = \int_0^\infty e^{-su} g(u) \, du.\] Then 
\begin{align*}
		F(s) G(s) & = \int_0^\infty e^{-s\eta} f(\eta) \, d\eta\; \int_0^\infty e^{-su} g(u) \, du\\
		&=\int_0^\infty g(u) \left( \int_0^\infty e^{-s(\eta+u)} f(\eta) \, d\eta \right) \, du\\
		& = \int_0^\infty g(u) \left( \int_u^\infty e^{-st} f(t - u) \, d\xi \right) \, du,
\end{align*}
where $t=\eta+u.$
By changing the order of integration, we obtain
\begin{align*} F(s) G(s)&=\int_0^\infty e^{-s t} \left( \int_0^t g(u)  f(t - u) \, du \right) \, dt \\
	&=\int_0^\infty e^{-st} (g\ast f)(t)\, dt\\
	& = \mathscr{L}\{(f\ast g)(t)\}.\qedhere
\end{align*}
\end{proof}
Some fundamental properties of the convolution operation are discussed in  Exercises~\ref{EX75}; see the problem~\ref{7-5-2}.

The following theorem concerns an interesting application of Theorem~\ref{thm:convolution-theorem} to linear initial value problems that involve the Dirac delta functions.  


\begin{theorem}\label{thm:IVP-convolution}
	Let $y_\delta(t)$ denote the solution of the initial value problem
 \begin{equation}\label{IVP-delta1}
	ay''+by'+cy = \delta(t), \quad y(0) = 0, y'(0) = 0.
\end{equation}
The the solution $y(t)$ of the initial value problem 
\begin{equation}\label{IVP-delta2}
	ay''+by'+cy = g(t), \quad y(0) = 0, y'(0) = 0
	\end{equation} where $a, b, c$ are constants with $a\ne 0,$ and $g$ is a piecewise continuous function that is of exponential order as $t\to\infty,$ is given by
	\[y(t) = (y_\delta \ast g)(t).\]
\end{theorem}

We first remark that $y'$ and $y''$ in (\ref{IVP-delta1}) are to be interpreted  differently (as \textit{distributional derivatives}) because of the presence of $\delta(t)$ on the right side of the equation. The fact beyond the scope of this book is that the Laplace transforms of  distributional derivatives coincide with the Laplace transforms of the classical derivatives.  


\begin{proof}
	Applying the Laplace transform $\L$ to the initial value problem (\ref{IVP-delta1}) and using $y(0)= y'(0)=0$  yields
	\[(as^2+bs +c)Y(s)= 1,\]that is,
	\[Y(s)= \frac{1}{as^2+bs +c}.\] Then 
	\[y_\delta(t) = \L^{-1}\left\{\frac{1}{as^2+bs +c}\right\}.\]
	
Now, taking the Laplace transform $\L$ of the initial value problem (\ref{IVP-delta2}) and using $y(0)= y'(0)=0$  yields
	\[(as^2+bs +c)Y(s)= G(s),\]that is,
	\[Y(s)= \frac{1}{as^2+bs +c}\; G(s).\] By Theorem~\ref{thm:convolution-theorem}, we obtain
	\[y(t) = \L^{-1}\left\{\frac{1}{as^2+bs +c}\right\}\ast \L^{-1}G(s) = y_\delta(t) \ast g(t).\qedhere\] 
\end{proof}
\begin{example}
	Consider the initial value problem $y''+ y= e^t,\; y(0) = y'(0)= 0.$
\end{example}
\begin{solution} In the context of Theorem~\ref{thm:IVP-convolution}, the solution to the initial value problem $y''+ y= \delta (t),\; y(0) = 0, \, y'(0)= 0,$ is given by 
	\[ y_\delta(t) = \L^{-1}\left\{\frac{1}{s^2+1}\right\} = \sin t.\]  We note here that $y'_\delta(t) =\cos t,$ and
	$y'_\delta (0+) = 1,$ which reflects  the behavior of $\delta(t)$ at $t = 0$ because  $y'_\delta $ would then have a jump discontinuity at $t = 0$, so that $y''_\delta $ would  be the Dirac delta $\delta(t)$ as a distribution.\footnote[1]{In the distribution theory, the distributional derivative of $\mathcal U(t)$ is the Dirac delta $\delta(t)$.} By Theorem~\ref{thm:IVP-convolution}, the solution to the given initial value problem is 
	\[ y(t) = y_\delta(t) \ast e^t = \sin t \ast e^t = -\frac{1}{2}\left(-e^t+\sin t+\cos t\right)\]
on $(-\infty, \infty).$
\end{solution}


\begin{Exercise}\label{EX75}
	\vspace{-\baselineskip}% <-- You don't need this line of code if there's some text here
	
	\Question\label{7-5-1}
For the functions $f$ and $g$ on $[0, \infty)$ given below, compute $f\ast g$, $\L\{(f\ast g)(t) \}$ and verify that $\L\{(f\ast g)(t) \} =\L\{f(t)\}\, \L\{g(t)\}.$
	
	\begin{tasks}(1)
		\task $ f(t) = g(t) = 1$ 
		\task $ f(t) = g(t) = e^{kt},$ where $k$ is any constant   
		\task $ f(t) = \sin t,\; g(t) = e^{kt},$ where $k$ is any real number
		\task $ f(t)= g(t) = \sin (\omega t),$ where $\omega$ is any real number
		\task $ f(t)= \sin(\omega t),\; g(t) = \cos(\omega t),$ where $\omega$ is any real number
	\end{tasks}
	
	\Question \label{7-5-2}
	Let $f, g, h$ be functions defined on the interval $[0, \infty)$, and assume that all the convolutions below are well-defined. Use the definition of the convolution  to prove the following properties.
		\begin{tasks}(1)
				\task\label{7-5-2(i)} $f\ast g= g\ast f$   
				\task $f\ast (g\ast h) = (f\ast g) \ast h$ 
				\task $f \ast (g+h)= f\ast g+ f\ast h$ 
				\task $f\ast k g= k( f\ast g),$ where $k$ is any constant
				\task\label{7-5-2(v)} $1\ast f(t) = \int_0^t f(u) \, du$
			\end{tasks}
	
	
	\Question \label{7-5-3}
	Use the convolution theorem (Theorem~\ref{thm:convolution-theorem}) to find the Laplace transform of  each of the following functions. Also,   find the convolution in  each one of \ref{7-5-3(i)}--\ref{7-5-3(v)}.

	\begin{tasks}(2)
		\task\label{7-5-3(i)}  $1\ast t\ast  t^2$
		\task $1\ast t^2\ast \sin t$
		\task  $1\ast \cos t\ast \sin t$
		\task $1\ast \sin t\ast \mathcal{U} (t-2)$ 
		\task\label{7-5-3(v)} $1\ast \mathcal{U} (t-2)\ast \mathcal{U} (t-5)$ 
		\task $\ds\int_0^t e^u\, du$
		\task $\ds\int_0^t \sin u\, du$
		\task $\ds\int_0^t u\sin u\, du$
		\task $\ds\int_0^t \sin u\,\cos(t-u)\, du$
		\task $\ds t\int_0^t \sin u\,\cos(t-u)\, du$
		
	\end{tasks}
\Question \label{7-5-4}
	Use the convolution theorem (Theorem~\ref{thm:convolution-theorem}) to find each of the following inverse Laplace transforms.
	\begin{tasks}[resume=false](2)
		\task $\ds\mathscr{L}^{-1}\left\{\frac{1}{(s+1)(s-1)}\right\}$
		\task $\ds\mathscr{L}^{-1}\left\{\frac{1}{s^3(s-1)}\right\}$
		\task $\ds\mathscr{L}^{-1}\left\{\frac{2}{(s^2+4)^2}\right\}$
		\task $\ds\mathscr{L}^{-1}\left\{\frac{s^2}{(s^2+4)^2}\right\}$
		\task $\ds\mathscr{L}^{-1}\left\{\frac{s^2}{(s^2-4)^2}\right\}$
		\task $\ds\mathscr{L}^{-1}\left\{\frac{1}{s^2(s^2+4)^2}\right\}$
	\end{tasks}

\Question \label{7-5-5} Solve the following integral or integro-differential equations using the Laplace transform.
\begin{tasks}[resume=false](1)
	\task $\ds y(t)+\int_0^t y(u)\, du = 1$
	\task $\ds y(t) = te^t+ \int_0^t u\,y(t-u)\, du$
	\task $\ds y'(t) = 1-\cos t - \int_0^t y(u)\, du, \quad y(0) =0$
	\task $\ds y'(t) +4 y(t) + 4 \int_0^t y(u)\, du=1, \quad y(0) =0$
\end{tasks}

\Question\label{7-5-6} Use Theorem~\ref{thm:IVP-convolution} to solve the following initial value problems.
\begin{tasks}[resume=false](1)
	\task $y'-3y= \sin t, \quad y(0)= 0$
	\task $y''+y= \sin t, \quad y(0)= 0, \; y'(0)=0$
	\task $y''+4y'+3y= e^t, \quad y(0)= 0, \; y'(0)=0$
	\task $y''+4y'+13y= \cos t, \quad y(0)= 0, \; y'(0)=0$
\end{tasks}
	
\Question \label{7-5-7} Let $f$ be a continuous function defined on the interval $[0, \infty)$, and assume that all the convolutions below are well-defined. Show the following properties of the convolution operation.
	\begin{tasks}(1)
		\task $f\ast \delta = f.$ (This means that $\delta$ is an identity element for the convolution operation.)
		\task $f(t)\ast \delta (t-a) =  \mathcal{U}(t-a) f(t-a),$ where $a\ge 0$ and is a fixed real number. 
	\end{tasks}	
	
\Question \label{7-5-8} 	Let $f, g$ be  functions with continuous derivatives  on the interval $[0, \infty)$ and of exponential order as $t\to\infty$.  Assume further that all the convolutions below are well-defined. Use the convolution theorem (Theorem~\ref{thm:convolution-theorem}) to prove the following properties.
	\begin{tasks}(1)
		\task $1\ast f'(t) = f(t) - f(0).$
		\task $(f\ast g)'(t) = g(0) f(t) + (g'\ast f)(t) = f(0) g(t) + (g\ast f')(t).$
	\end{tasks}	
The above properties are also valid for any differentiable functions $f$ and $g$, not necessarily of exponential order as $t\to\infty$, in light of the Leibniz rule\footnote{\scriptsize\begin{equation}\label{Leibniz-rule}
		\frac{d}{dt}\int_{a(t)}^{b(t)} f(u, t) \, du  = f(b(t), t) \frac{db(t)}{dt} - f(a(t), t)\frac{da(t)}{dt} +\int_{a(t)}^{b(t)} \frac{\partial f}{\partial t} (u, t)\, du
		\end{equation} } for differentiating an integral (see Theorem~\ref{thm:DUI-v2} in Appendix~\ref{Leibnitz Rule-Improper Integrals}).
	
	
	
\Question \label{7-5-9} By using the Leibniz rule (\ref{Leibniz-rule}), convert each of the following integral or integro-differential equations into an equivalent initial value problem and then solve it by using your favorite method. Compare your answers to those obtained in Problem~\ref{7-5-5}.
\begin{tasks}[resume=false](1)
	\task $\ds y(t)+\int_0^t y(u)\, du = 1$
	\task $\ds y(t) = te^t+ \int_0^t u\,y(t-u)\, du$
	\task $\ds y'(t) = 1-\cos t - \int_0^t y(u)\, du, \quad y(0) =0$
	\task $\ds y'(t) +4 y(t) + 4 \int_0^t y(u)\, du=1, \quad y(0) =0$
\end{tasks}


\Question\label{7-5-10} Solve the following initial value problems using the Laplace transform. Also, determine whether the solutions are continuous at the point(s) of concentration of the involved Dirac delta functions(s).
\begin{tasks}[resume=false](1)
	\task $y'-3y= \delta(t-1), \quad y(0)= 0$
	\task $y'+4y= \delta(t-2), \quad y(0)= 1$
	\task $y''+4y= \delta(t-2\pi), \quad y(0)= 0, \; y'(0)=0$
	\task $y''+4y'+3y= \delta(t-2\pi), \quad y(0)= 0, \; y'(0)=0$
	\task $y''+4y'+13y= \delta(t-2\pi), \quad y(0)= 0, \; y'(0)=0$
	\task $y''+4y'+3y= \delta(t-2\pi) +\delta(t-3\pi), \quad y(0)= 0, \; y'(0)=0$
	\task $y''+4y'+13y= \delta(t-2\pi) +\delta(t-3\pi), \quad y(0)= 0, \; y'(0)=0$
\end{tasks}

%\Question A 1 lb mass is suspended from the lower end of a vertical spring, stretching the spring 4 feet to reach its equilibrium position. The mass is then pulled downward an additional 1 foot and released from rest. Suppose that no damping and no external forces act on the system.   Let $y(t)$ denote the displacement of the mass at time $t$, with the convention that displacements measured downward from the equilibrium position are  positive. 
%\begin{enumerate}[label = (\roman*)]
%	\item Write down the initial value problem for the displacement $y(t).$
%	\item Find the first time at which the mass passes through the equilibrium position heading upward.
%	\item Determine the Dirac delta function  that models the impulsive force with which the mass must be struck with a hammer at the first upward passage through the equilibrium position, in such a way that the mass comes to rest instantly and remains at equilibrium thereafter.
%\end{enumerate}
 
 \Question\label{7-5-11} A 32 lb mass stretches a vertically hanging spring  4 feet to equilibrium. The mass is then pulled down 1 foot and released from rest. Assume no damping or external forces. Let $y(t)$ denote the displacement of the mass from equilibrium at time $t,$ measured positive in the downward direction. (Use $g= 32 ;\text{ft}/\text{s}^2$.)
\begin{enumerate}[label =(\roman*)]
	\item Set up an initial value problem for $y(t).$ 
	\item\label{partii} Determine a Dirac delta function that models the impulsive force on the mass exerted by striking it with a hammer to stop it at the first time it passes through equilibrium, keeping it at the equilibrium position thereafter.
	\item Repeat part~\ref{partii} to stop the mass at the second time it passes through equilibrium.
\end{enumerate}

 


\end{Exercise}

\setboolean{firstanswerofthechapter}{true}
\begin{multicols}{2}\scriptsize
	\begin{Answer}[ref={EX75}]
		\Question \label{7-5-1a}
		\begin{tasks}
			\task $t,\quad \frac{1}{s^2}$
			\task $t  e^{k t},\frac{1}{(k-s)^2}$
			\task $\frac{e^{k t}-k \sin t-\cos t}{k^2+1},\quad \frac{1}{\left(s^2+1\right) (s-k)}$
			\task $\frac{ \sin (\omega t )-\omega t \cos (\omega t )}{2 \omega },\frac{\omega ^2}{\left(s^2+\omega ^2\right)^2}$
			\task $\frac{1}{2} t \sin (\omega t),\frac{s \omega }{\left(s^2+\omega ^2\right)^2}$
		\end{tasks} 
		
	\Question\label{7-5-2a} Left as an exercise.
	
	\Question \label{7-5-3a}
	\begin{tasks}(1)
		\task $\frac{2}{s^6},\quad \frac{t^5}{60}$
		\task $\frac{1}{3} \left(\frac{6}{s^4}-\frac{6}{s^2}+\frac{6}{s^2+1}\right),\\ \frac{1}{3} \left(t^3-6 t+6 \sin t\right)$
		\task $\frac{1}{2} \left(\frac{1}{s^2+1}-\frac{s^2-1}{\left(s^2+1\right)^2}\right),\\ \frac{1}{2}  (\sin t-t \cos t)$
		\task $\frac{e^{-2 s}}{s^4+s^2},\quad \mathcal{U} (t-2) (t+\sin (2-t)-2)$
		\task $\frac{e^{-6 s}}{s^3},\quad \frac{1}{2} (t-6)^2 \mathcal{U}(t-6)$
		\task $\frac{1}{s(s-1)}$
		\task $\frac{1}{s^3+s}$
		\task $\frac{2}{\left(s^2+1\right)^2}$
		\task $\frac{s}{\left(s^2+1\right)^2}$
		\task $\frac{3 s^2-1}{\left(s^2+1\right)^3}$
	\end{tasks}
			
		\Question\label{7-5-4a}
	\begin{tasks}
		\task  $\frac{1}{2} \left(e^{ t}-e^{-t}\right) =\sinh t$
		\task	$e^t-\frac{t^2}{2}-t-1$
		\task $\frac{1}{8} (\sin (2 t)-2 t \cos (2 t))$
		\task $\frac{1}{4} (\sin (2 t)+2 t \cos (2 t))$
		\task $\frac{1}{8} e^{-2 t} \left(2 t+e^{4 t} (2 t+1)-1\right)$
		\task $\frac{1}{64} (4 t-3 \sin (2 t)+2 t \cos (2 t))$
	\end{tasks}
		
		\Question\label{7-5-5a}
		\begin{tasks}
			\task $y(t) = e^{-t}$
			\task $y(t) = \frac{1}{8}  \left((2 t^2 +3t+1)e^t -e^{-t}\right)$; use the property $f\ast g = g\ast f$	
			\task $y(t)= \frac{1}{2} (\sin t-t \cos t)$
			\task $y(t)=e^{-2 t} t$
		\end{tasks}	
		
			\Question\label{7-5-6a}
		\begin{tasks}
			\task $y(t) = -\frac{1}{10} \theta (t) \left(-e^{3 t}+3 \sin t+\cos t\right).$
			\task $y(t) = \left\{\frac{1}{2} (\sin t-t \cos t)\right\}$
			\task $y(t) = \frac{1}{8} e^{-3 t} \left(e^{2 t}-1\right)^2.$
			\task $\frac{1}{120} \big(3 \sin t+9 \cos t\\-e^{-2 t} (7 \sin (3 t)+9 \cos (3 t))\big)$
		\end{tasks}
		
			
			\Question\label{7-5-7a} 
		\begin{tasks}[resume=false]
			\task $(f\ast \delta)(t) =\int_0^t f(t-u) \delta(u)\,du =f(t)$
			\task $f(t)\ast \delta (t-a) =\int_0^t f(t-u) \delta(u-a)\,du\\
			 =\begin{cases}
			0 , \quad t<a\\
			\int_{-\infty}^\infty f(t-u) \delta(u-a)\, du,\;t\ge a
			\end{cases}\\ = \begin{cases}
			0 , \quad t<a\\
		f(t-a),\; t\ge a
			\end{cases}\\ = \mathcal{U}(t-a) f(t-a).$
		\end{tasks} 
			\Question\label{7-5-8a}
		\begin{tasks}
			\task $\L\{1\ast f'(t)\}= \frac{1}{s} [ s F(s) - f(0)] = F(s) - f(0)/ s.$ Taking $\L^{-1}$ of both sides yields the property.
			\task $\L\{(f\ast g)'(t)\}= s F(s) G(s) - (f\ast g)(0)] = sF(s) G(s) = [sF(s)- f(0)]G(s) + f(0) G(s).$ Taking $\L^{-1}$ of both sides yields the property.
		\end{tasks}
	

	
	\Question\label{7-5-9a} 
	\begin{tasks}[resume=false](1)
		\task Note that $y(0) = 1$.Differentiating with Leibniz rule gives:  $y'(t) +y(t) = 0.$ 
		\task Note that $y(0)=0.$ Differentiating with Leibniz rule gives:  $y'(t) = (t+1)e^t - t y(0) + \int_0^t y(u) \, du.$  Note that $y'(0) = 1.$ Differentiating again gives $y''(t) -y(t) = (t+2) e^t.$
		\task Note that $y'(0)= 0.$ Differentiating yields $y''(t)+y(t) = \sin t$. 
		\task Note that $y'(0)= 1.$ Differentiating gives $y''(t) + 4y'(t)+4 y(t) =0$. 
	\end{tasks}
	
	\Question\label{7-5-10a}
	\begin{tasks}
		\task $y(t) = e^{3 (t-1)} \mathcal{U} (t-1);$ the solution has a jump discontinuity at $t=1.$
		\task $y(t) = e^{-4 t} \left(e^8 \mathcal{U} (t-2)+1\right);$ the solution has a jump discontinuity at $t =2.$
		\task $y(t) = \frac12 \mathcal{U}(t-2\pi) \sin (2t);$ the solution is continuous everywhere including $t= 2\pi$.
		\task $y(t)= \frac{1}{2} e^{2 \pi -3 t} \left(e^{2 t}-e^{4 \pi }\right) \mathcal{U} (t-2 \pi ):$ the solution  is continuous everywhere including $t = 2\pi.$
		\task $y(t) = \frac{1}{3} e^{4 \pi -2 t} \mathcal{U} (t-2 \pi ) \sin (3 t);$ the solution  is continuous everywhere including $t = 2\pi.$
		\task $y(t) = \frac{1}{2}  \left(e^{3\pi-t }-e^{9\pi-3t }\right) \mathcal{U} (t-3 \pi )\\- \frac{1}{2} \left(e^{6 \pi-3t }-e^{2\pi- t}\right) \mathcal{U} (t-2 \pi );$
		the solution  is continuous everywhere including $t = 2\pi, 3\pi.$
		\task $y(t) = \frac{1}{3} e^{2 \pi -2 t} \Big(e^{2 \pi } \mathcal{U} (t-2 \pi )\\-\mathcal{U} (t-\pi )\Big)\sin (3 t);$
			the solution  is continuous everywhere including $t = \pi, 2\pi.$
	\end{tasks}


	\Question\label{7-5-11a}
	\begin{tasks}
	\task $y'' + 4y =0, \quad y(0) = 1, y'(0) =0.$
	\task $y'' + 4y =2 \delta(t- \pi/4), \quad y(0) = 1, y'(0) =0$; $y(t) = \cos(2t) [1- \mathcal{U}(t-\pi/4)].$
	\task$y'' + 4y =-2 \delta(t- 3\pi/4), \quad y(0) = 1, y'(0) =0$; $y(t) = \cos(2t) [1- \mathcal{U}(t-3\pi/4)].$
\end{tasks}

	\end{Answer}
\end{multicols}
\setboolean{firstanswerofthechapter}{false}



\section{Systems of Linear Differential Equations}
This chapter focuses on the applications of the Laplace transform  to solve systems of linear ordinary differential equations. Using the Laplace transform, we convert a system of differential equations  into a system of algebraic equations involving  Laplace transforms of the unknown functions (dependent variables), solve this algebraic system for the Laplace transforms , and then apply the inverse Laplace transform to recover the functions from their Laplace transforms.  These functions then become solutions of the differential equations in the system.


 To fix the ideas, we consider below a first-order linear system of ordinary differential equations in three variables $x(t), y(t), z(t)$, represented as a system of first-order equations of the form:

 \begin{equation}\label{linear-system}
 \begin{cases}
	 \displaystyle \frac{dx}{dt} = a_{11}x + a_{12}y + a_{13}z + f_1(t), \\[6pt]
	 \displaystyle \frac{dy}{dt} = a_{21}x + a_{22}y + a_{23}z + f_2(t), \\[6pt]
	 \displaystyle \frac{dz}{dt} = a_{31}x + a_{32}y + a_{33}z + f_3(t), 
	 \end{cases}
 \end{equation}
	where
	\begin{itemize}[noitemsep]
		\item \( x(t), y(t), z(t) \) are  unknown functions of time \( t \),
		\item the coefficients \( a_{ij} \) are fixed real numbers, and
		\item \( f_i(t) \) are given functions of \( t \), possibly zero.
	\end{itemize}
	
	The system (\ref{linear-system}) is said to be \textbf{\textit{decoupled}} if $a_{ij}= 0$ for all $i, j$ with $i\ne j$ and \textbf{\textit{coupled}} if for each $i = 1, 2, 3$ there is $j = 1, 2,3$ with $i\ne j$ and $a_{ij}\ne  0.$ The linear system (\ref{chap7-6-1}) is said to be \textbf{\textit{partially decoupled}} if it has a structure that allows solving one or more of the equations in isolation, and then using those solutions to simplify or solve the remaining equations.
	\subsection*{The Laplace Transform Method}
	
	To solve the system with the initial conditions
	\[
	x(0) = x_0, \quad y(0) = y_0, \quad z(0) = z_0,
	\]
	we apply the Laplace transform \( \L \) to each equation. The system is then transformed into
	
	\[
	\begin{cases}
		sX(s) - x_0 = a_{11}X(s) + a_{12}Y(s) + a_{13}Z(s) + F_1(s), \\
		sY(s) - y_0 = a_{21}X(s) + a_{22}Y(s) + a_{23}Z(s) + F_2(s), \\
		sZ(s) - z_0 = a_{31}X(s) + a_{32}Y(s) + a_{33}Z(s) + F_3(s),
	\end{cases}
	\]
	where \( X(s), Y(s), Z(s) \) are the Laplace transforms of \( x(t), y(t), z(t) \), and \( F_i(s) = \L\{f_i(t)\}, i =1, 2, 3 \). We solve the above system for  \( X(s), Y(s), Z(s) \)  using algebraic techniques, such as Cramer's rule, Gaussian elimination, etc. and then apply the inverse Laplace transform to obtain solutions \( x(t), y(t), z(t) \).
	
We discuss four main examples; namely, radioactive decay series, mixing  solutions, coupled spring-mass systems, and electrical circuits.
	%, and the double pendulum.}

\subsection{Radioactive Decay Series}
Consider a three-stage radioactive decay series of a radioactive element represented in a schematic diagram below:
\[
X \longrightarrow Y \longrightarrow Z
\]
where
\begin{itemize}[noitemsep]
	\item \( X \) is a parent radioactive isotope,
	\item \( Y \) is a radioactive daughter element, and
	\item \( Z\) is a stable (non-decaying) end element.
\end{itemize}
Let
\begin{align*}
	x(t) &=\ \text{the amount of isotope } X \text{ at time } t, \\
	y(t) &=\ \text{the amount of isotope } Y \text{ at time } t, \\
	z(t) &=\ \text{the amount of isotope } Z \text{ at time } t, \\
	a &=\ \text{the decay constant of } X, \text{    and} \\
	b &=\ \text{the decay constant of } Y.
\end{align*}    
We assume that the disintegration rate of $X$ at time $t$ is proportional to its amount at that time, and the same holds true for $Y$.
Then the system of differential equations modeling the decay series is
\begin{equation}\label{chap7-6-1}
\begin{cases}
	\smallskip
	\ds\frac{dx}{dt} &= -a x \\
	\smallskip
	\ds\frac{dy}{dt} &= ax - by\\
	\ds\frac{dz}{dt} &= by
\end{cases}
\end{equation}
subject to the initial conditions:
\[
x(0) = x_0, \quad y(0) = 0, \quad z(0) = 0.
\]

The system (\ref{chap7-6-1}) describes how the parent isotope \( X \) decays into \( Y \), which in turn decays into a stable isotope \( Z \). We can adopt this scheme to model real-world decay chains such as
\[
\text{Uranium-238} \rightarrow \text{Thorium-234}\rightarrow\cdots \rightarrow \text{Lead-206}.
\]
The linear system (\ref{chap7-6-1}) is an example of a \textbf{\textit{partially decoupled}} system. 
We observe that the  solutions of the differential equation in the system (\ref{chap7-6-1}) can be obtained in a sequence of steps: first solve the first equation for $x(t)$, substitute this $x(t)$ into the second equation to find $y(t)$ using the integrating factor method, and then substitute $y(t)$ into the third equation to solve for $z(t)$. Details are omitted, but the solutions can be found to be 
\[
	\begin{cases}
		x(t) &= x_0 e^{-a t}, \\[6pt]
		y(t) &= \frac{a x_0}{b - a} \left( e^{-a t} - e^{-b t} \right), \\[6pt]
		z(t) &= x_0 \left[ 1 + \frac{a e^{-b t} - b e^{-a t}}{b - a} \right].
	\end{cases}
\]
Let us obtain these solutions  by using the Laplace transform method.\\


\noindent\textbf{Step 1: Laplace Transforms}\\
Since
\[
	\L\left\{\frac{dx}{dt}\right\} = -a \L\{x(t)\},\] we have
	 \[s X(s) - x_0 = -a X(s),\] and therefore
 \[X(s) = \frac{x_0}{s + a}.\]
Similarly,
	
	\[\L\left\{\frac{dy}{dt}\right\} = a X(s) - b Y(s),\] which yields
	\[ s Y(s) = a X(s) - b Y(s),\] and therefore
\[Y(s) = \frac{a X(s)}{s + b} = \frac{a x_0}{(s + a)(s + b)}.\] 
Also, 
	\[\L\left\{\frac{dz}{dt}\right\} = b Y(s)\]  gives
	\[ Z(s) = \frac{b}{s} Y(s) = \frac{a b x_0}{s(s + a)(s + b)}\]


\noindent\textbf{Step 2: Inverse Laplace Transforms}\\
We have
\[
x(t) = \L^{-1}\left\{\frac{x_0}{s + a}\right\} = x_0 e^{-a t}.
\]
Since
\[
Y(s) = \frac{a x_0}{(s + a)(s + b)} 
= \frac{a x_0}{b - a} \left( \frac{1}{s + a} - \frac{1}{s + b} \right),\] we have
\[ y(t) = \frac{a x_0}{b - a} \left( e^{-a t} - e^{-b t} \right).
\]
Finally, since
\[
Z(s) = \frac{a b x_0}{s(s + a)(s + b)},\] we have
\[ z(t) = \int_0^t b y(\tau) \, d\tau 
= \frac{a b x_0}{b - a} \left( \frac{1 - e^{-a t}}{a} - \frac{1 - e^{-b t}}{b} \right)
= x_0 \left[ 1 + \frac{a e^{-b t} - b e^{-a t}}{b - a} \right]
\]
Thus, the solutions are given by

\[
	\begin{cases}
		x(t) &= x_0 e^{-a t}, \\[6pt]
		y(t) &= \frac{a x_0}{b - a} \left( e^{-a t} - e^{-b t} \right), \\[6pt]
		z(t) &= x_0 \left[ 1 + \frac{a e^{-b t} - b e^{-a t}}{b - a} \right],
	\end{cases}
\]
as expected.

\subsection{Mixing Tanks}


%	
%	\begin{tikzpicture}[thick, scale=1.2, every node/.style={scale=1}]
%		% Tank A
%		\draw[fill=blue!10] (0,0) rectangle (2,2);
%		\node at (1,1) {$\text{Tank A}$};
%		\node at (1,-0.5) {$x(t)$};
%		
%		% Tank B
%		\draw[fill=blue!10] (6,0) rectangle (8,2);
%		\node at (7,1) {$\text{Tank B}$};
%		\node at (7,-0.5) {$y(t)$};
%		
%		% Flow from A to B
%		\draw[->, thick] (2,1.5) -- node[above] {$k_1$} (6,1.5);
%		
%		% Flow from B to A
%		\draw[->, thick] (6,0.5) -- node[below] {$k_2$} (2,0.5);
%		
%		% Optional: inlet to A
%		\draw[->, thick] (-1,1.5) -- node[above] {$r_{\text{in}}$} (0,1.5);
%		\node at (-1.2,1.8) {In};
%		
%		% Optional: outlet from B
%		\draw[->, thick] (8,0.5) -- node[below] {$r_{\text{out}}$} (9,0.5);
%		\node at (9.3,0.8) {Out};
%		
%		% Labels
%		\node at (4,2.2) {Exchange of fluid between tanks};
%		
%	\end{tikzpicture}
	


%\textcolor{red}{open the figure below}
\begin{figure}[h]
	\centering
			\begin{tikzpicture}[thick, scale=1.2, every node/.style={scale=1}]
				
				% Tank A
				\shade[ball color=blue!5] (0,0) ellipse (1 and 0.3); % Top ellipse
				\draw[thick] (-1,0) -- (-1,-2); % Left side
				\draw[thick] (1,0) -- (1,-2);   % Right side
				\draw[thick] (-1,-2) arc[start angle=180,end angle=360,x radius=1, y radius=0.3]; % Bottom
				\draw[dashed] (-1,0) arc[start angle=180,end angle=360,x radius=1, y radius=0.3]; % Top (dashed back)
				\node at (0,-1) { Tank A};
				\node at (0,-1.5) { $x(t)$};
				
				% Tank B
				\shade[ball color=blue!5] (3.5,0) ellipse (1 and 0.3);
				\draw[thick] (2.5,0) -- (2.5,-2);
				\draw[thick] (4.5,0) -- (4.5,-2);
				\draw[thick] (2.5,-2) arc[start angle=180,end angle=360,x radius=1, y radius=0.3];
				\draw[dashed] (2.5,0) arc[start angle=180,end angle=360,x radius=1, y radius=0.3];
				\node at (3.5,-1) { Tank B};
				\node at (3.5,-1.5) { $y(t)$};
				
				% Pipe from B to A (top)
				\draw[thick] (1, -0.2) -- (2.5, -0.2);
				\draw[thick] (1, -0.5) -- (2.5, -0.5);
				\draw[->, very thick, blue!60!black] (2, -0.35) --(1.6, -0.35)  node[left] {$k_2$};
				
			
				% Pipe from A to B (bottom)
				\draw[thick] (2.5,-1.6) -- (1,-1.6);
				\draw[thick] (2.5,-1.9) -- (1,-1.9);
				\draw[->, very thick, blue!60!black](1.6,-1.75)  -- (2,-1.75) node[right] {$k_1$};
				
				% Inflow into Tank A
				\draw[thick] (-2,-0.2) -- (-1,-0.2);
				\draw[thick] (-2,-0.5) -- (-1,-0.5);
				\draw[->, very thick, teal] (-1.5, -0.35) -- (-1.25, -0.35);
				\node[left, teal] at (-1.5, -0.35) { $r_{\text{in}}$};
				\node[left, blue] at (-1.5, -0.05) { $c$};
				
				% Outflow from Tank B
		
				\draw[thick] (4.5,-1.9) -- (5.5,-1.9);
				\draw[thick] (4.5,-1.6) -- (5.5,-1.6);
				
				\draw[->, very thick, red] (4.5, -1.75) -- (4.8, -1.75);
				\node[right, red] at (4.8, -1.75) { $r_{\text{out}}$};
			\end{tikzpicture}
			\end{figure}
		
\[\displaystyle\begin{cases}
	\displaystyle x'(t) = r_{\text{in}}\; c+\frac{k_2y(t)}{V_2+t(k_1-k_2-r_{\text{out}})}-\frac{k_1x(t)}{V_1 +t(r_{\text{in}}+k_2-k_1)},\\[3ex]
	\displaystyle y'(t) = \frac{k_1x(t)}{V_1 +t(r_{\text{in}}+k_2-k_1)}-\frac{(k_2+r_{\text{out}})y(t)}{V_2+t(k_1-k_2-r_{\text{out}})}.
\end{cases}
\]
When \(r_{\text{in}} = k_1-k_2 = r_{\text{out}},\) the system becomes

\[\displaystyle\begin{cases}
	\displaystyle x'(t) = r_{\text{in}}\; c+\frac{k_2y(t)}{V_2} - \frac{k_1x(t)}{V_1 },\\[3ex]
	\displaystyle y'(t) = \frac{k_1x(t)}{V_1}-\frac{(k_2+r_{\text{out}})y(t)}{V_2}.
\end{cases}
\]

\subsection{Coupled Spring-Mass Systems} 
Suppose that two masses $m_1$ and $m_2$ are connected to two springs $A$ and $B$   hanging vertically from  a rigid support to form a coupled spring-mass system as shown in the figure below.  
%The figure includes labels for spring constants k₁ and k₂, equilibrium positions, displacements x₁(t) and x₂(t), and restoring forces.
% with spring constants $k_1$ and $k_2$, respectively. 
% 
% 
% 	\begin{tikzpicture}[scale=1.1, every node/.style={font=\small}]
% 		
% 		% Ceiling
% 		\draw[thick] (-1,0) -- (3,0);
% 		
% 		% Spring from ceiling to mass m1
% 		\draw[decorate,decoration={coil,aspect=0.5, segment length=5pt, amplitude=4pt}] 
% 		(1.0,0) -- (1.0,-1.5);
% 		\node at (1.2,-0.8) {$\quad k_1$};
% 		
% 		% Mass m1
% 		\draw[fill=gray!20] (0.7,-1.5) rectangle (1.3,-2.0);
% 		\node at (1,-1.9) {$m_1$};
% 		
% 		% Spring from m1 to m2
% 		\draw[decorate,decoration={coil,aspect=0.5, segment length=5pt, amplitude=4pt}] 
% 		(1.0,-2.0) -- (1.0,-3.5);
% 		\node at (1.2,-2.75) {$\quad k_2$};
% 		
% 		% Mass m2
% 		\draw[fill=gray!20] (0.7,-3.5) rectangle (1.3,-4.0);
% 		\node at (1,-3.9) {$m_2$};
% 		
% 		% Equilibrium lines
% 		\draw[dashed] (-0.5,-1.7) -- (2.5,-1.7);
% 		\node[left] at (-0.5,-1.7) {\(x_1=0\)};
% 		
% 		\draw[dashed] (-0.5,-3.7) -- (2.5,-3.7);
% 		\node[left] at (-0.5,-3.7) {\(x_2=0\)};
% 		
% 		% Displacement vectors
% 		\draw[->, thick] (1.6,-1.5) -- (1.6,-1.0);
% 		\node[right] at (1.6,-1.25) {$x_1(t)$};
% 		
% 		\draw[->, thick] (1.6,-3.5) -- (1.6,-3.0);
% 		\node[right] at (1.6,-3.25) {$x_2(t)$};
% 		
% 		% Restoring force vectors
% 		\draw[->, thick, red] (1.3,-1.75) -- (1.8,-1.75);
% 		\node[right] at (1.8,-1.75) {\textcolor{red}{$-k_1 x_1$}};
% 		
% 		\draw[->, thick, red] (1.3,-3.75) -- (1.8,-3.75);
% 		\node[right] at (1.8,-3.75) {\textcolor{red}{$-k_2(x_2 - x_1)$}};
% 		
% 	\end{tikzpicture}
 	
Let $x_1(t)$ and $x_2(t)$ denote the vertical displacements of the masses from their equilibrium positions.  We take  the positive direction of the displacements to be  vertically downward.
 When the masses are in motion, the spring $B$ is subject to both a stretch and a compression, so that its net stretch of the spring $B$ is given by 
 $x_2-x_1$ when the stretch of the spring $A$ is $x_1$. Let $k_1$ and $k_2$ be the constants of proportionality in Hooke's law for the springs $A$ and $B$, respectively. We assume that there is no damping or external force acting on the masses and that weights of the springs are negligible compared to  weights of the masses.
 
  The restoring force of the spring $A$ acting on the mass $m_1$ at time $t$  is  $-k_1 x_1(t),$ and the restoring force of the spring $B$  acting on the mass $m_2$ at time $t$ is  $-k_2(x_2(t)- x_1(t)).$  In addition to the restoring force of the spring $A$, the mass $m_1$ also experiences the force of $k_2(x_2(t)- x_1(t))$ at time $t$. Suppressing $t$ in $x_1(t)$ and $x_2(t)$ and using   Newton's law,
we see that the motion of the two masses in the  spring-mass system is described by the coupled linear system
 \begin{equation}\begin{cases}\label{spring-mass-1}
 	m_1x''_1 &= -k_1 x_1 + k_2 (x_2 - x_1) = -(k_1+k_2) x_1 +k_2 x_2,\\
 	m_2 x''_2 &= -k_2 (x_2 - x_1)= k_2 x_1-k_2 x_2,
 \end{cases}
 \end{equation}
which, together with the initial conditions  $x_1(t_0) = a, x'_1(t_0) = b$, $y_1(t_0) = c, y'_1(t_0) = d,$  forms an initial value problem.  

In the next example, we solve the initial value problem (\ref{spring-mass-1}) for  $ m_1=m_2=1, k_1=3,  k_2=2,$ and $x_1(0)=0, x'_1(0)=-1, x_2(0)=0, x'_2(0)=1.$


\begin{example} Solve the second-order coupled  system of linear differential equations with initial conditions
	\begin{equation}
\begin{cases}\label{spring-mass-2}
	x_1'' = -5 x_1 + 2 x_2 , \\
	x_2'' = 2 x_1 - 2 x_2, \\
	x_1(0) = 0, \quad x_1'(0) = -1, \\
	x_2(0) = 0, \quad x_2'(0) = 1.
\end{cases}
\end{equation}
\begin{solution}
		Taking the Laplace transform yields
		
		\[
		\begin{cases}
			s^2 X_1(s) - s x_1(0) - x_1'(0) = -5 X_1(s) + 2 X_2(s), \\
			s^2 X_2(s) - s x_2(0) - x_2'(0) = 2 X_1(s) - 2 X_2(s).
		\end{cases}
		\]
		Using the given initial conditions yields	
		\[
		\begin{cases}
			(s^2+5) X_1(s) + 1 =  2 X_2(s), \\
			(s^2+2) X_2(s) - 1 = 2 X_1(s).
		\end{cases}
		\]
%		Rewrite as:
%		
%		\[
%		\begin{bmatrix}
%			s^2 + 5 & -2 \\
%			-2 & s^2 + 2
%		\end{bmatrix}
%		\begin{bmatrix}
%			X_1(s) \\ X_2(s)
%		\end{bmatrix}
%		=
%		\begin{bmatrix}
%			-1 \\ 1
%		\end{bmatrix}.
%		\]	
	Solving for $X_1(s)$ and $X_2(s)$ gives
		
		\[
		X_1(s) = \frac{-s^2}{(s^2 + 1)(s^2 + 6)} \quad \text{and}\quad  
		X_2(s) = \frac{s^2 + 3}{(s^2 + 1)(s^2 + 6)}.
		\]
		Partial fraction decomposition of $X_1(s)$ and $X_2(s)$ are
		
		\[
		X_1(s) = \frac{1/5}{s^2 + 1} - \frac{6/5}{s^2 + 6}\quad \text{and}\quad
		X_2(s) = \frac{2/5}{s^2 + 1} + \frac{3/5}{s^2 + 6}.
		\]
		Taking the inverse Laplace transform yields
		\[
			\begin{cases}
				x_1(t) &= \frac{1}{5} \sin t - \frac{6}{5 \sqrt{6}} \sin(\sqrt{6} t), \\
				x_2(t) &= \frac{2}{5} \sin t + \frac{3}{5 \sqrt{6}} \sin(\sqrt{6} t),
			\end{cases}
		\]
		which render the solutions of the given initial value problem (\ref{spring-mass-2}).
		
%\textcolor{red}{Open this figure}
		
\begin{figure}[h]
	\centering
\includegraphics[scale=.8]{chap05/Mathematica/7-6-1}
\label{chap7:fig1}
\caption{Graphs of the solutions $x_1(t)$ and $x_2(t)$}\qedhere
\end{figure}	
\end{solution}
\end{example}
 

\subsection{Electrical Network Systems}
We discussed  series and parallel $LRC$ circuits in Section~\ref{subsec:electric-circuits}. In Figure~\ref{fig:systemLRC} below,  we have an $LRC$ circuit with a voltage source  \(e(t)\). We develop a system of linear differential equations to find the charge \(q_1(t)\) on the capacitor and currents $i(t)$ and $i_2(t)$  flowing through the resistor and inductor, respectively. Let $i_1(t)$ be the current flowing through the capacitor. 


%\textcolor{red}{open the figure below}
\begin{figure}[h]
	\begin{circuitikz}[american]
		% Current source from bottom to top
		\draw
		(0,0) to[sV, invert, v_<={}, i>=${i(t)}$] (0,4)
		to[american resistor, i_=${}$, l=$R$] (2,4)
		to[short] (2.5,4); % top horizontal wire
		\node[circle, fill=white, inner sep=1pt, label=above left: Voltage source \(e(t)\)] at (-0.4,1.75) {};
		% Split point at (2,4), three parallel branches
	
		\node[circle, fill=black, inner sep=1pt, label=above right: A] at (2.5,4) {};
		% Resistor branch (left)
		\draw
		(2.5,4)
		to[C=$C$, i>_=${i_1(t)}$ ] (2.5,.2)
		-- (2.5,0); % back to bottom of source
	
		% Inductor branch (right)
		\draw
		(2.5,4) -- (4.5,4)
		to[cute inductor, l=$L$, i>_=${i_2(t)}$] (4.5,0)
		-- (4.5,0);
		% Merge point at (0,4), three parallel branches
		\node[circle, fill=black, inner sep=1pt, label=above right:] at (2.5,0) {};
		\draw
		(4.5,0)	-- (0,0); % central node
	\end{circuitikz}
	\\
	\caption{$LRC$ circuit}
	\label{fig:systemLRC}
\end{figure}

We observe that the circuit contains three distinct closed loops:
\begin{enumerate}[label= ,noitemsep]
	\item \textbf{Loop 1}: consisting of the battery, resistor, and capacitor;
	\item \textbf{Loop 2}: consisting of the battery, resistor, and inductor; and
	\item \textbf{Loop 3}: formed by the capacitor and inductor alone.
\end{enumerate}

The system of corresponding equations for the loops are:
\[
\begin{cases}
\displaystyle R\, i(t)+\frac{1}{C} q_1(t) = e(t) & \qquad\text{(for Loop 1)}\\  
\displaystyle L\, i_2'(t) + R\, i(t) = e(t)  & \qquad \text{(for Loop 2)}   \\
\displaystyle L\, i_2'(t) - \frac{1}{C} q_1(t) = 0& \qquad \text{(for Loop 3)}  
\end{cases}
\]
We observe that each equation in the system can be obtained from the remaining two. For example, the equation for Loop 2 can be obtained by adding the equations for Loop 1 and Loop 3. For  the discussion that follows, we omit the equation for Loop 2.

Furthermore, by Kirchhoff's second law, the current split at the junction A is given by  
\begin{equation*}
i(t) = i_1(t) + i_2(t),
\end{equation*}
which, in view of \( i_1(t)= q_1'(t)\), becomes

 \begin{equation*}
q'_1(t) + i_2(t) -i(t) =0.
 \end{equation*}
Thus, a linear system in $i(t), q_1(t)$ and $i_2(t)$   is

\begin{numcases}\empty
	R\, i(t)+\frac{1}{C} q_1(t) = e(t) & \qquad\text{(for Loop 1)}  \label{eq:LRC-system-1} \\
	L\, i_2'(t) - \frac{1}{C} q_1(t) = 0& \qquad \text{(for Loop 3)}  \label{eq:LRC-system-2}\\
	q'_1(t) + i_2(t) -i(t) =0  & \qquad \text{(the current split)}   \label{eq:LRC-system-3}
\end{numcases}

 
 We now proceed to solve the system \eqref{eq:LRC-system-1}-\eqref{eq:LRC-system-3}.
Differentiating \eqref{eq:LRC-system-3} with respect to $t$ yields
 \begin{equation}\label{eq:LRC-system-4}
	i'(t) = q''_1(t) + i'_2(t).
\end{equation}
Solving for $i'_2(t)$ from \eqref{eq:LRC-system-2} and substituting it into \eqref{eq:LRC-system-4} gives
 \begin{equation*}
	i'(t) = q''_1(t) + \frac{1}{LC}q_1(t).
\end{equation*}
Substituting this expression for $i'(t)$ into the equation obtained by
differentiating \eqref{eq:LRC-system-1} with respect to $t$, we get
 \begin{equation}\label{eq:LRC-system-5}
 	LRC q''_1(t) +Lq'_1(t)+ Rq_1(t) =  LCe'(t),
\end{equation}
which can be solved for $q_1(t).$   Once $q_1(t)$ is determined, we find $i_1(t)$ by using \( i_1(t)= q_1'(t).\)  We then find \(i(t)\) from \eqref{eq:LRC-system-1} by using \(q_1(t)\) there. Finally, we find $i_2(t) = i(t) - i_1(t).$

%For analogy, we can interchange the spring-mass system parameters and the \(LRC\) circuit network parameters in \eqref{eq:LRC-system-5} as follows:
%\begin{center}
%	\begin{tabular}{cccc}
%		\hline
%		Spring-Mass  &     \(LRC\) circuit    \\ \hline
%		\(m\)  &\(R\)  \\
%		\(b\)     & \(\displaystyle\frac{1}{C}\) \\[2ex]
%		\(k\)     & \(\displaystyle\frac{R}{LC}\) \\[2ex]
%		\(\omega =\sqrt{k/m}\) &\(\omega =\sqrt{1/LC}\)   \\[2ex]
%		\hline
%	\end{tabular}
%\end{center}





\begin{Exercise}\label{EX76}
	\vspace{-\baselineskip}% <-- You don't need this line of code if there's some text here
	
	\Question\label{7-6-1}
	
	
	
	
\end{Exercise}

\setboolean{firstanswerofthechapter}{true}
\begin{multicols}{2}\scriptsize
	\begin{Answer}[ref={EX75}]
		\Question \label{7-6-1a}
		\begin{tasks}
			\task
		\end{tasks} 
		
	
		
	\end{Answer}
\end{multicols}
\setboolean{firstanswerofthechapter}{false}
       				%%% Open this one 
                                                                     

%\chapter{Introduction}\fakesections{}
%\chapter{Theory of Numbers}\fakesections{}
%\chapter{Irrational and Transcendent Numbers}\fakesections{}
%\chapter{Complex Numbers}\fakesections{}
%\chapter{Quaternions and Ausdehnungslehre}\fakesections{}
%\chapter{Theory of Equations}\fakesections{}
%\chapter{Substitutions and Groups}\fakesections{}
%\chapter{Determinants}\fakesections{}
%\chapter{Quantics}\fakesections{}
%\chapter{Calculus}\fakesections{}
%\chapter{Differential Equations}\fakesections{}
%\chapter{Infinite Series}\fakesections{}
%\chapter{Theory of Functions}\fakesections{}
%\chapter{Probabilities and Least Squares}\fakesections{}
%\chapter{Analytic Geometry}\fakesections{}
%\chapter{Modern Geometry}\fakesections{}
%\chapter{Trigonometry and Elementary Geometry}\fakesections{}
%\chapter{Non-Euclidean Geometry}\fakesections{}
%\chapter{Bibliography}\fakesections{}
%\chapter{General Tendencies}\fakesections{}

\chapter*{Answers to all problems}
\begin{multicols}{2}
	\raggedcolumns
	\shipoutAnswer
\end{multicols}

\begin{appendices}
	\appendixpage
	\noappendicestocpagenum
	\addappheadtotoc
	

\chapter{Differentiation Under the Integral Sign}\label{Leibnitz Rule-Improper Integrals}
We start with the following basic result from the multivariate calculus.

\begin{theorem}[Fubini Theorem-Version I]\label{fubini-v1}
	Let $f(x, y)$ be a continuous function on a rectangle $R=\{(x, y): a\le x \le b, \; c\le  y\le d\}.$ Then %$f$ is integrable on $R$ and its double integral on $R$ equals
	\[\int_a^b \int_c^d  f(x, y)\, dx\, dy = \int_c^d \int_a^b f(x, y) \, dy\, dx.\]
\end{theorem}
As an application of Theorem~\ref{fubini-v1} and fundamental theorem of calculus, we have a basic result for differentiating under the integral sign.

\begin{theorem}[Leibniz Rule]\label{DUI-v1}
	Let $f(x, y)$ be a  function on a rectangle $R=\{(x, y): a\le x \le b, \; c\le  y\le d\}$ such that both $f$ and $\frac{\partial f}{\partial y}$ are continuous on $R.$ Then 
	\[\frac{d}{dy}\int_a^b  f(x, y)\, dx =  \int_a^b \frac{\partial f}{\partial y}(x, y) \,  dx.\]
\end{theorem}
\begin{proof}
Applying Theorem~\ref{fubini-v1} to $\frac{\partial f}{\partial y},$ we have
\[\frac{d}{dy}\int_c^y \int_a^b \frac{\partial f}{\partial z}(x, z) \, dz\, dx=\frac{d}{dy}\int_a^b \int_c^y  \frac{\partial f}{\partial z}(x, z)\, dx\, dz \]
for all $y\in [c, d].$  Applying the fundamental theorem of calculus, we obtain
\[\int_a^b \frac{\partial f}{\partial y}(x, y) dx =\frac{d}{dy}\int_a^b   \big(f(x, y) - f(x, c)\big)\, dx  =\frac{d}{dy}\int_a^b   f(x, y) \, dx. \qedhere\]
\end{proof}

\begin{theorem}[General Leibniz Rule]\label{thm:DUI-v2}
	Let $f(x, y)$ be a  function on a rectangle $R=\{(x, y): a\le x \le b, \; c\le  y\le d\}$ such that both $f$ and $\frac{\partial f}{\partial y}$ are continuous on $R.$  Let $a$ and $b$ be differentiable functions on $[c, d]$ such that $a\le a(y), b(y) \le b$ for all $y$ in $[c, d].$
	Then 
	\[\frac{d}{dy}\int_{a(y)}^{b(y)}  f(x, y)\, dx =   b'(y)\, f(b(y), y) - a'(y) \,f((y), y)+\int_{a(y)}^{b(y)} \frac{\partial f}{\partial y}(x, y) \,  dx.\]
\end{theorem}

\begin{proof}
	Define  a function $J:  [a, b]\times [a, b]\times [c, d]\to \mathbf R$ by
	\[J(u, v, y) = \int_u^v f(x, y)\, dx. \]
	Then, by Theorem~\ref{DUI-v1}, we have
	\[\frac{\partial J}{\partial y} = \int_u^v \frac{\partial f}{\partial y} f(x, y) \, dx.\]
	Also, we have
	\[\frac{\partial J}{\partial u}=  - f(u, y) \quad\text{ and }\quad 
	\frac{\partial J}{\partial v}=   f(v, y).\]
	By the chain rule, we have
	\[\begin{split}
		\frac{dJ}{dy} (a(y), b(y), y)& =\frac{\partial J}{\partial u} (a(y), b(y), y)\, a'(y) + \frac{\partial J}{\partial v} (a(y), b(y), y)\, b'(y) \\
		&+ \frac{\partial J}{\partial y} (a(y), b(y), y),
	\end{split}
	\]
	which gives
		\[\frac{d}{dy}\int_{a(y)}^{b(y)}  f(x, y)\, dx =   b'(y) f(b(y), y) - a'(y) f(a(y), y) +\int_{a(y)}^{b(y)} \frac{\partial f}{\partial y}(x, y) \,  dx.\qedhere\]
	\end{proof}
	
	\begin{theorem}[Leibniz Rule for Improper Integrals]\label{thm:Leibnitz Improper Integrals}
		Let $f(x, y)$ be a  function on a rectangle $R=\{(x, y): a\le x <\infty, \; c\le  y\le d\}$ such that, for each $y$ in $I,$ both $f$ and $\frac{\partial f}{\partial y}$ have  improper integrals on $[a, \infty)$ and such that, for each $x$ in $[a, \infty),$ $f(x, y)$ is differentiable in $y$ for each $y$ in $I.$ Suppose, further, that there exists a function $g$ on $[a, \infty)$ such that 
		\[\abs{\frac{\partial f}{\partial y}(x, y)}\le g(x)\] for all $x\in [a, \infty)$ and for all $y$ in $I$ and such that 
		\[\int_a^\infty g(x) \,dx\] is convergent.
	Then 
	\[\frac{d}{dy}\int_a^\infty  f(x, y)\, dx =  \int_a^\infty \frac{\partial f}{\partial y}(x, y) \,  dx.\]
	\end{theorem}
	

	The dominated convergence theorem (whose proof is beyond the scope of this book) for improper integrals over $[a, \infty)$ is required to prove Theorem~\ref{thm:Leibnitz Improper Integrals}. The reader may consult the book\footnote{J.~W. Lewin and M. Lewin, {\it An introduction to mathematical analysis}, The Random House/Birkh\"auser Mathematics Series, Random House, New York, 1988; MR1019088} for the dominated convergence theorem for improper integrals.





\chapter{Partial Fraction Decomposition}\label{partial-fraction-decomposition}


%
%% --------------------------------------------------------------------
%% Packages
%% --------------------------------------------------------------------
%\usepackage{amsmath,amssymb,amsfonts}
%\usepackage{enumitem}
%\usepackage{tikz}
%\usetikzlibrary{arrows.meta, positioning, shapes.geometric}
%
%% Hyperbolic operators (for consistency in examples)
%\DeclareMathOperator{\sech}{sech}
%\DeclareMathOperator{\csch}{csch}
%
%% Counter for auto-numbered items
%\newcounter{tableitem}
%\newcommand{\resetitems}{\setcounter{tableitem}{0}}
%\newcommand{\nextitem}{\stepcounter{tableitem}\arabic{tableitem}.}
%
%\section*{Summary of Partial Fraction Decomposition Techniques}

Let $P$ and $Q$ be general polynomials in $x,$  and let $m$ and $n$ be  the degrees of $P$ and $Q$, respectively. Then the \textbf{\textit{rational}} function
\[
R(x)=\frac{P(x)}{Q(x)}
\]
is said to be \textbf{\textit{proper }} if \(m<n\) and \textbf{\textit{improper}} if  \(m\ge n.\)
When  $R(x)$ is improper, we   perform polynomial division to obtain
\[
R(x)=q(x)+\frac{r(x)}{Q(x)},
\]
where $q(x)$ is the \textbf{\textit{quotient}} polynomial of degree  $m-n,$ and $r(x)$ is called the \textbf{\textit{remainder}} polynomial and it is of degree less than $n.$ For example, if $P(x) = x^3, Q(x) = x^2-1,$ then 
\[\frac{x^3+2}{x^2-1} = x+ \frac{x+2}{x^2-1},\] so that \(q(x) = x\) and $r(x) = x+2.$

After performing polynomial division, the term \(q(x)\) is already a polynomial and requires no  decomposition. The  term \(\dfrac{r(x)}{Q(x)}\) is a proper rational function and the only part of $R(x)$ that is usually decomposed into partial fractions. Therefore, in what follows, we assume that \(R(x)\) itself is  proper. The  method of partial fraction decomposition  then depends entirely  on the factorization of the
denominator $Q(x)$.  We now examine  all possible cases that may occur.

\bigskip

%--------------------------------------------------------------
\subsection*{1. Linear Factors \texorpdfstring{$(x-a)$}{(x-a)}}

If the denominator contains distinct linear factors,
\[
Q(x)=(x-a_1)(x-a_2)\cdots(x-a_n),
\]
then the decomposition has the form
\[
\frac{P(x)}{Q(x)}
= \frac{A_1}{x-a_1}+\frac{A_2}{x-a_2}+\cdots+\frac{A_n}{x-a_n},
\]
where  $A_1, \dots, A_n$ are constants to be determined. Let us illustrate this with an example below.

\paragraph{Example 1.} Perform  partial fraction decomposition of 
\[
\frac{3x+5}{(x-1)(x+2)}.
\]
Suppose there are constants $A$ and $B$ such that 
\[
\frac{3x+5}{(x-1)(x+2)}
= \frac{A}{x-1}+\frac{B}{x+2}.
\]
Multiplying through by $(x-1)(x+2)$ gives
\[
3x+5 = A(x+2)+B(x-1)
\] which must be an identity in $x.$ For convenience, we do allow $x=1$ or $x=-2$ in this identity.
We find $A=2$, $B=1$, so
\[
\frac{3x+5}{(x-1)(x+2)}
= \frac{2}{x-1}+\frac{1}{x+2}.
\]

\bigskip

%--------------------------------------------------------------
\subsection*{2. Repeated Linear Factors \texorpdfstring{$(x-a)^n$}{(x-a)^n}}

If 
\[
Q(x)=(x-a)^n,
\]
then the decomposition includes every power of the linear factor:
\[
\frac{P(x)}{(x-a)^n}
= \frac{A_1}{x-a} + \frac{A_2}{(x-a)^2}
+ \cdots + \frac{A_n}{(x-a)^n},
\]
$A_1, \dots, A_n$ are constants to be determined.

\paragraph{Example 2.}  Perform the partial fraction decomposition of 
\[
\frac{x+1}{(x-1)^2}.
\]
Suppose there exist constants $A$ and $B$ such that
\[\frac{x+1}{(x-1)^2}= \frac{A}{x-1}+\frac{B}{(x-1)^2}.\]
Multiplying through by  $(x-1)^2$ gives
\[
x+1 = A(x-1) + B.
\]
We find $A=1$ and $B=2.$ Then
\[
\frac{x+1}{(x-1)^2}
= \frac{1}{x-1}+\frac{2}{(x-1)^2}.
\]

\bigskip

%--------------------------------------------------------------
\subsection*{3. Irreducible Quadratic Factors \texorpdfstring{$(x^2+bx+c)$}{(x²+bx+c)}}

Suppose that $Q(x)$ contains  a factor $x^2+bx+c$ that is irreducible  over the real numbers.  The partial fraction decomposition of \(\displaystyle\frac{P(x)}{Q(x)}\) contains a term of the form
\[
\frac{Ax+B}{x^2+bx+c},
\]  where $A$ and $B$ are constants to be determined.
We illustrate this in the next an example.

\paragraph{Example 3.} To perform the partial fraction decomposition of
\[
\frac{x}{(x^2+1)(x-2)},
\]
we find $A$,$B$ and $C$ such that 
\[\frac{x}{(x^2+1)(x-2)}= \frac{Ax+B}{x^2+1} + \frac{C}{x-2}.\] Notice the term \(\dfrac{C}{x-2}\) corresponding to the linear factor $x-2$ of the denominator.
Multiplying through by $(x^2+1)(x-2)$ yields
\[
x = (Ax+B)(x-2) + C(x^2+1).
\] 
Comparing coefficients gives
\[
A = \dfrac13, \qquad B = \dfrac23, \qquad C = -\dfrac13,
\]
so
\[
\frac{x}{(x^2+1)(x-2)}
= \frac{\tfrac13 x + \tfrac23}{x^2+1} - \frac{1}{3(x-2)}.
\]

\bigskip

%--------------------------------------------------------------
\subsection*{4. Repeated Irreducible Quadratic Factors}

If \(Q(x) = (x^2+bx+c)^k,\) where the quadratic $x^2+bx+c$ is irreducible and repeated with multiplicity $k,$  
then the decomposition \(\displaystyle\frac{P(x)}{Q(x)}\)  is of the form
\[
\frac{P(x)}{(x^2+bx+c)^k}
= \frac{A_1x+B_1}{x^2+bx+c}
+ \frac{A_2x+B_2}{(x^2+bx+c)^2}
+ \cdots +
\frac{A_kx+B_k}{(x^2+bx+c)^k},
\]
with the constants $A_1, \dots, A_k$ and $B_1, \dots, B_k$ are to be determined. We illustrate this in the next example.

\paragraph{Example 4.} To perform the partial fraction decomposition of 
\[
\frac{3x+5}{(x^2+1)^2},
\]
we find constants $A, B, C, D$ such that 
\[\frac{3x+5}{(x^2+1)^2}= \frac{Ax+B}{x^2+1} + \frac{Cx+D}{(x^2+1)^2}.\]
Multiplying both sides by $(x^2+1)^2$ gives
\[
3x+5 = (Ax+B)(x^2+1) + (Cx+D).
\]
Expanding and matching coefficients yields
\[
A=0,\quad B=3,\quad C=-3,\quad D=5,
\]
and hence
\[
\frac{3x+5}{(x^2+1)^2}
= \frac{3}{x^2+1} + \frac{-3x+5}{(x^2+1)^2}.
\]

\bigskip

%--------------------------------------------------------------
\subsection*{5. Mixed Factors (General Case)}

If the denominator $Q(x)$ factors as\footnote[1]{The symbol $\prod_{i=1}^\ell p_i$ denotes product $p_1p_2 \dots p_\ell$.}
\[
Q(x)=
\prod_{i=1}^{\ell_1} (x-a_i)^{k_i}
\prod_{j=1}^{\ell_2} (x^2 + b_jx + c_j)^{m_j},
\] 
then the full partial fraction decomposition of \(\displaystyle\frac{P(x)}{Q(x)}\) contains terms

\begin{itemize}
	\item    $ \dfrac{A}{x-a}$, 
	$\dfrac{A}{(x-a)^2}$, etc., and
	\item  $ \dfrac{Ax+B}{x^2+bx+c}$,
	$\dfrac{Ax+B}{(x^2+bx+c)^2}$, etc.
\end{itemize}
Let us illustrate this in the next example

\paragraph{Example 5.} To perform the partial fraction decomposition of 
\[
\frac{2x^2+3x+1}{(x-1)(x^2+1)},
\]
we find constants $A, B, C$ such that 
\[\frac{2x^2+3x+1}{(x-1)(x^2+1)}= \frac{A}{x-1} + \frac{Bx+C}{x^2+1}.\]
Multiplying both sides by $(x-1)(x^2+1)$ gives
\[
2x^2+3x+1 = A(x^2+1) + (Bx+C)(x-1).
\]
Expanding and comparing coefficients yields
\[
A = 1,\qquad B = 1,\qquad C = 2,
\]
so
\[
\frac{2x^2+3x+1}{(x-1)(x^2+1)}
= \frac{1}{x-1} + \frac{x+2}{x^2+1}.
\]

\bigskip



% ====================================================================
\section*{Exercises}
% ====================================================================

\resetitems
\textbf{Decompose each rational function into partial fractions.}

\begin{enumerate}
	\item $\displaystyle \frac{5x+3}{(x-1)(x+4)}$
	\item $\displaystyle \frac{7}{(x+2)^2}$
	\item $\displaystyle \frac{3x^2+2x+1}{x(x^2+4)}$
	\item $\displaystyle \frac{x^2+5}{(x^2+1)^2}$
	\item $\displaystyle \frac{4x+9}{(x-1)^3}$
	\item $\displaystyle \frac{2x^2-1}{(x^2+1)(x-3)}$
	\item $\displaystyle \frac{x^3+1}{x^2-1}$
	\item $\displaystyle \frac{6x}{(x^2+2x+5)(x-1)}$
\end{enumerate}

\subsection*{Answers}

\begin{enumerate}[label=\arabic*.]
	\item $\displaystyle \frac{2}{x-1}+\frac{3}{x+4}$
	\item $\displaystyle \frac{7}{(x+2)^2}$
	\item $\displaystyle \frac{1}{x}+\frac{x+2}{x^2+4}$
	\item $\displaystyle \frac{5}{x^2+1}-\frac{x}{(x^2+1)^2}$
	\item $\displaystyle \frac{A}{x-1}+\frac{B}{(x-1)^2}+\frac{C}{(x-1)^3}$ (determine $A, B, C$)
	\item $\displaystyle \frac{1}{x-3}+\frac{x+3}{x^2+1}$
	\item Divide first: $x+1+\frac{2}{x^2-1}$
	\item $\displaystyle \frac{2}{x-1}+\frac{-2x+1}{x^2+2x+5}$
\end{enumerate}
%	
%	% ====================================================================
%	\section*{Flashcard Summary (Cheat Sheet)}
%	% ====================================================================
%	
%	\textbf{Goal:} Decompose $\displaystyle \frac{P(x)}{Q(x)}$ into partial fractions when the degree of $P(x)$ is less than that of $Q(x).$
%	
%	\begin{itemize}[noitemsep]
	%		\item Distinct linear factors:  
	%		$\displaystyle \frac{A}{x-a}$
	%		\item Repeated linear factors:  
	%		$\displaystyle \frac{A_1}{x-a}+\frac{A_2}{(x-a)^2}+\cdots$
	%		\item Irreducible quadratic factor:  
	%		$\displaystyle \frac{Ax+B}{x^2+bx+c}$
	%		\item Repeated irreducible quadratic:  
	%		$\displaystyle \sum \frac{A_ix+B_i}{(x^2+bx+c)^i}$
	%	\end{itemize}
%	
%\noindent\textbf{Steps:}
%	\begin{enumerate}[noitemsep]
	%		\item Factor denominator.
	%		\item Write correct partial fraction decomposition  form.
	%		\item Multiply through to eliminate denominators.
	%		\item Solve for unknown coefficients.
	%		\item Write down the full decomposition.
	%	\end{enumerate}

% ====================================================================




\chapter{Formulas}
\section{Fundamental Laplace Transforms}

%\begin{table}[htbp]
	%\centering
	\renewcommand{\arraystretch}{2.25}
	\setlength{\tabcolsep}{6pt}
	%\setcounter{rownum}{5}
	\rowcolors{2}{tablegray}{white}

	\begin{tabular}{llc} 
	%\begin{tabular}{ll@{\hskip .4cm}c}
		%\toprule
		\specialrule{1pt}{5pt}{-2pt} % Thick rule, 5pt space above, -2pt below (example)
 		& $f(t)$  & $\mathscr{L}\{f(t)\}=F(s)$ \\ [-1ex]
		\midrule
		%\specialrule{0.4pt}{0pt}{1pt}
		\nextitem  & $1$   &$\dfrac{1}{s},$ \quad $s>0$ \\ 
		\nextitem  & $t$ & $\dfrac{1}{s^{2}},$ \quad $s>0$  \\ 
		\nextitem  & $t^{n},\;  n\mbox{ being a natural number}$ & $\dfrac{n!}{s^{\,n+1}},$ \quad $s>0$  \\
		\nextitem   & $t^{-1/2}$ & $\dfrac{\sqrt{\pi}}{\sqrt{s}},$ \quad $s>0$  \\
		\nextitem   & $t^{1/2}$ & $\dfrac{\sqrt{\pi}}{2\,s^{3/2}},$ \quad $s>0$  \\
		\nextitem   & $t^{\alpha},\ \alpha>-1$ & $\dfrac{\Gamma(\alpha+1)}{s^{\alpha+1}},$ \quad $s>0$  \\
		\nextitem   & $\sin(kt)$ & $\dfrac{k}{s^{2}+k^{2}},$ \quad $s>0$  \\
		\nextitem  & $\cos(kt)$ & $\dfrac{s}{s^{2}+k^{2}},$ \quad $s>0$ \\
		\nextitem   & $\sin^{2}(kt)$ & $\dfrac{2k^{2}}{s(s^{2}+4k^{2})},$ \quad $s>0$  \\
		\nextitem  & $\cos^{2}(kt)$ & $\dfrac{s^{2}+2k^{2}}{s(s^{2}+4k^{2})},$ \quad $s>0$  \\
		%\bottomrule
	\end{tabular}
	%\caption{Basic Laplace transforms (including error-function-related formulas).}
%\end{table}

%\begin{table}[htbp]
\renewcommand{\arraystretch}{2.25}
\rowcolors{2}{tablegray}{white}
\setlength{\tabcolsep}{6pt}
	\begin{tabular}{llc} 
		%\begin{tabular}{ll@{\hskip .4cm}c}
		%\toprule 
		\specialrule{1pt}{5pt}{-2pt} % Thick rule, 5pt space above, -2pt below (example)
		 & $f(t)$  & $\mathscr{L}\{f(t)\}=F(s)$ \\[-1ex]
		\midrule
		\nextitem  & $e^{at}$ & $\dfrac{1}{s-a},$ \quad $s>a$  \\
		\nextitem  & $\sinh(kt)$ & $\dfrac{k}{s^{2}-k^{2}},$ \quad $s>\abs{k}$  \\ 
		\nextitem  & $\cosh(kt)$ & $\dfrac{s}{s^{2}-k^{2}},$ \quad $s>\abs{k}$ \\ 
		\nextitem  & $\sinh^{2}(kt)$ & $\dfrac{2k^{2}}{s(s^{2}-4k^{2})},$ \quad $s>\abs{k}$ \\ 
		\nextitem  & $\cosh^{2}(kt)$ & $\dfrac{s^{2}-2k^{2}}{s(s^{2}-4k^{2})},$ \quad $s>\abs{k}$ \\
		\nextitem  & $t e^{at}$ & $\dfrac{1}{(s-a)^{2}},$ \quad $s>a$ \\ 
		\nextitem & $t^{n}e^{at},\;  n\mbox{ being a natural number}$ & $\dfrac{n!}{(s-a)^{n+1}},$ \quad $s>a$ \\ 
		\nextitem & $e^{at}\sin(kt)$ & $\dfrac{k}{(s-a)^{2}+k^{2}},$ \quad $s>a$ \\ 
		\nextitem & $e^{at}\cos(kt)$ & $\dfrac{s-a}{(s-a)^{2}+k^{2}},$ \quad $s>a$ \\ 
		\nextitem & $e^{at}\sinh(kt)$ & $\dfrac{k}{(s-a)^{2}-k^{2}},$ \quad $s>a+\abs{k}$ \\ 
		\nextitem & $e^{at}\cosh(kt)$ & $\dfrac{s-a}{(s-a)^{2}-k^{2}},$ \quad $s>a+\abs{k}$ \\ 
		\nextitem & $t\sin(kt)$ & $\dfrac{2ks}{(s^{2}+k^{2})^{2}},$ \quad $s>0$  \\ 
		\nextitem & $t\cos(kt)$ & $\dfrac{s^{2}-k^{2}}{(s^{2}+k^{2})^{2}},$ \quad $s>0$  \\ 
		\nextitem & $\sin(kt)+kt\cos(kt)$ &
		$\dfrac{2k s^{2}}{(s^{2}+k^{2})^{2}},$ \quad $s>0$  \\ 
		\nextitem & $\sin(kt)-kt\cos(kt)$ &
		$\dfrac{2k^{3}}{(s^{2}+k^{2})^{2}},$ \quad $s>0$  \\ 
			\nextitem & $t\sinh(kt)$ & $\dfrac{2ks}{(s^{2}-k^{2})^{2}},$ \quad $s>\abs{k}$ \\ 
		\nextitem & $t\cosh(kt)$ & $\dfrac{s^{2}+k^{2}}{(s^{2}-k^{2})^{2}},$ \quad $s>\abs{k}$ \\ 
		\nextitem & $\dfrac{e^{at}-e^{bt}}{a-b}$ &
		$\dfrac{1}{(s-a)(s-b)},$ \quad $s>\max\{a, b\}$  \\ 
\end{tabular}
%\caption{Basic Laplace transforms (including error-function-related formulas).}
%\end{table}

%\begin{table}[htbp]
	\renewcommand{\arraystretch}{2.25}
	\rowcolors{2}{tablegray}{white}
	\setlength{\tabcolsep}{10pt}
	\begin{tabular}{llc} 
		%\begin{tabular}{ll@{\hskip .4cm}c}
		%\toprule
		\specialrule{1pt}{5pt}{-2pt} % Thick rule, 5pt space above, -2pt below (example) 
		& $f(t)$  & $\mathscr{L}\{f(t)\}=F(s)$ \\[-1ex]
		\midrule
	
		\nextitem & $\dfrac{ae^{at}-be^{bt}}{a-b}$ &
		$\dfrac{s}{(s-a)(s-b)},$ \quad $s>\max\{a, b\}$ \\ 
		\nextitem & $1-\cos(kt)$ & $\dfrac{k^{2}}{s(s^{2}+k^{2})}, $ \quad $s>0$ \\ 
		\nextitem & $kt-\sin(kt)$ & $\dfrac{k^{3}}{s^{2}(s^{2}+k^{2})}, $ \quad $s>0$ \\ 
		\nextitem & $\dfrac{a\sin(bt)-b\sin(at)}{ab(a^{2}-b^{2})}$ &
		$\dfrac{1}{(s^{2}+a^{2})(s^{2}+b^{2})}, $ \quad $s>0$\\ 
		\nextitem  & $\dfrac{\cos(bt)-\cos(at)}{a^{2}-b^{2}}$ &
		$\dfrac{s}{(s^{2}+a^{2})(s^{2}+b^{2})},$ \quad $s>0$ \\ 
		\nextitem & $\sin(kt)\sinh(kt)$ &
		$\dfrac{2k^{2}s}{s^{4}-4k^{4}}$ \\ 
		\nextitem & $\sin(kt)\cosh(kt)$ &
		$\dfrac{k(s^{2}+2k^{2})}{s^{4}-4k^{4}}$ \\ 
		\nextitem & $\cos(kt)\sinh(kt)$ &
		$\dfrac{k(s^{2}-2k^{2})}{s^{4}-4k^{4}}$ \\ 
		\nextitem & $\cos(kt)\cosh(kt)$ &
		$\dfrac{s^{3}}{s^{4}-4k^{4}}$ \\ 
%		\nextitem & $J_{0}(kt)$ &
%		$\dfrac{1}{\sqrt{s^{2}+k^{2}}}$ \\ 
		\nextitem & $\dfrac{e^{bt}-e^{at}}{t}$ &
		$\ln\!\left(\dfrac{s-a}{s-b}\right)$ \\ 
		\nextitem & $\dfrac{2(1-\cos(kt))}{t}$ &
		$\ln\!\left(\dfrac{s^{2}+k^{2}}{s^{2}}\right)$ \\ 
		\nextitem & $\dfrac{2(1-\cosh(kt))}{t}$ &
		$\ln\!\left(\dfrac{s^{2}}{s^{2}-k^{2}}\right)$ \\ 
		\nextitem & $\dfrac{\sin(at)}{t}$ &
		$\arctan\!\left(\dfrac{a}{s}\right)$ \\ 
		\nextitem & $\dfrac{\sin(at)\cos(bt)}{t}$ &
		$\dfrac12\!\left[
		\arctan\!\left(\dfrac{a+b}{s}\right)
		+\arctan\!\left(\dfrac{a-b}{s}\right)
		\right]$ \\ 
		\nextitem & $\displaystyle \operatorname{erf}\!\left(\frac{a}{2\sqrt{t}}\right)$ &
		$\dfrac{1 - e^{-a\sqrt{s}}}{s}$ \\ 
		\nextitem & $\displaystyle \operatorname{erfc}\!\left(\frac{a}{2\sqrt{t}}\right)$ &
		$\dfrac{e^{-a\sqrt{s}}}{s}$ \\ 
		\nextitem & $\displaystyle \frac{1}{\sqrt{\pi t}}\,
		\exp\!\left(-\frac{a^{2}}{4t}\right)$ &
		$\dfrac{e^{-a\sqrt{s}}}{\sqrt{s}}$ \\ 
%		\nextitem & $\displaystyle t^{-1/2}\exp\!\left(-\frac{a^{2}}{4t}\right)$ &
% 	$\sqrt{\pi}\,\dfrac{e^{-a\sqrt{s}}}{\sqrt{s}}$ \\ 
%		\nextitem & $\displaystyle \frac{a}{2\sqrt{\pi}}\,
%		t^{-3/2}\exp\!\left(-\frac{a^{2}}{4t}\right)$ &
%		$e^{-a\sqrt{s}}$ \\ 
%		\nextitem & $\displaystyle
%		\left(\frac{a^{2}}{4t}-\frac12\right)
%		\frac{\exp\!\left(-\frac{a^{2}}{4t}\right)}%
%		{\sqrt{\pi}\,t^{3/2}}$ &
%		$\sqrt{s}\,e^{-a\sqrt{s}}$ \\ 
		%\bottomrule
\end{tabular}
%\caption{Basic Laplace transforms (including error-function-related formulas).}
%\end{table}

%\begin{table}[htbp]
\renewcommand{\arraystretch}{2.25}
\rowcolors{2}{tablegray}{white}
\setlength{\tabcolsep}{6pt}
	\begin{tabular}{cll}
		%\toprule
		\specialrule{1pt}{5pt}{-2pt} % Thick rule, 5pt space above, -2pt below (example)
			& $f(t)$  & $\mathscr{L}\{f(t)\}=F(s)$ \\[-1ex]
		\midrule
	\nextitem & $e^{at}f(t)$ &
		$F(s-a)$ \\ 
		\nextitem & $\mathcal{U}(t-a)$ &
		$\dfrac{e^{-as}}{s}$ \\ 
		\nextitem & $f(t-a)\,\mathcal{U}(t-a)$ &
		$e^{-as}F(s)$ \\ 
		\nextitem & $g(t)\,\mathcal{U}(t-a)$ &
		$e^{-as}\,\mathscr{L}\{g(t+a)\}$ \\ 
		\nextitem & $y^{(n)}(t)$ &
		$s^{n}Y(s)-s^{n-1}y(0)-\cdots-y^{(n-1)}(0)$ \\ 
		\nextitem & $t^{n}f(t)$ &
		$(-1)^{n}F^{(n)}(s)$ \\ 
		\nextitem &$\displaystyle\frac{f(t)}{t}$& $\displaystyle\int_s^\infty F(p)\,dp$\\
		\nextitem & $(f*g)(t)=\displaystyle\int_{0}^{t}
		f(\tau)g(t-\tau)\,d\tau$ &
		$F(s)G(s)$ \\ 
		\nextitem & $f\left( {t + T} \right) = f\left( t \right)$&$\displaystyle\frac{1}{1- e^{-sT}}\int_0^T f(t) e^{-st}\, dt$\\
		\nextitem & $\delta(t)$ & $1$ \\ 
		\nextitem & $\delta(t-a)$ & $e^{-as}$ \\ 
		%\bottomrule
		\specialrule{1pt}{5pt}{-2pt} % Thick rule, 5pt space above, -2pt below (example)
	\end{tabular}
	%\caption{Operational properties of the Laplace transform.}
%\end{table}



\clearpage

\section{Derivatives}

We notations $\dfrac{d}{dx}f(x)$ and $f'(x)$  synonymously denote the derivative of $f(x)$ with respect to $x.$
\resetitems
\vspace{1em}

%-------------------------------------------------
\noindent\textbf{Differentiation Rules}\\
\renewcommand{\arraystretch}{2.25}
\normalsize
\begin{tabular}{l}
	\nextitem\ 
	Constant:\quad
	$\displaystyle \frac{d}{dx}c = 0$\\
	\nextitem\ 
	Constant Multiple:\quad
	$\displaystyle \frac{d}{dx}\big(cf(x)\big) = c f'(x)$\\
	
	\nextitem\ 
	Sum:\quad
	$\displaystyle \frac{d}{dx}\!\big(f(x)\pm g(x)\big)
	= f'(x)\pm g'(x)$\\	
	\nextitem\ 
	Product:
	$\displaystyle \frac{d}{dx}\big(f(x)g(x)\big)
	= f'(x)g(x) + g'(x)f(x)$\\

	\nextitem\ 
	Quotient:\quad
	$\displaystyle \frac{d}{dx}\left(\frac{f(x)}{g(x)}\right)
	= \frac{g(x)f'(x)-f(x)g'(x)}{[g(x)]^{2}}$\\
	
	\nextitem\ 
	Chain:\quad
	$\displaystyle \frac{d}{dx}f(g(x)) = f'(g(x))\,g'(x)$\\
	
	\nextitem\ 
	Power:\quad
	$\displaystyle \frac{d}{dx}x^{n} = n x^{n-1}$\\
	\nextitem\ 
	Power:\quad
	$\displaystyle \frac{d}{dx}[g(x)]^{n}
	= n[g(x)]^{n-1}g'(x)$
\end{tabular}

\vspace{1em}

%-------------------------------------------------

\noindent\textbf{Derivatives of Trigonometric Functions}\\
\newcolumntype{?}{!{\vrule width 1.5pt}}
\begin{tabular}{@{}l?l@{}}
	\nextitem\ $\displaystyle \frac{d}{dx}\sin x = \cos x$	&
	\nextitem\ $\displaystyle \frac{d}{dx}\cos x = -\sin x$\\

	\nextitem\ $\displaystyle \frac{d}{dx}\tan x = \sec^{2}x$
&
	\nextitem\ $\displaystyle \frac{d}{dx}\cot x = -\csc^{2}x$\\
	
	\nextitem\ $\displaystyle \frac{d}{dx}\sec x = \sec x\tan x$
	&
	\nextitem\ $\displaystyle \frac{d}{dx}\csc x = -\csc x\cot x$\\
\end{tabular}

\vspace{0.8em}

\noindent\textbf{Derivatives of Inverse Trigonometric Functions}\\
\newcolumntype{?}{!{\vrule width 1.5pt}}
\begin{tabular}{@{}l?l@{}}
	\nextitem\ $\displaystyle \frac{d}{dx}\sin^{-1}x
	= \frac{1}{\sqrt{1-x^{2}}}$
	&
	\nextitem\ $\displaystyle \frac{d}{dx}\cos^{-1}x
	= -\frac{1}{\sqrt{1-x^{2}}}$\\

	\nextitem\ $\displaystyle \frac{d}{dx}\tan^{-1}x
	= \frac{1}{1+x^{2}}$
	&
	
	\nextitem\ $\displaystyle \frac{d}{dx}\cot^{-1}x
	= -\frac{1}{1+x^{2}}$\\
	
	\nextitem\ $\displaystyle \frac{d}{dx}\sec^{-1}x
	= \frac{1}{|x|\sqrt{x^{2}-1}}$
	&
	\nextitem\ $\displaystyle \frac{d}{dx}\csc^{-1}x
	= -\frac{1}{|x|\sqrt{x^{2}-1}}$
\end{tabular}

\vspace{0.8em}
\clearpage
\noindent\textbf{Derivatives of Hyperbolic Functions}\\
\newcolumntype{?}{!{\vrule width 1.5pt}}
\begin{tabular}{@{}l?l@{}}
	\nextitem\ $\displaystyle \frac{d}{dx}\sinh x = \cosh x$
	&
	\nextitem\ $\displaystyle \frac{d}{dx}\cosh x = \sinh x$\\
	
	\nextitem\ $\displaystyle \frac{d}{dx}\tanh x = \sech^{2}x$
	&
	
	\nextitem\ $\displaystyle \frac{d}{dx}\coth x = -\csch^{2}x$\\
	
	\nextitem\ $\displaystyle \frac{d}{dx}\sech x = -\sech x\tanh x$
	&
	\nextitem\ $\displaystyle \frac{d}{dx}\csch x = -\csch x\coth x$
\end{tabular}

\vspace{0.8em}

\noindent\textbf{Derivatives of Inverse Hyperbolic Functions}\\
\newcolumntype{?}{!{\vrule width 1.5pt}}
\begin{tabular}{@{}l?l@{}}
	\nextitem\ $\displaystyle \frac{d}{dx}\sinh^{-1}x
	= \frac{1}{\sqrt{x^{2}+1}}$
	&
	\nextitem\ $\displaystyle \frac{d}{dx}\cosh^{-1}x
	= \frac{1}{\sqrt{x^{2}-1}}$\\
	
	\nextitem\ $\displaystyle \frac{d}{dx}\tanh^{-1}x
	= \frac{1}{1-x^{2}}$
	&
	
	\nextitem\ $\displaystyle \frac{d}{dx}\coth^{-1}x
	= \frac{1}{1-x^{2}}$\\
	
	\nextitem\ $\displaystyle \frac{d}{dx}\sech^{-1}x
	= -\frac{1}{x\sqrt{1-x^{2}}}$
	&
	\nextitem\ $\displaystyle \frac{d}{dx}\csch^{-1}x
	= -\frac{1}{|x|\sqrt{x^{2}+1}}$
\end{tabular}

\vspace{0.8em}

\noindent\textbf{Derivatives of Exponential Functions}\\
\begin{tabular}{@{}ll@{}}
	\nextitem\ $\displaystyle \frac{d}{dx}e^{x} = e^{x}$
	&
	\nextitem\ $\displaystyle \frac{d}{dx}b^{x} = b^{x}\ln b$
\end{tabular}

\vspace{0.8em}

\noindent\textbf{Derivatives of Logarithmic Functions}\\
\begin{tabular}{@{}l l@{}}
	\nextitem\ $\displaystyle \frac{d}{dx}\ln|x| = \frac{1}{x}$
	&
	\nextitem\ $\displaystyle \frac{d}{dx}\log_{b}x = \frac{1}{x\ln b}$
\end{tabular}
\clearpage

\section{Integrals}
\resetitems
\newcolumntype{?}{!{\vrule width 1.5pt}}
\begin{tabular}{|@{}l?l@{}}
%\begin{tabular}{l}
	% 1–2
	\nextitem\ $\displaystyle
	\int u^{n}\,du = \frac{u^{n+1}}{n+1}+C,\quad n\neq -1$
	&
	\nextitem\ $\displaystyle
	\int \frac{1}{u}\,du = \ln|u| + C$\\
	
	% 3–4
	\nextitem\ $\displaystyle
	\int e^{u}\,du = e^{u}+C$
	&
	\nextitem\ $\displaystyle
	\int a^{u}\,du = \frac{1}{\ln a}\,a^{u}+C$\\
	
	% 5–6
	\nextitem\ $\displaystyle
	\int \sin u\,du = -\cos u + C$
	&
	\nextitem\ $\displaystyle
	\int \cos u\,du = \sin u + C$\\
	
	% 7–8
	\nextitem\ $\displaystyle
	\int \sec^{2}u\,du = \tan u + C$
	&
	\nextitem\ $\displaystyle
	\int \csc^{2}u\,du = -\cot u + C$\\
	
	% 9–10
	\nextitem\ $\displaystyle
	\int \sec u \tan u\,du = \sec u + C$
	&
	\nextitem\ $\displaystyle
	\int \csc u \cot u\,du = -\csc u + C$\\
	
	% 11–12
	\nextitem\ $\displaystyle
	\int \tan u\,du = -\ln|\cos u| + C$
	&
	\nextitem\ $\displaystyle
	\int \cot u\,du = \ln|\sin u| + C$\\
	
	% 13–14
	\nextitem\ $\displaystyle
	\int \sec u\,du = \ln|\sec u + \tan u| + C$
	&
	\nextitem\ $\displaystyle
	\int \csc u\,du = \ln|\csc u - \cot u| + C$\\
	
	% 15–16
	\nextitem\ $\displaystyle
	\int u\sin u\,du = \sin u - u\cos u + C$
	&
	\nextitem\ $\displaystyle
	\int u\cos u\,du = \cos u + u\sin u + C$\\
	
	% 17–18
	\nextitem\ $\displaystyle
	\int \sin^{2}u\,du = \frac12 u - \frac14 \sin 2u + C$
	&
	\nextitem\ $\displaystyle
	\int \cos^{2}u\,du = \frac12 u + \frac14 \sin 2u + C$\\
	
	% 19–20
	\nextitem\ $\displaystyle
	\int \tan^{2}u\,du = \tan u - u + C$
	&
	\nextitem\ $\displaystyle
	\int \cot^{2}u\,du = -\cot u - u + C$\\
	
	
	% 31–32
	\nextitem\ $\displaystyle
	\int \sinh u\,du = \cosh u + C$
	&
	\nextitem\ $\displaystyle
	\int \cosh u\,du = \sinh u + C$\\
	
	% 33–34
	\nextitem\ $\displaystyle
	\int \sech^{2}u\,du = \tanh u + C$
	&
	\nextitem\ $\displaystyle
	\int \csch^{2}u\,du = -\coth u + C$\\
	
	% 35–36
	\nextitem\ $\displaystyle
	\int \tanh u\,du = \ln(\cosh u) + C$
	&
	\nextitem\ $\displaystyle
	\int \coth u\,du = \ln|\sinh u| + C$\\

 \multicolumn{2}{@{}l@{}}{\nextitem\
		  $\displaystyle
		 \int \sin^{3}u\,du =
		 -\frac13\bigl(2+\sin^{2}u\bigr)\cos u + C$}\\
  \multicolumn{2}{@{}l@{}}{\nextitem\
	$\displaystyle
	\int \cos^{3}u\,du =
	\frac13\bigl(2+\cos^{2}u\bigr)\sin u + C$ }\\
\multicolumn{2}{@{}l@{}}{\nextitem\
	$\displaystyle
	\int \tan^{3}u\,du =
	\frac12\tan^{2}u + \ln|\cos u| + C$}\\
\multicolumn{2}{@{}l@{}}{\nextitem\
	$\displaystyle
	\int \cot^{3}u\,du =
	-\frac12\cot^{2}u - \ln|\sin u| + C$}\\
\multicolumn{2}{@{}l@{}}{\nextitem\
	$\displaystyle
	\int \sec^{3}u\,du =
	\frac12\sec u\tan u + \frac12\ln|\sec u + \tan u| + C$}\\
\end{tabular}



\begin{tabular}{l}
\nextitem\ $\displaystyle
\int \csc^{3}u\,du =
-\frac12\csc u\cot u + \frac12\ln|\csc u - \cot u| + C$\\

\nextitem\ $\displaystyle
\int \sin(au)\cos(bu)\,du =
\frac{\sin(a-b)u}{2(a-b)} -
\frac{\sin(a+b)u}{2(a+b)} + C$\\
\nextitem\ $\displaystyle
\int \cos(au)\cos(bu)\,du =
\frac{\sin(a-b)u}{2(a-b)} +
\frac{\sin(a+b)u}{2(a+b)} + C$\\

% 29–30
\nextitem\ $\displaystyle
\int e^{au}\sin(bu)\,du =
\frac{e^{au}}{a^{2}+b^{2}}
\bigl(a\sin bu - b\cos bu\bigr) + C$\\
\nextitem\ $\displaystyle
\int e^{au}\cos(bu)\,du =
\frac{e^{au}}{a^{2}+b^{2}}
\bigl(a\cos bu + b\sin bu\bigr) + C$\\



% 37–38
\nextitem\ $\displaystyle
\int \ln u\,du = u\ln u - u + C$\\
\nextitem\ $\displaystyle
\int u\ln u\,du =
\frac12 u^{2}\ln u - \frac14 u^{2} + C$\\

% 39–40
\nextitem\ $\displaystyle
\int \frac{1}{\sqrt{a^{2}-u^{2}}}\,du =
\frac{1}{a}\sin^{-1}\!\frac{u}{a} + C$\\
\nextitem\ $\displaystyle
\int \frac{1}{\sqrt{a^{2}+u^{2}}}\,du =
\ln\bigl|u+\sqrt{a^{2}+u^{2}}\bigr| + C$\\

% 41–42
\nextitem\ $\displaystyle
\int \sqrt{a^{2}-u^{2}}\,du =
\frac12 u\sqrt{a^{2}-u^{2}}
+ \frac{a^{2}}{2}\sin^{-1}\!\frac{u}{a} + C$\\
\nextitem\ $\displaystyle
\int \sqrt{a^{2}+u^{2}}\,du =
\frac12 u\sqrt{a^{2}+u^{2}}
+ \frac{a^{2}}{2}
\ln\bigl|u+\sqrt{a^{2}+u^{2}}\bigr| + C$\\

% 43–44
\nextitem\ $\displaystyle
\int \frac{1}{a^{2}+u^{2}}\,du =
\frac{1}{a}\tan^{-1}\!\frac{u}{a} + C$\\
\nextitem\ $\displaystyle
\int \frac{1}{a^{2}-u^{2}}\,du =
\frac{1}{2a}\ln\left|\frac{a+u}{\,a-u\,}\right| + C$\\


\end{tabular}



%\begin{tabular}{lll}
%	\hline
%	\multicolumn{2}{l}{Merged Row Content} & Column 2, Row 1 Column 3, Row 1 \\
%	\cline{1-2} % Partial line for columns 2 and 3
%	& Column 2, Row 2 & Column 3, Row 2 \\
%	\hline
%	Regular Row & More Data & Even More Data \\
%	\hline
%\end{tabular}
%\begin{tabular}{ |p{3cm}||p{3cm}|p{3cm}|p{3cm}|  }
%	\hline
%	\multicolumn{4}{|l|}{Country List} \\
%	\hline
%	Country Name or Area Name& ISO ALPHA 2 Code &ISO ALPHA 3 Code&ISO numeric Code\\
%	\hline
%	Afghanistan   & AF    &AFG&   004\\
%	Aland Islands&   AX  & ALA   &248\\
%	Albania &AL & ALB&  008\\
%	Algeria    &DZ & DZA&  012\\
%	American Samoa&   AS  & ASM&016\\
%	Andorra& AD  & AND   &020\\
%	Angola& AO  & AGO&024\\
%	\hline
%\end{tabular}
	

 %Open it
\end{appendices}


\addcontentsline{toc}{chapter}{Index}
    \printindex   
\backmatter
\end{document}